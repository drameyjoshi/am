\documentclass{article}
\usepackage{amsmath}
\usepackage{graphicx}
\begin{document}
The miscellaneous problem in Simmons' book are like problems in combinatorics.
It is easy to understand the statement but it takes some effort to get the
solution. They are not hard but neither are they trivial. I had to spend quite
some time on each one of them to get a headway. The challenge was in getting 
the correct differential equation. As a physicist, I think that translating a
problem statement into a mathematical one is as important as solving the 
mathematical problem.

\begin{enumerate}
\item[Prob 1:] It began to snow on a certain morning and the snow continued 
to fall at a steady rate through out the day. At noon the snowplough started 
to clear a road at a constant rate in terms of volume of snow removed in an 
hour. It cleared 2 miles by 2 PM and another mile by 4 PM. When did it start 
snowing?
\item[Soln: ] Let $t = 0$ be the time at which it started snowing and $t = T$
be when the ploughing began. The snow plougher clears snow at a constant rate.
That is, the volume of snow cleared per unit time is constant. Let the plough
be at a position $x$ at time $t$. In a time interval $\delta t$, let $\delta V$
be the volume cleared. If $h$ is the height of snow at $x(t)$ then it is $h(t)
+ r\delta t$ at at the position $x(t + \delta t)$. If $w$ is the width of the
road then 
\begin{equation}\label{e1}
\delta V = (\delta x)ah(t) + \frac{1}{2}(\delta x)a\delta h.
\end{equation}
In this expression, the first term is the volume of a slice of thickness 
$\delta x$, width $a$ and height $h$. The second term is the wedge of the snow
accumulated on top of this slice as the plough was at work. The cross section 
of the wedge is a triangle of base $\delta x$ and height $h(t + \delta t) - 
h(t)$. If the snow falls at a constant rate $\lambda$, $\delta h = \lambda
\delta t$ so that
\begin{equation}\label{e2}
\delta V = (\delta x)ah(t) + \frac{1}{2}(\delta x)a\lambda(\delta t).
\end{equation}
Dividing both sides by $\delta t$, and taking the limit $\delta t \rightarrow 
0$,
\begin{equation}\label{e3}
\frac{dV}{dt} = ah\frac{dx}{dt} + \frac{\lambda}{2} a\delta x.
\end{equation}
As $\delta t \rightarrow 0, \delta x \rightarrow 0$ so that
\begin{equation}\label{e4}
\mu = ah\frac{dx}{dt}
\end{equation}
Now, $h(t) = \lambda t$, so that
\begin{equation}\label{e5}
\frac{dx}{dt} = \frac{\mu}{a\lambda t} = \frac{\nu}{t},
\end{equation}
where 
\begin{equation}\label{e6}
\nu = \frac{\mu}{a\lambda}
\end{equation}
is a constant. The solution of \eqref{e5} is
\begin{equation}\label{e7}
x(t) = \nu\ln t + \nu\ln\alpha = \nu\ln(\alpha t),
\end{equation}
where $\alpha$ is a constant of integration. Since $x(T) = 0$, $\alpha = 1/T$.
We now use the two observations,
\begin{eqnarray}
x(T + 2) &=& \nu\ln\left(\frac{T + 2}{T}\right) = 2 \label{e8} \\
x(T + 4) &=& \nu\ln\left(\frac{T + 4}{T}\right) = 3 \label{e9}
\end{eqnarray}
The ratio of these equations gives
\begin{equation}\label{e10}
2\ln\frac{T + 4}{T} - 3\ln\frac{T + 2}{T} = 0.
\end{equation}
This equation can be solved numerically or graphically to get $T$. A graph of
the left hand side is shown in figure \ref{f1}.
\begin{figure}
\centering
\includegraphics[scale=0.7]{simmons-p1.png}
\caption{Graphical solution of equation \eqref{e10}}
\label{f1}
\end{figure}
A numerical solution to the equation gives $T \approx 1.236$. As the snow 
plough started working at noon, it started snowing roughly $1.236$ hours prior
to noon, that is around $10:46$ AM.

\end{enumerate}
\end{document}
