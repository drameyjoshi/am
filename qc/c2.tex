\chapter{Linear Vector Spaces in Quantum Mechanics}\label{c2}
Finite dimensional vector spaces play an important role in quantum computation
and information. The previous chapter introduced the linear vector spaces over
fields and their dimensions. In this chapter we will build on those ideas and
use a notation that is more prevalent in quantum computation.

\section{Dirac notation}\label{c2s1}
When we introduced vector spaces in the previous chapter we remarked that 
several objects can be added together and multiplied by numbers and therefore
are eligible to be called vectors. Because of the wide variety of objects that
have characteristics of a vector, mathematicians do not use special notation
to tell vectors from other quantities. On the other hand, physicists 
differentiate between scalars and vectors by typesetting the latter in bold
or putting an arrow on the top. However, the arrow notation is suitable only
for vectors in three dimensions. For the kind of vectors we will encounter
in quantum mechanics it is more convenient to use the notation developed by
P.A.M. Dirac.

If $V$ is a vector space over the field $\soc$ then its members are denoted
by $\ket{v}$. Thus, the defining properties of the vector space are expressed
as $\ket{v_1} + \ket{v_2} \in V$ and $\alpha\ket{v} \in V$ for all $\ket{v},
\ket{v_1}, \ket{v_2} \in V$ and $\alpha \in \soc$. The term $\ket{v}$ is read
as `ket v'. The reason for this unusual name will be clear a little later. In 
the discussion that follows, we will assume that all our vector spaces are
defined over the field of complex numbers.

The reader is probably familiar with the `scalar product' of two vectors in
$\sor^3$. If $\vec{a} = a_x\hat{i} + a_y\hat{j} + a_z\hat{k}$ and $\vec{b} =
b_x\hat{i} + b_y\hat{j} + b_z\hat{k}$ then their scalar product is $\vec{a}
\cdot\vec{b} = a_xb_x + a_yb_y + a_zb_z$. As the name suggests, $\vec{a}\cdot
\vec{b}$ is indeed a scalar. This idea can be extended to general vector spaces.
If $\ket{v_1}, \ket{v_2} \in V$ then their scalar product is denoted by 
$(\ket{v_1}, \ket{v_2})$ and it has the following properties,
\begin{eqnarray}
(\ket{v_1}, \ket{v_2}) &=& (\ket{v_2}, \ket{v_1})^\ast \label{c2s1e1} \\
(\ket{v_1}, \alpha\ket{v_2}) &=& \alpha (\ket{v_1}, \ket{v_2}) \label{c2s1e2} 
\end{eqnarray}
Equation \eqref{c2s1e2} tells that the inner product of $\ket{v_1}$ and
$\alpha\ket{v_2}$ is just $\alpha$ times the inner product of $\ket{v_1}$
and $\ket{v_2}$. In order to find the inner product of $\alpha\ket{v_1}$
and $\ket{v_2}$, we start with
\begin{eqnarray}
(\alpha\ket{v_1}, \ket{v_2}) &=& (\ket{v_2}, \alpha\ket{v_1})^\ast \nonumber \\
 &=& \left\{\alpha(\ket{v_2}, \ket{v_1})\right\}^\ast \nonumber \\
 &=& \alpha^\ast(\ket{v_2}, \ket{v_1})^\ast \nonumber \\
 &=& \alpha^\ast (\ket{v_1}, \ket{v_2}) \label{c2s1e3}
\end{eqnarray}


A `linear functional' over a vector space $V$ over a field $F$ is a mapping
$\phi: V \rightarrow F$ such that $\phi(\alpha_1\ket{v_1} + \alpha_2\ket{v_2})
= \alpha_1\phi_1(\ket{v_1}) + \alpha_2\phi_1(\ket{v_2})$ for all $\ket{v_1},
\ket{v_2} \in V$ and $\alpha_1, \alpha_2 \in F$. Thus, a functional is a linear
mapping. It maps every vector in $V$ to a scalar in $F$. There is a close
connection between linear functionals and inner products. The Riesz representation
theorem \cite{kreyszig1978introductory} guarantees that for every $\ket{v} \in V$
there is a linear functional $\phi_v \in V^\dagger$ (the space $V^\dagger$ is
defined in problem \ref{c2s1p1}) such that for all $\ket{u} \in V$,
\begin{equation}\label{c2s1e4}
\phi_v(\ket{u}) = (\ket{v}, \ket{u}).
\end{equation}
The functional $\phi_v$ is denoted by $\bra{v}$, read as `bra v', in Dirac 
notation and equation \eqref{c2s1e4} is written more compactly as
\begin{equation}\label{c2e1e5}
\phi_v(\ket{u}) = (\ket{v}, \ket{u}) = \braket{v}{u}.
\end{equation}
Thus, the ket $\ket{u}$ is a vector, the bra $\bra{v}$ is a linear functional 
and the `bra(c)ket' $\braket{v}{u}$ is the action of the linear functional on
the vector. $\braket{v}{u}$ is a scalar, that is, a complex number.

If $\phi_u$ is the linear functional corresponding to $\ket{u}$ then
\[
\phi_u(\ket{v}) = (\ket{u}, \ket{v}) = \braket{u}{v}.
\]
But we know that $(\ket{u}, \ket{v}) = (\ket{v}, \ket{u})^\ast$, while the 
latter, from \eqref{c2e1e5}, is $(\braket{v}{u})^\ast$. Thus, we have the 
important relation
\begin{equation}\label{c2e1e6}
\braket{u}{v} = \braket{v}{u}^\ast.
\end{equation}

In terms of the bras and the kets we can state Riesz representation theorem
as `for every $\ket{v}\in V$ there is a linear functional $\bra{v} \in 
V^\dagger$ such that
\begin{equation}\label{c2s1e7}
\bra{v}\left(\ket{u}\right) = (\ket{v}, \ket{u}) = \braket{v}{u}.
\end{equation}
Problem \ref{c2s1p3} shows that the bra corresponding to $\ket{v_1} + \ket{v_2}$
is $\bra{v_1} + \bra{v_2}$ and that corresponding to $\alpha\ket{v}$ is 
$\alpha^\ast\bra{v}$.


\subsection{Problems}
\begin{enumerate}
\item\label{c2s1p1} Let $V^\dagger$ be the set of all linear functionals of the 
vector space $V$ over the field F. Show that $V^\dagger$ is also a vector space 
over the same field. It is called the `dual space' of the space $V$.
\item Explain why each step leading to equation \eqref{c2s1e3} is correct.  
\item\label{c2s1p3} Using the argument preceeding equation \eqref{c2s1e7} show that the bra
corresponding to $\ket{v_1} + \ket{v_2}$ is $\bra{v_1} + \bra{v_2}$. Extend the
argument to show that the bra corresponding to $\alpha\ket{v}$ is $\alpha^\ast
\bra{v}$.
\end{enumerate}

\section{Operators}\label{c2s2}
