\chapter{Mathematical Preliminaries}\label{c1}
Quantum computation is a mathematical model of computing based on quantum mechanics.
A thorough understanding of quantum computing needs familiarity with the language of
mathematics. This chapter will cover all the mathematics we will need in our course.
\section{Sets}\label{c1s1}
We will take a naive view of sets as collections of objects. We will not give a 
definition of a set but rather take it as a concept that we understand. If $A$ is a
set and an element $x$ is a member of $A$ then we express it as $x \in A$. If $y$
is not a member of $A$ then we express it as $y \notin A$. A set $B$ is a subset of
a set $A$ if for all $x \in B, x \in A$. It is written as $B \subset A$. Sets $A$ and
$B$ are equal if $A \subset B$ and $B \subset A$. Sets are either defined by listing
all their elements or using the notation
\[
A = \{x : P(x) \text{ is true.}\},
\]
where $P$ is some property of the elements $x$ that is either true or false. The 
union of sets $A$ and $B$ is
defined as
\begin{equation}\label{c1s1e1}
A \cup B = \{x : x \in A \text{ or } x \in B\}
\end{equation}
and their intersection is defined as
\begin{equation}\label{c1s1e2}
A \cap B = \{x : x \in A \text{ and } x \in B\}.
\end{equation}
The difference of sets $A$ and $B$ is defined as
\begin{equation}\label{c1s1e3}
A - B = \{x : x \in A \text{ and } x \notin B\}.
\end{equation}
The symmetric differences of $A$ and $B$ is defined as
\begin{equation}\label{c1s2e4}
A \Delta B = (A - B) \cup (B - A).
\end{equation}
Often times $A$ is a subset of a bigger set $X$, in which case we define
the complement of $A$ as
\begin{equation}\label{c1s2e5}
A^c = X - A.
\end{equation}
A set with no elements is called an empty set and is denoted by $\varnothing$.

Set theory starts off with simple concepts like these and suddenly takes plunge 
into very deep concepts. Although very interesting, we will stay focused on our
agenda and will only refer to excellent texts like \cite{halmos1960naive} and
\cite{lewis1981elements}.

\subsection{Problems}
$A, B, C$ and $X$ are sets in all the problems in this subsection.
\begin{enumerate}
\item If $A = \{1, 3, 4, 5, 9\}, B = \{2, 3, 5, 10\}$ find $A \cup B, A \cap B,
A - B, B - A, A \Delta B$.
\item Show that $A \subset A \cup B$ and $A \cap B \subset A$, where $A$ and $B$ are
sets.
\item Express the set difference operations $A - B, B - A, A \Delta B$ in terms of
relational operations of SQL.
\item Show that $A \Delta B = B \Delta A$.
\item If $A, B \in X$ show that $(A \cup B)^c = A^c \cap B^c$ and $(A \cap B)^c 
= A^c \cup B^c$. These relations are called De Morgan's laws.
\item If $P$ and $Q$ are logical statements, that is they are either true or false, 
then we say that $P \Rightarrow Q$ if either $Q$ is true or $P$ is false. If $P$ is
the statement $x \in \varnothing$ and $Q$ is the statement $x \in A$, where $A$ is 
a set then show that $P \Rightarrow Q$. This proves that an empty set is a subset 
of any set.
\item If $A$ has $n$ elements then show that it has $2^n$ subsets.
\end{enumerate}

\section{Numbers}\label{c1s2}
It is possible to begin with set theory and define natural numbers, the integers,
the rational numbers, the real numbers and the complex numbers. The reader may
refer to \cite{tao2009analysis} to see how this is done. We will just describe these
sets informally as
\begin{eqnarray}
\son &=& \{1, 2, 3, \ldots\} \label{c1s2e1} \\
\soi &=& \{\ldots, -2, -1, 0, 1, 2, \ldots\} \label{c1s2e2} \\
\soq &=& \{m/n : m, n \in \soi, n \ne 0\} \label{c1s2e3} 
\end{eqnarray}
$\son$ is the set of natural numbers, $\soi$ that of integers and $\soq$ that of
rational numbers. The adjective `rational' means that the number is expressed as 
a ratio of two integers. It is easy to check that $\son \subset \soi \subset \soq$. 

Some authors consider the element $0$ to be in $\son$ but we subscribe to the more 
traditional view. It can be shown that common numbers like $\sqrt{2}$ do not belong 
to $\soq$. We therefore need a bigger set which admits numbers like these. The set 
of real numbers, $\sor$ is defined as a union of the set $\soq$ of rational numbers 
and the  set of irrational numbers like $\sqrt{2}, e, \pi$ etc.

One can view the introduction of a bigger set of numbers stemming from a desire to 
solve equations. Thus, the integers were created to that an equation of the type
$x + a = b$ could be solved when $a > b$. The rational numbers were created to solve
equations of the type $ax + b = c$ where $a \ne 1$ and the real numbers were created
so that an equation of the type $x^2 = a$ has a solution for all $a > 0$. The set of
complex numbers, $\soc$, was created so that equations of the type $x^2 = -a$ have a 
solution for all $a > 0$. In fact, if $\sor[x]$ is the set of all polynomials in $x$ 
with real coefficients then, it can be shown that, the polyomials can be solved in terms
of complex numbers. That is, if $p(x) \in \sor[x]$ then the equation $p(x) = 0$ has
roots in complex numbers.

Although it is not at all obvious, one can show that the size of $\son$ is the same
as the size of $\soi$ and also that of $\soq$. The proof consists in demonstrating
a one-to-one relation between the elements of the three sets. The set $\sor$ is,
however, much bigger than these. $\sor$ and $\soc$, however, have the same size.

\subsection{Problems}
\begin{enumerate}
\item Assume that $\sqrt{2}$ is a rational number. That is, write $\sqrt{2} = m/n$,
where $m, n \in \soi$, $n \ne 0$ and we can cancelled the common factors between $m$
and $n$. Show that this leads to a contradiction and therefore conclude that $\sqrt{2}$
is not a rational number \cite{hardy1992mathematician}.
\end{enumerate}

\section{Properties of complex numbers}\label{c1s3}
Complex numbers play a significant role in quantum computation. We will review their
properties in this section. One can consider a complex number $z$ to be an ordered
pair $(x, y)$ of real numbers. $x$ is called the real part of $z$ and $y$ its imaginary
part. It is common to write $x = \re(z)$ and $y = \im(z)$. It is convenient to write
\begin{equation}\label{c1s3e1}
z = x + iy,
\end{equation}
where the number $i = \sqrt{-1}$. Note that $z \in \soc$ while $x, y \in \sor$. If $y=0$
then $z$ is a real number and if $x = 0$ then $z$ is an imaginary number. The terms
`real' and `imaginary' are just the names given to the numbers. One should refrain from
attaching the meaning of the words to them. 

Common arithmetic operations between complex numbers can be defined in terms of similar
operations between real numbers. If $z_1 = x_1 + iy_1$ and $z_2 = x_2 + iy_2$ then
\begin{enumerate}
\item $z_1 = z_2$ if and only if $x_1 = x_2$ and $y_1 = y_2$,
\item $-z1 = -x_1 - iy_1$,
\item $z_1 \pm z_2 = (x_1 \pm x_2) + i(y_1 \pm y_2)$,
\item $z_1z_2 = (x_1x_2 - y_1y_2) + i(x_1y_2 + x_2y_1)$,
\item If $z_2 \ne 0$ then
\[
\frac{z_1}{z_2} = \frac{(x_1x_2 + y_1y_2) + i(x_2y_1 - x_1y_2)}{\sqrt{x_2^2 + y_2^2}}.
\]
\end{enumerate}

The conjugate of a complex number $z = x + iy$ is defined as
\begin{equation}\label{c1s3e2}
z^\ast = x - iy.
\end{equation}


It is easy to check that every complex number corresponds to a point in the plane and
conversely every point on the plane can be considered to be a complex number. One can
also check that a complex number can be considered as the `radius vector' of a point
in a plane and that the sum of complex numbers can be interpreted as a sum of vectors.

A point in a plane can be described either by its Cartesian coordinates $(x, y)$ or
its planar coordinates $(r, \theta)$ where
\begin{eqnarray}
r &=& \sqrt{x^2 + y^2} \label{c1s3e3} \\
\theta &=& \tan^{-1}\left(\frac{y}{x}\right). \label{c1s3e4}
\end{eqnarray}
Given $(x, y)$ one can find $(r, \theta)$ using equations \eqref{c1s3e3} and 
\eqref{c1s3e4}. If, instead, we were given $(r, \theta)$ then we can get $(x, y)$
using the equations
\begin{eqnarray}
x &=& r\cos\theta \label{c1s3e5} \\
y &=& r\sin\theta. \label{c1s3e6}
\end{eqnarray}
This suggests that one can express a complex number either as 
\begin{equation}\label{c1s3e7}
z = x + iy
\end{equation}
or 
\begin{equation}\label{c1s3e8}
z = r(\cos\theta + i\sin\theta).
\end{equation}
Using the Taylor series for $\cos\theta, \sin\theta$ and $e^{i\theta}$ one can 
show that
\begin{equation}\label{c1s3e9}
e^{i\theta} = \cos\theta + i\sin\theta.
\end{equation}
Equation \eqref{c1s3e7} is called Euler's formula. From equations \eqref{c1s3e8}
and \eqref{c1s3e9} one can infer the `polar-form' of complex numbers
\begin{equation}\label{c1s3e10}
z = re^{i\theta}.
\end{equation}
The number $r \ge 0$ is called its modulus and $\theta$ is called its argument.

\subsection{Problems}
\begin{enumerate}
\item If $z = 0$ then show that $\re(z) = 0$ and $\im(z) = 0$.
\item If $z_1 = 3 + 5i$ and $z_2 = 1 - 2i$, calculate $z_1 + z_2, z_1 - z_2, z_1z_2$
and $z_1/z_2$.
\item If $a \in \sor, z = x + iy \in \soc$ then show that $az = ax + iay$ follows from
the definition of two complex numbers.
\item Show that $(z_1 \pm z_2)^\ast = z_1^\ast \pm z_2^\ast$.
\item Show that $zz^\ast \ge 0$ for all $z \in \soc$. 
\item Show that $e^{i\pi} = -1$ and hence $\ln(-1) = i\pi$. Although logarithms 
of negative numbers do not exist in the set of real numbers, they do in $\soc$. In
fact, $\ln(-1)$ has infinitely many values of the form $i\pi + 2i\pi m$, where 
$m \in \soi$.
\end{enumerate}
\section{Fields}\label{c1s4}
\section{Linear spaces}\label{c1s5}