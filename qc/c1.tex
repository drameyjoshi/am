\chapter{Mathematical Preliminaries}\label{c1}
Quantum computation is a mathematical model of computing based on quantum mechanics.
A thorough understanding of quantum computing needs familiarity with the language of
mathematics. This chapter will cover all the mathematics we will need in our course.
\section{Sets}\label{c1s1}
We will take a naive view of sets as collections of objects. We will not give a 
definition of a set but rather take it as a concept that we understand. If $A$ is a
set and an element $x$ is a member of $A$ then we express it as $x \in A$. If $y$
is not a member of $A$ then we express it as $y \notin A$. A set $B$ is a subset of
a set $A$ if for all $x \in B, x \in A$. It is written as $B \subset A$. Sets $A$ and
$B$ are equal if $A \subset B$ and $B \subset A$. Sets are either defined by listing
all their elements or using the notation
\[
A = \{x : P(x) \text{ is true.}\},
\]
where $P$ is some property of the elements $x$ that is either true or false. The 
union of sets $A$ and $B$ is
defined as
\begin{equation}\label{c1s1e1}
A \cup B = \{x : x \in A \text{ or } x \in B\}
\end{equation}
and their intersection is defined as
\begin{equation}\label{c1s1e2}
A \cap B = \{x : x \in A \text{ and } x \in B\}.
\end{equation}
The difference of sets $A$ and $B$ is defined as
\begin{equation}\label{c1s1e3}
A - B = \{x : x \in A \text{ and } x \notin B\}.
\end{equation}
The symmetric differences of $A$ and $B$ is defined as
\begin{equation}\label{c1s2e4}
A \Delta B = (A - B) \cup (B - A).
\end{equation}
Often times $A$ is a subset of a bigger set $X$, in which case we define
the complement of $A$ as
\begin{equation}\label{c1s2e5}
A^c = X - A.
\end{equation}
A set with no elements is called an empty set and is denoted by $\varnothing$.

Set theory starts off with simple concepts like these and suddenly takes plunge 
into very deep concepts. Although very interesting, we will stay focused on our
agenda and will only refer to excellent texts like \cite{halmos1960naive} and
\cite{lewis1981elements}.

\subsection{Problems}
$A, B, C$ and $X$ are sets in all the problems in this subsection.
\begin{enumerate}
\item If $A = \{1, 3, 4, 5, 9\}, B = \{2, 3, 5, 10\}$ find $A \cup B, A \cap B,
A - B, B - A, A \Delta B$.
\item Show that $A \subset A \cup B$ and $A \cap B \subset A$, where $A$ and $B$ are
sets.
\item Express the set difference operations $A - B, B - A, A \Delta B$ in terms of
relational operations of SQL.
\item Show that $A \Delta B = B \Delta A$.
\item If $A, B \in X$ show that $(A \cup B)^c = A^c \cap B^c$ and $(A \cap B)^c 
= A^c \cup B^c$. These relations are called De Morgan's laws.
\item If $P$ and $Q$ are logical statements, that is they are either true or false, 
then we say that $P \Rightarrow Q$ if either $Q$ is true or $P$ is false. If $P$ is
the statement $x \in \varnothing$ and $Q$ is the statement $x \in A$, where $A$ is 
a set then show that $P \Rightarrow Q$. This proves that an empty set is a subset 
of any set.
\item If $A$ has $n$ elements then show that it has $2^n$ subsets.
\end{enumerate}

\section{Real numbers}\label{c1s2}
It is possible to begin with set theory and define natural numbers, the integers,
the rational numbers, the real numbers and the complex numbers. The reader may
refer to \cite{tao2009analysis} to see how this is done. We will just describe these
sets informally as
\begin{eqnarray}
\son &=& \{1, 2, 3, \ldots\} \label{c1s2e1} \\
\soi &=& \{\ldots, -2, -1, 0, 1, 2, \ldots\} \label{c1s2e2} \\
\soq &=& \{m/n : m, n \in \soi, n \ne 0\} \label{c1s2e3} 
\end{eqnarray}
$\son$ is the set of natural numbers, $\soi$ that of integers and $\soq$ that of
rational numbers. The adjective `rational' means that the number is expressed as 
a ratio of two integers. It is easy to check that $\son \subset \soi \subset \soq$. 

Some authors consider the element $0$ to be in $\son$ but we subscribe to the more 
traditional view. It can be shown that common numbers like $\sqrt{2}$ do not belong 
to $\soq$. We therefore need a bigger set which admits numbers like these. The set 
of real numbers is defined as a union of the set $\soq$ of rational numbers and the 
set of irrational numbers like $\sqrt{2}, e, \pi$ etc. 

Although it is not at all obvious, one can show that the size of $\son$ is the same
as the size of $\soi$ and also that of $\soq$. The proof consists in demonstrating
a one-to-one relation between the elements of the three sets. The set $\sor$ is,
however, much bigger than these.

\subsection{Problems}
\begin{enumerate}
\item Assume that $\sqrt{2}$ is a rational number. That is, write $\sqrt{2} = m/n$,
where $m, n \in \soi$, $n \ne 0$ and we can cancelled the common factors between $m$
and $n$. Show that this leads to a contradiction and therefore conclude that $\sqrt{2}$
is not a rational number \cite{hardy1992mathematician}.
\end{enumerate}

\section{Complex numbers}\label{c1s3}
\section{Fields}\label{c1s4}
\section{Linear spaces}\label{c1s5}