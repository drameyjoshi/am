\chapter{Set Theory}\label{c2}

Set theory starts off being extremely simple to the point of appearing
trivial but soon gets very subtle. This chapter introduces axiomatic set
theory just enough to start a serious study of analysis. 

\section{Axioms of set theory}\label{c2s1}
We begin with an informal definition of a set as a well-defined, unordered
collection of objects. The terms 'collection' and 'object' are interpreted
as they are in common language. We assume that sets satisfy the following
axioms:
\begin{enumerate}
\item[(A1)] Sets too are objects. If $A$ is a set then it also an object.
In particular, given two sets $A$ and $B$ it is meaningful to ask if $A$
is a member of $B$.
\item[(A2)] Two sets $A$ and $B$ are equal iff every element of $A$ is an
element of $B$ and \emph{vice versa}.
\item[(A3)] There exists a set $\varnothing$ called the empty set which
contains no elements. In particular, for every object $x$, $x \notin 
\varnothing$.
\item[(A4)] If $x$ is an object then there exists a set $\{x\}$ whose only
element is $x$. A set with just a single element is called a singleton set.
If $x$ and $y$ are objects then there exits a set whose only elements are
$x$ and $y$.
\item[(A5)] If $X$ and $Y$ are sets then there exists a set $X \cup Y$, 
called the union of $X$ and $Y$, which consists of all the elements which
belong to $X$ or $Y$. (The word 'or' is always in the inclusive sense.)
\item[(A6)] If $X$ is a set and if $P(x)$ is a logical proposition for 
each element $x$ of $X$ then there exists a set $\{x \in A \;|\; P(x)\}$
whose elements are precisely those elements of $X$ for which $P(x)$ is 
true.
\item[(A7)] Let $X$ be a set and $x$ be an arbitrary member of the set.
If $y$ is any object and we have a statement $P(x, y)$ which is true for
at most $y$. Then there exists a set $Y = \{y: P(x, y) \text{ is true for 
some } x \in A\}$ such that for any object $z$, $z \in Y \Leftrightarrow
P(x, z)$ is true for some $x \in A$.
\item[(A8)] There exists a set $\son$, whose elements are called natural
numbers. There is an object $0 \in \son$ and an object $S(n)$ assigned to
every natural number $n \in \son$ such that the axioms in section 
\ref{c1s1} hold.
\item[(A9)] The axiom of universal specification and existence of a 
universal set. This axiom is dropped in Zermelo-Frankael Set Theory and
the next one is included to avoid Russell's paradox.
\item[(A10)] (Regularity) If $A$ is a non-empty set then there is at least 
one element $x$ of $A$ which is either not a set or is disjoint from $A$.
\item[(A11)] Let $X$ and $Y$ be sets. Then there exists a set $Y^{X}$ which
consists of all the functions from $X$ to $Y$.
\item[(A12)] Let $X$ be a set all of whose elements are themselves sets.
Then there exists a set $\cup X$ whose elements are precisely those objects
which are \underline{elements of elements} of $X$. Thus,
\[
x \in \cup X \Leftrightarrow x \in S \in X.
\]
\end{enumerate}

We now prove a few basic lemmas that follow from these axioms.
\begin{lem}\label{c2s1l1}
The empty set is unique.
\end{lem}
\begin{proof}
Let if possible $\varnothing^\op$ be another empty set. Then, for any 
object $x$, $x \in \varnothing \Leftrightarrow x \in \varnothing^\op$,
making them equal.
\end{proof}

\begin{lem}[Single choice]\label{c2s1l2}
If $A$ is a non-empty set then there exists an object $x$ such that $x \in
A$.
\end{lem}
\begin{proof}
Assume to the contrary. Then for every object $x$, $x \notin A$, which 
makes $A$ an empty set.
\end{proof}

\begin{lem}\label{c2s1l3}
If $x$ is an object then the singleton set containing $x$ is unique.
\end{lem}
\begin{proof}
Let $X = \{x\}$ and $Y = \{y\}$ be another singleton set containing $x$.
Since $Y$ is a singleton containing $x$, $x \in Y$ and hence $x = y$. Since
$y = x$, $y \in X$ is also true. Therefore $X = Y$.
\end{proof}

\begin{lem}\label{c2s1l4}
$\{x, y\} = \{y, x\}$ for any objects $x, y$.
\end{lem}
\begin{proof}
$x \in \{y, x\}, y \in \{y, x\}$ and $y \in \{x, y\}, x \in \{x, y\}$.
\end{proof}

\begin{lem}\label{c2s1l5}
If $X, Y, X^\op$ are sets, $X = X^\op$ then $X \cup Y = X^\op \cup Y$.
\end{lem}
\begin{proof}
Let $a \in X \cup Y$. Then $a \in X$ or $a \in Y$. Since $X = X^\op$, we
have $a \in X^\op$ or $a \in Y$, that is $a \in X^\op \cup Y$.

We can similarly show that if $b \in X^\op \cup Y$, $b \in X \cup Y$.
\end{proof}

\begin{lem}\label{c2s1l6}
If $x$ and $y$ are objects then $\{x\} \cup \{y\} = \{x, y\}$.
\end{lem}
\begin{proof}
$x \in \{x\} \cup \{y\}$ and $x \in \{x, y\}$. Similarly for $y$. 
Therefore, every element of $\{x\} \cup \{y\}$ is an element of $\{x, y\}$.
We can similarly show that every element of $\{x, y\}$ is an element of
$\{x\} \cup \{y\}$.
\end{proof}

\begin{lem}\label{c2s1l7}
If $X$ and $Y$ are sets then $X \cup Y = Y \cup X$.
\end{lem}
\begin{proof}
Let $a$ be an arbitrary element of $X \cup Y$. Then
$a \in X \cup Y \Rightarrow a \in X \;\lor\; a \in Y \Rightarrow a \in Y
\;\lor\; a \in X \Rightarrow a \in Y \cup X$.

Likewise, if $b$ is an arbitrary element of $Y \cup X$ then we can show 
that $b \in X \cup Y$.
\end{proof}

We can use the same idea to show that
\begin{lem}\label{c2s1l8}
If $X, Y, Z$ are sets then $X \cup (Y \cup Z) = (X \cup Y) \cup Z$.
\end{lem}

\begin{lem}\label{c2s1l9}
If $X$ is a set then $X \cup \varnothing = X$.
\end{lem}
\begin{proof}
If $a \in X$ then clearly $a \in X \cup \varnothing$. If $b \in X \cup
\varnothing$ then $b \in X$ or $b \in \varnothing$. But the statement $b 
\in \varnothing$ is always false. Therefore, the only way $b \in X 
\;\lor\; b \in \varnothing$ can be true is if $b \in X$ is true.
\end{proof}

From lemmas \ref{c2s1l7} and \ref{c2s1l9} we conclude that
\begin{lem}\label{c2s1l10}
If $X$ is a set then $\varnothing \cup X = X$.
\end{lem}

If $P$ is a proposition then $P \Leftrightarrow P \;\lor\; P$. Using this
we can conclude 
\begin{lem}\label{c2s1l11}
$X = X \cup X$ for any set $X$.
\end{lem}

\begin{defn}\label{c2s1d1}
If $X$ and $Y$ are sets then $X$ is a subset of $Y$ iff every element of 
$X$ is also an element of $Y$. The relation is denoted by $X \subseteq Y$.
\end{defn}

Proofs of lemmas \ref{c2s1l4} tp \ref{c2s1l11} can be written more 
elegantly if we use the notation of subsets because
\begin{lem}\label{c2s1l12}
$X = Y$ iff $X \subseteq Y$ and $Y \subseteq X$.
\end{lem}
\begin{proof}
$X = Y$ means every element of $X$ is an element of $Y$, that is, $X 
\subseteq Y$ and every element of $Y$ is an element of $X$, that is $Y 
\subseteq X$.
\end{proof}

Since $X = X$ for every set $X$ (follows from the logical idea of equality
of objects), lemma \ref{c2s1l12} leads us to
\begin{lem}\label{c2s2l13}
$X \subseteq X$ for all sets $X$.
\end{lem}

\begin{defn}\label{c2s1d2}
If $X \subseteq Y$ and $X \ne Y$ then $X$ is called a proper subset of 
$Y$ and is denoted by $X \subset Y$.
\end{defn}

\begin{lem}\label{c2s1l14}
If $X \subset Y$ and $Y \subset Z$ then $X \subset Z$.
\end{lem}
\begin{proof}
$Y \subset Z \Rightarrow \forall\; y \in Y, y \in Z$. Since $X \subset
Y$, $x \in Y, \forall\; x \in X$. Therefore, $\forall\; x \in X, x \in Z$.
\end{proof}

Since $X \subset Y \Rightarrow X \subseteq Y$, we also have
\begin{lem}\label{c2s1l15}
If $X \subseteq Y$ and $Y \subseteq Z$ then $X \subset Z$.
\end{lem}

\begin{lem}\label{c2s1l16}
$X = Y$ iff $X \subseteq Y$ and $Y \subseteq X$.
\end{lem}
\begin{proof}
$X = Y$ means,
\begin{itemize}
\item $\forall\; x \in X, x \in Y$. Therefore, $X \subseteq Y$ and
\item $\forall\; y \in Y, y \in X$. Therefore, $Y \subseteq X$.
\end{itemize}

Conversely,
\begin{itemize}
\item $X \subseteq Y \Rightarrow \forall\; x \in X, x \in Y$ and
\item $Y \subseteq X \Rightarrow \forall\; y \in Y, y \in X$.
\end{itemize}
Therefore, $X = Y$.
\end{proof}

\begin{lem}\label{c2s1l17}
$X \subseteq X \cup Y$ for any sets $X$ and $Y$.
\end{lem}
\begin{proof}
If $x \in X$ then $x \in X \cup Y$.
\end{proof}

Axiom (A6) allows us to define
\begin{defn}\label{c2s1d3}
The intersection of sets $X$ and $Y$ is the set $X \cap Y := \{x \in X
\;|\; x \in Y\}$.
\end{defn}
Note that although axiom (A6) allows us to define the intersection of two
sets it does not suffice to introduce union. That is why the operation of 
union has to be introduced as a separate axiom. Suppose we define, $X \cup
Y = \{x \in X, y \in Y \;|\; x \in X \;\lor\; y \in Y\}$. The clause $x \in
X, y \in Y$ is itself a union and we are using axiom (A6) to define a union
in terms of itself!

\begin{defn}\label{c2s1d4}
The sets $X$ and $Y$ are said to be disjoint if $X \cap Y = \varnothing$.
\end{defn}
Note that $X \cap Y \varnothing$ does not mean $X \ne Y$. In fact, $X
\cap Y = \varnothing \Rightarrow X \ne Y$ but not conversely.

A concept closely related to intersection is
\begin{defn}\label{c2s1d5}
The difference of sets $X$ and $Y$ is the set $X - Y := \{x \in X
\;|\; x \notin Y\}$.
\end{defn}

\begin{prop}\label{c2s1p1}
If $X, Y, Z$ are sets and if $U$ is a set such that $X \cup Y \cup Z 
\subseteq U$ then
\begin{enumerate}
\item $X \cup \varnothing = X$.
\item $X \cap \varnothing = \varnothing$.
\item $X \cup U = U$.
\item $X \cap U = X$.
\item $X \cup X = X$.
\item $X \cap X = X$.
\item $X \cup Y = Y \cup X$.
\item $X \cap Y$ = $Y \cap X$.
\item $X \cup (Y \cup Z) = (X \cup Y) \cup Z$.
\item $X \cap (Y \cap Z) = (X \cap Y) \cap Z$.
\item $X \cap (Y \cup Z) = X \cap Y \cup X \cap Z$.
\item $X \cup (Y \cap Z) = X \cup Y \cap X \cup Z$.
\item $X \cup (U - X) = U$.
\item $X \cap (U - X) = \varnothing$.
\item $U - (X \cap Y) = (U - X) \cup (U - Y)$.
\item $U - (X \cup Y) = (U - X) \cap (U - Y)$.
\end{enumerate}
\end{prop}
\begin{proof}
\begin{enumerate}
\item $X \cup \varnothing = \{x\;|\; x \in X \;\lor\; x \in varnothing\}$.
Since the statement $x \in \varnothing$ is vacuously false, $X \cup 
\varnothing = \{x\;|\; x \in X\} = X$.
\item $X \cap \varnothing = \{x\;|\; x \in X \;\land\; x \in varnothing\}$.
Since the statement $x \in \varnothing$ is vacuously false, there is no
$x$ such that $x \in X \cap \varnothing$, making it an empty set. Lemma
\ref{c2s1l1} assures us that the empty set is unique.
\item From lemma \ref{c2s1l17}, $U \subset X \cup U$. Since $X \subseteq
U$ by hypothesis, every element of $X \cup U$ is a subset of $U$ or $X 
\cup U \subseteq U$.
\item Since $X \subseteq U$ by hypothesis, $X \cap U = \{x\;|\; x \in X 
\;\land \; x \in U\} = \{x\;|\; x \in X\} = X$.
\item Same as lemma \ref{c2s1l11}.
\item $X \cap X = \{x \in X \;|\; x \in X\} = \{x \in X\} = X$.
\item Same as lemma \ref{c2s1l7}.
\item $a \in X \cap Y \Leftrightarrow a \in X \;\land\; a \in Y 
\Leftrightarrow a \in Y \;\land\; a \in X \rightarrow Y \cap X$.
\item Same as lemma \ref{c2s1l8}.
\item $a \in X \cap (Y \cap Z) \Leftrightarrow a \in X \;\land\; a \in 
Y \cap Z \Leftrightarrow a \in X \;\land a \in Y \;\land\; a \in Z
\Leftrightarrow a \in X \cap Y \;\land\; a \in Z \Leftrightarrow a \in
(X \cap Y) \cap Z$.
\item Let $a \in X \cap (Y \cup Z)$. Then $a \in X$ and $a \in Y \cup Z$.
If $a \in Y$ then $a \in X \cap Y$. If $a \in Z$ then $a \in X \cap Z$. In
any case, $a \in X \cap Y$ or $a \in X \cap Z$. That is $a \in (X \cap Y) 
\cup (X \cap Z)$.

Now let $a \in (X \cap Y) \cup (X \cap Z)$. Then $a \in X \cap Y$ or $a
\in X \cap Z$, that is $a \in X$ and $a \in Y$ or $a \in Z$.
\item $a \in X \cup (Y \cap Z) \Leftrightarrow a \in X \;\lor\; a \in Y 
\cap Z \Leftrightarrow (a \in X \;\lor\; a \in Y) \;\land\; (a \in X 
\;\lor\; a \in Z) \Leftrightarrow a \in (X \cup Y) \cap (X \cup Z)$.

\item $a \in X \cup (U - X) \Leftrightarrow a \in X \;\lor\; a \in U, a 
\notin X$. Since $X \subseteq U$, $a \in X \Rightarrow a \in U$. Therefore,
$a \in X \cup (U - X) \Leftrightarrow (a \in X, a \in U) \;\lor\; (a \notin
X, a \in ) \Leftrightarrow a \in U$.

\item $a \in X \cap (U - X) \Leftrightarrow a \in X \;\land\; a \in U - X
\Leftrightarrow a \in X \;\land\; a \in U, a \notin X$. Now, $X \subseteq
U \Rightarrow a \in U \;\forall\; a \in X$. Therefore $a \in X \cap (U - X)
$ will require $a$ to be simultaneously in and not in $X$. There is no such
$a$. Therefore $X \cap (U - X) = \varnothing$.

\item Let $a \in U - (X \cap Y)$. Then $a \in U$ and $a \notin X \cap Y$.
Since $X, Y \subseteq U$, this means that $a$ is $a \in X, a \notin Y$ or
$a \notin X, a \in Y$ or $a \notin X, a \notin Y$. $a \in X$ and $a \notin
Y \Rightarrow a \in U - Y$. Similarly, $a \in Y, a \notin X \Rightarrow
a \in U - X$. Therefore, the three possibilities are $a \in U - Y$ or $a
\in U - X$ or $a \in U - Y \;\land\; a \in U - X$. Therefore, $U - (X \cap
Y) \subseteq (U - X) \cup (U - Y)$.

Now let $a \in (U - X) \cup (U - Y)$. Then $a \in U - X$ or $a \in U - Y$ 
or both. This can happen if 1) $a \notin X, a \in Y$ or 2) $a \in X, a 
\notin Y$ or 3) $a \notin X, a \notin Y$. These three possibilities can
be combined as $a$ is not a member of $X \cap Y$. That is $(U - X) \cup
(U - Y) \subseteq U - (X \cap Y)$.

\item Let $a \in U - (X \cup Y)$. Then $a$ is not a member of $X$ and
neither is it a member of $Y$. That is $a \in U - X$ and $a \in U - Y$. 
Therefore $U - (X \cup Y) \subseteq (U - X) \cap (U - Y)$.

If $a \in (U - X) \cap (U - Y)$ then $a \in U - X$ and $a \in U - Y$, that
is $a \notin X$ and $a \notin Y$, which is same as $a \in U - (X \cap Y)$.
\end{enumerate}
\end{proof}

Statements in the above proposition are collectively called as the laws
of Boolean algebra.

\subsection{Exercises}
\begin{enumerate}
\item[Ex 3.1.5 ] Let $A, B$ be sets. Show that the three statements $A
\subseteq B$, $A \cup B = B$, $A \cap B = A$ are logically equivalent.
\footnote{Exercise numbers are same as the one in Tao's books 
\cite{tao2014a1} and \cite{tao2014a2}.}
\item[Solution: ] Assume that $A \subseteq B$. Then for all $a \in A, a
\in B$ so that $A \cup B = \{x \;|\; x \in A \;\lor\; x \in B\} = 
\{x \:|\; x \in B\} = B$.

Assume that $A \cup B = B$. Then every element of $A \cup B$ is an element
of $B$. In particular, every element of $A$ is also an element of $B$. 
Therefore, the set $A \cap B = \{x \;|\; x \in A \;\land\; x \in B\} = 
\{x \;|\; x \in A\} = A$. (Note that every element of $A$ is an element of
$B$ does not mean that the converse is also true. That is why in the
second last set cannot be written as $\{x \;|\; x \in B\}$.)

Now assume that $A \cap B = A$. $A \cap B = B \cap A = \{x \in B | x \in
A\}$. If this set is same as $A$ then every element of $A$ is an element
of $B$. That is $A \subseteq A$.

\item[Ex 3.1.9 ] Let $A, B, X$ be sets such that $A \cup B = X$ and $A
\cap B = \varnothing$. Show that $A = X - B$ and $B = X - A$.
\item[Solution: ] Let $a \in X - B$, that is $a \in (A \cup B) - B$. Thus,
$a \in A \cup B \;\land\; a \notin B$. Therefore, $a \in A \;\land\; A 
\notin B$, that is $a \in A$ alone. Since $a$ was an arbitrary element
of $X - B$, this just means that $X - B \subseteq A$.

Now, let $a \in A$ so that $a \in X$. Since $A \cap B = \varnothing$, $a
\notin B$. Thus, $a \in X \;\land\; a \notin B$ or $a \in X - B$. Since
$a$ was an arbitrary element of $A$ this just means that $A \subseteq X -
B$.

\item[Ex 3.1.10 ] Let $A$ and $B$ be sets. Show that the three sets $A - 
B$, $A \cap B$ and $B - A$ are disjoint and that their union is $A \cup B$.
\item[Solution: ] If $a \in A - B$ then $a \in A \;\land\; a \notin B$. 
Therefore, $a \notin A \cap B$ and $a \notin B - A$. Likewise, if $b \in
B - A$ then $b \notin A \cap B$ and $b \notin A$.

Let $X = (A - B) \cup (A \cap B) \cup (B - A)$. If $x \in X$, then $x$ is
a member of one of the three disjoint sets, $A - B, A \cap B$ and $B - A$,
each of which is a subset of $A \cup B$. Therefore $X \subseteq A \cup B$.

Let $x \in A \cup B$. If $x \in A$ then $x \in A - B \cup A \cap B$, making
$x$ a member of $X$. We can similarly show that if $x \in B$ then $x \in
X$. Thus, $x \in A \cup B \Rightarrow x \in X$ or $A \cup B \subseteq X$.

\item[Ex 3.1.11 ] Show that the axiom of replacement implies the axiom of
specification.
\item[Solution: ] Put $y = x$ in the axiom of replacement.
\end{enumerate}

\section{Russell's Paradox}\label{c2s2}
If one introduces an axiom of universal specification like
\begin{enumerate}
\item If for every object $x$ we have a logical
proposition $P(x)$ then there exists a set $U = \{x \;|\; P(x)\}$ such that
for all $y \in U \Rightarrow P(y)$ is true
\end{enumerate}
then one is led to a logical inconsistency. As a set is also an object, we
can construct sets of whose members are sets. As an example, let $P$ be
a proposition which is true for sets that do not contain themselves and
let
\begin{equation}
V = \{x \;|\; P(x)\}.
\end{equation}
If $V \notin V$ then $P(V)$ is true making $V \in V$. If $V \in V$ then
$P(V)$ is false making $V \notin V$. This is \emph{Russell's paradox} also
known to philosophers before Bertrand Russell. To avoid it, we augment the
axioms of section \ref{c2s1} with
\begin{enumerate}
\item[(A9)] If $A$ is a non-empty set, then there is at least one element
$x$ of $A$ which is either not a set, or is disjoint from $A$.
\end{enumerate}

Let us consider the set $A = \{3, 4, \{3, 4\}\}$. Then the elements $3$
and $4$ are not sets but $\{3, 4\}$ is. Now consider 
\[
\{3, 4\} \cap A = \{x \in \{3, 4\} \;|\; x \in A\} = \{3,  4\} \ne
\varnothing.
\]
However, this is not a problem because $A$ has elements $3$ and $4$ that
are not sets.

\subsection{Exercises}
\begin{enumerate}
\item[3.2.1] Show that the universal specification axiom implies several
other axioms of section \ref{c2s1}.
\item[Solution:] Let AUS stand for the axiom of universal specification.
Then,
\begin{enumerate}
\item Choose $P$ to be $P(x)$ is false for every object. Then the set it
leads to is the empty set. Thus UAS $\Rightarrow$ (A3).
\item Choose $P_x$ to be such that $P_x(x)$ is true only for $x$ and if 
$y \ne x$ then $P_x(y)$ is false. This leads to the singleton sets. Thus 
UAS $\Rightarrow$ (A4).
\item Let $P_{xy}$ be the proposition that is true only for $x$ and $y$.
Then using it in UAS gives us the set $\{x, y\}$.
\item Axiom (A6) is UAS restricted to elements of a certain set $X$.
\item Axiom (A7) is UAS restricted to elements of a certain set $X$ and
and object $y$.
\end{enumerate}

\item[3.2.2] Use the axiom of regularity to show that if $A$ is a set
then $A \notin A$. Furthermore, show that if $A$ and $B$ are sets then
either $A \notin B$ or $B \notin A$ or both.
\item[Solution:] The axiom of regularity applies to non empty sets. Let
$A$ be a non-empty set then there is at least one element $x$ in $A$. 
Using axiom (A4), there exists a singleton set ${x}$. If $A \in A$, then
there is a singleton set $\{A\}$. The axiom of regularity assures that
$A \cap \{A\} = \varnothing$. Therefore, no element of $\{A\}$ is an 
element of $A$. That is $A \notin A$.

To prove the second statement, assume to the contrary. That is, assume that
there are sets $A$ and $B$ such that $A \in B$ and $B \in A$. Construct
a set $C = \{A, B\}$. No element of $C$ is 'not a set'. Therefore, it
must be true that $C \cap A = \varnothing$ and $C \cap B = \varnothing$.
The first relation leads us to conclude that $B \notin A$ and the second
one $A \notin B$, a contradiction.

\item[3.2.3] Show that the axiom of universal specification is equivalent
to the existance of a universal set.
\item[Solution:] If we choose the logical proposition $P$ to be true for
all objects then we get the universal set.

On the other hand, if $\Omega$ were a universal set then we can use axiom
(A6) to define a set $U = \{x \in \Omega \;|\; P(x)\}$. Since $\Omega$ is 
the universal set, this is equivalent to $U = \{x\;|\; P(x)\}$, which is
what UAS states.
\end{enumerate}

The crucial difference between UAS and (A6) is that the latter is 
restricted to members of a certain set $X$ while the former is applicable 
to everything. That is why it is called the axiom of 'universal' 
specification.

\section{Functions}\label{c2s3}
So far I have come across definitions of functions as a special kind of
relation. Tao \cite{tao2014a1} defines it in terms of logical propositions.
The two definitions are equivalent.
\begin{defn}\label{c2s3d1}
Let $X, Y$ be sets and $P(x, y)$ be a logical proposition for an $x \in X$
and $y \in Y$. If for every $x \in X$, there is a unique $y \in Y$ for 
which $P(x, y)$ is true then we define the function $f: X \rightarrow Y$ 
to be the object that assigns a value $y$ to every $x \in X$. Thus, for $x 
\in X, y \in Y$,
\[
y = f(x) \Leftrightarrow P(x, y) \text{ is true.}
\]
The set $X$ is called the domain of $f$ and $Y$ the codomain. The set $\{y
\in Y \;|\; y = f(x) \;\forall\; x \in X\}$ is called the range of the 
function.
\end{defn}

Tao has used this definition because he has not defined Cartesian products
by the time he introduced functions.

\begin{defn}\label{c2s3d2}
Two functions $f: X \rightarrow Y$ and $g: X \rightarrow Y$ are said to be
equal if their domain and codomain are equal and for all $x \in X$, $f(x) 
= g(x)$.
\end{defn}

\begin{defn}\label{c2s3d3}
If $f: X \rightarrow Y$ and $g: Y \rightarrow W$ are functions then the
composition $g \circ f: X \rightarrow Z$ is the function $(g \circ f)(x)
= g(f(x))$.
\end{defn}

\begin{lem}\label{c2s3p1}
Let $f: U \rightarrow V, g: V \rightarrow W$ and $h: W \rightarrow X$ be
functions. Then $h \circ (g \circ f) = (h \circ g) \circ f$.
\end{lem}
\begin{proof}
We first confirm that the two functions have the same domain and codomain. 
$g \circ f$ is a function with $U$ as domain and $W$ as codomain and $h 
\circ (g \circ f)$ has $U$ as domain and $X$ as codomain. Likewise, the 
function $h \circ g$ has $V$ as domain and $X$ as codomain and the function
$(h \circ g) \circ f$ has $U$ as fomain and $X$ as codomain. 

Now consider, for some $u \in U$,
\begin{eqnarray*}
(h \circ (g \circ f))(u) &=& h((g \circ f)(u)) \\
 &=& h(g(f(u)) \\
 &=& (h \circ g)(f(u)) \\
 &=& ((h \circ g) \circ f)(u)
\end{eqnarray*}
\end{proof}

\begin{defn}\label{c2s3d4}
A function $f: X \rightarrow Y$ is said to be one-to-one, or injective, 
if $x_1 \ne x_2 \Rightarrow f(x_1) \ne f(x_2)$ for all $x_1, x_2 \in X$.
\end{defn}

\begin{defn}\label{c2s3d5}
A function $f: X \rightarrow Y$ is said to be onto, or surjective, if
for all $y \in Y$, there exists an $x \in X$ such that $y = f(x)$.
\end{defn}

\begin{defn}\label{c2s3d6}
A function $f: X \rightarrow Y$ is said to be bijective if it is injective
and surjective.
\end{defn}

If $f$ is bijective then for every $y \in Y$ there is a unique $x \in X$
such that $y = f(x)$. We can as well define a function $f^{-1}: Y 
\rightarrow X$ such that $f^{-1}(y) = x$ iff $y = f(x)$ for all $x \in X$
and $y \in Y$.

\subsection{Exercises}
These are problems at the end of section 3.3 of Tao's first volume
\cite{tao2014a1}.
\begin{enumerate}
\item[Ex 1:] Show that the equality of functions defined in \ref{c2s3d2}
is reflexive, symmetric and transitive.
\item[Solution:] All three properties follow from the corresponding 
properties of equality of elements of the domain and the codomain.

\item[Ex 2:] If $f_1, f_2: X \rightarrow Y$ and $g_1, g_2: Y \rightarrow
Z$ are such that $f_1 = f_2$ and $g_1 = f_2$ then $g_1 \circ f_1 = g_2
\circ f_2$.
\item[Solution:] The functions $g_1 \circ f_1$ and $g_2 \circ f_2$ both
have $X$ as their domain and $Y$ as their codomain. For some $x \in X$
consider $(g_1 \circ f_1)(x) = g_1(f_1(x))$. Since $f_1 = f_2$, 
$g_1(f_1(x)) = g_1(f_2(x))$. Since $g_1 = g_2$, $g_1(f_2(x)) = g_2f_2(x))$.

\item[Ex 3:] If $f: X \rightarrow Y$ and $g: Y \rightarrow Z$ are both
injective then so is $g \circ f$.
\item[Solution:] Let $x_1, x_2 \in X$. Then $x_1 \ne x_2 \Rightarrow
f(x_1) \ne f(x_2) \Rightarrow g(f(x_1)) \ne g(f(x_2))$. The first 
implication follows from injectivity of $f$ and the second one from that of
$g$.

\item[Ex 4:] If $f: X \rightarrow Y$ and $g: Y \rightarrow Z$ are both
surjective then so is $g \circ f$.
\item[Solution:] Since $g$ is surjective, for all $z \in Z$, there exists
a $y \in Y$ such that $z = g(y)$. Since $f$ is also surjective, for all $y
\in Y$, there exists an $x \in X$ such that $y = f(x)$. Therefore, for all
$z \in Z$, we can find an $x \in X$ such that $z = g(f(x)) = (g \circ f)(x)
$.

\item[Ex 5:] The function $e: \varnothing \rightarrow Y$ is called an
empty function. When is it injective, surjective and bijective?
\item[Solution:] $e$ will be injective if $x_1 \in \varnothing \;\land\;
x_2 \in \varnothing \;\land\; x_1 \ne x_2 \Rightarrow e(x_1)
\ne e(x_2)$. Since $x_1 \in \varnothing$ is false, the implication is true
making $e$ an injective function.

$e$ will be surjective if for all $y \in Y$, there exists an $x \in 
\varnothing$ such that $e(x) = y$. Since the statement $x \in \varnothing$
is false, the implication is also false. Therefore $e$ is not surjective
and hence is not bijective.

\item[Ex 6:] Let $f_1, f_2: X \rightarrow Y$ and $g_1, g_2: Y \rightarrow
Z$. Suppose $g_1 \circ f_1 = g_1 \circ f_2$. Show that $f_1 = f_2$ if $g_1$
is injective. What if it is not injective.
\item[Solution:] Let $x$ be an arbitrary member of $X$. Then, $g_1 \circ
f_1 = g_1 \circ f_2 \Rightarrow (g_1 \circ f_1)(x) = (g_1 \circ f_2)(x)
\Rightarrow g_1(f_1(x)) = g_1(f_2(x))$. But $g_1$ is injective. Therefore
from the previous inequality we conclude that $f_1(x) = f_2(x)$. Since $x$
was chosen arbitrarily, this implies $f_1 = f_2$.

If $g_1$ is not injective then we cannot proceed beyond $g_1(f_1(x)) = 
g_1(f_2(x))$. For instance, $g_1$ can map its entire domain to a single
member of its codomain.

\item[Ex 6:] Let $f_1, f_2: X \rightarrow Y$ and $g_1, g_2: Y \rightarrow
Z$. Suppose $g_1 \circ f_1 = g_2 \circ f_1$. Show that $g_1 = g_2$ if $f_1$
is surjective. What if it is not surjective.
\item[Solution:] Let $x \in X$ be arbitrary. Then $g_1 \circ f_1 = g_2
\circ f_2 \Rightarrow g_1(f_1(x)) = g_2(f_1(x))$. Since $f_1$ is 
surjective, every $y \in Y$ has an $x \in X$ such that $y = f_1(x)$.
Therefore, the previous inequality can also be written as $g_1(y) = g_2(y)$
for all $y \in Y$.

If $f_1$ were not surjective then there will be at least one $y \in Y$ for
which $g_1(y) = g_2(y)$ cannot be concluded from $g_1(f_1(x)) = 
g_2(f_1(x))$.

\item[Ex 7:] Let $f: X \rightarrow Y$ be bijective with $f^{-1}: Y
\rightarrow X$ as its inverse. Show that $f^{-1}(f(x)) = x, \;\forall\;
x \in X$ and $f(f^{-1}(y)) = y, \;\forall\; y \in Y$.
\item[Solution:] Fix a $y \in Y$. Since $f$ is bijective, there is a unique
$x \in X$ such that $y = f(x)$. This unique $x$ is defined to be 
$f^{-1}(y)$. Therefore, $y = f(f^{-1}(y))$. Since $y$ was chosen 
arbitrarily, this statement is true for all $y \in Y$.

Since $y = f(x) \Rightarrow x = f^{-1}(y) = f^{-1}(f(x))$.

Remark: The definition of $f^{-1}$ as $f^{-1}(y) = x$ iff $y = f(x)$ is 
important here. Mere bijectivity is not enough to prove this. If $X = {1, 
2}, Y = {3, 4}$ then $f(1) = 3, f(2) = 4$ is bijective. So is $g(3) = 2$
and $g(4) = 1$. But the two are not inverse of each other.

\item[Ex 8:] If $f: X \rightarrow Y$ and $g: Y \rightarrow Z$ are bijective
then show that $g \circ f$ is also bijective and $(g \circ f)^{-1} = f^{-1}
\circ g^{-1}$.
\item[Solution:] We showed in exercises 3(4) that if $f$ and $g$ are
injective(surjective) then so is their composition. From this it follows 
that if $f$ and $g$ are bijective then so is $g \circ f$. Therefore,
$(g \circ f)^{-1}$ exists and is such that $(g \circ f)^{-1}(z) = x$ iff
$z = (g \circ f)(x)$. Thus, 
\begin{equation}\label{c2s3e1}
(g \circ f)^{-1}(z) = x \Leftrightarrow z = g(f(x)). 
\end{equation}

Since $g$ is invertible, $z = g(f(x)) \Rightarrow g^{-1}(z) = f(x)$. Since
$f$ is invertible, $g^{-1}(z) = f(x) \Rightarrow f^{-1}(g^{-1}(z)) = x$.
Thus, we have 
\begin{equation}\label{c2s3e2}
z = g(f(x)) \Rightarrow (f^{-1} \circ g^{-1})(z) = x
\end{equation}

From \eqref{c2s3e1} and \eqref{c2s3e2} we get
\begin{equation}\label{c2s3e3}
(g \circ f)^{-1}(z) = x \Leftrightarrow (f^{-1} \circ g^{-1})(z) = x.
\end{equation}
Since $x$ was chosen arbitrarily this just means that
\begin{equation}\label{c2s3e4}
(g \circ f)^{-1}(z) = (f^{-1} \circ g^{-1})(z)
\end{equation}
is true for all $z \in Z$.

\item[Ex 9:] If $X \subseteq Y$ then $\iota_{X \rightarrow Y}:X 
\rightarrow Y$, called the inclusion map, is defined by the mapping 
$\iota(x) = x$ for all $x \in X$. The map $\iota_{X \rightarrow X}$ is
called the identity map.
\begin{enumerate}
\item[(a)] Show that if $X \subseteq Y \subseteq Z$ then 
\[
\iota_{Y \rightarrow Z} \circ \iota_{X \rightarrow Z} = 
\iota_{X \rightarrow Z}.
\]
\item[Solution:] $\iota_{X \rightarrow Z}(x) = x$ for all $x \in X 
\subseteq Z$. Likewise $(\iota_{Y \rightarrow Z} \circ \iota_{X \rightarrow
Z})(x) = \iota_{Y \rightarrow Z}(\iota_{X \rightarrow Y}(x)) =
\iota_{Y \rightarrow Z}(x)$, for all $x \in X \subseteq Y$. Since $X
\subseteq Z$ as well, the rhs of the previous equation evaluates to $x$.

\item[(b)] If $f: A \rightarrow B$ then $f \circ \iota_{A \rightarrow A} = 
\iota_{B \rightarrow B}f$.
\item[Solution:] Let $a \in A$ and $b \in B$ be such that $b = f(a)$.
Then $(f \circ \iota_{A \rightarrow A})(a) = f(\iota_{A \rightarrow A}(a))
= f(a) = b$ and $(\iota_{B \rightarrow B} \circ f)(a) = 
\iota_{B \rightarrow B}(f(a))$ or 
\[
\iota_{B \rightarrow B}(b) = b.
\]

\item[(c)] Show that if $f$ is a bijective function then $f \circ f^{-1}
= \iota_{B \rightarrow B}$ and $f^{-1} \circ f = \iota_{A \rightarrow A}$.
\item[Solution:] Follows from the fact that $(f \circ f^{-1})(b) = b$ and
$(f^{-1} \circ f)(a) = a$.

\item[(d)] Show that if $X$ and $Y$ are disjoint sets, $f: X \rightarrow Z$
and $g: Y \rightarrow Z$ are functions then there is a unique function
$h: X \cup Y \rightarrow Z$ such that $h \circ \iota_{X \rightarrow X \cup
Y} = f$ and $h \circ \iota_{Y \rightarrow X \cup Y} = g$.
\item[Solution:] Since $X$ and $Y$ are disjoint, we can define a function
$h: X \cup Y \rightarrow Z$ such that
\[
h(u) = 
\begin{cases} f(u) & \text{ if } u \in X \\
g(u) & \text{ if } u \in Y.
\end{cases}
\]
Note that the domain of $h \circ \iota_{X \rightarrow X \cup Y}$ is $X$
and its range is $Z$. Further, from the definition of $h$ it is clear that
for all $u \in X$, $h(u) = f(u)$. Similarly, we can show that $h \circ
\iota_{Y \rightarrow X \cup Y} = g$.

Let, if possible, $k: X \cup Y \rightarrow Z$ be a function such that $k
\circ \iota_{X \rightarrow X \cup Y} = f$ and $k \circ
\iota_{Y \rightarrow X \cup Y} = g$. Then if $u \in X$, $f(u) = (k \circ 
\iota_{X \rightarrow X \cup Y})(u) = k(\iota_{X \rightarrow X \cup Y}(u))
= k(u)$. Similarly, if $u \in Y$, $g(u) = (k \circ 
\iota_{Y \rightarrow X \cup Y})(u)$. Thus, $h$ and $k$ are identical maps.
\end{enumerate}
\end{enumerate}

\section{Images and inverse images}\label{c2s4}
\begin{defn}\label{c2s4d1}
Let $f: X \rightarrow Y$ and $S \subseteq X$ then $f(S) = \{y \in Y\;|\;
y = f(x), x \in S\}$. $f(S)$ is called the image of $S \subseteq X$ under
the function $f: X \rightarrow Y$.
\end{defn}

The set $f(S)$ is well defined by axiom (A6) if we let $P$ be the 
proposition that $P(y)$ is true if $y = f(x)$ for some $x \in S$.

\begin{defn}\label{c2s4d2}
Let $f: X \rightarrow Y$ and $T \subseteq Y$ then $f^{-1}(T) = \{x \in X
\;|\; f(x) \in T\}$. $f^{-1}(T)$ is called the preimage of $T \subseteq
Y$ under the function $f: X \rightarrow Y$.
\end{defn}
\begin{itemize}
\item $T$ may have some elements for which there is no preimage. 
$f^{-1}(T)$ is just that subset of $X$ which maps to $T$.
\item It is not always true that $f(f^{-1}(T)) = T$ although one can always
be assured of $f(f^{-1}(T)) \subseteq T$. This is because some elements of
$T$ will not have a preimage and they will be not be recovered by applying
$T$ to $f^{-1}(T)$.
\end{itemize}

Axiom (A12) allows us to contruct the unions based on an index set. If $I$
is a set and if for every $\alpha \in I$ there is a set $A_\alpha$ then 
\begin{equation}\label{c2s4e1}
\bigcup_{\alpha \in I} = \cup \{A_\alpha \;|\; \alpha \in I\}.
\end{equation}
The set $I$ is called the index set, its elements $\alpha$ are called 
labels and the sets $A_\alpha$ is called a family of sets indexed by label
$\alpha$.

\begin{lem}\label{c2s4l1}
If $I$ is empty then the set defined by $\cup_{\alpha \in I}A_\alpha$ is 
empty.
\end{lem}
\begin{proof}
\[
x \in \cup \{A_\alpha \;|\; \alpha \in I\} \Leftarrow x \in A_\alpha
\text{ for some } \alpha \in I.
\]
Since $I$ is empty, there is no set $A_\alpha$ labelled by an element of 
$I$. Therefore the union is empty.
\end{proof}

If $I$ is a non-empty set of labels $\alpha$ and if $A_\alpha$ is a family
of labelled sets then
\begin{equation}\label{c2s4e2}
\bigcap_{\alpha \in I} A_\alpha := \{x \in A_\beta \;|\; x \in A_\alpha
\forall \alpha \in I\}.
\end{equation}
We require the set $I$ to be non-empty because the definition selects one
element $\beta \in I$ and the set labelled $A_\beta$ by it and chooses
all elements of $A_\beta$ which are common to other labelled sets in the 
family.

\subsection{Exercises}
\begin{enumerate}
\item[Ex 1:] Let $f: X \rightarrow Y$ be a bijective function and let 
$g$ be its inverse. Show that the image of $V$ under $g$ is the same as 
the preimage of $V$ under $f$.
\item[Solution:] Let $V \subseteq Y$. Then $g(V) = \{g(y) \;|\; y \in V\}$.
Since $f$ is a bijective map, every element of $V$ is mapped to a unique
element of $X$. The preimage of $V$ under $f$ is the set $f^{-1}(V) = \{x
\in X \;|\; f(x) \in V$.

We will first show that $g(V) \subseteq f^{-1}(V)$. Let $x \in g(V)$. Then
there exists $x = g(y) = f^{-1}(y)$. Equivalently, $x \in f^{-1}(V)$.

Let $x \in f^{-1}(V)$. Then $y = f(x) \in V$ and $x = g(y)$. Therefore,
$x \in g(V)$.

Therefore, $f^{-1}(V) = g(V)$.

\item[Ex 2:] Let $f: X \rightarrow Y, S \subseteq X, T \subseteq Y$. What
can one say about $f^{-1}(f(S))$ and $S$. What about $f(f^{-1}(T))$ and 
$T$?
\item[Solution:] $f^{-1}(f(S)) \subseteq S$ and $f(f^{-1}(T)) \subseteq T$.
If $f$ is bijective then $f^{-1}(f(S)) = S$ and $f(f^{-1}(T)) = T$.

\item[Ex 3:] Let $f: X \rightarrow Y$, $A, B \subseteq X$. Show that
\begin{enumerate}
\item[(a)] $f(A \cap B) \subseteq f(A) \cap f(B)$.
\item[Solution:] Let $x \in f(A \cap B)$. Then $x \in f(A \cap B), x \in
f(A), x \in f(B)$. In particular, $x \in f(A \cap B)$.

\item[(b)] $f(A) - f(B) \subseteq f(A - B)$.
\item[Solution:] Let $y \in f(A) - f(B)$. That is, $y = f(x)$ for some $x
\in A$ and $x \notin B$. That is, $y = f(x)$ for some $x \in A - B$, or
$y \in f(A - B)$.

\item[(c)] $f(A \cup B) = f(A) \cup f(B)$.
\item[Solution:] Let $y \in f(A \cup B)$. Then there exists $x \in A \cup
B$ such that $y = f(x)$. Since $x \in A \cup B \Rightarrow x \in A$, $y \in
f(A \cup B) \Rightarrow y \in f(A) \Rightarrow y \in f(A) \cup f(B)$.
Therefore, $f(A \cup B) \subseteq f(A) \cup f(B)$.

Now let $y \in f(A) \cup f(B)$. If $y \in f(A)$ then there exists an $x
\in A$ such that $y = f(x)$. In particular, for the chosen $y \in f(A)$,
there is an $x \in A \cup B$ such that $y = f(x)$. Therefore, $y$ is also
a member of $f(A \cup B)$. We can show the same thing for $y \in f(B)$.
Thus $y \in f(A) \cup f(B) \Rightarrow y \in f(A \cup B)$.

\item[(d)] Can we replace $\subseteq$ with $=$ for the inequalities in
(a) and (b)?
\item[Solution:] If $y \in f(A) \cap f(B)$ then $y \in f(A)$ and $y \in 
f(B)$. However, if $f$ is a many to one function then distinct elements of 
$X$ may may to $y$. Therefore, one cannot confirm that the preimage of $y$
under $f$ belongs to both $A$ and $B$.

If $y \in f(A - B)$ then there exists $x \in A - B$ such that $y = f(x)$.
Although $x \in A - B$, there could be an $x^\op \in B$ with $f(x^\op) = 
y$. Therefore, $y \in f(A)$ and $y \in f(B)$ and hence $y \notin f(A) - 
f(B)$.
\end{enumerate}

\item[4:] Let $f: X \rightarrow Y$ and $S, T \subseteq Y$. Show that
\begin{enumerate}
\item[(a)] $f^{-1}(S \cup T) = f^{-1}(S) \cup f^{-1}(T)$.
\item[Solution:] Let $x \in f^{-1}(S \cup T)$. Then $f(x) \in S \cup T$.
If $f(x) \in S$, then $x \in f^{-1}(S)$. If $f(x) \in T$, then $x \in 
f^{-1}(T)$. Since at least one of the possibilities is true, $x \in f^{-1}
(S) \cup f^{-1}(T)$.

Let $x \in f^{-1}(S) \cup f^{-1}(T)$. If $x \in f^{-1}(S)$, then $x \in
f^{-1}(S \cup T)$. Likewise, if $x \in f^{-1}(T)$, then $x \in f^{-1}(S
\cup T)$. Since both possibilities lead to the same conclusion, $x \in
f^{-1}(S \cup T)$ is always true.

\item[(b)] $f^{-1}(S \cap T) = f^{-1}(S) \cap f^{-1}(T)$.
\item[Solution:] Let $x \in f^{-1}(S \cap T)$. Then $y = f(x) \in S \cap T
\Rightarrow y = f(x) \in S \;\land\; y = f(x) \in T \Rightarrow x \in
f^{-1}(S) \;\land\; x \in f^{-1}(T) \Rightarrow x \in f^{-1}(S) \cap
f^{-1}(T)$.

Let $x \in f^{-1}(S) \cap f^{-1}(T)$. Then there exists $y = f(x)$ such 
that $y \in S \;\land\; y \in T \Rightarrow y \in S \cap T$ and hence $x
\in f^{-1}(S \cap T)$.

\item[(c)] $f^{-1}(S - T) = f^{-1}(S) - f^{-1}(T)$.
\item[Solution:] Let $x \in f^{-1}(S - T)$. Then there exists $y \in S - T$
such that $y = f(x)$. $y \in S \;\land\; y \notin T \Rightarrow x \in
f^{-1}(S) \;\land\; x \notin f^{-1}(T) \Rightarrow x \in f^{-1}(S) - 
f^{-1}(T)$.

Now let $x \in f^{-1}(S) - f^{-1}(T)$. Then $x \in f^{-1}(S)$ and $x \notin
f^{-1}(T)$. Therefore, if $y = f(x)$ then $y \in S$ and $y \notin T$, or
$y \in S - T$, that is $x \in f^{-1}(S - T)$.
\end{enumerate}
I think relations between preimages are exact equalities while those 
between images need not be because preimages are subsets of domain while 
images are subsets of the range. Elements of the domain map to a unique 
element in the range but the converse is not true. Further, not all 
elements of the codomain are mapped to an element in the domain.

\item[5:] Let $f: X \rightarrow Y, S \subseteq X, T \subseteq Y$. Then
$f(f^{-1}(T)) = T$ iff $f$ is surjective and $f^{-1}(f(S)) = S$ iff $f$
is injective.
\item[Solution:] Let $f$ be surjective. Then for all $t \in T$, there 
exists at least one $x \in X$ such that $t = f(x)$. Further, all these
$x$ lie in $f^{-1}(T)$ and their image lies in $f(f^{-1}(T))$. Thus, $T
\subseteq f(f^{-1}(T))$. We proved in exercise 2 that $f(f^{-1}(T)) 
\subseteq T$. Therefore, $T = f(f^{-1}(T))$.

Let $f$ be injective and $S \subset X$. Every $y \in Y$ will have at most
one preimage in $X$. However, every $y \in f(S) \subseteq Y$ will have
exactly one preimage in $S$. If $s \in S$ then $f(s) \in f(S)$ and 
because $f$ is injective $s \in f^{-1}(f(S))$ so that $S \subseteq 
f^{-1}(f(S))$. We proved in exercise 2 that $f^{-1}(f(S)) \subseteq S$.
Therefore $S = f^{-1}(f(S))$.

\item[6:] Let $X$ be a set. Show that $\{Y \;|\; Y \subseteq X\}$ is
also a set.
\item[Solution:] Choose $Z = \{0, 1\}$. Axiom (A9) assures us that there
exists a set $\{0, 1\}^X$ which consists of all functions from $f: X 
\rightarrow \{0, 1\}$. If $f \in \{0, 1\}^X$ then let $P_f(x)$ be true if
$f(x) = 1$ and false otherwise. Define the set $Y_f = \{x \in X \;|\;
P_f(x)\}$. The existence of $Y_f$ follows from axiom (A6). For every $f \in
\{0, 1\}^X$ there exists a unique $Y_f$. If $Q(f, Y_f)$ is true if $Y_f$
corresponds to $f$ then axiom (A7) assures us of the set
\[
\{Y_f: Q(f, Y_f), f \in \{0, 1\}^X\}.
\]

\item[7:] Let $X$ and $Y$ be sets. A partial function from $X$ to $Y$ is
a function $f: X^\op \rightarrow Y^\op$, where $X^\op \subseteq X$ and
$Y^\op \subseteq Y$. Show that the set of all partial functions is a set.
\item[Solution:] For a fixed choice of $X^\op, Y^\op$, axiom (A9) assures
us of the existence of the set ${Y^\op}^{X^\op}$. The previous exercise
proved the existence of set of all subsets of a set. Then by axiom (A6)
\[
Z = \{{Y^\op}^{X^\op} \;|\; X^\op \in 2^X, Y^\op \in 2^Y\}
\]
is a set whose elements are sets. Finally (A10) guarantees that the union
of all members of $Z$ is also a set. It is the set of all partial functions
from $X$ to $Y$.

\item[8:] Show that axiom (A5) follows from (A1), (A4) and (A12).
\item[Solution:] Let $A$ and $B$ be sets. Then by axiom if $a \in A$ and
$b \in B$, then by (A4) we have a set $\{a, b\}$. Now consider the set
whose elements are such pair sets. We can define it as
\[
X = \{ \{a, b\} \;|\; \forall\; a \in A, b \in B\}
\]
Then by axiom (A12), $\cup X$ to be the union of elements of elements of 
$X$, which is precisely the union of $A$ and $B$.

\item[9:] Show that if $\beta$ and $\beta^\op$ are two elements of a set
$I$, and to each $\alpha \in I$ we assign a set $A_\alpha$, then
\[
\{x \in A_\beta \;|\; x \in A_\alpha \forall\; \alpha \in I\} =
\{x \in A_{\beta^\op} \;|\; x \in A_\alpha \forall\; \alpha \in I\}
\]
so the definition in \eqref{c2s4e2} does not depend on the choice of 
$\beta$.
\item[Solution:] Let 
\[
y \in \{x \in A_\beta \;|\; x \in A_\alpha \forall\; \alpha \in I\}
\]
Then $y \in A_\alpha \forall \alpha \in I$. In particular $y \in 
A_{\beta^\op}$ and $y \in A_\alpha \forall \alpha \in I$. Therefore,
\[
y \in \{x \in A_{\beta^\op} \;|\; x \in A_\alpha \forall\; \alpha \in I\}
\]
Therefore,
\[
\{x \in A_\beta \;|\; x \in A_\alpha \forall\; \alpha \in I\} \subseteq
\{x \in A_{\beta^\op} \;|\; x \in A_\alpha \forall\; \alpha \in I\}
\]
We can similarly show that
\[
\{x \in A_{\beta^\op} \;|\; x \in A_\alpha \forall\; \alpha \in I\} 
\subseteq
\{x \in A_\beta \;|\; x \in A_\alpha \forall\; \alpha \in I\} \subseteq
\]

\item[9:] Suppose that $I$ and $J$ are two sets and for all $\alpha \in 
I \cup J$ let $A_\alpha$ be a set. Show that
\[
\left(\bigcup_{\alpha \in I} A_\alpha\right) \cup
\left(\bigcup_{\alpha \in J} A_\alpha\right) = 
\bigcup_{\alpha \in I \cup J} A_\alpha.
\]
If $I$ and $J$ are non-empty then show that
\[
\left(\bigcap_{\alpha \in I} A_\alpha\right) \cap
\left(\bigcap_{\alpha \in J} A_\alpha\right) = 
\bigcap_{\alpha \in I \cap J} A_\alpha.
\]
\item[Solution:] Let 
\[
x \in \left(\bigcup_{\alpha \in I} A_\alpha\right) \cup
\left(\bigcup_{\alpha \in I} A_\alpha\right)
\]
Then $x \in \cup_{\alpha \in I} A_\alpha$ or $x \in \cup_{\alpha \in J}
A_\alpha$ If $x \in A_{\bar{\alpha}}$ for some $\bar{\alpha} \in I$ then 
since $\bar{\alpha} \in I \cup J$, $x \in \cup_{\alpha \in I \cup J} 
A_\alpha$. Similarly,
\[
x \in \cup_{\alpha \in J}A_\alpha \Rightarrow x \in \cup_{\alpha \in I 
\cup J} A_\alpha
\]
so that
\[
\left(\bigcup_{\alpha \in I} A_\alpha\right) \cup
\left(\bigcup_{\alpha \in J} A_\alpha\right) \subseteq
\bigcup_{\alpha \in I \cup J} A_\alpha.
\]

Now let $x \in \cup_{\alpha \in I \cup J} A_\alpha$. Then there exists
$\bar{\alpha} \in I \cup J$ such that $x \in A_{\bar{\alpha}}$. If 
$\bar{\alpha} \in I$ then $x \in \cup_{\alpha \in I \cup J}A_\alpha$. The
same conclusion holds good if $\bar{\alpha} \in J$. Therefore we have
\[
\bigcup_{\alpha \in I \cup J} A_\alpha \subseteq
\left(\bigcup_{\alpha \in I} A_\alpha\right) \cup
\left(\bigcup_{\alpha \in J} A_\alpha\right).
\]

We will now prove similar result for the intersection of index sets. Given
that $I$ and $J$ are non-empty, there is $\beta \in I, \bar{\beta} \in J$
such that if
\[
x \in 
\left(\bigcap_{\alpha \in I} A_\alpha\right) \cap
\left(\bigcap_{\alpha \in J} A_\alpha\right) 
\]
then 
\[
x \in \left(\bigcap_{\alpha \in I} A_\alpha\right) \;\land\;
x \in \left(\bigcap_{\alpha \in J} A_\alpha\right) \Rightarrow 
\]
\[
x \in A_\alpha \;\forall\; \alpha \in I \;\land\; x \in A_\alpha
\;\forall\; \alpha \in J \Rightarrow
\]
\[
x \in A_\alpha \;\forall\; \alpha \in I \;\land\; \alpha \in J 
\Rightarrow x \in A_\alpha \;\forall\; \alpha \in I \cup J \Rightarrow
x \in \bigcap_{\alpha \in I \cup J} A_\alpha.
\]
Therefore,
\[
\left(\bigcap_{\alpha \in I} A_\alpha\right) \cap
\left(\bigcap_{\alpha \in J} A_\alpha\right) \subseteq
\bigcap_{\alpha \in I \cup J} A_\alpha.
\]
Now let
\[
x \in \bigcap_{\alpha \in I \cup J} A_\alpha.
\]
Then $x \in A_\alpha$ for all $\alpha \in I \cup J$. In particular, $x
\in A_\alpha$ for all $\alpha \in I$ and $\alpha \in J$. Therefore,
\[
x \in \bigcap_{\alpha \in I}A_\alpha \;\land\;
x \in \bigcap_{\alpha \in J}A_\alpha \Rightarrow 
x \in \left(\bigcap_{\alpha \in I}A_\alpha\right) \cap
\left(\bigcap_{\alpha \in J}A_\alpha\right)
\]
Therefore,
\[
\bigcap_{\alpha \in I \cup J} A_\alpha \subseteq
\left(\bigcap_{\alpha \in I}A_\alpha\right) \cap
\left(\bigcap_{\alpha \in J}A_\alpha\right)
\]

\item[11:] Let $X$ be a set. $I$ be a non-empty set, and for all $\alpha
\in I$ let $A_\alpha$ be a subset of $X$. Show that
\[
X - \bigcup_{\alpha \in I} A_\alpha = \bigcap_{\alpha \in I} (X - A_\alpha)
\]
and
\[
X - \bigcap_{\alpha \in I} A_\alpha = \bigcup_{\alpha \in I} (X - A_\alpha)
\]
\item[Solution:] Let
\[
x \in X - \bigcup_{\alpha \in I} A_\alpha
\]
then 
\[
x \notin \bigcup_{\alpha \in I} A_\alpha \Rightarrow x \notin A_\alpha
\;\forall\; \alpha \in I \Rightarrow x \in X - A_\alpha \;\forall\;
\alpha \in I \Rightarrow x \in \bigcap_{\alpha \in I}(X - A_\alpha)
\]
therefore,
\begin{equation}\label{c2s4e3}
X - \bigcup_{\alpha \in I} A_\alpha \subseteq \bigcap_{\alpha \in I}
(X - A_\alpha).
\end{equation}
If 
\[
x \in \bigcap_{\alpha \in I}(X - A_\alpha)
\]
then $x \in (X - A_\alpha)$ for all $\alpha \in I$ that is $x \notin 
A_\alpha \;\forall\; \alpha \in I$ or 
\[
x \notin \bigcup_{\alpha \in I} A_\alpha
\]
or
\[
x \in X - \bigcup_{\alpha \in I} A_\alpha.
\]
so that
\begin{equation}\label{c2s4e4}
\bigcap_{\alpha \in I}(X - A_\alpha) \subseteq X - \bigcup_{\alpha \in I} 
A_\alpha.
\end{equation}
From \eqref{c2s4e3} and \eqref{c2s4e4} we conclude that
\[
\bigcap_{\alpha \in I}(X - A_\alpha) = X - \bigcup_{\alpha \in I} 
A_\alpha.
\]

Next consider
\[
y \in \bigcap_{\alpha \in I}(X - A_\alpha).
\]
Therefore,
\[
y \in (X - A_\alpha) \;\forall\; \alpha \in I \Rightarrow y \notin
A_\alpha \;\forall\; \alpha \in I \Rightarrow y \notin \bigcup_{\alpha
\in I}A_\alpha \Rightarrow y \in X - \bigcup_{\alpha \in I}A_\alpha
\]
that is,
\begin{equation}\label{c2s4e5}
\bigcap_{\alpha \in I}(X - A_\alpha) \subseteq X - \bigcup_{\alpha \in I}
A_\alpha.
\end{equation}
On the other hand, if
\[
y \in X - \bigcup_{\alpha \in I}A_\alpha,
\]
$y$ is in none of $A_\alpha$ and hence it is in all of $X - A_\alpha$ and
hence it is a member of $\cap_{\alpha \in I}(X - A_\alpha)$ so that
\begin{equation}\label{c2s4e6}
X - \bigcup_{\alpha \in I}A_\alpha \subseteq \cap_{\alpha \in I}
(X - A_\alpha).
\end{equation}
The equality of the sets follows from \eqref{c2s4e5} and \eqref{c2s4e6}.
\end{enumerate}

\section{Cartesian products}\label{c2s5}
In the spirit of this chapter, we will prefer a set theoretic definition
of the cartesian product. It is given as an alternative to the more
conventional one in terms of an ordered pair. It is part of exercise
(3.5.1) of Tao's \cite{tao2014a1} book.
\begin{defn}\label{c2s5d1}
For objects $x$ and $y$ define the ordered pair 
\[
(x, y) := \{\{x\}, \{x,y\}\}
\]
\end{defn}

\begin{lem}\label{c2s5l1}
The set in definition \ref{c2s5d1} can be constructed from the axioms of
set theory.
\end{lem}
\begin{proof}
Since $x$ and $y$ are objects, by axiom (A4), $\{x\}$ is a set. By axiom
(A4) again, $\{x, y\}$ is also a set. Applying (A4) to the objects $\{x\}$
and $\{x, y\}$ again we get the set on the rhs of the definition. That
sets are themselves objects follows from (A1).
\end{proof}

\begin{lem}\label{c2s5l2}
$(x, y) = (x^\op, y^\op)$ iff $x = x^\op$ and $y = y^\op$.
\end{lem}
\begin{proof}
Let $(x, y) = (x^\op, y^\op)$. Then $\{{x}, \{x, y\}\} = \{{x^\op}, 
\{x^\op , y^\op \}\}$. Two sets are equal iff they have the same elements.
Therefore, $\{x\} \in \{\{x^\op\}, \{x^\op, y^\op\}\}$ so that either
$\{x\} = \{x^\op\}$ or $\{x\} = \{x^\op, y^\op\}$. The second possibility
is ruled out because, $x^\op \ne y^\op$ does not permit $x = x^\op$ and
$x = y^\op$. The first possibility gives $x = x^\op$. We are also required
to have $\{x, y\} = \{x^\op, y^\op\} = \{x, y^\op\}$. Since $x \ne y$
we must have $y = y^\op$.

To prove the converse, we note that $x = x^\op$ and $y = y^\op$ implies 
$\{x\} = \{x^\op\}$ and $\{x, y\} = \{x^\op, y^\op\}$, so that 
$\{\{x\}, \{x, y\}\} = \{\{x^\op\}, \{x^\op, y^\op\}\}$.
\end{proof}

Note that the definition cleverly differentiates $(x, y) = \{\{x\}, \{x,
y\}\}$ from $(y, x) = \{\{y\}, \{y, x\}\} = \{\{y\}, \{x, y\}\}$.

An alternative definition of an ordered pair is
\begin{defn}\label{c2s5d2}
If $x$ and $y$ are objects then their ordered pair is defined as
\[
(x, y) := \{x, \{x. y\}\}.
\]
\end{defn}
\begin{lem}\label{c2s5l3}
The set in definition \ref{c2s5d2} can be constructed from the axioms of
set theory.
\end{lem}
\begin{proof}
Since $x$ and $y$ are objects $\{x, y\}$ is a set by axiom (A4). Axiom
(A10) permits us to construct a set $\{x, \{x, y\}\}$, one of whose 
element is not a set.
\end{proof}

\begin{lem}\label{c2s5l4}
Using definition \ref{c2s5d2}, $(x, y) = (x^\op, y^\op)$ iff $x = x^\op$
and $y = y^\op$.
\end{lem}
\begin{proof}
Let $(x, y) = (x^\op, y^\op)$. Then $\{x, \{x, y\}\} = \{x^\op, \{x^\op,
y^\op\}\}$. Now $x$ is an object that is not a set and $\{x^\op, y^\op\}$ 
is a set. Therefore the two cannot be equal. That leaves the only 
possibility that $x = x^\op$. This, coupled with the other conclusion
$\{x, y\} = \{x^\op, y^\op\}$ gives us $y = y^\op$.

Conversely, if $x = x^\op$ and $y = y^\op$, $\{x, y\} = \{x^\op, y^\op\}$
and hence $\{x, \{x, y\}\} = \{x^\op, \{x^\op, y^\op\}\}$.
\end{proof}

\begin{defn}\label{c2d5d3}
If $X$ and $Y$ are sets then their cartesian product $X \times Y$ is 
defined as 
\[
a \in X \times Y \Leftrightarrow a = (x, y) \text{ for some} x \in X,
y \in Y.
\]
\end{defn}

We can generalise the concept of an ordered pair to an ordered n-tuple
using both a rigorous set theoretic definition and an axiomatic one. We
will start with the later and prove in the exercise that the axiom can 
be dropped.
\begin{defn}\label{c2s5d4}
If $n$ is a natural number then an ordered $n$-tuple is $(x_1, \ldots,
x_n)$ such that $(x_1, \ldots, x_n) = (y_1, \ldots, y_n)$ iff $x_i = y_i$
for all $1 \le i \le n$.
\end{defn}

We can also introduce an n-fold cartesian product as 
\begin{defn}\label{c2s5d5}
If $n$ is a natural number and $X_1, \ldots, X_n$ are sets than
\[
a \in X_1 \times \cdots \times X_n \Leftrightarrow a = (x_1, \ldots, x_n)
\text{ for some } x_i \in X_i \;\forall\; 1 \le i \ne n.
\]
\end{defn}

\begin{lem}\label{c2s5l5}
Let $n \ge 1$ be a natural number. Let $X_i$ be a non-empty set for each
$1 \le i \le n$. There exists an n-tuple $(x_1, \ldots, x_n)$ such that
$x_i \in X_i \;\forall\; 1 \le i \le n$. This makes $X_1 \times \cdots
\times X_n$ non-empty.
\end{lem}
\begin{proof}
Since $X_1$ is non empty there is at least one element $x_1 \in X_1$,
which can for a tuple $(x_1)$. This proves the base case. Assume that the
induction hypothesis is true for $n = k$. Now consider the sets $X_1, 
\ldots, X_k, X_{k+1}$. By induction hypothesis, there is at least one
tuple $(x_1, \ldots x_k)$. If $Y = X_1 \times X_k$ then the sets $Y$ and
$X_{k+1}$ are non empty and hence there is a tuple $((x_1, \ldots, x_k),
x_{k+1})$ for some $x_{k+1} \in X_{k+1}$. We once again used the induction
hypothesis. We can identify the 2-tuple $((x_1, \ldots, x_k), x_{k+1})$
with the (k+1)-tuple $(x_1, \ldots, x_k, x_{k+1})$ and close the induction.
\end{proof}

\subsection{Exercises}
\begin{enumerate}
\item[1:] We define a function $x: \{i \in \son\;|\; 1 \le i \le n\}
\rightarrow X$, where $X$ is an arbitrary set. Then each $x(i) \in X$. If
we denote $x(i)$ by $x_i$ then we see that the function $x$ is just an
n-tuple in disguise. Each n-tuple is a different function and the 
cartesian product is just a set of these functions. Show that if $x$ and 
$y$ are two such functions then $x = y \Leftrightarrow x_i = y_i \;\forall
\; 1 \le i \le n$.
\item[Solution:] If $x = y$ then by the definition of equality of 
functions, their ranges are identical and $x(i) = y(i)$ for all $1 \le i
\le n$. Conversely, if $x_i = y_i$, that is $x(i) = y(i)$ then the two
functions have same range and same values for same arguments.

\item[2:] Show that is $(X_1, \ldots, X_n)$ is an ordered n-tuple of sets
then the cartesian product defined in \ref{c2s5d5} is indeed a set.
\item[Solution:] For any n-tuple $x = (x_1, \ldots, x_n)$ one can construct
the set $X = \{x_i | x_i \in X_i \;\forall\; 1 \le i \le n\}$. The function
$x$ is a partial function from $\{1, \ldots, n\}$ to the set $X$. By 
exercise 7 of the previous section the collection of all partial functions
is itself a set. It is indeed the set of all n-tuples.

\item[3:] Show that the definition of equality of ordered pairs obeys
the reflexivity, symmetry and transitivity properties.
\item[Solution:] Since the equality of elements of sets obeys reflexivity,
for any $(x, y)$, $x = x, y = y \Rightarrow (x, y) = (x, y)$. Likewise,
since $x = u \Rightarrow u = x$ and $y = v \Rightarrow v = y$, $(x, y)
= (u, v) \Rightarrow (u, v) = (x, y)$. Further, since $x = u, u = s 
\Rightarrow x = s$ and $y = v, v = r \Rightarrow y = r$ we have $(x, y)
= (u, v)$ and $(u, v) = (s, r)$ implies $(x, y) = (s, r)$.

The proof can be easily extended to n-tuples.

\item[4:] Let $A, B, C$ be sets. Then show that
\begin{enumerate}
\item[(a)] $A \times (B \cup C) = A \times B \cup A \times C$. 
\item[Solution:] Let $(x, y)
\in A \times (B \cup C)$. Then $x \in A$ and $y \in B \cup C$. If $y \in
B$, $(x, y) \in A \times B$. If $y \in C$, $(x, y) \in A \times C$. In
any case, $(x, y) \in A \times B \cup A \times C$ so that $A \times (B
\cup C) \subseteq A \times B \cup A \times C$.

If $(x, y) \in A \times B \cup A \times C$, $(x, y) \in A \times B$ or
$(x, y) \in A \times C$. That is $x \in A$ and $y \in B$ or $y \in C$.
That is, $x \in A, y \in B \cup C$ or that $(x, y) \in A \times (B \cup C)$
so that $A \times B \cup A \times C \subseteq A \times (B \cup C)$.

\item[(b)] $A \times (B \cap C) = A \times B \cap A \times C$. 
\item[Solution:] Let $(x, y)
\in A \times (B \cap C)$. Then $x \in A$ and $y \in B \cap C$. Therefore,
$y \in B$ and $y \in C$ so that $(x, y) \in A \times B$ and $(x, y) \in A 
\times C$. That is, $(x, y) \in A \times B \cap A \times C$ so that
$A \times (B \cap C) \subseteq A \times B \cap A \times C$.

If $(x, y) \in A \times B \cap A \times C$, $(x, y) \in A \times B$ and
$(x, y) \in A \times C$. That is $x \in A$ and $y \in B$ and $y \in C$.
That is, $x \in A, y \in B \cap C$ or that $(x, y) \in A \times (B \cap C)$
so that $A \times B \cap A \times C \subseteq A \times (B \cap C)$.

\item[(c)] $A \times (B - C) = A \times B - A \times C$.
\item[Solution:] $(x, y) \in A \times (B - C) \Rightarrow x \in A, y \in
B - C \Rightarrow x \in A, y \in B, y \notin C \Rightarrow (x, y) \in A
\times B \;\land\; (x, y) \notin A \times C \Rightarrow (x, y) \in A 
\times B - A \times C$ so that $A \times (B - C) \subseteq A \times B - A 
\times C$.

If $(x, y) \in A \times B - A \times C$ then $(x, y) \in A \times B$ and
$(x, y) \notin A \times C$. $(x, y) \in A \times B \Rightarrow x \in A,
y \in B$. $(x, y) \notin A \times C$ means either $x \notin A$ or $y \notin
C$. But the former property is ruled out. Therefore, $y \notin C$, or 
that $y \in B - C$. Therefore, $(x, y) \in A \times (B - C)$ so that
$A \times B - A \times C \subseteq A \times (B - C)$.
\end{enumerate}

\item[5:] If $A, B, C, D$ are sets then
\begin{enumerate}
\item[(a)] Show that $A \times B \cap C \times D = A \cap C \times B \cap
D$.
\item[Solution:] If $(x, y) \in A \times B \cap C \times D$ then $(x, y)
\in A \times B$ and $(x, y) \in C \times D$ so that $x \in A, x \in C,
y \in B, y \in D$, that is $x \in A \cap C, y \in B \cap D$ or $(x, y)
\in A \cap C \times B \cap D$.

If $(x, y) \in A \cap C \times B \cap D$ then $x \in A \cap C, y \in B 
\cap D$ so that $x \in A, y \in B, x \in C, y \in D$ or $(x, y) \in A 
\times B, (x, y) \in C \times D$. That is, $(x, y) \in A \times C \cap
B \times D$.

\item[(b)] Is $A \times B \cup C \times D = A \cup C \times B \cup D$?
\item[Solution:] If $(x, y) \in A \times B \cup C \times D$, $(x, y) \in
A \times B$ or $(x, y) \in C \times D$. That is, $x \in A, y \in B$ or 
$x \in C, y \in D$. That is, $x \in A \cup C, y \in B \cup D$. Therefore,
$A \times B \cup C \times D \subseteq A \cup C \times B \cup D$.

Now let $(x, y) \in A \cup C \times B \cup D$. Therefore, $x \in A \cup C$
and $y \in B \cup D$. It is quite possible that $x \in C$ and $y \in B$ so
that $(x, y) \in C \times B$. Therefore, in general $(x, y) \notin 
A \times B \cup C \times D$.

\item[(c)] Is $A \times B - C \times D = A - C \times B - D$?
\item[Solution:] Let $(x, y) \in A \times B - C \times D$. Then $(x, y)
\in A \times B$ and $(x, y) \notin C \times D$. Thus, $x \in A, y \in B$
but either $x \notin C$ or $y \notin D$ or both. If both possibilities
are true then $x \in A - C, y \in B - D$ making $(x, y)$ a member of
$A - C \times B - D$. But if $x \in C, y \notin D$ then $x \notin A - C$
or $(x, y) \notin A - C \times B - D$.

If $(x, y) \in A - C \times B - D$ then $x \in A - C, y \in B - D$ so
that $x \in A, y \in B$ and $x \notin C, y \notin D$, that is $(x, y) \in
A \times B, (x, y) \notin C \times D$ or $(x, y) \in A \times B - C \times
D$. Thus, $A - C \times B - D \subseteq A \times B - C \times D$ but not
the other way round.
\end{enumerate}

\item[6:] Let $A, B, C, D$ be non-empty sets.
\begin{enumerate}
\item[(a)] Show that $A \times B \subseteq C \times D$ iff $A \subseteq
C$ and $B \subseteq D$.
\item[Solution:] Let $A \times B \subseteq C \times D$. Then $(x, y)
\in A \times B \Rightarrow (x, y) \in C \times D$. That is, $x \in A
\Rightarrow x \in C$ and $y \in B \Rightarrow y \in D$. That is, $A 
\subseteq C$ and $B \subseteq D$.

If $A \subseteq C$ and $B \subseteq D$, $x \in A \Rightarrow x \in B$
and $y \in C \Rightarrow y \in D$. That is $(x, y) \in A \times B 
\Rightarrow (x, y) \in C \times D$ or that $A \times B \subseteq C \times
D$.

\item[(b)] $A \times B = C \times D$ iff $A = C$ and $B = D$.
\item[Solution:] If $(x, y) \in A \times B \Leftrightarrow (x, y) \in
C \times D$ then $x \in A, y \in B \Leftrightarrow x \in C, y \in D$
which means that $A = C, B = D$.

The converse is trivially true.

\item[(c)] What happens if any of the sets are allowed to be empty?
\item[Solution:] If $A$ is an empty set then $A \subseteq B, C, D$ and
$A \times B = \varnothing \subset C \times D$. Same is true if $B$ is
empty.

If $C$ is empty and $A \subseteq C$ then $A$ is empty. Therefore $A \times
B$ and $C \times D$ are both empty and thus equal. However, $B$ and $D$
may be completely unrelated sets and yet the relation between $A \times B$
and $C \times D$ will be true.
\end{enumerate}

\item[7:] Let $X_1, X_2$ be sets and let $\pi_1: X_1 \times X_2 
\rightarrow X_1$ and $\pi_2: X_1 \times X_2 \rightarrow X_2$ be maps such 
that if $x = (x_1, x_2)$ then $\pi_1(x) := x_1, \pi_2(x) := x_2$. They
are called the coordinate functions on $X_1 \times X_2$. Show that for
any function $f_1: Z \rightarrow X_1, f_2: Z \rightarrow X_2$ there is a
unique function $f: Z \rightarrow X_1 \times X_2$ such that $\pi_1 \circ
f = f_1$ and $\pi_2 \circ f = f_2$. The function $f$ is known as the
direct sum of functions $f_1$ and $f_2$. It is written as $f = f_1 \oplus
f_2$.
\item[Solution:] For all $z \in Z$, we have $f_1(z) \in X_1$ and $f_2(z)
\in X_2$ or $(f_1(z), f_2(z)) \in X_1 \times X_2$. This defines the
function $f: Z \rightarrow X_1 \times X_2$ such that $f(z) = (f_1(z),
f_2(z))$ for all $z \in Z$. Further $(\pi_i \circ f)(z) = \pi_i(f(z))
= \pi_i((f_1(z), f_2(z)) = f_i(z)$ for $i = 1, 2$. This shows the 
existence of the direct sum.

To prove uniqueness, let $g: Z \times X_1 \times X_2$ be such that $\pi_i
\circ g = f_i$. Then for all $z \in Z$, $(\pi_i \circ g)(z) = f_i(z)$.
Therefore, $g(z) = (f_1(z), f_2(z)) = f(z)$ for all $z \in Z$ or that 
$g = f$.

\item[8:] Let $X_1, \ldots, X_n$ be sets then the cartesian product
$\prod_{i=1}^n X_i$ is empty iff at lease one of $X_i$ is empty.
\item[Solution:] If at least one of $X_i$ is empty then clearly the
Cartesian product is empty.

To prove the converse, we prove the contrapositive. If none of the $X_i$
are empty, that is all are non-empty, then by lemma \ref{c2s5l5} the
cartesian product is non-empty.

\item[9:] Let $I$ and $J$ be two sets and for all $\alpha \in I$ let
$A_\alpha$ be a set and for all $\beta \in J$ let $B_\beta$ be a set.
Show that $(\cup_{\alpha \in I} A_\alpha) \cap (\cup_{\beta \in J }
B_\beta) = \cup_{(\alpha, \beta) \in I \times J} (A_\alpha \cap B_\beta)$.
\item[Solution:] Let $x \in (\cup_{\alpha \in I} A_\alpha) \cap 
(\cup_{\beta \in J}B_\beta)$. Therefore, there exists at least one $\alpha
\in I, \beta \in J$ such that $x \in A_\alpha, x \in B_\beta$ or that
$(\alpha, \beta) \in I \times J$ such that $x \in A_\alpha \cap B_\beta$.
Therefore, $x \in \cup_{(\alpha, \beta) \in I \times J} (A_\alpha \cap
B_\beta)$.

Now let $x \in \cup_{(\alpha, \beta) \in I \times J} (A_\alpha \cap 
B_\beta)$. Then there exists $\alpha \in I, \beta \in J$ such that
$(\alpha, \beta) \in I \times J$ and $x \in A_\alpha, x \in B_\beta$.
Thus, $\alpha \in I \;\land\; x \in A_\alpha$ and $\beta \in J \;\land\;
x \in B_\beta$. Therefore $x \in \cup_{\alpha \in A} A_\alpha$ and $x
\in \cup_{\beta \in B} B_\beta$. That is, $x \in (\cup_{\alpha \in I} 
A_\alpha) \cap (\cup_{\beta \in J } B_\beta)$.

\item[10:] If $f:X \rightarrow Y$ is a function then its graph is the
subset of $X \times Y$ defined by $\{(x, f(x)) \;|\; x \in X\}$.
\begin{enumerate}
\item[(a)] Show that two functions $f: X \rightarrow Y, \tilde{f}: X
\rightarrow Y$ are equal iff they have the same graph.
\item[Solution:] If $f = \tilde{f}$ then $f(x) = \tilde{f}(x)$ for all 
$x \in X$. Therefore, $(x, f(x)) = (x, \tilde{f}(x))$ for all $x \in X$,
so that their graphs are identical.

Now assume that the graphs of the functions are identical. That is $\{(x,
f(x) \;|\; \forall x \in X\} = \{(x, \tilde{f}(x)\;|\; \forall x \in X\}$.
Thus, for every $(x, f(x))$ in the first set there is $(x, \tilde{f}(x))$
in the second one. For exery $x$ there is exactly one $f(x)$ and one 
$\tilde{f}(x)$. Therefore, we have $f(x) = \tilde{f}(x)$ for all $x$.

\item[(b)] Let $G \subseteq X \times Y$ such that for all $x \in X$, the
set $\{y \in Y: (x, y) \in G\}$ has exactly one element. Then show that 
there is exactly one function $f: X \rightarrow Y$ whose graph is $G$.
\item[Solution:] Define a function using the proposition $P(x, y)$ which
is defined for all $x \in X$ and which is true if and only if $(x, y) \in 
G$. Since there is a unique $y$ for each $x$, this proposition defines a
function $f: X \rightarrow Y$. The graph of the function is then the set
$(x, f(x)) \;|\; \forall x \in X\}$ and this is same as the set $G$. We
proved in part (a) that if there is a unique function that produces this
set.

\item[(c)] Suppose we define a function $f: X \rightarrow Y$ to be an 
ordered triple $f = (X, Y, G)$ where $X, Y$ are sets and $G$ is a subset 
of $X \times Y$ that obeys the conditions of problem 10(b). We define $X$
to be the domain of $f$, $Y$ to be its codomain and for every $x \in X$, we
define $f(x)$ to be the unique $y \in Y$ such that $(x, y) \in G$. Show
that this definition is compatible with definition \ref{c2s3d1}.
\item[Solution:] Assume the definition of a function as the ordered triple
$f = (X, Y, G)$. Let $P(x, y)$ be a logical proposition that for each 
$x \in X$ there is a unique $y \in Y$ such that $(x, y) \in G$. Then, by
our choice $P(x, y)$ is defined for all $x \in X$ and there is exactly
one $y \in Y$ for which $P(x, y)$ is true. Therefore, the statment
\[
y = f(x) \Leftrightarrow P(x, y) \text{ is true.}
\]
is true. Conversely, define $G$ to be a subset of $X \times Y$ as
\[
G = \{(x, y) \;|\; x \in X, P(x, y) \text{ is true.}\}
\]
Since for a given $x$, $P(x, y)$ is true for exactly one $y \in Y$, the
set $G$ satisfies the conditions of 10(b). Therefore $(X, Y, G)$ is indeed
a function and we call $y = f(x)$.
\end{enumerate}

\item[11:] Show that the power set axiom (A11) can be deduced from the
statement proved in problem 6 of the exercises in section \ref{c2s4} and
the other axioms of set theory.
\item[Solution:] I cannot solve this problem.

\end{enumerate}
