\chapter{Set Theory}\label{c2}

Set theory starts off being extremely simple to the point of appearing
trivial but soon gets very subtle. This chapter introduces axiomatic set
theory just enough to start a serious study of analysis. 

\section{Axioms of set theory}\label{c2s1}
We begin with an informal definition of a set as a well-defined, unordered
collection of objects. The terms 'collection' and 'object' are interpreted
as they are in common language. We assume that sets satisfy the following
axioms:
\begin{enumerate}
\item[(A1)] Sets too are objects. If $A$ is a set then it also an object.
In particular, given two sets $A$ and $B$ it is meaningful to ask if $A$
is a member of $B$.
\item[(A2)] Two sets $A$ and $B$ are equal iff every element of $A$ is an
element of $B$ and \emph{vice versa}.
\item[(A3)] There exists a set $\varnothing$ called the empty set which
contains no elements. In particular, for every object $x$, $x \notin 
\varnothing$.
\end{enumerate}

We now prove a few basic lemmas that follow from these axioms.
\begin{lem}\label{c2s1l1}
The empty set is unique.
\end{lem}
\begin{proof}
Let if possible $\varnothing^\op$ be another empty set. Then, for any 
object $x$, $x \in \varnothing \Leftrightarrow x \in \varnothing^\op$,
making them equal.
\end{proof}

\begin{lem}[Single choice]\label{c2s1l2}
If $A$ is a non-empty set then there exists an object $x$ such that $x \in
A$.
\end{lem}
\begin{proof}
Assume to the contrary. Then for every object $x$, $x \notin A$, which 
makes $A$ an empty set.
\end{proof}
