\chapter{Set Theory}\label{c2}

Set theory starts off being extremely simple to the point of appearing
trivial but soon gets very subtle. This chapter introduces axiomatic set
theory just enough to start a serious study of analysis. 

\section{Axioms of set theory}\label{c2s1}
We begin with an informal definition of a set as a well-defined, unordered
collection of objects. The terms 'collection' and 'object' are interpreted
as they are in common language. We assume that sets satisfy the following
axioms:
\begin{enumerate}
\item[(A1)] Sets too are objects. If $A$ is a set then it also an object.
In particular, given two sets $A$ and $B$ it is meaningful to ask if $A$
is a member of $B$.
\item[(A2)] Two sets $A$ and $B$ are equal iff every element of $A$ is an
element of $B$ and \emph{vice versa}.
\item[(A3)] There exists a set $\varnothing$ called the empty set which
contains no elements. In particular, for every object $x$, $x \notin 
\varnothing$.
\item[(A4)] If $x$ is an object then there exists a set $\{x\}$ whose only
element is $x$. A set with just a single element is called a singleton set. 
\item[(A5)] If $X$ and $Y$ are sets then there exists a set $X \cup Y$, 
called the union of $X$ and $Y$, which consists of all the elements which
belong to $X$ or $Y$. (The word 'or' is always in the inclusive sense.)
\item[(A6)] If $X$ is a set and if $P(x)$ is a logical proposition for 
each element $x$ of $X$ then there exists a set $\{x \in A \;|\; P(x)\}$
whose elements are precisely those elements of $X$ for which $P(x)$ is 
true.
\item[(A7)] Let $X$ be a set and $x$ be an arbitrary member of the set.
If $y$ is any object and we have a statement $P(x, y)$ which is true for
at most $y$. Then there exists a set $Y = \{y: P(x, y) \text{ is true for 
some } x \in A\}$ such that for any object $z$, $z \in Y \Leftrightarrow
P(x, z)$ is true for some $x \in A$.
\item[(A8)] There exists a set $\son$, whose elements are called natural
numbers. There is an object $0 \in \son$ and an object $S(n)$ assigned to
every natural number $n \in \son$ such that the axioms in section 
\ref{c1s1} hold.
\end{enumerate}

We now prove a few basic lemmas that follow from these axioms.
\begin{lem}\label{c2s1l1}
The empty set is unique.
\end{lem}
\begin{proof}
Let if possible $\varnothing^\op$ be another empty set. Then, for any 
object $x$, $x \in \varnothing \Leftrightarrow x \in \varnothing^\op$,
making them equal.
\end{proof}

\begin{lem}[Single choice]\label{c2s1l2}
If $A$ is a non-empty set then there exists an object $x$ such that $x \in
A$.
\end{lem}
\begin{proof}
Assume to the contrary. Then for every object $x$, $x \notin A$, which 
makes $A$ an empty set.
\end{proof}

\begin{lem}\label{c2s1l3}
If $x$ is an object then the singleton set containing $x$ is unique.
\end{lem}
\begin{proof}
Let $X = \{x\}$ and $Y = \{y\}$ be another singleton set containing $x$.
Since $Y$ is a singleton containing $x$, $x \in Y$ and hence $x = y$. Since
$y = x$, $y \in X$ is also true. Therefore $X = Y$.
\end{proof}

\begin{lem}\label{c2s1l4}
$\{x, y\} = \{y, x\}$ for any objects $x, y$.
\end{lem}
\begin{proof}
$x \in \{y, x\}, y \in \{y, x\}$ and $y \in \{x, y\}, x \in \{x, y\}$.
\end{proof}

\begin{lem}\label{c2s1l5}
If $X, Y, X^\op$ are sets, $X = X^\op$ then $X \cup Y = X^\op \cup Y$.
\end{lem}
\begin{proof}
Let $a \in X \cup Y$. Then $a \in X$ or $a \in Y$. Since $X = X^\op$, we
have $a \in X^\op$ or $a \in Y$, that is $a \in X^\op \cup Y$.

We can similarly show that if $b \in X^\op \cup Y$, $b \in X \cup Y$.
\end{proof}

\begin{lem}\label{c2s1l6}
If $x$ and $y$ are objects then $\{x\} \cup \{y\} = \{x, y\}$.
\end{lem}
\begin{proof}
$x \in \{x\} \cup \{y\}$ and $x \in \{x, y\}$. Similarly for $y$. 
Therefore, every element of $\{x\} \cup \{y\}$ is an element of $\{x, y\}$.
We can similarly show that every element of $\{x, y\}$ is an element of
$\{x\} \cup \{y\}$.
\end{proof}

\begin{lem}\label{c2s1l7}
If $X$ and $Y$ are sets then $X \cup Y = Y \cup X$.
\end{lem}
\begin{proof}
Let $a$ be an arbitrary element of $X \cup Y$. Then
$a \in X \cup Y \Rightarrow a \in X \;\lor\; a \in Y \Rightarrow a \in Y
\;\lor\; a \in X \Rightarrow a \in Y \cup X$.

Likewise, if $b$ is an arbitrary element of $Y \cup X$ then we can show 
that $b \in X \cup Y$.
\end{proof}

We can use the same idea to show that
\begin{lem}\label{c2s1l8}
If $X, Y, Z$ are sets then $X \cup (Y \cup Z) = (X \cup Y) \cup Z$.
\end{lem}

\begin{lem}\label{c2s1l9}
If $X$ is a set then $X \cup \varnothing = X$.
\end{lem}
\begin{proof}
If $a \in X$ then clearly $a \in X \cup \varnothing$. If $b \in X \cup
\varnothing$ then $b \in X$ or $b \in \varnothing$. But the statement $b 
\in \varnothing$ is always false. Therefore, the only way $b \in X 
\;\lor\; b \in \varnothing$ can be true is if $b \in X$ is true.
\end{proof}

From lemmas \ref{c2s1l7} and \ref{c2s1l9} we conclude that
\begin{lem}\label{c2s1l10}
If $X$ is a set then $\varnothing \cup X = X$.
\end{lem}

If $P$ is a proposition then $P \Leftrightarrow P \;\lor\; P$. Using this
we can conclude 
\begin{lem}\label{c2s1l11}
$X = X \cup X$ for any set $X$.
\end{lem}

\begin{defn}\label{c2s1d1}
If $X$ and $Y$ are sets then $X$ is a subset of $Y$ iff every element of 
$X$ is also an element of $Y$. The relation is denoted by $X \subseteq Y$.
\end{defn}

Proofs of lemmas \ref{c2s1l4} tp \ref{c2s1l11} can be written more 
elegantly if we use the notation of subsets because
\begin{lem}\label{c2s1l12}
$X = Y$ iff $X \subseteq Y$ and $Y \subseteq X$.
\end{lem}
\begin{proof}
$X = Y$ means every element of $X$ is an element of $Y$, that is, $X 
\subseteq Y$ and every element of $Y$ is an element of $X$, that is $Y 
\subseteq X$.
\end{proof}

Since $X = X$ for every set $X$ (follows from the logical idea of equality
of objects), lemma \ref{c2s1l12} leads us to
\begin{lem}\label{c2s2l13}
$X \subseteq X$ for all sets $X$.
\end{lem}

\begin{defn}\label{c2s1d2}
If $X \subseteq Y$ and $X \ne Y$ then $X$ is called a proper subset of 
$Y$ and is denoted by $X \subset Y$.
\end{defn}

\begin{lem}\label{c2s1l14}
If $X \subset Y$ and $Y \subset Z$ then $X \subset Z$.
\end{lem}
\begin{proof}
$Y \subset Z \Rightarrow \forall\; y \in Y, y \in Z$. Since $X \subset
Y$, $x \in Y, \forall\; x \in X$. Therefore, $\forall\; x \in X, x \in Z$.
\end{proof}

Since $X \subset Y \Rightarrow X \subseteq Y$, we also have
\begin{lem}\label{c2s1l15}
If $X \subseteq Y$ and $Y \subseteq Z$ then $X \subset Z$.
\end{lem}

\begin{lem}\label{c2s1l16}
$X = Y$ iff $X \subseteq Y$ and $Y \subseteq X$.
\end{lem}
\begin{proof}
$X = Y$ means,
\begin{itemize}
\item $\forall\; x \in X, x \in Y$. Therefore, $X \subseteq Y$ and
\item $\forall\; y \in Y, y \in X$. Therefore, $Y \subseteq X$.
\end{itemize}

Conversely,
\begin{itemize}
\item $X \subseteq Y \Rightarrow \forall\; x \in X, x \in Y$ and
\item $Y \subseteq X \Rightarrow \forall\; y \in Y, y \in X$.
\end{itemize}
Therefore, $X = Y$.
\end{proof}

\begin{lem}\label{c2s1l17}
$X \subseteq X \cup Y$ for any sets $X$ and $Y$.
\end{lem}
\begin{proof}
If $x \in X$ then $x \in X \cup Y$.
\end{proof}

Axiom (A6) allows us to define
\begin{defn}\label{c2s1d3}
The intersection of sets $X$ and $Y$ is the set $X \cap Y := \{x \in X
\;|\; x \in Y\}$.
\end{defn}
Note that although axiom (A6) allows us to define the intersection of two
sets it does not suffice to introduce union. That is why the operation of 
union has to be introduced as a separate axiom. Suppose we define, $X \cup
Y = \{x \in X, y \in Y \;|\; x \in X \;\lor\; y \in Y\}$. The clause $x \in
X, y \in Y$ is itself a union and we are using axiom (A6) to define a union
in terms of itself!

\begin{defn}\label{c2s1d4}
The sets $X$ and $Y$ are said to be disjoint if $X \cap Y = \varnothing$.
\end{defn}
Note that $X \cap Y \varnothing$ does not mean $X \ne Y$. In fact, $X
\cap Y = \varnothing \Rightarrow X \ne Y$ but not conversely.

A concept closely related to intersection is
\begin{defn}\label{c2s1d5}
The difference of sets $X$ and $Y$ is the set $X - Y := \{x \in X
\;|\; x \notin Y\}$.
\end{defn}

\begin{prop}\label{c2s1p1}
If $X, Y, Z$ are sets and if $U$ is a set such that $X \cup Y \cup Z 
\subseteq U$ then
\begin{enumerate}
\item $X \cup \varnothing = X$.
\item $X \cap \varnothing = \varnothing$.
\item $X \cup U = U$.
\item $X \cap U = X$.
\item $X \cup X = X$.
\item $X \cap X = X$.
\item $X \cup Y = Y \cup X$.
\item $X \cap Y$ = $Y \cap X$.
\item $X \cup (Y \cup Z) = (X \cup Y) \cup Z$.
\item $X \cap (Y \cap Z) = (X \cap Y) \cap Z$.
\item $X \cap (Y \cup Z) = X \cap Y \cup X \cap Z$.
\item $X \cup (Y \cap Z) = X \cup Y \cap X \cup Z$.
\item $X \cup (U - X) = U$.
\item $X \cap (U - X) = \varnothing$.
\item $U - (X \cap Y) = (U - X) \cup (U - Y)$.
\item $U - (X \cup Y) = (U - X) \cap (U - Y)$.
\end{enumerate}
\end{prop}
\begin{proof}
\begin{enumerate}
\item $X \cup \varnothing = \{x\;|\; x \in X \;\lor\; x \in varnothing\}$.
Since the statement $x \in \varnothing$ is vacuously false, $X \cup 
\varnothing = \{x\;|\; x \in X\} = X$.
\item $X \cap \varnothing = \{x\;|\; x \in X \;\land\; x \in varnothing\}$.
Since the statement $x \in \varnothing$ is vacuously false, there is no
$x$ such that $x \in X \cap \varnothing$, making it an empty set. Lemma
\ref{c2s1l1} assures us that the empty set is unique.
\item From lemma \ref{c2s1l17}, $U \subset X \cup U$. Since $X \subseteq
U$ by hypothesis, every element of $X \cup U$ is a subset of $U$ or $X 
\cup U \subseteq U$.
\item Since $X \subseteq U$ by hypothesis, $X \cap U = \{x\;|\; x \in X 
\;\land \; x \in U\} = \{x\;|\; x \in X\} = X$.
\item Same as lemma \ref{c2s1l11}.
\item $X \cap X = \{x \in X \;|\; x \in X\} = \{x \in X\} = X$.
\item Same as lemma \ref{c2s1l7}.
\item $a \in X \cap Y \Leftrightarrow a \in X \;\land\; a \in Y 
\Leftrightarrow a \in Y \;\land\; a \in X \rightarrow Y \cap X$.
\item Same as lemma \ref{c2s1l8}.
\item $a \in X \cap (Y \cap Z) \Leftrightarrow a \in X \;\land\; a \in 
Y \cap Z \Leftrightarrow a \in X \;\land a \in Y \;\land\; a \in Z
\Leftrightarrow a \in X \cap Y \;\land\; a \in Z \Leftrightarrow a \in
(X \cap Y) \cap Z$.
\item Let $a \in X \cap (Y \cup Z)$. Then $a \in X$ and $a \in Y \cup Z$.
If $a \in Y$ then $a \in X \cap Y$. If $a \in Z$ then $a \in X \cap Z$. In
any case, $a \in X \cap Y$ or $a \in X \cap Z$. That is $a \in (X \cap Y) 
\cup (X \cap Z)$.

Now let $a \in (X \cap Y) \cup (X \cap Z)$. Then $a \in X \cap Y$ or $a
\in X \cap Z$, that is $a \in X$ and $a \in Y$ or $a \in Z$.
\item $a \in X \cup (Y \cap Z) \Leftrightarrow a \in X \;\lor\; a \in Y 
\cap Z \Leftrightarrow (a \in X \;\lor\; a \in Y) \;\land\; (a \in X 
\;\lor\; a \in Z) \Leftrightarrow a \in (X \cup Y) \cap (X \cup Z)$.

\item $a \in X \cup (U - X) \Leftrightarrow a \in X \;\lor\; a \in U, a 
\notin X$. Since $X \subseteq U$, $a \in X \Rightarrow a \in U$. Therefore,
$a \in X \cup (U - X) \Leftrightarrow (a \in X, a \in U) \;\lor\; (a \notin
X, a \in ) \Leftrightarrow a \in U$.

\item $a \in X \cap (U - X) \Leftrightarrow a \in X \;\land\; a \in U - X
\Leftrightarrow a \in X \;\land\; a \in U, a \notin X$. Now, $X \subseteq
U \Rightarrow a \in U \;\forall\; a \in X$. Therefore $a \in X \cap (U - X)
$ will require $a$ to be simultaneously in and not in $X$. There is no such
$a$. Therefore $X \cap (U - X) = \varnothing$.

\item Let $a \in U - (X \cap Y)$. Then $a \in U$ and $a \notin X \cap Y$.
Since $X, Y \subseteq U$, this means that $a$ is $a \in X, a \notin Y$ or
$a \notin X, a \in Y$ or $a \notin X, a \notin Y$. $a \in X$ and $a \notin
Y \Rightarrow a \in U - Y$. Similarly, $a \in Y, a \notin X \Rightarrow
a \in U - X$. Therefore, the three possibilities are $a \in U - Y$ or $a
\in U - X$ or $a \in U - Y \;\land\; a \in U - X$. Therefore, $U - (X \cap
Y) \subseteq (U - X) \cup (U - Y)$.

Now let $a \in (U - X) \cup (U - Y)$. Then $a \in U - X$ or $a \in U - Y$ 
or both. This can happen if 1) $a \notin X, a \in Y$ or 2) $a \in X, a 
\notin Y$ or 3) $a \notin X, a \notin Y$. These three possibilities can
be combined as $a$ is not a member of $X \cap Y$. That is $(U - X) \cup
(U - Y) \subseteq U - (X \cap Y)$.

\item Let $a \in U - (X \cup Y)$. Then $a$ is not a member of $X$ and
neither is it a member of $Y$. That is $a \in U - X$ and $a \in U - Y$. 
Therefore $U - (X \cup Y) \subseteq (U - X) \cap (U - Y)$.

If $a \in (U - X) \cap (U - Y)$ then $a \in U - X$ and $a \in U - Y$, that
is $a \notin X$ and $a \notin Y$, which is same as $a \in U - (X \cap Y)$.
\end{enumerate}
\end{proof}

Statements in the above proposition are collectively called as the laws
of Boolean algebra.

\section{Exercises}
\begin{enumerate}
\item[Ex 3.1.5 ] Let $A, B$ be sets. Show that the three statements $A
\subseteq B$, $A \cup B = B$, $A \cap B = A$ are logically equivalent.
\footnote{Exercise numbers are same as the one in Tao's books 
\cite{tao2014a1} and \cite{tao2014a2}.}
\item[Solution: ] Assume that $A \subseteq B$. Then for all $a \in A, a
\in B$ so that $A \cup B = \{x \;|\; x \in A \;\lor\; x \in B\} = 
\{x \:|\; x \in B\} = B$.

Assume that $A \cup B = B$. Then every element of $A \cup B$ is an element
of $B$. In particular, every element of $A$ is also an element of $B$. 
Therefore, the set $A \cap B = \{x \;|\; x \in A \;\land\; x \in B\} = 
\{x \;|\; x \in A\} = A$. (Note that every element of $A$ is an element of
$B$ does not mean that the converse is also true. That is why in the
second last set cannot be written as $\{x \;|\; x \in B\}$.)

Now assume that $A \cap B = A$. $A \cap B = B \cap A = \{x \in B | x \in
A\}$. If this set is same as $A$ then every element of $A$ is an element
of $B$. That is $A \subseteq A$.

\item[Ex 3.1.9 ] Let $A, B, X$ be sets such that $A \cup B = X$ and $A
\cap B = \varnothing$. Show that $A = X - B$ and $B = X - A$.
\item[Solution: ] Let $a \in X - B$, that is $a \in (A \cup B) - B$. Thus,
$a \in A \cup B \;\land\; a \notin B$. Therefore, $a \in A \;\land\; A 
\notin B$, that is $a \in A$ alone. Since $a$ was an arbitrary element
of $X - B$, this just means that $X - B \subseteq A$.

Now, let $a \in A$ so that $a \in X$. Since $A \cap B = \varnothing$, $a
\notin B$. Thus, $a \in X \;\land\; a \notin B$ or $a \in X - B$. Since
$a$ was an arbitrary element of $A$ this just means that $A \subseteq X -
B$.

\item[Ex 3.1.10 ] Let $A$ and $B$ be sets. Show that the three sets $A - B$,
$A \cap B$ and $B - A$ are disjoint and that their union is $A \cup B$.
\item[Solution: ] If $a \in A - B$ then $a \in A \;\land\; a \notin B$. 
Therefore, $a \notin A \cap B$ and $a \notin B - A$. Likewise, if $b \in
B - A$ then $b \notin A \cap B$ and $b \notin A$.

Let $X = (A - B) \cup (A \cap B) \cup (B - A)$. If $x \in X$, then $x$ is
a member of one of the three disjoint sets, $A - B, A \cap B$ and $B - A$,
each of which is a subset of $A \cup B$. Therefore $X \subseteq A \cup B$.

Let $x \in A \cup B$. If $x \in A$ then $x \in A - B \cup A \cap B$, making
$x$ a member of $X$. We can similarly show that if $x \in B$ then $x \in
X$. Thus, $x \in A \cup B \Rightarrow x \in X$ or $A \cup B \subseteq X$.

\item[Ex 3.1.11 ] Show that the axiom of replacement implies the axiom of
specification.
\item[Solution: ] Put $y = x$ in the axiom of replacement.
\end{enumerate}
