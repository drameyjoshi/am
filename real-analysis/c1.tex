\chapter{The Natural Numbers}\label{c1}
We begin with a systematic study of the natural numbers. We will prove their
properties, familiar to us for a long time, using a set of axioms. We do so
to learn the art of proving properties from a limited set of axioms. In the 
case of natural numbers, we know the properties. Therefore it is just a 
matter of getting the proofs right. Later on, the properties may not be 
obvious and we will bank on the skills of writing proofs.

\section{Axioms defining natural numbers}\label{c1s1}
We start with the naive notion of a set and denote the set of natural 
numbers by $\son$. Members of $\son$ satisfy the following axioms:
\begin{enumerate}
\item[(A1)] $0$ is a natural number.
\item[(A2)] If $n$ is a natural number, so is its successor, denoted by 
$S(n)$.
\item[(A3)] $0$ is not a successor of any natural number.
\item[(A4)] $S(n) = S(m)$ iff $n = m$, for all natural numbers $n$ and $m$.
\item[(A5)] If $K$ is a set such that $0$ is a member of $K$ and if a 
natural number $n$ is its member then so is $S(n)$ then $K$ contains all 
natural numbers.
\end{enumerate}
The axiom (A5) is called the principle of mathematical induction.

\section{Addition of natural numbers}\label{c1s2}
\begin{defn}\label{c1s2d1}
Let $m$ be a natural number. Then $0 + m := m$ and $S(n) + m := S(n + m)$.
\end{defn}

We will now prove the familiar properties of addition. We will first show 
that the order of operations in the definition \ref{c1s2d1} does not matter.
\begin{prop}\label{c1s2p1}
$m + 0 = m$ for every natural number.
\end{prop}
\begin{proof}
We will use induction on $m$. Since $0$ is a natural number, from 
\ref{c1s2d1}, $0 + 0 = 0$. Suppose that $n + 0 = n$. Then $S(n) + 0 = 
S(n + 0)$ by the definition \ref{c1s2d1}. However, by induction hypothesis, 
$n + 0 = n$. Therefore, we have $S(n) + 0 = S(n)$.
\end{proof}

\begin{prop}\label{c1s2p2}
$n + S(m) = S(n + m)$ for all natural numbers $n$ and $m$.
\end{prop}
\begin{proof}
We will use induction on $n$. For $n = 0$, the lhs is $0 + S(m)$, which
is $S(m)$ be definition \ref{c1s2d1}. The rhs is $S(0 + m)$ which is $S(m)$
once again by definition \ref{c1s2d1}. Assume that the hypothesis is true
for $n$ and consider $S(n) + S(m) = S(n + S(m)) = S(S(n+m))$ by induction
hypothesis. However, by definition \ref{c1s2d1}, $S(n + m) = S(n) + m$ so
that $S(n) + S(m) = S(S(n) + m)$, closing the induction.
\end{proof}

\begin{rem}
In this proof we cannot induct on $m$. While trying to prove the base case
we consider $S(n + 0)$. From proposition \ref{c1s2p1}, $n + 0 = n$. 
Therefore, $S(n + 0) = S(n)$. However, we still have not proved that $S(n)
= n + 1$.
\end{rem}

\begin{prop}\label{c1s2p3}
For any pair $m, n$ of natural numbers $m + n = n + m$.
\end{prop}
\begin{proof}
We fix $n$ and induct on $m$. For the base case of $m = 0$, $m + 0 = m$ by
proposition \ref{c1s2p1} while $0 + m = m$ be definition of addition. 
Assume that the hypothesis is true for $m$ and consider $S(m) + n = S(m+n)$
by definition. By induction hypothesis, $S(m + n) = S(n + m)$, which is $
n + S(m)$ by proposition \ref{c1s2p2}. Thus, we have $S(m) + n = n + S(m)$,
closing the induction.
\end{proof}

\begin{defn}\label{c1s2d2}
A natural number $n$ is said to be positive iff $n \ne 0$.
\end{defn}

\begin{prop}\label{c1s2p4}
If $m$ is a natural number and $n$ is positive then $m + n$ is positive.
\end{prop}
\begin{proof}
Fix $n$ and induct on $m$. When $m = 0$, $m + n = 0 + n = n$, by definition
\ref{c1s2d1} of addition. $n$ is given to be positive. Therefore, $0 + n$
is positive.

Assume that the hypothesis is true for $m$ and consider $S(m) + n = S(m + 
n)$. Thus, $S(m) + n$ is a successor of a natural number. By axiom (A3), 
$S(m) + n$ is not zero and therefore is positive.
\end{proof}

\begin{cor}\label{c1s2c1}
If $a + b = 0$ for natural numbers $a$ and $b$ then $a = 0$ and $b = 0$.
\end{cor}
\begin{proof}
If any one of them were positive then proposition \ref{c1s2p4} guarantees 
that $a + b$ would be positive.
\end{proof}

\begin{prop}\label{c1s2p5}
Let $a$ be a positive number then there exists exactly one natural number 
$b$ such that $S(b) = a$.
\end{prop}
\begin{proof}
That every positive number can be written as $S(b)$ for some natural number
$b$ follows from the axioms defining the natural numbers. Let, if possible,
there be a natural number $c$ such that $S(b) = S(c)$ then axiom (A4)
assures us that $b = c$.
\end{proof}

\begin{prop}\label{c1s2p6}
If $a, b, c$ are natural numbers then $a + (b + c) = (a + b) + c$.
\end{prop}
\begin{proof}
We will fix $a$ and $b$ and induct on $c$. When $c = 0$, $a + (b + 0) = a
+ b$ and $(a + b) + 0 = a + b$. Assume that the hypothesis is true for a 
$c$ and consider, $a + (b + S(c))$. By proposition \ref{c1s2p2}, $b + S(c)
= S(b + c)$. Therefore, $a + (b + S(c)) = a + S(b + c) = S(a + (b + c))$. 
Using the induction hypothesis, $a + (b + S(c)) = S((a + b) + c)$. Using
proposition \ref{c1s2p2}, $S((a + b) + c) = (a + b) + S(c)$. Thus, we have,
$a + (b + S(c)) = (a + b) + S(c)$, closing the induction.
\end{proof}

\begin{prop}\label{c1s2p7}
If $a + b = a + c$ then $b = c$.
\end{prop}
\begin{proof}
We will fix $b$ and $c$ and induct on $a$. If $a = 0$ then $a + b = a + c
\Rightarrow 0 + b = 0 + c \Rightarrow b = c$. Assume that the hypothesis is
true for some natural number $a$ and consider $S(a) + b = S(a) + c 
\Rightarrow S(a + b) = S(a + c)$. From axiom (A4) we have $a + b = a + c$
which implies $b = c$, closing the induction.
\end{proof}

\begin{prop}\label{c1s2p8}
For any natural number $n$, $S(n)$ is positive.
\end{prop}
\begin{proof}
$0$ is not a successor of any number and all other natural numbers are
successors of a natural number.
\end{proof}

From now on we will assume that $0$ is the additive identity in $\son$, 
addition of natural numbers is a natural number, addition is commutative 
and associative. The set $\son$ is thus a commutative monoid.

\section{Ordering of natural numbers}\label{c1s3}
\begin{defn}\label{c1s3d1}
Let $a, b$ be natural numbers. We say that $a \ge b$ if there exists a 
natural number $c$ such that $a = b + c$. We say that $a > b$ if $a \ge b$
and $a \ne b$.
\end{defn}

\begin{rem}
$a \ge b$ can also be written as $b \le a$ and $a > b$ can be written as 
$b < a$.
\end{rem}

\begin{lem}\label{c1s3l1}
For all natural numbers $a$, $a \ge a$.
\end{lem}
\begin{proof}
Follows from proposition \ref{c1s2p1} that $a = a + 0$ for all natural
numbers.
\end{proof}

\begin{lem}\label{c1s3l2}
If $a \ge b$ and $b \ge c$ then $a \ge c$.
\end{lem}
\begin{proof}
$a \ge b \Rightarrow \exists m$ such that $a = b + m$. Likewise, there 
exists a natural number $n$ such that $b = c + n$. Thus $a = (c + n) + m
= c + (n + m) \Rightarrow a \ge c$.
\end{proof}

\begin{lem}\label{c1s3l3}
If $a \ge b$ and $b \ge a$ then $a = b$.
\end{lem}
\begin{proof}
We have natural numbers $m$ and $n$ such that $a = b + m$ and $b = a + n$.
Combining the two, we have $a = a + (n + m)$. Since $a = a + 0$, by
proposition \ref{c1s2p1}, we also have $a + 0 = a + (n + m)$. From 
proposition \ref{c1s2p7} we get $n + m = 0$ and from corollary \ref{c1s2c1}
we get $m = 0$ and $n = 0$.
\end{proof}

\begin{lem}\label{c1s3l4}
$a \ge b$ iff $a + c \ge b + c$.
\end{lem}
\begin{proof}
Assume $a \ge b$ so that there exists a natural number $n$ such that $a = 
b + n$. Therefore, $a + c = b + n + c \Rightarrow a + c = b + c + n
\Rightarrow a + c \ge b + c$.

Now assume that $a + c \ge b + c$. Therefore, there exists a natural number
$n$ such that $a + c = b + c + n \Rightarrow a + c = b + n + c$. Using the
cancellation law proved by \ref{c1s2p7}, we have $a = b + n \Rightarrow
a \ge b$.
\end{proof}

\begin{lem}\label{c1s3l5}
$a > b$ iff $a \ge S(b)$.
\end{lem}
\begin{proof}
Assume that $a \ge S(b)$ so that there exists a natural number $n$ such 
that $a = S(b) + n \Rightarrow S(b + n) \Rightarrow b + S(n)$. For any 
natural number $n$, $S(n)$ is positive. Therefore $a > b$.

Now assume that $a > b$ so that there exists a positive number $n$ such that
$a = b + n$. Since $n$ is a positive number there exists a natural number
$m$ such that $n = S(m)$. Therefore, we have $a = b + S(m) = S(b + m) = 
S(b) + m$ so that $a \ge S(b)$.
\end{proof}

\begin{lem}\label{c1s3l6}
$a > b$ iff there exists a positive $n$ such that $a = b + n$.
\end{lem}
\begin{proof}
The 'if' part is just the definition of a positive number. Now assume that
$a > b$ so that by lemma \ref{c1s3l6}, $a \ge S(b)$ so that there exists 
a natural number $m$ such that $a \ S(b) + m = S(b + m) = b + S(m)$. For 
any natural number $m$, $S(m)$ is positive from proposition \ref{c1s2p8}.
Therefore that $a = b + n$, where $n =S(m)$ is positive.
\end{proof}
