\chapter{Integers and Rationals}\label{c3}
Natural numbers admit addition and multiplication but not their inverse
operations. Integers permit us to define subtraction, which is inverse of 
addition and rationals allow us to introduce division as an inverse of
multiplication. We start with constructing the integers.

\section{The Integers}\label{c3s1}
\begin{defn}\label{c3s1d1}
We define integers as an ordered pair of natural numbers and denote the 
set of all integers as $\soi$. Two integers $(a, b)$ and $(c, d)$ are 
equal iff $a + d = c + b$.
\end{defn}

\begin{lem}\label{c3s1l1}
The equality of integers is an equivalance relation.
\end{lem}
\begin{proof}
Let $a. b, c, d, e, f \in \son$. Clearly $(a, b) = (a, b)$ because $a + b
= a + b$, the last equality follows from the properties of equality of 
natural numbers. Let $(a, b) = (c, d)$, that is $a + d = c + b$. But the
equality of natural numbers is symmetric. Therefore, $c + b = a + d$ so
that $(c, d) = (a, b)$.

In order to prove transitivity, let $(a, b) = (c, d)$ and $(c, d) = 
(e, f)$. Therefore, $a + d = c + b$ and $c + f = d + e$. Therefore $a + d
+ c + f = c + b + d + e$. Using commutativity and associativity of addition
of natural numbers, $a + f + c + d = e + b + c + d$. Using proposition 
\ref{c1s2p7}, we get $a + f = e + b$ or that $(a, b) = (e, f)$.
\end{proof}

\begin{defn}\label{c3s1d2}
The sum of two integers $(a, b)$ and $(c, d)$ is defined as $(a + c, 
b + d)$.
\end{defn}
\begin{rem}
This definition is very similar to that of addition of two dimensional 
vectors.
\end{rem}

\begin{lem}\label{c3s2l2}
Addition of integers is a well-defined operation.
\end{lem}
\begin{proof}
Let $(a, b), (c, d), (e, f)$ be integers such that $(a, b) = (c, d)$. Then
we will show that $(a, b) + (e, f) = (c, d) + (e, f)$. The lhs of this 
claim is $(a + e, b + f)$ and rhs is $(c + e, d + f)$. The two will be equal
if $a + e + d + f = c + e + b + f$. Now, $(a, b) = (c, d) \Rightarrow 
a + d = b + c$. Therefore, using the law of cancellation of natural numbers
in proposition \ref{c1s2p7} and the associativity of natural numbers we 
get $e + f = e + f$, which is true from the equality of natural numbers.
\end{proof}

\begin{defn}\label{c3s1d3}
The product of two integers $(a, b)$ and $(c, d)$ is defined as $(ac + bd,
ad + bc)$.
\end{defn}

\begin{lem}\label{c3s3l3}
Multiplication of integers is a well-defined operation.
\end{lem}
\begin{proof}
Let $(a, b), (c, d), (e, f)$ be integers such that $(a, b) = (c, d)$. Then
we will show that $(a, b) \times (e, f) = (c, d) \times (e, f)$. That is,
we will prove that
\[
(ae + bf, af + be) = (ce + df, cf + de),
\]
or
\[
ae + bf + cf + de = ce + df + af + be
\]
or
\[
(a + d)e + (b + c)f = (c + b)e + (d + a)f.
\]
This is indeed true because $(a, b) = (c, d) \Rightarrow a + d = c + b$ and
because addition is commutative in natural numbers.
\end{proof}

\begin{lem}\label{c3s1l4}
There is a bijection between the set of integers $\{(a, 0) \;|\; a \in 
\son\}$ and $\son$ that preserves addition and multitplication. That is, 
if we denote the bijection by $f$ then $f(x + y) = f(x) + f(y)$ and $f(xy)
= f(x)f(y)$.
\end{lem}
\begin{proof}
Let $f: \{(a, 0) \;|\; a \in \son\} \rightarrow \son$ be such that $f(a) =
a$. Then for any $n \in \son$, there is an integer $(n, 0)$ such that
$f((n, 0)) = n$ making $f$ surjective. If $f((a, 0)) = f((b, 0))$ then
$a = b$ and $a + 0 = b + 0$ and hence $(a, 0) = (b, 0)$. Thus $f$ is also
injective.

We will now show that $f$ preserves addition and multiplication. $(a, 0) +
(b, 0) = (a + b, 0) \Rightarrow f((a, 0) + (b, 0)) = f((a+b,0)) = a+b
= f(a) + f(b)$. Likewise, $(a, 0)(b, 0) = (ab+0,0+0)=(ab,0)$ so that
$f((a, 0)(b, 0)) = f((ab, 0)) = ab = f(a, 0)f(b, 0)$.

We also note that $f(0, 0) = 0$ and $f(1, 0) = 1$
\end{proof}

This bijection allows us to identify an integer $(a, 0)$ with natural 
numbers $a$. In an algebraic setting such a bijection would be called an
\emph{isomorphism}.

\begin{defn}\label{c3s1d4}
The negation of an integer $(a, b)$ is defined as $(b, a)$ and is denoted
by $-(a, b)$.
\end{defn}

\begin{lem}\label{c3s1l5}
Negation is a well-defined operation.
\end{lem}
\begin{proof}
Let $(a, b) = (c, d)$ that $a + d = b + c$. Then $-(a, b) = (b, a)$ and
$-(c, d) = (d, c)$. They will be equal iff $b + c = d + a$, which is 
indeed true.
\end{proof}

\begin{lem}\label{c3s1l6}
Let $x$ be an integer then exactly one of the three statements is true:
(a) $x$ is zero; (b) $x$ is equal to $(a, 0)$ or (c) $x$ is equal to 
$(0, a)$, for some $a \in \son$.
\end{lem}
\begin{proof}
Let $x = (m, n)$. By the law of trichotomy for the natural numbers, 
\ref{c1s3t1}, either $m = n$ or $m < n$ or $m > n$. If $m = n$ then we
observe that $(m, n) = (0, 0)$ because $m + 0 = n + 0$. In this case, $x$
is indeed $(0, 0)$. If $m > n$ then there is a natural number $a$ such
that $m = n + a$ that is $m + 0 = n + a$ of $(m, n) = (a, 0)$. If $n > m$
then there is a natural number $a$ such that $n + 0 = m + a$ or $(n, m) =
(a, 0)$ or $(m, n) = (0, a) = -(a, 0)$.
\end{proof}

\begin{rem}
Note that we have not defined order relation between integers yet. That is
why we did not case this lemma as a statment about two rationals $x$ and 
$y$.
\end{rem}

Combining this lemma with lemma \ref{c3s1l4} we see that any integer is
either zero or equal to a natural number $n > 0$ or its negation is equal 
to a natural number $n > 0$. These integers are called zero, positive and
negative.

We will now prove the laws of algebra for integers in a series of lemmas.
In each of these $x, y, z$ will denote integers.
\begin{lem}\label{c3s1l7}
Addition of integers is commutative.
\end{lem}
\begin{proof}
Let $x = (a, b), y = (c, d)$ then $x + y = (a + c, b + d) = (c + a, d + b)$
because addition is commutative in $\son$. But the last integer is $y + x$.
\end{proof}

\begin{lem}\label{c3s1l8}
Addition of integers is associative.
\end{lem}
\begin{proof}
Let $x = (a, b), y = (c, d), z = (e, f)$. Then $(a, b) + ((c, d) + (e, f))
= (a, b) + (c + e, d + f) = (a + (c + e), b + (d + f)) = ((a + c) + e, 
(b + d), f) = (a + c, b + d) + (e, f) = ((a, b) + (c, d)) + (e, f)$.
\end{proof}

\begin{lem}\label{c3s1l9}
$0$ is an additive identity for integers.
\end{lem}
\begin{proof}
Let $x = (a, b)$, Then $x + 0 = (a, b) + (0, 0) = (a + 0, b + 0) = (a, b)
= x$. By lemma \ref{c3s1l7} $0 + x = x$.
\end{proof}

\begin{lem}\label{c3s1l10}
$-x$ is the additive inverse of $x$.
\end{lem}
\begin{proof}
The negation of an integer $x = (a, b)$ was defined as $-x = (b, a)$. Then
$-x + x = (b, a) + (a, b) = (a + b, a + b) = (0, 0) = 0$. By lemma 
\ref{c3s1l7} we also have $x + (-x) = 0$.
\end{proof}

\begin{rem}
Lemmas \ref{c3s1l7} to \ref{c3s1l10} make the set of integers an abelian
group.
\end{rem}

\begin{lem}\label{c3s1l11}
Multiplication of integers is commutative.
\end{lem}
\begin{proof}
Let $x = (a, b), y = (c, d)$ then $xy = (a, b)(c, d) = (ac + bd, ad + bc)
= (ca + db, da + cb) = (c, d)(a, b)$, where we used the commutativity of
multiplication in $\son$.
\end{proof}

\begin{lem}\label{c3s1l12}
Multiplication of integers is associative.
\end{lem}
\begin{proof}
Let $x = (a, b), y = (c, d), z = (e, f)$. Then $x(yz) = (a, b)(cf + de, ce
+ df) = (a(ce + df) + b(cf + de), a(cf + de) + b(ce + df)) = ((ad + bc)f +
(ac + bd)e, (ad + dc)e + (ac + bd)f) = ((ad + bc), (ac + bd))(e, f) = 
((a, b)(c, d))(e, f) = (xy)z$. We used the associativity of natural numbers
and proposition \ref{c1s5p5} concerning the distributivity of 
multiplication over addition.
\end{proof}

\begin{lem}\label{c3s1l13}
$1$ is a multiplicative identity in integers.
\end{lem}
\begin{proof}
Let $x = (a, b)$. Then $1 \times x = (1, 0)(a, b) = (1 \times a + 0 \times 
b, 1 \times b + 0 \times a) = (a, b) = x$. Commutativity of multiplication,
\ref{c3s1l11}, assures us that $x \times 1 = x$.
\end{proof}

