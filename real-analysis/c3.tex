\chapter{Integers and Rationals}\label{c3}
Natural numbers admit addition and multiplication but not their inverse
operations. Integers permit us to define subtraction, which is inverse of 
addition and rationals allow us to introduce division as an inverse of
multiplication. We start with constructing the integers.

\section{The Integers}\label{c3s1}
\begin{defn}\label{c3s1d1}
We define integers as an ordered pair of natural numbers and denote the 
set of all integers as $\soi$. Two integers $(a, b)$ and $(c, d)$ are 
equal iff $a + d = c + b$.
\end{defn}

\begin{lem}\label{c3s1l1}
The equality of integers is an equivalance relation.
\end{lem}
\begin{proof}
Let $a. b, c, d, e, f \in \son$. Clearly $(a, b) = (a, b)$ because $a + b
= a + b$, the last equality follows from the properties of equality of 
natural numbers. Let $(a, b) = (c, d)$, that is $a + d = c + b$. But the
equality of natural numbers is symmetric. Therefore, $c + b = a + d$ so
that $(c, d) = (a, b)$.

In order to prove transitivity, let $(a, b) = (c, d)$ and $(c, d) = 
(e, f)$. Therefore, $a + d = c + b$ and $c + f = d + e$. Therefore $a + d
+ c + f = c + b + d + e$. Using commutativity and associativity of addition
of natural numbers, $a + f + c + d = e + b + c + d$. Using proposition 
\ref{c1s2p7}, we get $a + f = e + b$ or that $(a, b) = (e, f)$.
\end{proof}

\begin{defn}\label{c3s1d2}
The sum of two integers $(a, b)$ and $(c, d)$ is defined as $(a + c, 
b + d)$.
\end{defn}
\begin{rem}
This definition is very similar to that of addition of two dimensional 
vectors.
\end{rem}

\begin{lem}\label{c3s1l2}
Addition of integers is a well-defined operation.
\end{lem}
\begin{proof}
Let $(a, b), (c, d), (e, f)$ be integers such that $(a, b) = (c, d)$. Then
we will show that $(a, b) + (e, f) = (c, d) + (e, f)$. The lhs of this 
claim is $(a + e, b + f)$ and rhs is $(c + e, d + f)$. The two will be equal
if $a + e + d + f = c + e + b + f$. Now, $(a, b) = (c, d) \Rightarrow 
a + d = b + c$. Therefore, using the law of cancellation of natural numbers
in proposition \ref{c1s2p7} and the associativity of natural numbers we 
get $e + f = e + f$, which is true from the equality of natural numbers.
\end{proof}

\begin{defn}\label{c3s1d3}
The product of two integers $(a, b)$ and $(c, d)$ is defined as $(ac + bd,
ad + bc)$.
\end{defn}

\begin{lem}\label{c3s1l3}
Multiplication of integers is a well-defined operation.
\end{lem}
\begin{proof}
Let $(a, b), (c, d), (e, f)$ be integers such that $(a, b) = (c, d)$. Then
we will show that $(a, b) \times (e, f) = (c, d) \times (e, f)$. That is,
we will prove that
\[
(ae + bf, af + be) = (ce + df, cf + de),
\]
or
\[
ae + bf + cf + de = ce + df + af + be
\]
or
\[
(a + d)e + (b + c)f = (c + b)e + (d + a)f.
\]
This is indeed true because $(a, b) = (c, d) \Rightarrow a + d = c + b$ and
because addition is commutative in natural numbers.
\end{proof}

\begin{lem}\label{c3s1l4}
There is a bijection between the set of integers $\{(a, 0) \;|\; a \in 
\son\}$ and $\son$ that preserves addition and multitplication. That is, 
if we denote the bijection by $f$ then $f(x + y) = f(x) + f(y)$ and $f(xy)
= f(x)f(y)$.
\end{lem}
\begin{proof}
Let $f: \{(a, 0) \;|\; a \in \son\} \rightarrow \son$ be such that $f(a) =
a$. Then for any $n \in \son$, there is an integer $(n, 0)$ such that
$f((n, 0)) = n$ making $f$ surjective. If $f((a, 0)) = f((b, 0))$ then
$a = b$ and $a + 0 = b + 0$ and hence $(a, 0) = (b, 0)$. Thus $f$ is also
injective.

We will now show that $f$ preserves addition and multiplication. $(a, 0) +
(b, 0) = (a + b, 0) \Rightarrow f((a, 0) + (b, 0)) = f((a+b,0)) = a+b
= f(a) + f(b)$. Likewise, $(a, 0)(b, 0) = (ab+0,0+0)=(ab,0)$ so that
$f((a, 0)(b, 0)) = f((ab, 0)) = ab = f(a, 0)f(b, 0)$.

We also note that $f(0, 0) = 0$ and $f(1, 0) = 1$
\end{proof}

This bijection allows us to identify an integer $(a, 0)$ with natural 
numbers $a$. In an algebraic setting such a bijection would be called an
\emph{isomorphism}.

\begin{defn}\label{c3s1d4}
The negation of an integer $(a, b)$ is defined as $(b, a)$ and is denoted
by $-(a, b)$.
\end{defn}

\begin{lem}\label{c3s1l5}
Negation is a well-defined operation.
\end{lem}
\begin{proof}
Let $(a, b) = (c, d)$ that $a + d = b + c$. Then $-(a, b) = (b, a)$ and
$-(c, d) = (d, c)$. They will be equal iff $b + c = d + a$, which is 
indeed true.
\end{proof}

\begin{lem}\label{c3s1l6}
Let $x$ be an integer then exactly one of the three statements is true:
(a) $x$ is zero; (b) $x$ is equal to $(a, 0)$ or (c) $x$ is equal to 
$(0, a)$, for some $a \in \son$.
\end{lem}
\begin{proof}
Let $x = (m, n)$. By the law of trichotomy for the natural numbers, 
\ref{c1s3t1}, either $m = n$ or $m < n$ or $m > n$. If $m = n$ then we
observe that $(m, n) = (0, 0)$ because $m + 0 = n + 0$. In this case, $x$
is indeed $(0, 0)$. If $m > n$ then there is a natural number $a$ such
that $m = n + a$ that is $m + 0 = n + a$ of $(m, n) = (a, 0)$. If $n > m$
then there is a natural number $a$ such that $n + 0 = m + a$ or $(n, m) =
(a, 0)$ or $(m, n) = (0, a) = -(a, 0)$.
\end{proof}

\begin{rem}
Note that we have not defined order relation between integers yet. That is
why we did not case this lemma as a statment about two rationals $x$ and 
$y$.
\end{rem}

Combining this lemma with lemma \ref{c3s1l4} we see that any integer is
either zero or equal to a natural number $n > 0$ or its negation is equal 
to a natural number $n > 0$. These integers are called zero, positive and
negative.

We will now prove the laws of algebra for integers in a series of lemmas.
In each of these $x, y, z$ will denote integers.
\begin{lem}\label{c3s1l7}
Addition of integers is commutative.
\end{lem}
\begin{proof}
Let $x = (a, b), y = (c, d)$ then $x + y = (a + c, b + d) = (c + a, d + b)$
because addition is commutative in $\son$. But the last integer is $y + x$.
\end{proof}

\begin{lem}\label{c3s1l8}
Addition of integers is associative.
\end{lem}
\begin{proof}
Let $x = (a, b), y = (c, d), z = (e, f)$. Then $(a, b) + ((c, d) + (e, f))
= (a, b) + (c + e, d + f) = (a + (c + e), b + (d + f)) = ((a + c) + e, 
(b + d), f) = (a + c, b + d) + (e, f) = ((a, b) + (c, d)) + (e, f)$.
\end{proof}

\begin{lem}\label{c3s1l9}
$0$ is an additive identity for integers.
\end{lem}
\begin{proof}
Let $x = (a, b)$, Then $x + 0 = (a, b) + (0, 0) = (a + 0, b + 0) = (a, b)
= x$. By lemma \ref{c3s1l7} $0 + x = x$.
\end{proof}

\begin{lem}\label{c3s1l10}
$-x$ is the additive inverse of $x$.
\end{lem}
\begin{proof}
The negation of an integer $x = (a, b)$ was defined as $-x = (b, a)$. Then
$-x + x = (b, a) + (a, b) = (a + b, a + b) = (0, 0) = 0$. By lemma 
\ref{c3s1l7} we also have $x + (-x) = 0$.
\end{proof}

\begin{rem}
Lemmas \ref{c3s1l7} to \ref{c3s1l10} make the set of integers an abelian
group.
\end{rem}

\begin{lem}\label{c3s1l11}
Multiplication of integers is commutative.
\end{lem}
\begin{proof}
Let $x = (a, b), y = (c, d)$ then $xy = (a, b)(c, d) = (ac + bd, ad + bc)
= (ca + db, da + cb) = (c, d)(a, b)$, where we used the commutativity of
multiplication in $\son$.
\end{proof}

\begin{lem}\label{c3s1l12}
Multiplication of integers is associative.
\end{lem}
\begin{proof}
Let $x = (a, b), y = (c, d), z = (e, f)$. Then $x(yz) = (a, b)(cf + de, ce
+ df) = (a(ce + df) + b(cf + de), a(cf + de) + b(ce + df)) = ((ad + bc)f +
(ac + bd)e, (ad + dc)e + (ac + bd)f) = ((ad + bc), (ac + bd))(e, f) = 
((a, b)(c, d))(e, f) = (xy)z$. We used the associativity of natural numbers
and proposition \ref{c1s5p5} concerning the distributivity of 
multiplication over addition.
\end{proof}

\begin{lem}\label{c3s1l13}
$1$ is a multiplicative identity in integers.
\end{lem}
\begin{proof}
Let $x = (a, b)$. Then $1 \times x = (1, 0)(a, b) = (1 \times a + 0 \times 
b, 1 \times b + 0 \times a) = (a, b) = x$. Commutativity of multiplication,
\ref{c3s1l11}, assures us that $x \times 1 = x$.
\end{proof}

\begin{lem}\label{c3s1l14}
Multiplication is distributive over addition. That is, $x(y + z) = xy + xz$
and $(x + y)z = xz + yz$.
\end{lem}
\begin{proof}
Let $x = (a, b), y = (c, d), z = (e, f)$. Then $x(y+z) = (a, b)(c+e, d+f)
= (ad + af + bc + be, ac + ae + bd + bf) = (ad+bc, ac+bd) + (af+be, ae+bf)
= (a, b)(c, d) + (a, b)(e, f) = xy + xz$. The other equality can be 
similarly proved.
\end{proof}

Lemmas \ref{c3s1l7} to \ref{c3s1l14} make the set of integers a commutative
ring.

\begin{defn}\label{c3s1d5}
Subtraction of an integer $b$ from an integer $a$ is defined as $a - b = 
a + (-b)$.
\end{defn}
Since addition and negation are well-defined operations on the set of 
integers so is subtraction.

We next observe that
\begin{prop}
If the product of two integers $x$ and $y$ is zero then at least one of 
them is zero.
\end{prop}
\begin{proof}
Let $x = (a, b), y = (c, d)$ so that $xy = (ad + bc, ac + bd)$. If $xy = 0$
then $(ad + bc, ac + bd) = (0, 0)$ that is $ad + bc + 0 = ac + bd + 0$ or
$ad + bc = ac + bd$. Now let us assume that $x \ne 0$ that is $a \ne b$. 
Without loss of generality, assume that $a > b$ or that $a = b + k$, where
$k > 0$ is a natural number. Then $ad + bc = ac + bd \Rightarrow bd + bc +
kd = bc + kc + bd \Rightarrow kd = kc \Rightarrow d = c$ or $y = 0$. If
$a < b$ then $b = a + k$ and we get similar conclusion. Thus if $xy = 0,
x \ne 0$ then $y = 0$. We can similarly show that if $xy = 0, y \ne 0$ 
then $x = 0$.
\end{proof}

\begin{lem}\label{c3s1l15}
If $x, y, z \in \soi$ are such that $xz = yz$ and if $z \ne 0$ then $x=y$.
\end{lem}
\begin{proof}
$xz = xz \Rightarrow xz - xz = 0 \Rightarrow (x - y)z = 0$. Since $z \ne 0$
we have $x - y = 0$ or $x = y$.
\end{proof}

\begin{defn}\label{c3s1d6}
If $x, y \in \soi$ we say $x \ge y$ or $y \le x$ if there is $n \in \son$
such that $x = y + n$. If $n > 0$ then $x > y$, or $y < x$.
\end{defn}

We will now prove a few properties of the order relation.
\begin{lem}\label{c3s1l16}
If $x, y \in \soi$, $x > y$ iff $x - y$ is a positive natural number.
\end{lem}
\begin{proof}
$x > y \Rightarrow x = y + k$ where $k \in \son, k > 0$. Therefore $x +
(-y) = y + k + (-y) \Rightarrow x - y = k > 0$.
\end{proof}

\begin{lem}\label{c3s1l17}
If $x, y, z \in \soi$ and $x > y$ then $x + z > y + z$.
\end{lem}
\begin{proof}
$x > y \Rightarrow x = y + k$ where $k \in \son, k > 0$. Then $x + z =
y + z + k$ or $x + z > y + z$.
\end{proof}

\begin{lem}\label{c3s1l18}
If $x > y$, $z > 0$ then $xz > yz$.
\end{lem}
\begin{proof}
If $z > 0$ then $z = m > 0$ where $m \in \son$. Note that we are treating
$z = (m, 0) = m$. $x > y \Rightarrow x = y + k$, where $k \in \son, k > 0$.
Therefore $xz = (y + k)z = yz + kz = yz + km$. If $k > 0, m > 0$ then $km
> 0$. Therefore $xz > yz$.
\end{proof}

\begin{lem}\label{c3s1l19}
If $x > y$, $z < 0$ then $xz < yz$.
\end{lem}
\begin{proof}
If $z < 0$ then $z = (0, m)$ for some natural number $m$. Then $zk = (0, m)
(k, 0) = (0 + 0, 0 + mk) = (0, mk) < 0$ and hence $-zk = (mk, 0) > 0$. Now
$x > y$ implies $x = y + k$ where $k \in \son, k > 0$. Therefore, $xz = yz
+ zk \Rightarrow xz + (-zk) = yz$. Since $(-zk) > 0$, $xz < yz$.
\end{proof}

\begin{lem}\label{c3s1l20}
If $x > y$ then $-x < -y$.
\end{lem}
\begin{proof}
$x > y \Rightarrow x = y + k \Rightarrow 0 = -x + y + k \Rightarrow -y = 
-x + k \Rightarrow -y > -x$.
\end{proof}

\begin{lem}\label{c3s1l21}
If $x < y$ and $y < z$ then $x < z$.
\end{lem}
\begin{proof}
$x < y \Rightarrow y = x + k$, $y < z \rightarrow z = y + l = x + k + l$.
If $k > 0, l > 0$ then $k + l > 0$ by proposition \ref{c1s2p4}. Therefore,
$z > x$.
\end{proof}

\begin{lem}\label{c3s1l22}
If $x, y \in \soi$ then exactly one of the three possibilities: $x > y$,
$x < y$, $x = y$, is true.
\end{lem}
\begin{proof}
Let $z = x - y$. Then by lemma \ref{c3s1l6} applied to $z$ we have exactly
one of the possibilities $z > 0, z < 0, z = 0$. If $z > 0$ then $x - y > 0$
or, by lemma \ref{c3s1l17} $x > y$. Similarly, we can show that $z < 0
\Rightarrow x < y$ and $z = 0 \Rightarrow x = y$.
\end{proof}

\subsection{Problems}
\begin{enumerate}
\item[1:] Show that for all $x \in \soi$, $(-1)x = -x$.
\item[Solution:] Let $x = (a, b)$. Then $(-1)x = (0, 1)(a, b) = (0 + b, 0 +
a) = (b, a) = -(a, b) = -x$.

\item[2:] Show that for any integer $x$, $x^2 \ge 0$.
\item[Solution:] If $x > 0$ then $x \times x > 0 \times x = 0$, by lemma
\ref{c3s1l18}. If $x < 0$ then $x \times x > 0 \times x = 0$ by lemma 
\ref{c3s1l19}.
\end{enumerate}

\section{The rational numbers}\label{c3l2}
The way integers were built from natural numbers as a certain subset of 
$\son \times \son$, we will construct rational numbers from integers as a
subset of $\soi \times \soi$.

\begin{defn}\label{c3l2d1}
We define rational numbers as an ordered pair $(a, b)$ of integers such 
that $b \ne 0$ and denote the set of all rationals as $\soq$. Two 
rationals $(a, b)$ and $(c, d)$ are equal iff $ad = bc$.
\end{defn}

\begin{lem}\label{c3s2l1}
The equality of rationals is an equivalence relation.
\end{lem}
\begin{proof}
If $(a, b)$ is a rational number then $(a, b) = (b, a)$ because $ab = ba$
is true for $a, b \in \soi$. If $(a, b) = (c, d)$ then $ad = bc 
\Rightarrow cb = da \Rightarrow (c, d) = (a, b)$. Once again we have used
the commutativity of multiplication of integers. If $(a, b) = (c, d)$ and
$(c, d) = (e, f)$ then $ad = cb$ and $cf = de$ and hence $adcf = cbde$. 
Since $d \ne 0$, we have $acf = cbe$. If $c \ne 0$ we have $af = be$ 
which implies $(a, b) = (e, f)$. If $c = 0$ then we necessarily have $a=0$
because $b \ne 0$ and $e = 0$ because $f \ne 0$. Once again we have $(a, b)
= (e, f)$.
\end{proof}

\begin{defn}\label{c3s2d2}
The sum of two rationals $(a, b)$ and $(c, d)$ is defined as
$(ad + bc, bd)$.
\end{defn}

\begin{lem}\label{c3s2l2}
Addition of rationals is a well-defined operation.
\end{lem}
\begin{proof}
Let $(a, b), (c, d), (e, f)$ be rationals such that $(a, b) = (c, d)$.
Then $(a, b) + (e, f) = (af + eb, bf)$ and $(c, d) + (e, f) = (cf+ed, df)$.
The two sums will be equal iff $(af + eb)df = (cf + ed)bf \Leftrightarrow
adf^2 + bdef = bcf^2 + bdef \Rightarrow adf^2 = bcf^2$. Since $f \ne 0,
f^2 \ne 0$ and the two sums will be equal iff $ad = bc$. However that is 
true because $(a, b) = (c, d)$.
\end{proof}

\begin{defn}\label{c3s2d3}
The product of two rationals $(a, b)$ and $(c, d)$ is defined to be $(ac,
bd)$.
\end{defn}

\begin{lem}\label{c3s2l3}
Product of rationals is a well-defined operation.
\end{lem}
\begin{proof}
Let $(a, b), (c, d), (e, f)$ be rationals such that $(a, b) = (c, d)$.
Then $(a, b) \times (e, f) = (ae, bf)$ and $(c, d) \times (e, f)=(ce, df)$.
The two products will be equal iff $(ae, bf) = (ce, df) \Leftrightarrow
adef = bcef$. Since $f \ne 0$, the products will be equal iff $ade = bce$.
Now $e$ must non-zero, otherwise we have $x\times 0 = y\times 0$ for 
rationals $x \ne y$. With that assumption, the products will be equal iff
$ad = bc$. However, that is true because $(a, b) = (c, d)$.
\end{proof}

\begin{defn}\label{c3s2d4}
The negation of a rational number $(a, b)$ is defined to be the rational
number $(-a, b)$.
\end{defn}

\begin{lem}\label{c3s2l4}
Negation of rationals is a well-defined operation.
\end{lem}
\begin{proof}
Let $(a, b) = (c, d)$. Then their negations will be equal if $-ad = -cb$,
which is indeed true because the equality of integers gives $ad = bc$.
\end{proof}

\begin{lem}\label{c3s2l5}
There is a bijection between the set of rationals $\{(a, 1) \;|\; a \in 
\soq\}$ and $\soi$ that preserves addition and multiplication. That is, if
we denote the bijection by $f$ then $f(x + y) = f(x) + f(y)$ and $f(xy)
= f(x)f(y)$ for all $x, y \in \{(a, 1) \;|\; a \in \soq\}$.
\end{lem}
\begin{proof}
Let $f: \{(a, 1) \;|\; a \in \soq\} \rightarrow \soi$ such that $f((a, 1))
= a$. Then $f((a, 1) + (b, 1)) = f((a + b, 1)) = a + b = f((a, 1)) + f((b,
1))$. Similarly, $f((a, 1)(b, 1)) = f((ab, 1)) = ab = f((a, 1))f((b, 1))$.
\end{proof}

\begin{defn}\label{c3s2d5}
Subraction of rationals is defined as $(a, b) - (c, d) := (a, b) + 
(-(c, d))$.
\end{defn}

\begin{lem}\label{c3s2l6}
A rational number $(a, b)$ is zero iff $a = 0$.
\end{lem}
\begin{proof}
$(a, b) = (0, 1)$ iff $a \times 1 = b \times 0$ iff $a = 0$.
\end{proof}

\begin{defn}\label{c3s2d6}
If $x = (a, b)$ is a rational number with $a \ne 0$ (and $b \ne 0$) then
its reciprocal is defined as $x^{-1} := (b, a)$.
\end{defn}

We will now prove a series of lemmas proving elementary properties of
rationals.
\begin{lem}\label{c3s2l7}
Addition of rationals is commutative.
\end{lem}
\begin{proof}
Let $x = (a, b), y = (c, d)$ so that $x + y = (ad + cb, bd) = (da + bc, db)
= (c, d) + (a, b)$.
\end{proof}

\begin{lem}\label{c3s2l8}
Addition of rationals is associative.
\end{lem}
\begin{proof}
Let $x = (a, b), y = (c, d), z = (e, f)$ then $x + (y + z) = (a, b) + 
(cf + ed, df) = (adf + bcf + bed, bdf) = (ad + bc, bd) + (e, f) = ((a, b)
+ (c, d)) + (e, f) = (x + y) + z$.
\end{proof}

\begin{lem}\label{c3s2l9}
$0 = (0, 1)$ is the additive identity.
\end{lem}
\begin{proof}
Let $x = (a, b)$. Then $x + 0 = (a, b) + (0, 1) = (a + 0, b) = (a, b)$.
$0 + x = x + 0$ follows from \ref{c3s2l7}.
\end{proof}

\begin{lem}\label{c3s2l10}
For any $x \in \soq$, $-x$ is the additive inverse.
\end{lem}
\begin{proof}
Let $x = (a, b)$. Then $-x = (-a, b)$ and $x + (-x) = (a, b) + (-a, b) = 
(ab - ba, b^2) = (0, b^2) = 0$. From \ref{c3s2l7} $(-x) + x = 0$ is also
true.
\end{proof}

\begin{lem}\label{c3s2l11}
Multiplication of rationals is commutative.
\end{lem}
\begin{proof}
Let $x = (a, b), y = (c, d)$ so that $xy = (a, b)(c, d) = (ac, bd) = (ca,
db) = (c, d)(a, b) = yx$. We used commutativity of multiplication of 
integers in this proof.
\end{proof}

\begin{lem}\label{c3s2l12}
Multiplication of rationals is associative.
\end{lem}
\begin{proof}
Let $x = (a, b), y = (c, d), z = (e, f)$ so that $x(yz) = (a(ce), b(df)) =
((ac)e, (bd)f) = (ac, bd)(e, f) = ((a, b)(c, d))(e, f) = (xy)z$
\end{proof}

\begin{lem}\label{c3s2l13}
$1 = (1, 1)$ is a multiplicative identity.
\end{lem}
\begin{proof}
Let $x = (a, b)$ so that $1 \times x = (1, 1)(a, b) = (a, b) = (a, b)(1, 1)
$ so that we have $1x = x = x1$.
\end{proof}

\begin{lem}\label{c3s2l14}
Multiplication distributes over addition.
\end{lem}
\begin{proof}
Let $x = (a, b), y = (c, d), z = (e, f)$. Then $x(y + z) = (a, b)((c, d)
+ (e, f)) = (a, b)(cf + de, df) = (acf + ade, bdf) = (bacf + bade, bdbf) =
(ac, bd) + (ae, bf) = (a, b)(c, d) + (a, b)(e, f) = xy + xz$. We can 
similarly show that $(x + y)z = xz + yz$.
\end{proof}

\begin{lem}\label{c3s2l15}
If $x \ne 0$ then $xx^{-1} = x^{-1}x = 1$.
\end{lem}
\begin{proof}
Let $x = (a, b)$ where $a \ne 0, b \ne 0$. Then $x^{-1} = (b, a)$ so that
$xx^{-1} = (a, b)(b, a) = (ab, ba) = (ab, ab) = (1, 1)$. Similarly $x^{-1}x
= (b, a)(a, b) = (ba, ab) = (ab, ab) = (1, 1)$. We used commutativity of
multiplication of integers in the proof.
\end{proof}

Lemmas \ref{c3s2l7} to \ref{c3s2l15} make $\soq$ a field.

\begin{defn}\label{c3s2d7}
A rational number $x = (a, b)$ is said to be positive of both $a$ and $b$
are positive. It is said to be negative if $x = -y$ for some positive 
number $y$
\end{defn}

\begin{lem}\label{c3s2l16}
If $x = (-a, -b)$ for $a, b \in \soi$ and $a, b > 0$ then $x$ is positive.
\end{lem}
\begin{proof}
$(a, b) = (-a, -b)$ because $-ab = -ba$.
\end{proof}

\begin{lem}\label{c3s2l17}
If $x = (-a, b)$ where $a, b \in \soi$ and both a positive then $x$ is 
negative. Moreover, $x$ is also equal to $(a, -b)$.
\end{lem}
\begin{proof}
Equality of $(a, -b)$ and $(-a, b)$ follows from the definition of equality
of rationals. $x$ is negative because $-(-a, b) = (a, b)$ is positive. The
equality follows from (a) definition \ref{c3s2d4} of negativen numbers and
(b) $-(-a) = a$ for any integer, being a consequence of lemma 
\ref{c3s1l10}.
\end{proof}

\begin{lem}\label{c3s2l18}
If $x$ is a rational number then exactly one of the following it true (a)
$x = 0$, (b) $x$ is a positive rational number or (c) $x$ is a negative
rational number.
\end{lem}
\begin{proof}
Let $x = (a, b)$. The law of trichotomy for integers assures us that $a$
and $b$ can be one of (a) zero, (b) positive or (c) negative. The 
definition of rational assures us that $b \ne 0$. If $a = 0$ then lemma
\ref{c3s2l6} tells us that $x = 0$. That leaves us with four possibilities:
if both $a$ and $b$ are both positive or both negative then by definition
or by lemma \ref{c3s2l16} $x$ is positive. If one of $s$ or $b$ is positive
and the other is negative then by lemma \ref{c3s2l7} $x$ is negative. Thus
we have shown that at least one of the three possibilities is true.

We next show that at most one of them is true. If $x = (a, b)$ is positive
then $a, b > 0$ so that $x$ is neither zero, nor negative. If $x = (a, b)$
is negative, then one of $a$ or $b$ is negative. Therefore, $x$ is not
positive. Neither is it zero. If $x = 0$ then $a = 0$. That makes $x$ 
neither positive nor negative.
\end{proof}

\begin{defn}\label{c3s2d8}
Let $x, y \in \soq$. Then $x > y$ iff $x - y$ is a positive rational number,
$x < y$ is $x - y$ is negative rational number. We write $x \ge y$ if 
$x > y$ or $x = y$ and $x \le y$ if $x < y$ or $x = y$.
\end{defn}

\begin{lem}\label{c3s2l19}
If $x, y \in \soq$ then exactly one of the three statements $x = y, x < y$
or $x > y$ is true.
\end{lem}
\begin{proof}
Let $z = x - y$. Then by lemma \ref{c3s2l18} exactly one of the following
statements is true $z = 0, z < 0$ or $z > 0$. By definition \ref{c3s2d8}
these correspond to $x = y, x < y$ and $x > y$ respectively.
\end{proof}

\begin{lem}\label{c3s2l20}
For $x, y \in \soq$, $x < y$ iff $y > x$.
\end{lem}
\begin{proof}
If $y > x$ then $y - x > 0$ so that $(-1)(y - x) = -y -(-x) = x - y < 0$ so
that $x < y$. 

Interchanging $x$ and $y$ in the above argument proves the `only if' part.
\end{proof}

\begin{lem}\label{c3s2l21}
For $x, y, z \in \soq$, $x < y, y < z \Rightarrow x < z$.
\end{lem}
\begin{proof}
Let $z_1 = y - x, z_2 = z - y$. Then, by definition, $z_1 > 0, z_2 > 0$.
Therefore, both numbers are of the form $z_1 = (a_1, b_1), z_2 = (a_2, 
b_2)$, where $a_1, a_2, b_1, b_2 \in \soi$ are all positive integers. 
Therefore, $z_1 + z_2 = y - x + z - y = z - x$, following commutativity
and associativity of addition of rationals. Further, $z_1 + z_2 = (a_1b_2
+ a_2b_1, b_1b_2)$. Since $a_1, a_2, b_1, b_2 > 0$ so are $a_1b_2 + 
a_2b_1$ and $b_1b_2$. Therefore, $z_1 + z_2 > 0$ or that $z > x$.
\end{proof}

\begin{rem}
By lemma \ref{c3s1l18} $a_1 > 0, b_2 > 0 \Rightarrow a_1b_2 > 0$. 
Similarly, $a_2b_1 > 0$ and $b_1b_2 > 0$. By lemma \ref{c3s1l17}, $a_1b_2 
> 0 \Rightarrow a_1b_2 + a_2b_1 > a_2b_1$. By lemma \ref{c3s1l21}, the 
previous inequality along with $a_2b_1 > 0$ implies $a_1b_2 + a_2b_1 > 0$.
\end{rem}

\begin{lem}\label{c3s2l22}
If $x, y, z \in \soq$ and $x < y$ then $x + z < y + z$.
\end{lem}
\begin{proof}
$y + z - (x + z) = y + z - x - z = y - x > 0$ because $y > x$. Therefore,
$y +z > x + z$.
\end{proof}

\begin{lem}\label{c3s2l23}
If $x, y \in \soq, x, y > 0$ then $xy > 0$.
\end{lem}
\begin{proof}
Let $x = (a, b), y = (c, d)$. $x > 0, y > 0 \Rightarrow a, b, c, d > 0$
so that $ac > 0, bd > 0$ by lemma \ref{c3s1l18}. Therefore, $(ac, bd) =
xy > 0$.
\end{proof}

\begin{lem}\label{c3s2l24}
If $x, y, z \in \soq$, $z > 0$ then $x < y \Rightarrow xz < yz$.
\end{lem}
\begin{proof}
$x < y \Rightarrow y - x > 0$. Therefore $z(y - x) > 0 \Rightarrow zy - zx
> 0 \Rightarrow yz - xz > 0 \Rightarrow xz < yz$.
\end{proof}

\subsection{Exercises}
\begin{enumerate}
\item[1:] Show that if $x, y, z \in \soq$ and $z < 0$ then $x < y 
\Rightarrow xz > xy$.
\item[Solution:] $x < y \Rightarrow y - x > 0$. Likewise, $z < 0 
\Rightarrow -z > 0$. Therefore, using lemma \ref{c3s2l23}, we get $(-z)
(y - x) > 0 \Rightarrow -zy + zx > 0 \Rightarrow zx > zy$.
\end{enumerate}

\section{Absolute value and exponentiation}\label{c3s3}
\begin{defn}\label{c3s3d1}
If $x \in \soq$ then its absolute value $|x|$ is defined as
\[
|x| := \begin{cases}
x & \text{ if } x \ge 0 \\
-x & \text{ if } x < 0.
\end{cases}
\]
\end{defn}

\begin{defn}\label{c3s3d2}
If $x, y \in \soq$ then the distance between $x$ and $y$ is defined as 
\[
d(x, y) := |x - y|.
\]
\end{defn}

We will now prove a few basic properties of absolute value and distance.
\begin{lem}\label{c3s3l1}
For all $x \in \soq$, $|x| \ge 0$. Further, $|x| = 0$ iff $x = 0$.
\end{lem}
\begin{proof}
If $x = (-a, b)$ is a negative rational, that is, if $a, b \in \soi$ are 
positive integers then $-x = (-1, 1)(-a, b) = (a, b) > 0$. Therefore, by
definition \ref{c3s3d1}, $|x| \ge 0$, whatever be the value of $x$.

If $x = 0$ then by definition $|x| = 0$. Now let $|x| = 0$. Since $|x| = 
x$ if $x \ge 0$ this gives us $x = 0$.
\end{proof}

\begin{lem}\label{c3s3l2}
$\pm x \le |x|$ for all $x \in \soq$.
\end{lem}
\begin{proof}
If $x \ge 0$, $|x| = x$ therefore, $|x| \ge x$. In this case, we also have
$-x \le 0 \le x \le |x|$. Therefore $\pm x \le |x|$.

If $x < 0$, $|x| = -x$ so that $|x| \ge -x > 0$. This, coupled with the 
fact that $0 > x$ gives us $|x| \ge \pm x$.
\end{proof}

\begin{lem}\label{c3s3l3}
If $a, b, c, d \in \soq$ be such that $a < b$ and $c < d$ then we have
$a + c < b + d$.
\end{lem}
\begin{proof}
In this proof we used the fact that $a < b, c < d \Rightarrow a + c < b + 
d$. We prove it as follows: $a < b \Rightarrow a + c < b + c$. $c < d
\Rightarrow b + c < b + d$. Therefore, we have $a + c < b + c < b + d$.
We used lemma \ref{c3s3l2} over here.
\end{proof}

\begin{lem}\label{c3s3l4}
If $x, y \in \soq$, $|x + y| \le |x| + |y|$.
\end{lem}
\begin{proof}
From lemma \ref{c3s3l2} we have $\pm x \le |x|$ for all $x \in \soq$. From
lemma \ref{c3s3l3} we also have $x \le |x|$ and $y \le |y|$ implies $x + y 
\le |x| + |y|$. Likewise, $-(x + y) = -x - y \le |x| + |y|$. Since $|x + y|
= x + y$ or $-(x+y)$, we have $|x + y| \le |x| + |y|$.
\end{proof}

\begin{lem}\label{c3s3l5}
If $x, y \in \soq$ then $-y \le x \le y$ iff $y \ge |x|$. In particular,
we have $-|x| \le x \le |x|$.
\end{lem}
\begin{proof}
Let $y \ge |x|$. By lemma \ref{c3s3l2} $|x| \ge x$. Therefore 
\begin{equation}\label{c3s3e1}
y \ge x.
\end{equation}

$y \ge |x|$ also means $-y \le -|x|$. By lemma \ref{c3s3l2} $|x| \ge \pm x$.
Therefore, $-|x| \le \mp x$. Therefore 
\begin{equation}\label{c3s3e2}
-y \le -|x| \le x \Rightarrow -y \le x.
\end{equation}
The lemma follows from \eqref{c3s3e1} and \eqref{c3s3e2}.

Since $|x| \ge x$, we can put $y = |x|$ and get $-|x| \le x \le |x|$.
\end{proof}

\begin{lem}\label{c3s3l6}
If $x, y \in \soq$ then $|xy| = |x||y|$.
\end{lem}
\begin{proof}
If $x \ge 0, y \ge 0$ then $xy \ge 0$. We also have $x = |x|, y = |y|, xy
= |xy|$ that is $|x||y| = |xy|$.

If $x \le 0, y \le 0$, then $xy \ge 0$. We also have $-x = |x|, -y = |y|,
xy = |xy|$ that is $(-|x|)(-|y|) = |x||y| = |xy|$.

If $x \le 0, y \ge 0$ then $xy \le 0$. We also have $-x = |x|, y = |y|,
-xy = |xy|$. Therefore, we have $|x||y| = |xy|$. The same is true if $x \ge
0, y \le 0$.
\end{proof}

\begin{lem}\label{c3s3l7}
For all $x, y \in \soq$, $d(x, y) \ge 0$. Further, $d(x, y) = 0$ iff 
$x = y$.
\end{lem}
\begin{proof}
$d(x, y) = |x - y| \ge 0$ be definition of the absolute value. By lemma
\ref{c3s3l1} $|x - y| = 0$ iff $x - y = 0$, therefore, $d(x, y) = 0$ iff
$x = y$.
\end{proof}

\begin{lem}\label{c3s3l8}
For all $x \in \soq$, $|x| = |-x|$.
\end{lem}
\begin{proof}
By definition $|-1| = 1 = |1|$. Therefore, $|-1||x| = |-x| = |1||x| = |x|$.
\end{proof}

\begin{lem}\label{c3s3l9}
For all $x, y \in \soq$, $d(x, y) = d(y, x)$.
\end{lem}
\begin{proof}
Follows immediately from the fact that $|y - x| = |-(y - x)| = |x - y|$.
\end{proof}

\begin{lem}\label{c3s3l10}
For all $x, y, z \in \soq$, $d(x, z) \le d(x, y) + d(y, z)$.
\end{lem}
\begin{proof}
Follows immediately from $|z - x| = |z - y + y - x| \le |z - y| + |y - x|$. 
\end{proof}

