\chapter{Integers and Rationals}\label{c3}
Natural numbers admit addition and multiplication but not their inverse
operations. Integers permit us to define subtraction, which is inverse of 
addition and rationals allow us to introduce division as an inverse of
multiplication. We start with constructing the integers.

\section{The Integers}\label{c3s1}
\begin{defn}\label{c3s1d1}
We define integers as an ordered pair of natural numbers and denote the 
set of all integers as $\soi$. Two integers $(a, b)$ and $(c, d)$ are 
equal iff $a + d = c + b$.
\end{defn}

\begin{lem}\label{c3s1l1}
The equality of integers is an equivalance relation.
\end{lem}
\begin{proof}
Let $a. b, c, d, e, f \in \son$. Clearly $(a, b) = (a, b)$ because $a + b
= a + b$, the last equality follows from the properties of equality of 
natural numbers. Let $(a, b) = (c, d)$, that is $a + d = c + b$. But the
equality of natural numbers is symmetric. Therefore, $c + b = a + d$ so
that $(c, d) = (a, b)$.

In order to prove transitivity, let $(a, b) = (c, d)$ and $(c, d) = 
(e, f)$. Therefore, $a + d = c + b$ and $c + f = d + e$. Therefore $a + d
+ c + f = c + b + d + e$. Using commutativity and associativity of addition
of natural numbers, $a + f + c + d = e + b + c + d$. Using proposition 
\ref{c1s2p7}, we get $a + f = e + b$ or that $(a, b) = (e, f)$.
\end{proof}

\begin{defn}\label{c3s1d2}
The sum of two integers $(a, b)$ and $(c, d)$ is defined as $(a + c, 
b + d)$.
\end{defn}
\begin{rem}
This definition is very similar to that of addition of two dimensional 
vectors.
\end{rem}

\begin{lem}\label{c3s2l2}
Addition of integers is a well-defined operation.
\end{lem}
\begin{proof}
Let $(a, b), (c, d), (e, f)$ be integers such that $(a, b) = (c, d)$. Then
we will show that $(a, b) + (e, f) = (c, d) + (e, f)$. The lhs of this 
claim is $(a + e, b + f)$. Now, $(a, b) = (c, d) \Rightarrow a + d = b + 
c$.
\end{proof}

\begin{defn}\label{c3s1d3}
The product of two integers $(a, b)$ and $(c, d)$ is defined as $(ac + bd,
ad + bc)$.
\end{defn}

