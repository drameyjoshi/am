\chapter{Limits of sequences}\label{c5}
\section{Convergence and limit laws}\label{c5s1}
\begin{defn}\label{c5s1d1}
Given two real numbers $x$ and $y$, we define the distance between them as
$d(x, y) = |x - y|$.
\end{defn}
\begin{defn}\label{c5s1d2}
Let $\epsilon > 0$ be a real number. We say that two real numbers $x$ and 
$y$ are $\epsilon$-close iff $d(x, y) \le \epsilon$.
\end{defn}
\begin{defn}\label{c5s1d3}
A sequence $\seq{a_n}$ of real number is a Cauchy sequence if for every
real $\epsilon > 0$ there exists a natural number $N$ such that for all
$n, m \ge N$, $d(a_n, a_m) \le \epsilon$.
\end{defn}
\begin{defn}\label{c5s1d4}
A sequence $\seq{a_n}$ of real numbers converges to $L \in \sor$ iff for
every $\epsilon > 0$ we can find $N \in \son$ such that for all $n \ge N$,
$d(a_n, L) \le \epsilon$. A sequence that converges to a real number is 
called convergent. Otherwise it is called divergent.
\end{defn}

\begin{prop}\label{c5s1p1}
If a sequence is convergent then it converges to a unique limit.
\end{prop}
\begin{proof}
Let, if possible, $\seq{a_n}$ converge to $L_1$ and $L_2$. Hence, for any 
$\epsilon > 0$, we can find $N_1, N_2 \in \son$ such that for all $n \ge 
N_1$, $d(L_1, a_n) < \epsilon/2$ and for all $n \ge N_2$, $d(L_2, a_n) <
\epsilon/2$. If $N = \max(N_1, N_2)$, then both inequalities are true for
all $n \ge N$. Now consider $|L_1 - L_2| = |L_1 - a_n + a_n - L_2| \le
|L_1 - a_n| + |L_2 - a_n| \le \epsilon$. Therefore, $L_1 = L_2$.
\end{proof}
\begin{rem}
If $x, y \in \sor$ are such that for any $\epsilon > 0$, $|x - y| < \epsilon$
then $x = y$.
\end{rem}

When a sequence $\seq{a_n}$ converges to a limit $L$ we write it as
\[
\lim_{n \rightarrow \infty} a_n = L
\]
or simply $\lim a_n = L$.

\begin{prop}\label{c5s1p2}
\[
\lim_{n \rightarrow \infty} \frac{1}{n} = 0.
\]
\end{prop}
\begin{proof}
Let $\epsilon > 0$. Then if $N > \epsilon^{-1}$, $n > N$, we have 
\[
\frac{1}{n} < \frac{1}{N} < \epsilon \Rightarrow \left|\frac{1}{n} - 0\right|
< \epsilon.
\]
\end{proof}

\begin{prop}\label{c5s1p3}
A convergent sequence is a Cauchy sequence.
\end{prop}
\begin{proof}
Let 
\[
\lim_{n \rightarrow \infty} a_n = L.
\]
Then for any $\epsilon > 0$ we can find $N \in \son$ such that for all
$n > N$, $|a_n - L| < \epsilon/2$. If $m, n > N$, then $|a_n - a_m| <
|a_n - L + L - a_m| \le |a_n - L| + |a_m - L| > \epsilon$.
\end{proof}

\begin{prop}\label{c5s1p4}
If a sequence of rationals is a Cauchy sequence in the sense of chapter 
\ref{c4} then it also a Cauchy sequence in the sense of definition 
\ref{c5s1d3}. The converse is also true.
\end{prop}
\begin{proof}
If a sequence is Cauchy in the sense of chapter \ref{c4} then for any real
$\epsilon > 0$ we find a rational $\epsilon_r > 0$ such that $\epsilon_r <
\epsilon$. For this choice of $\epsilon_r$ we can find $N \in \son$ such that
for all $n, m > N$, $d(a_n, a_m) < \epsilon_r < \epsilon$.

To prove the converse, interchange the role of $\epsilon$ and $\epsilon_r$.
\end{proof}

\begin{prop}\label{c5s1p5}
Let $\seq{a_n}$ be a Cauchy sequence of rational and $L = 
\overline{\seq{a_n}}$ then 
\[
\lim_{n \rightarrow \infty} a_n = L.
\]
\end{prop}
\begin{proof}
Let $L$ be represented by a sequence $\seq{b_n}$ in the equivalance class
$\overline{\seq{a_n}}$. Therefore, $\overline{\seq{a_n}} = 
\overline{\seq{b_n}}$. Let $\epsilon > 0$ be a given real number. Let 
$\epsilon > \epsilon_r > 0$ be a positive rational number. Then
\[
|L - a_N| = |\overline{\seq{b_n}} - a_N|
\]
For a fixed $N$, $a_N = \overline{\seq{a_N}}$ where all terms of the sequence
are $a_N$. Therefore,
\[
|L - a_N| = |\overline{\seq{b_n}} - \seq{a_N}| = |\overline{b_n - a_N}| <
\epsilon
\]
for $n, N > M$, where the number $M$ depends on $\epsilon_r$. Since 
$\seq{a_N}$ is also a member of $\overline{\seq{b_n}}$.
\end{proof}

\begin{defn}\label{c5s1d5}
A sequence $\seq{a_n}$ is said to be bounded if there is a real number $M>0$
such that $|a_n| < M$ for all $n$. The number $M$ is called a bound of the
sequence.
\end{defn}

\begin{prop}\label{c5s1p6}
A convergent sequence is bounded.
\end{prop}
\begin{proof}
Let $\seq{a_n}$ converge to $L$. For a fixed $\epsilon > 0$, we can find
$N \in \son$ such that, for all $n > N$, $|a_n - L| < \epsilon$. Therefore,
for all $n > N$, $L - \epsilon < a_n < L + \epsilon$. Let $M_1$ be the 
maximum number in the finite set $|a_1|, \ldots, |a_n|$. Then $M = M_1 + 
L + \epsilon$ is such that $|a_n| < M$ for all $n$.
\end{proof}

We will now prove a few elementary properties of limits of Cauchy sequences.
In each of these we assume that the sequences $\seq{a_n}$ and $\seq{b_n}$
converge to $x$ and $y$ respectively.
\begin{lem}\label{c5s1l1}
$\seq{a_n + b_n}$ converges to $x + y$.
\end{lem}
\begin{proof}
Fix an $\epsilon > 0$. Then we can find $N \in \son$ such that $|a_n - x| <
\epsilon/2$ and $|b_n - y| < \epsilon/2$ for all $n \ge N$. Therefore,
$|a_n + b_n - (x + y)| \le |a_n - x| + |b_n - y| < \epsilon$.
\end{proof}

\begin{lem}\label{c5s1l2}
$\seq{a_n - b_n}$ converges to $x - y$.
\end{lem}
\begin{proof}
$|a_n - b_n - (x - y)| = |a_n - x - (b_n - y)| \le |a_n - x| + |b_n - y| 
< \epsilon$.
\end{proof}

\begin{lem}\label{c5s1l3}
If $c \in \sor$ then $\seq{ca_n}$ converges to $cx$.
\end{lem}
\begin{proof}
Fix an $\epsilon > 0$. If $c = 0$, the sequence is a constant sequence of all
zeros and it converges to $0$. Let us assume that $c \ne 0$. Then there 
exists $N \in \son$ such that for all $n \ge N$, $|a_n - x| < \epsilon/|c|$. 
Then $|ca_n - cx| = |c||a_n - x| < \epsilon$.
\end{proof}

\begin{lem}\label{c5s1l4}
$\seq{a_nb_n}$ converges to $xy$.
\end{lem}
\begin{proof}
Let us first consider the case $x \ne 0$.
Since $\seq{b_n}$ is convergent, by proposition \ref{c5s1p6}, it is bounded.
Let $M$ be its upper bound. Fix an $\epsilon > 0$. We can find an $N \in 
\son$ such that for all $n \ge N$,
\[
|a_n - x| \le \frac{\epsilon}{M} \;\land\; |b_n - y| \le \frac{\epsilon}{|x|}
\]
Therefore,
\[
|a_nb_n - xy| = |a_nb_n - xb_n + xb_n - xy| \le |a_n - x||b_n| + |x||b_n - y|
< \epsilon.
\]

If $x = 0$ then we will show that $\seq{a_nb_n}$ converges to $0$. Fix an
$\epsilon > 0$. Then $|a_nb_n - 0| = |a_n||b_n| < |a_n|M$. If we choose an
$N \in \son$ such that $|a_n - 0| < \epsilon/M$ for all $n \ge N$ then we
also have $|a_nb_n - 0| < \epsilon$ for all $n \ge N$.
\end{proof}

\begin{lem}\label{c5s1l5}
If $y \in 0$ and $b_n \ne 0$ for all $n \in \son$ then $\seq{b_n^{-1}}$
converges to $y^{-1}$.
\end{lem}
\begin{proof}
Since $b_n \ne 0$ for all $m$, the sequence $\seq{|b_n|}$ is bounded away
from zero and there exists $0 < m$ such that $m < |b_n|$ for all $n \in
\son$. Fix an $\epsilon > 0$. Then we can find an $N \in \son$ such that
for all $n \ge N$, 
\[
|b_n - y| < \epsilon|y|m.
\]
Now consider
\[
\left|\frac{1}{b_n} - \frac{1}{y}\right| = \frac{|b_n - y|}{|y||b_n|}.
\]
Since $m < |b_n|$, $|b_n|^{-1} < m^{-1}$. Therefore,
\[
\left|\frac{1}{b_n} - \frac{1}{y}\right| < \frac{|b_n - y|}{|y|m} <
\epsilon.
\]
\end{proof}

\begin{lem}\label{c5s1l6}
If $y \ne 0$ and $b_n \ne 0$ for all $n$ then $\seq{a_n/b_n}$ converges
to $x/y$.
\end{lem}
\begin{proof}
Follows immediately from lemmas \ref{c5s1l5} and \ref{c5s1l4}.
\end{proof}

\begin{lem}\label{c5s1l7}
$\seq{\max(a_n, b_n)}$ converges to $\max(x, y)$.
\end{lem}
\begin{proof}
Without loss of generality assume that $\max(x, y) = x$. Let $d = |x - y|$. 
Fix an $\epsilon > 0$ and let $N \in \son$ be such that for all $n \ge N$,
$|a_n - x| < \epsilon/2, |b_n - y| < \epsilon/2$. Thus $a_n - \epsilon/2 < 
x < a_n + \epsilon/2$ and $b_n - \epsilon/2 < y < b_n + \epsilon/2$. From
these two equations we get $x - y - \epsilon < a_n - b_n < x - y + \epsilon$.
If we choose $\epsilon > d$ then we have $a_n - b_n > 0$ for all $n \in 
\son$ or that $\max(a_n, b_n) = a_n$ for all $n \ge N$. Therefore, for 
all $n \ge N$, $|\max(a_n, b_n) - x| = |a_n - x| < \epsilon/2 < \epsilon$.
\end{proof}

\begin{lem}\label{c5s1l8}
$\seq{\min(a_n, b_n)}$ converges to $\min(x, y)$.
\end{lem}
\begin{proof}
Without loss of generality assume that $\min(x, y) = x$. Let $d = |y - x|$. 
Fix an $\epsilon > 0$ and let $N \in \son$ be such that for all $n \ge N$,
$|a_n - x| < \epsilon/2, |b_n - y| < \epsilon/2$. Thus $a_n - \epsilon/2 < 
x < a_n + \epsilon/2$ and $b_n - \epsilon/2 < y < b_n + \epsilon/2$. From
these two equations we get $x - y - \epsilon < a_n - b_n < x - y + \epsilon$.
If we choose $\epsilon > d$ then we have $a_n - b_n < 0$ for all $n \in 
\son$ or that $\min(a_n, b_n) = a_n$ for all $n \ge N$. Therefore, for 
all $n \ge N$, $|\min(a_n, b_n) - x| = |a_n - x| < \epsilon/2 < \epsilon$.
\end{proof}

\subsection{Exercices}
\begin{enumerate}
\item[1:] Explain why lemma \ref{c5s1l6} fails when $\seq{b_n}$ converges 
to $0$.
\item[Solution:] We will show that when
\[
\lim_{n \rightarrow \infty}b_n = 0
\]
then $\seq{b_n^{-1}}$ does not converge. Fix an $\epsilon > 0$, then we can
find $n \in \son$ such that for all $n \ge N$, $|b_n| < \epsilon$. Therefore,
\[
\frac{1}{|b_n|} = \left|\frac{1}{b_n}\right| > \frac{1}{\epsilon}
\]
Thus, as $n \rightarrow \infty$, the terms of $\seq{b_n^{-1}}$ rise without
limit. It is not a bounded sequence and therefore not convergent.

\begin{rem}
For any $M > 0$, we can find $N \in \son$ such that $|b_n| < M^{-1}$ or that
$|b_n|^{-1} > M$.
\end{rem}
\end{enumerate}

\section{The extended real number system}\label{c5s2}
\begin{defn}\label{c5s2d1}
The extended real number system $\soe$ is the set $\sor$ with two additional
symbols $+\infty$ and $-\infty$. They are distinct from each other and from
every real numbers. $x \in \soe$ is called finite iff it is a real number
and infinite iff it is $\pm\infty$.
\end{defn}

\begin{defn}\label{c5s2d2}
The operation of negation $x \mapsto -x$ on $\sor$ is extended to $\soe$ by
letting $-(+\infty) := -\infty$ and $-(-\infty) = \infty$.
\end{defn}

\begin{defn}\label{c5s2d3}
Let $x, y \in \soe$. Then $x \le y$ iff one of the following three statements
is true
\begin{enumerate}
\item $x, y \in \sor$ and $x \le y$,
\item $y = +\infty$.
\item $x = -\infty$.
\end{enumerate}
\end{defn}

\begin{lem}\label{c5s2l1}
Let $x, y, z \in \soe$. Then the following statements are true.
\begin{enumerate}
\item $x \le x$.
\item Exactly one of the statements $x < y, x = y$ of $x > y$ is true.
\item $x \le y, y \le z \Rightarrow x \le z$.
\item $x \le y \Rightarrow -x \le -y$.
\end{enumerate}
\end{lem}
\begin{proof}
If $x, y, z \in \sor$ then we have already proved these statements. If $x = 
\infty$ then $x \le x$ because of statement 2 in definition \ref{c5s2d3}. If
$x = -\infty$ then $x \le x$ because of statement 3 in definition 
\ref{c5s2d3}

If $x = \infty$ then $x > y$ is true for all $y \in \soe - \{\infty\}$. 
Likewise, if $x = -\infty$ then $x < y$ is true for all $y \in \soe - 
\{-\infty\}$. $x = y$ is true of $x = \infty, y = \infty$ or $x = -\infty,
y = -\infty$ or $x = y \in \sor$.

$x \le y$ if $x \le y \;\land\; x, y \in \sor$ or $y = \infty$ or $x = 
-\infty$. Likewise, $y \le z$ if $y \le z \;\land\; y, z \in \sor$ or $z =
\infty$ of $y = -\infty$. If $x = -\infty$ then $x \le z$ for all $z \in 
\soe$. If $z = \infty$, $x \le z$ for all $x \in \infty$. If $y = \infty$
then $z = \infty$ and hence $x \le z$ is true. If $y = -\infty$ then $x =
-\infty$ and hence $x \le z$ is true. 

Let $x \le y$. Then either $x, y \in \sor \;\land\; x \le y$ of $y = \infty$
or $x = -\infty$. In the first case, we have already proved that $x \le y
\Rightarrow -x \ge -y$. If $y = \infty$ then by definition \ref{c5s2d2},
$-y = -\infty$ and hence $-y \le -x$ for all $x \in \soe$. If $x = -\infty$
then by definition \ref{c5s2d2}, $-x = \infty$ and hence $-x \ge -y$ for all
$y \in \soe$.
\end{proof}

\begin{defn}\label{c5s2d4}
Let $E \subset \soe$. Then we define the supremum of $E$ using the following
rule.
\begin{enumerate}
\item If $E \subset \sor$ then its supremum is as defined as in definition 
\ref{c4s5d2}.
\item If $\infty \in E$ then $\sup(E) := \infty$.
\item If $\infty \notin E$ but $-\infty \in E$ then $\sup(E) := 
\sup(E - \{-\infty\})$.
\end{enumerate}
\end{defn}

\begin{defn}\label{c5s2d5}
If $E \subset \soe$ then $\inf(E) = -\sup(-E)$, where $-E = \{-x \;|\; x
\in E\}$.
\end{defn}

\begin{thm}\label{c5s2t1}
Let $E \subset \soe$ then the following statements are true.
\begin{enumerate}
\item $\forall\; x \in E$, $x \le \sup(E)$ and $x \ge \inf(E)$.
\item If $M \in \soe$ is an upper bound for $E$ then $\sup(E) \le M$.
\item If $m \in \soe$ is an lower bound for $E$ then $\inf(E) \ge m$.
\end{enumerate}
\end{thm}
\begin{proof}
If $\sup(E) \in \sor$ then $\sup(E)$ is an upper bound of $E$ so that
$x \le \sup(E)$ is true. Further, it is a least upper bound so that if
$M \in \sor$ is any other upper bound then $\sup(E) \le M$. If $M = \infty$
then $\sup((E) \le M$ continues to be true.

If $\sup(E) = \infty$ then $x \le \sup(E)$ is true by definition 
\ref{c5s2d3}. Further, $M \in \soe$ can be an upper bound of $E$ iff $M =
\infty$. Therefore, $\sup(E) \le M$ by lemma \ref{c5s2l1}.

The statements about infimum can be proved analogously.
\end{proof}

It is important to note that the supremum or the infimum of a set need not
be a member of a set.

\section{Suprema and infima of sequences}\label{c5s3}
\begin{defn}\label{c5s3d1}
Let $\seq{a_n}$ be a sequence of real numbers. Then $\sup\seq{a_n} :=
\sup\{a_n \;|\; n \in \son\}$ and $\inf\seq{a_n} := \inf\{a_n \;|\; n 
\in \son\}$.
\end{defn}

\begin{prop}\label{c5s3p1}
Let $\seq{a_n}$ be a sequence of real numbers and let $x = \sup\seq{a_n}
\in \soe$. Then $a_n \le x$ for all $n \in \son$. If $M \in \soe$ is an
upper bound for $\{a_n \;|\; n \in \son\}$ then we have $x \le M$. Finally, 
for every extended
real number $y$ for which $y < x$ there exists at least one $n \in \son$
such that $y < a_n \le x$.
\end{prop}
\begin{proof}
Let $E = \{a_n \;|\; n \in \son\}$. Then $x = \sup\seq{a_n} \Rightarrow
x = \sup(E)$. By theorem \ref{c5s2t1} $a_n \le x$ for all $n \in \son$.
and if $M \in \soe$ is an upper bound of $E$ then $x \le M$.

Now let $y < x$. Then $y \in \sor$ and $y \notin \soe$. If there was no $a_n$
greater than $y$ then $y$ will be an upper bound of $E$ and we would have $
x \ge y$, a contradiction.
\end{proof}

The corresponding proposition for infima is
\begin{prop}\label{c5s3p2}
Let $\seq{a_n}$ be a sequence of real numbers and let $x = \inf\seq{a_n}
\in \soe$. Then $a_n \ge x$ for all $n \in \son$. If $m \in \soe$ is a
lower bound for $\{a_n \;|\; n \in \son\}$ then we have $x \ge m$. Finally, 
for every extended real number $y$ for which $y > x$ there exists at least 
one $n \in \son$ such that $y > a_n \ge x$.
\end{prop}
The proof is similar to that of proposition \ref{c5s3p1}.

In proposition \ref{c5s1p6} we proved that convergent sequences are bounded.
However, the converse is not true. The sequence $\{1, -1, 1, -1, \ldots\}$
is bounded but not convergent. However, if a sequence is bounded and monotone
then we show that it converges. 
\begin{prop}\label{c5s3p3}
Let $\seq{a_n}$ be a sequence of real numbers and let $M \in \sor$ be its
upper bound. If $a_{n+1} \ge a_n$ for all $n \ge N$ for some $N \in \son$
then $\seq{a_n}$ converges and
\[
\lim_{n \rightarrow \infty}a_n = \sup\seq{a_n} \le M.
\]
\end{prop}
\begin{proof}
Let us first assume that the sequence is positive after some index $m$. 
We will show that the sequence has to be Cauchy. For if were not then there
will be an $\epsilon_0 > 0$ such that $a_{n+1} - a_n \ge \epsilon_0$ for
all $n \ge N > m$ and some $N \in \son$. Therefore $a_{n + k} \ge a_n + 
k\epsilon_0$. By 
Archimedean property of real numbers (\ref{c4s4p3}) there will be a positive 
integer $K$ such that $M < K\epsilon$ or that $a_{n + K} \ge a_n + 
K\epsilon > a_n + M > M$, contradicting the requirement that $M$ is an upper
bound of $\seq{a_n}$.

Now let us consider the case where $a_n$ are all negative. For some $N \in
\son$ we can find $K \in \son$ such that $-a_N + M < K\epsilon_0$. Therefore
$a_{N + K} > a_N + K\epsilon>0 > -a_N + M > M$, again a contradiction. Note
that $-a_N > 0$.

Since $\seq{a_n}$ is Cauchy, it converges and as a limit, say $L$. We will
now show that $L = \sup\seq{a_n}$. We first show that $L$ is an upper bound
of $E = \{a_n \;|\; n \in \son$. If it was not then there will be some $N
\in \son$ for which $a_N > L$ and if $\epsilon = a_N - L$. Then, for $n > N$,
$|a_n - L| = a_n - L > a_N - L$, as the sequence is monotonic. Therefore,
$|a_n - L| > \epsilon$ so that $L$ is not a limit of the sequence, a
contradiction. 

We will next show that $L$ is the least upper bound of $E$. If it were
not then there would be a real number $K < L$ such that $a_n \le K < L$ for
all $n \in \son$. If $0 < \epsilon < L - K$ then $|a_n - L| = L - a_n =
L - K + K - a_n > \epsilon + (K - a_n) > \epsilon$. Therefore $L$ cannot be
a limit of the sequence.

Thus, we have shown that $L = \sup(E)$. Since $M$ is an upper bound of $E$
we surely have $L \le M$.
\end{proof}

We can analogously show that
\begin{prop}\label{c5s3p4}
Let $\seq{a_n}$ be a sequence of real numbers and let $m \in \sor$ be its
lower bound. If $a_{n+1} \le a_n$ for all $n \ge N$ for some $N \in \son$
then $\seq{a_n}$ converges and
\[
\lim_{n \rightarrow \infty}a_n = \inf\seq{a_n} \le m.
\]
\end{prop}

\begin{prop}\label{c5s3p5}
Let $0 < x < 1$ be a real number. Then we have
\[
\lim_{n \rightarrow \infty}x^n = 0,
\]
\end{prop}
\begin{proof}
We first note that $x^n > 0$ and $x^{n+1} < x^n$ for all $n \in \son$. 
Therefore the sequence $\seq{x^n}$ is monotonically decreasing and bounded
below. Therefore, by proposition \ref{c5s3p5} it converges to a limit $L$.
We will show that $L = 0$. The sequence $\seq{x^{n+1}}$ is also monotonically
decreasing and bounded below and its limit will be $Lx$. However both the
sequences are the same so that $Lx = L$, Since $0 < x < 1$, this is possible
only if $L = 0$.
\end{proof}

\subsection{Exercices}
\begin{enumerate}
\item[1:] Explain why proposition \ref{c5s3p5} fails when $x > 1$.
\item[Solution:] The sequence $\seq{x^n}$ is monotonically increasing and
unbounded. Therefore it is not Cauchy and it does not converge.
\end{enumerate}

\section{Limsup, liminf and limit points}\label{c5s4}
\begin{defn}\label{c5s4d1}
Let $\seq{a_n}$ be a sequence of real numbers, $x \in \sor$ and $\epsilon
> 0$ be a real number. We say that $x$ is $\epsilon$-adherent to $\seq{a_n}$
iff there exists an $n \ge 0$ such that $|a_n - x| < \epsilon$. We say
that $x$ is continually $\epsilon$-adherent to $\seq{a_n}$ iff there exists
an $N \in \son$ such that $|a_n - x| < \epsilon$ for all $n \ge N$. We say
that $x$ a limit point of $\seq{a_n}$ iff $x$ is continually $\epsilon$-
adherent to $\seq{a_n}$ for all $\epsilon > 0$.
\end{defn}

\begin{rem}
$x$ is $\epsilon$-adherent to $\seq{a_n}$ iff there is at least one point
$a_m$ for which $|a_m - x| < \epsilon$.
\end{rem}

\begin{rem}
$x$ is continually $\epsilon$-adherent to $\seq{a_n}$ means that beyond a
certain $a_N$, one can always find at least one $a_m$ such that $|a_m - x|
< \epsilon$. This \emph{does not} mean that for all $m \ge N$, $|a_m - x|
< \epsilon$. The latter situation is described by saying that $\seq{a_n}$
is \emph{eventuall} $\epsilon$-close to $x$.
\end{rem}

\begin{rem}
$x$ is a limit point iff for all $\epsilon > 0$, there exists $N \in \son$
such that for there is at least one $m \ge N$ for which $|a_m - x| < 
\epsilon$.
\end{rem}

Limits are limit points but not conversely.
\begin{prop}\label{c5s4p1}
Let $\seq{a_n}$ converge to a $c \in \sor$. Then $c$ is a unique limit point 
of $\seq{a_n}$.
\end{prop}
\begin{proof}
$\seq{a_n}$ converges to $c$ implies that for every $\epsilon > 0$ there 
exists an $N \in \son$ such that $|a_n - x| < \epsilon$ for all $n \ge N$.
Thus, $\seq{a_n}$ is continually $\epsilon$-adherent to $x$ and hence $x$
is a limit point of $\seq{a_n}$.

Now let $y \in \sor$ be another limit point of $\seq{a_n}$. Fix an $\epsilon
> 0$. Then we can find an $N \in \son$ such that $|a_n - y| < \epsilon/2$ for
at least one $n \ge N$. But we also have $|a_n - x| < \epsilon/2$ for all 
$n \ge N$. Therefore, $|x - y| = |x - a_n + a_n - y| \le |x - a_n| + 
|y - a_n| < \epsilon$. Therefore $x = y$.
\end{proof}

\begin{rem}
Recall that the claim `$|x - y| < \epsilon$ for all $\epsilon > 0$ implies
$x = y$' follows from the fact that both $x$ and $y$ belong to the same
equivalence class of Cauchy sequence of rationals.
\end{rem}

\begin{defn}\label{c5s4d2}
We define the limit superior of a sequence $\seq{a_n}$ as
\[
\limsup\seq{a_n} = \inf(\seq{\sup(\seq{a_n}_{n \ge m})}). 
\]
\end{defn}
Informally, we start with $\seq{a_n}$ and for each $m \ge 0$, we look at 
suprema beyond $a_m$. This is the sequence $\seq{b_m} = \seq{
\sup(\seq{a_n}_{n \ge m})}$. $\inf(b_m)$ is the limit supremum of 
$\seq{a_n}$.

Likewise,
\begin{defn}\label{c5s4d3}
We define the limit inferior of a sequence $\seq{a_n}$ as
\[
\liminf\seq{a_n} = \sup(\seq{\inf(\seq{a_n}_{n \ge m})}).
\]
\end{defn}
Informally, we start with $\seq{a_n}$ and form a sequence $\seq{b_m}$ where
$\seq{b_m} = \seq{\sup(\seq{a_n}_{n \ge m})}$. $\sup{b_m}$ is the limit 
inferior of $\seq{a_n}$.

A few observations about the sequences $\seq{b_m}$ in the above definition:
\begin{lem}\label{c5s4l1}
Let $\seq{a_n}$ be a sequence of real numbers and let $b_m = \sup\seq{
{a_n}}_{n \ge m}$. Then the sequence $\seq{b_m}$ is monotone decreasing.
\end{lem}
\begin{proof}
$b_{m + 1} = \sup(\seq{a_n}_{n \ge m + 1})$ so that $b_m = \max(a_m, 
b_{m+1})$ or that $b_m \ge b_{m+1}$.
\end{proof}

\begin{rem}
The sequence $\seq{b_m}$ where $b_m$ is defined in lemma \ref{c5s4l1} may
not be bounded below. If $\seq{a_n}$ is a monotone increasing sequence then
$\seq{b_m} = \{\infty, \infty, \ldots\}$ and $\limsup\seq{a_n} = \infty$.
\end{rem}

Likewise,
\begin{lem}\label{c5s4l2}
Let $\seq{a_n}$ be a sequence of real numbers and let $b_m = \inf\seq{
{a_n}}_{n \ge m}$. Then the sequence $\seq{b_m}$ is monotone increasing.
\end{lem}
\begin{proof}
$b_{m + 1} = \inf(\seq{a_n}_{n \ge m + 1})$ so that $b_m = \min(a_m, 
b_{m+1})$ or that $b_m \le b_{m+1}$.
\end{proof}

\begin{rem}
The sequence $\seq{b_m}$ where $b_m$ is defined in lemma \ref{c5s4l1} will
not be bounded above. If $\seq{a_n}$ is a stricly monotone increasing 
sequence, that is $a_n < a_{n + 1}$ for all $n \in \son$, then
$\seq{b_m} = \{a_1, a_2, \ldots\}$ and $\liminf\seq{a_n} = \infty$.
\end{rem}

Let $\seq{a_n}$ be a sequence of real numbers, $L = \limsup\seq{a_n}$ and i
$l = \liminf\seq{a_n}$. Then the following statements are true:
\begin{lem}\label{c5s4l3}
For every $x > L$, there exists an $N \in \son$ such that $a_n < x$ for
all $n \ge N$.
\end{lem}
\begin{proof}
Given that $L = \inf(\seq{\sup(\seq{a_n}_{n \ge m})})$. Let $b_m = 
\sup(\seq{a_n}_{n \ge m}) > x$. By lemma \ref{c5s4l1}, $\seq{b_m}$ is 
monotone decreasing. Since $L = \inf\seq{b_m}$, if $x > L$ then there are
infinitely many $b_m < x$ for all $m \ge N$ for some $N \in \son$. Thus,
$\sup(\seq{a_n}_{n \ge m}0 < x$ for all $m \ge N$. Therefore $a_m < x$ for
all $m \ge N$.
\end{proof}

Likewise,
\begin{lem}\label{c5s4l4}
For every $x < l$, there exists an $N \in \son$ such that $a_n > x$ for
all $n \ge N$.
\end{lem}
\begin{proof}
Given that $l = \sup(\seq{\inf(\seq{a_n}_{n \ge m})})$. Let $b_m = \inf(\seq
{a_n}_{n \ge m})$. By lemma \ref{c5s4l2}, $\seq{b_m}$ is a monotone
increasing sequence. Therefore, if $x < l$ then there are an infinitely many
$b_m$ beyond a certain $N \in \son$ such that $x < b_m$. That is $x < \inf(
\seq{a_n}_{n \ge m})$ or that $a_n > x$ for all $n \ge N$.
\end{proof}

\begin{lem}\label{c5s4l5}
For every $x < L$, for all $N \in \son$, there exists $n \ge N$ such that
$a_n > x$.
\end{lem}
\begin{proof}
Given that $L = \inf(\seq{\sup(\seq{a_n}_{n \ge m})})$ and $x < L$. If $b_m
= \sup(\seq{a_n}_{n \ge m})$ then by lemma \ref{c5s4l1} $\seq{b_m}$ is 
monotone decreasing. Since $x < L$, there exists $N \in \son$ such that $
b_m > x$ for all $m \ge N$. Thus, $\sup(\seq{a_n}_{n \ge m}) > x$ for all
$m \ge N$. Therefore, there exists at least one $a_n$, $n \ge M$ such that
$a_n > x$.
\end{proof}

Likewise,
\begin{lem}\label{c5s4l6}
For every $x > l$, for all $N \in \son$. there exists $n \ge N$ such that 
$a_n < x$.
\end{lem}
\begin{proof}
Given that $l = \sup(\seq{\inf(\seq{a_n}_{n \ge m)}})$. If $b_m = \inf(\seq{
a_n}_{n \ge n})$ then by lemma \ref{c5s4l2}, $\seq{b_m}$ is monotone 
increasing. If $x > l$ then for all $N \in \son$, $x > b_N$. Therefore,
there exists at least one $a_n$ for $n \ge N$ such that $x > a_n$.
\end{proof}

\begin{lem}\label{c5s4l7}
$\inf(\seq{a_n}) \le l \le L \le \sup(\seq{a_n})$.
\end{lem}
\begin{proof}
By definition $l = \sup(\seq{\inf(\seq{a_n}_{n \ge m})})$. In particular,
$l \ge \inf(\seq{a_n}_{n \ge 0})$. Therefore, $\inf(\seq{a_n}) \le l$. 
Similarly $L = \inf(\seq{\sup(\seq{a_n}_{n \ge m})}) \Rightarrow L \le 
\sup(\seq{a_n}_{n \ge 0})$ or that $L \le \sup(\seq{a_n})$. 

Let $B_m = \sup({a_n}_{n \ge m})$ and $b_m = \inf(\seq{a_n}_{n \ge m})$. 
Then $B_m \ge b_m$ for all $m \in \son$. Further, $\seq{B_m}$ is monotone
decreasing and $\seq{b_m}$ is monotone increasing. Therefore, 
$\inf{\seq{B_m}} \ge \sup(\seq{b_m})$
\end{proof}

\begin{lem}\label{c5s4l8}
If $c$ is any limit point of $\seq{a_n}$ then $l \le c \le L$.
\end{lem}
\begin{proof}
$c$ is a limit point of $\seq{a_n}$ implies that for any $\epsilon > 0$ 
and for all $m \in \son$, there is at least one $N \ge m$ such that $|a_N
- c| < \epsilon$. That is, $a_N - \epsilon < c < a_N + \epsilon$. 

If $B_m = \sup(\seq{a_n}_{n \ge m})$ then $B_m \ge a_n$ for all $n \ge m$. 
In particular, $B_m \ge a_N$. If $B_m > a_N$, choose $\epsilon < B_m - a_N$
so that we have $c < B_m$. Otherwise $c = B_m$. In all situations, $c 
\le B_m$. Therefore, $c \le \inf(\seq{B_m})$ or $c \le L$.

If $b_m = \inf(\seq{a_n}_{n \ge m})$ then $b_m \le a_n$ for all $n \ge m$.
In particular, $b_m \le a_N$. If $b_m < a_N$, choose $\epsilon < a_N - b_m$
so that $a_N - \epsilon > b_m$ so that we have $c > b_m$. Otherwise $c = 
b_m$. In all situations, $c \ge b_m$, so that $c \ge \sup(\seq{b_m}) = l$.
\end{proof}

\begin{lem}\label{c5s4l9}
If $L \in \sor$ then $L$ is a limit point of $\seq{a_n}$.
\end{lem}
\begin{proof}
Given that $L = \inf(\seq{\sup(\seq{a_n}_{n \ge m})}) \in \sor$. Therefore
$L \le B_m$ for all $m \ge 0$ where $B_m = \sup(\seq{a_n}_{n \ge m})$. Since
$\seq{B_m}$ is monotone decreasing and $L$ is its infimum, for any $\epsilon
> 0$, there is an $N \in \son$ such that $B_m - L \le \epsilon$ for all 
$m \ge N$. Since $L \in \sor$, $B_m \in \sor$. $B_m = \sup(\seq{a_n}_
{n \ge m}) \Rightarrow B_m \ge a_n \;\forall\; n \ge m$. Therefore, $a_n -
L \le B_m - L \le \epsilon \;\forall\; n \ge m$. Therefore, $L$ is a limit
point of $\seq{a_n}$.
\end{proof}

\begin{lem}\label{c5s4l10}
If $l \in \sor$ then $l$ is a limit point of $\seq{a_n}$.
\end{lem}
\begin{proof}
Given that $l = \sup(\seq{\inf(\seq{a_n}_{n \ge m})}) \in \sor$. Therefore,
$l \ge b_m$ for all $m \ge 0$, where $b_m = \inf(\seq{a_n}_{n \ge m})$. Since
$\seq{b_m}$ is monotone increasing and $l$ is its supremum, for any $\epsilon
> 0$, there is an $N \in \son$ such that $l - b_m \ge \epsilon$ for all
$m \ge N$. Since $l \in \sor$, $b_m \in \sor$. $b_m = \inf(\seq{a_n}_
{n \ge m} \Rightarrow b_m \le a_n \;\forall\; n \ge m$. Therefore, $l - b_m
\ge l - a_n$ or that $l - a_n \le \epsilon \;\forall\; n \ge m$, making $l$
a limit point of $\seq{a_n}$.
\end{proof}

\begin{lem}\label{c5s4l11}
Let $c \in \sor$. If $\lim a_n = c$ then we must have $l = L = c$. 
\end{lem}
\begin{proof}
Let $\lim a_n = c$. Therefore, for a given $\epsilon > 0$ we can find an
$N \in \son$ such that $|a_n - c| \le \epsilon/2 \;\forall\; n \ge N$. Let
$B_m = \sup(\seq{a_n}_{n \ge m})$. Then there exists at least one $r \ge m$
such that $B_m - a_r < \epsilon/2$. Therefore $|B_m - c| = |B_m - a_r + a_r
- c| \le B_m - a_r + |a_r - c| < \epsilon$.

Similarly, if $b_m = \inf(\seq{a_n}_{n \ge m})$ then there exists at least
one $r \ge m$ such that $a_r - b_m < \epsilon/2$. Therefore, $|c - b_m| =
|c - a_r| + a_r - b_m < \epsilon$.
\end{proof}

\begin{lem}\label{c5s4l12}
If $l = L = c$ then $\lim a_n = c$.
\end{lem}
\begin{proof}
Follows immediately from lemma \ref{c5s4l8}.
\end{proof}

The next set of assertions are about comparison of limit suprema and infima
of two sequences. Let $\seq{a_n}$ and $\seq{b_n}$ be sequences of reals
such that $a_n \le b_n$ for all $n \in \son$. Then, we have:
\begin{lem}\label{c5s4l13}
$\sup(\seq{a_n}) \le \sup(\seq{b_n})$.
\end{lem}
\begin{proof}
Let $A = \sup(\seq{a_n})$ and $B = \sup(\seq{b_n})$. Further, let $E_a = \{
a_n \;|\; n \in \son\}, E_b = \{b_n \;|\; n \in \son\}$. If $A > B$ then $B$
is not an upper bound of $E_a$. Therefore, there exists at least one $m \in
\son$ such that $B < a_m \le b_m$. But this makes $B$ not an upper bound of
$E_b$ as well.
\end{proof}

\begin{lem}\label{c5s4l14}
$\inf(\seq{a_n}) \le \inf(\seq{b_n})$.
\end{lem}
\begin{proof}
Let $a = \inf(\seq{a_n}), b = \inf(\seq{b_n}), E_a =\{a_n \;|\; n \in \son\},
E_b = \{b_n \;|\; n \in \son\}$. If $a > b$ then $a$ is not a lower bound
of $E_b$. Therefore, there exists $m \in \son$ such that $b_m < a$. Therefore
$a_m \le b_m < a$ so that $a$ is not a lower bound of $E_a$.
\end{proof}

\begin{rem}
The above to lemmas are true even if we consider sequences to start from
$n = m$ instead of $n = 0$.
\end{rem}

\begin{lem}\label{c5s4l15}
$\limsup\seq{a_n} \le \limsup\seq{b_n}$.
\end{lem}
\begin{proof}
$a_n \le b_n \Rightarrow \sup(\seq{a_n}_{n \ge m}) \le 
\sup(\seq{b_n}_{n \ge m})$. If $A_m = \sup(\seq{a_n}_{n \ge m}$ and $B_m =
\sup(\seq{b_n}_{n \ge m}$ then $A_m \le B_m \Rightarrow \inf(\seq{A_m}) \le
\inf(\seq{B_m}) \Rightarrow \limsup\seq{a_n} \le \limsup\seq{b_n}$.
\end{proof}

\begin{lem}\label{c5s4l16}
$\liminf\seq{a_n} \le \liminf\seq{b_n}$.
\end{lem}
\begin{proof}
Let $A_m = \inf(\seq{a_n}_{n \ge m}), B_m = \inf(\seq{b_n}_{n \ge m})$.
Then $A_m \le B_m$, by lemma \ref{c5s4l14}. Therefore, $\sup(\seq{A_m}) 
\le \sup(\seq{B_m})$ by lemma \ref{c5s4l13}.
\end{proof}

\begin{lem}\label{c5s4l17}
Let $\seq{a_n}, \seq{b_n}, \seq{c_n}$ be such that $a_n \le b_n \le c_n$
for all $n \in \son$. If $\lim\seq{a_n} = \lim\seq{c_n} = L$ then $\lim\seq
{c_n} = L$.
\end{lem}
\begin{proof}
By lemmas \ref{c5s4l16} and \ref{c5s4l17}, $\limsup\seq{a_n} \le \limsup\seq
{b_n} \le \limsup\seq{c_n}$ and $\liminf\seq{a_n} \le \liminf\seq{b_n} \le
\liminf\seq{c_n}$. By lemma \ref{c5s4l11}, $\limsup\seq{a_n} = 
\liminf\seq{a_n} = L$ and $\limsup\seq{b_n} = \liminf\seq{b_n} = L$. 
Therefore, we have $L \le \limsup\seq{b_n} \le L$ and $L \le \liminf\seq{b_n}
\le L$ so that $\limsup\seq{b_n} = \liminf\seq{b_n} = L$. By lemma 
\ref{c5s4l11} this means that $\lim\seq{b_n} = L$.
\end{proof}

\begin{lem}\label{c5s4l18}
If $x \in \sor$, $-|x| \le x \le |x|$ and $-x \le |x| \le x$.
\end{lem}
\begin{proof}
Recall that
\[
|x| = \begin{cases} x & \text{ if } x \ge 0 \\
-x & \text{ if } x < 0.
\end{cases}
\]
Therefore, $|x| \le x$ and $|x| \ge -x$ so that $-x \le |x| \le x$. We also
have $x \le |x|$ and $-x \ge |x| \Rightarrow -|x| \le x$. Therefore, $-|x|
\le x \le |x|$.
\end{proof}

An immediate corollary of \ref{c5s4l17} is
\begin{cor}\label{c5s4c1}
Let $\seq{a_n}$ be a sequence of real numbers. Then $\lim\seq{a_n}$ exists
and is equal to zero iff $\lim\seq{|a_n|}$ exists and is equal to zero.
\end{cor}
\begin{proof}
$-|a_n| \le a_n \le |a_n| \Rightarrow \lim\seq{-|a_n|} \le \lim\seq{a_n} \le
\lim\seq{|a_n|} \Rightarrow -\lim\seq{|a_n|} \le \lim\seq{a_n} \le \lim\seq
{|a_n|}$. If $\lim\seq{|a_n|} = 0$ then be lemma \ref{c5s4l17}, we have $
\lim\seq{a_n} = 0$.

Similarly, $-a_n \le |a_n| \le a_n$ implies that $\lim\seq{a_n} = 0 
\Rightarrow \lim\seq{|a_n|} = 0$.
\end{proof}

Proposition \ref{c5s1p3} showed that a convergent sequence is Cauchy. We
will now prove that a Cauchy sequence of reals converges to a real number.
This fact is called the completeness property of real numbers. Before
proving it we need an analogue of \ref{c4s1l2} for real numbers.
\begin{lem}\label{c5s4l19}
A Cauchy sequence of real numbers is bounded.
\end{lem}
\begin{proof}
If $\seq{a_n}$ is a Cauchy sequence then for any $\epsilon > 0$ there exists
an $N \in \son$ such that for all $n \ge N$, $|a_n - a_N| \le \epsilon$. Let
$a = \min\{a_1, \ldots, a_N\}$ and $A = \max\{a_1, \ldots, a_N\}$. Then
$-|a| \le a \le a_n \le A \le |A|$ for all $n \in \{1, \ldots, N\}$. For 
$n \ge N$, we have $a_N - \epsilon \le a_n \le a_N + \epsilon \le |a_N| + 
\epsilon$. Therefore, $-|a| - |a_N| - \epsilon \le a_n \le |A| + |a_N| + 
\epsilon$.
\end{proof}

\begin{prop}\label{c5s4p2}
A Cauchy sequence of real numbers converges to a real number.
\end{prop}
\begin{proof}
By lemma \ref{c5s4l19} a Cauchy sequence is bounded. That is, it has lower 
and upper bounds. If $l = \liminf\seq{a_n}$ and $L = \limsup\seq{a_n}$
then $\inf(\{a_n\}) \le l \le L \le \sup(\{a_n\})$ by lemma \ref{c5s4l7}.
Therefore, $l$ and $L$ are both finite.

If $\seq{a_n}$ is Cauchy then for a given $\epsilon > 0$ there exists $N \in
\son$ such that $|a_n - a_N| \le \epsilon/2$ for all $n \ge N$. Therefore,
$a_N - \epsilon/2 \le a_n \le a_N + \epsilon/2$ for all $n \ge N$. Thus, 
for the sequence $\seq{a_n}_{n \ge N}$, $a_N - \epsilon/2$ is a lower bound 
and $a_N + \epsilon/2$ is an upper bound. Therefore, $a_N - \epsilon/2 \le l
\le a_n \le L \le a_N + \epsilon/2$. Now,
\[
L \le a_N + \epsilon/2 \Rightarrow L - l \le a_N - l + \epsilon/2.
\]
Similarly,
\[
a_N - \epsilon/2 \le l \Rightarrow -l \le -a_N + \epsilon/2.
\]
Therefore, $L - l \le \epsilon$. Since $\epsilon$ is arbitrary, this leads
us to $l = L$, which is the limit of the sequence.
\end{proof}

\subsection{Exercises}
\begin{enumerate}
\item[1:] Give an example of two bounded sequences $\seq{a_n}$ and 
$\seq{b_n}$ such that $a_n < b_n$ for all $n \in \son$ yet $\sup\seq{a_n} 
\nless \sup \seq{b_n}$.
\item[Solution:] Let $a_n = 3 - 1/n$ and $b_n = 3 - 1/(2n)$ so that $a_n \le 
b_n$ and yet $\sup\seq{a_n} = \sup\seq{b_n} = 3$.

\item[2:] Is corollary \ref{c5s4l1} true if the sequences $\seq{a_n}$ or
$\seq{|a_n|}$ converged to a non-zero numbers?
\item[Solution:] Let $\lim\seq{|a_n|} = c \ne 0$. Therefore, $-|a_n| \le a_n
\le |a_n| \Rightarrow -c \le \lim\seq{a_n} \le c$. Thus, all we get is a
bound for $\lim\seq{a_n}$ not its exact value.

\item[3:] Construct a sequence which has exactly three limit points at
$-\infty, 0$ and $\infty$.
\item[Solution:] Let
\begin{eqnarray*}
a_{2m} &=& \frac{1}{2m + 1} \\
a_{2m+1} &=& (-1)^m m
\end{eqnarray*}
Thus, $a_0 = 1, a_1 = 1, a_2 = 1/3, a_3 = -1, a_4 = 1/5, a_5 = 2, a_6 = 1/7,
a_7 = -3, a_8 = 1/9, a_9 = 4, \ldots$. The subsequence $a_0, a_2, a_4, 
\ldots$ has a limit point $0$. The subsequence $a_3, a_7, a_{11}, a_{15}, 
\ldots$ has $-\infty$ as a limit point. The subsequence $a_1, a_5, a_9, 
\ldots$ has $\infty$ as a limit point. There are none other.

\item[4:] Let $\seq{a_n}$ be a sequence of real numbers and $\seq{b_n}$ be
another sequence of real numbers such that each $b_n$ is a limit point
of $\seq{a_n}$. If $c$ is a limit point of $\seq{b_n}$ then it is also a
limit point of $\seq{a_n}$.
\item[Solution:] Recall that a real number $x$ is a limit point of 
$\seq{a_n}$ if for any $\epsilon > 0$ there exists $N \in \son$ such that 
there is atleast one $m \ge N$ such that $|a_m - x| \le \epsilon$. Thus,
for all $b_m$, for any $\epsilon > 0$, there exists $N_m \in \son$ such that
$|b_m - a_n| < \epsilon$ for at least one $n \ge N_m$.

Let $c$ be a limit point of $\seq{b_m}$. Fix an $\epsilon > 0$. Then there
exists $N \in \son$ such that for atleast one $m \ge N$, $|b_m - c| \le 
\epsilon/2$.
Since $b_m$ is a limit point of $\seq{a_n}$, there exists $N_m \in \son$
such that for at least one $n \ge N_m$, $|a_n - b_m| \le \epsilon/2$. For
this element $|a_n - c| \le |a_n - b_m| + |b_m - c| \le \epsilon$. Thus, for
every $b_m, m \ge 0$ there exists at least one $a_n$ for which $|a_n - c|
\le \epsilon$. Thus $c$ is a limit point of $\seq{a_n}$.
\end{enumerate}

\section{Some standard limits}\label{c5s5}
In this section we will prove a few lemmas that state limits of commonly
found sequences.
\begin{lem}\label{c5s5l1}
If $a_n = c, \;\forall\; n \in \son$ then $\lim\seq{a_n} = c$.
\end{lem}
\begin{proof}
Let $\epsilon > 0$. Then for all $n \in \son$, $|a_n - c| = 0 < \epsilon$.
\end{proof}

\begin{lem}\label{c5s5l2}
Show that for all $k \ge 1$, $k \in \son$, $\lim 1/(n+1)^{1/k} = 0$.
\end{lem}
\begin{proof}
For a fixed $k$, $(n+1)^{1/k}$ is an increasing sequence by lemma 
\ref{c4s6l5}. Therefore $1/(n+1)^{1/k}$ is a decreasing sequence. It is 
bounded from below by zero because no element of the sequence is negative.
Therefore, by proposition \ref{c5s3p4}, $\seq{1/(n+1)^{1/k}}$ converges.
Let $L$ be its limit. Using lemma \ref{c5s1l4} $k$ times, we get $L^k$
as the limit of the sequence $\seq{1/n}$. By proposition \ref{c5s1p2} $L^k
 = 0$. Therefore $L = 0$.
\end{proof}

\begin{lem}\label{c5s5l3}
Let $x \in \sor$. Then $\lim\seq{x^n} = 0$ when $|x| < 1$.
\end{lem}
\begin{proof}
Since $0 < |x| < 1$, by proposition \ref{c5s3p5}, $\lim\seq{|x|^n} = 0$.
$-|x| \le x \le |x| \Rightarrow -|x|^n \le \le x^n \le |x|^n$. Therefore,
$-\lim\seq{|x|^n} \le \lim\seq{x^n} \le \lim\seq{|x|^n}$. The lemma follows
from \ref{c5s4l17}.
\end{proof}

\begin{lem}\label{c5s5l4}
If $x = 1$, $\lim\seq{x^n} = 1$.
\end{lem}
\begin{proof}
Follows from lemma \ref{c5s5l1}.
\end{proof}

\begin{lem}\label{c5s5l5}
If $x = -1$ or when $|x| > 1$ the sequence $\seq{x^n}$ does not converge.
\end{lem}
\begin{proof}
If $x = -1$, the sequence is $\{1, -1, 1, -1, \ldots\}$. Both $1$ and $-1$ 
are its limit points but neither is a limit. 

If $|x| > 1$ then $|x|^n$ is a monotonically increasing sequence without an
upper bound. Therefore, it does not converge.
\end{proof}

\begin{lem}\label{c5s5l6}
For any $x > 0$, $\lim x^{1/n} = 1$. The sequence starts at $n = 1$.
\end{lem}
\begin{proof}
By lemma \ref{c4s6l7} $x^{1/n}$ is a decreasing function of $n$ if $x > 1$,
be lemma \ref{c4s6l8} it is an increasing function of $n$ if $x < 1$ and it
is equal to $1$ for all $n$ if $x = 1$. The third case is the constant series
with limit $1$.

If $x > 1$, $\seq{x^{1/n}}$ is a monotonically decreasing sequence with $1$ 
as a lower bound. We will show that $1$ is also its infimum. For if it were
a number $c > 1$ then we would have $x^{1/n} \ge c$ for all $n \in \son, n 
\ge 1$. But this is false for $x = (1 + c)/2 > 1$. By proposition 
\ref{c5s3p4} the infimum is the limit.

If $x < 1$, $\seq{x^{1/n}}$ is a monotonically increasing sequence with $1$
as an upper bound. It is also its supremum. For if it were a number $c < 1$
then we would have $x^{1/n} \le c$ for all $n$, that is $x \le c^n$ for all
$n$. But this is false for $x = (c + 1)/2 < 1$. By proposition \ref{c5s3p3}
the supremum is the limit.
\end{proof}

\subsection{Exercices}
\begin{enumerate}
\item[1:] Show that $\lim n^{-q} = 0$ for any rational $q > 0$. Conclude
 that $\lim n^q$ does not exist.
\item[Solution:] Let $q = a/b$ where $a, b > 0$. Then, 
\[
n^{-q} = \frac{1}{n^q} = \frac{1}{n^{a/b}} = 
\left(\frac{1}{n^{1/b}}\right)^{a}
\]
so that if $L = \lim 1/n^{1/b}$ then $\lim n^{-q} = L^a$. By lemma 
\ref{c5s5l2}, $L = 0$.

Let, if possible, $\lim n^q = L \in \sor$. Since $n^q \ne 0$ for all $n$,
by lemma \ref{c5s1l5}, $\lim n^{-q} = L^{-1} \ne 0$, a contradiction.
\end{enumerate}

\section{Subsequences}\label{c5s6}
\begin{defn}\label{c5s6d1}
Let $\seq{a_n}$ and $\seq{b_n}$ be sequences of real numbers. We say that 
$\seq{b_n}$ is a subsequence of $\seq{a_n}$ iff there exists a function
$f: \son \rightarrow \son$ which is strictly increasing, that is, $f(n+1)
> f(n)$ for all $n \in \son$, such that $b_n = a_{f(n)}$ for all $n \in
\son$.
\end{defn}

\begin{lem}\label{c5s6l1}
Every sequence is a subsequence of itself.
\end{lem}
\begin{proof}
Choose $f$ in definition \ref{c5s6d1} to be the identity function.
\end{proof}

\begin{lem}\label{c5s6l2}
If $\seq{b_n}$ is a subsequence of $\seq{a_n}$ and $\seq{c_n}$ is a
subsequence of $\seq{b_n}$ then $\seq{c_n}$ is also a subsequence of
$\seq{a_n}$.
\end{lem}
\begin{proof}
$\seq{b_n}$ is a subsequence of $\seq{a_n}$ iff there exists a function
$f: \son \rightarrow \son$ such that $f(n + 1) > f(n)$ for all $n \in
\son$ and $b_n = a_{f(n)}$ for all $n \in \son$. Similarly, there exists
a function $g: \son \rightarrow \son$ such that $g(n + 1) > g(n)$ for all
$n \in \son$ such that $c_n = b_{g(n)}$. Therefore, $c_n = a_{g(f(n)} =
a_{(g \circ f)(n)}$. Clearly $g \circ f: \son \rightarrow \son$. If we 
also show that it is a strictly increasing function then we can conclude
that $\seq{c_n}$ is a subsequence of $\seq{a_n}$. Now $f(n + 1) > f(n)
\Rightarrow g(f(n + 1)) > g(f(n))$ so that $g \circ f$ is an increasing
function of $n$.
\end{proof}

\begin{prop}\label{c5s6p1}
Let $\seq{a_n}$ be a sequence of real numbers and $\seq{b_n}$ be its
subsequence. Then $\lim\seq{a_n} = L$ iff $\lim\seq{b_n} = L$ for every
subseequence of $\seq{a_n}$.
\end{prop}
\begin{proof}
Let $\seq{b_n}$ be such that $b_n = a_{f(n)}$ for all $n \in \son$ and $f:
\son \rightarrow \son$ is a strictly increasing function. If $\lim\seq{a_n}
= L$ then for any $\epsilon > 0$ there exists $N \in \son$ such that $|a_n
- L| < \epsilon$ for all $n \ge N$. Since $f(n) \ge n$ for all $n \in \son$,
we also have $|a_{f(n)} - L| < \epsilon$ for all $n \ge N$. Therefore, $
|b(n) - L| < \epsilon$ for all $n > f(N)$.

To prove the converse, we will prove its contrapositive. The converse is
`if all subsequences converge to $L$ then the sequence converges to $L$'.
Its contrapositive is `if the sequence does not converge to $L$ then
there is at least one subsequence that does not converge to $L$'. Since 
a sequence is a subsequence of itself, the contrapositive it trivially 
true.
\end{proof}

\begin{prop}\label{c5s6p2}
Let $\seq{a_n}$ be a sequence of real numbers and $L$ be a real number.
Then $L$ is a limit point of $\seq{a_n}$ iff there is a subsequence of
$\seq{b_n}$ of $\seq{a_n}$ which converges to $L$.
\end{prop}
\begin{proof}
Let $L$ be a limit point of $\seq{a_n}$. Then for any $\epsilon > 0$, we
can find $N \in \son$ such that there is at least one $n > N$ for which
$|a_n - L| < \epsilon$. Define the function $f: \son \rightarrow \son$
such that
\[
f(n) = \begin{cases}
 & n \text{ if } n \le N \\
 & \text{first } n > f(n - 1) \text{ for which } |a_n - L| < \epsilon
\end{cases}
\]
By definition $f$ is an increasing function. If $b_n = a_{f(n)}$ then $|b_n
- L| < \epsilon$ for all $n \ge N$, so that $\lim\seq{b_n} = L$.

Now assume that $b_n = a_{f(n)}$ is such that $\lim\seq{b_n} = L$. Then
for a given $\epsilon > 0$ there exists $N \in \son$ such that $|b_n - L|
< \epsilon \Rightarrow |a_{f(n)} - L| < \epsilon$ for all $n \ge N$. In
particular, for any $\epsilon > 0$ there exists $N \in \son$ such that
$|a_{f(N)} - L| < \epsilon$ where $f(N) \ge N$ ($f$ has to be an increasing
function of $N$). Therefore, $L$ is a limit point of $\seq{a_n}$.
\end{proof}

\begin{thm}[Bolzano-Weierstrass]\label{c5s6t1}
Every bounded sequence has a convergent subsequence.
\end{thm}
\begin{proof}
Let $\seq{a_n}$ be a bounded sequence. Therefore, there exists $M \in \sor$
such that $|a_n| \le M$ for all $n \in \son$. By lemmas \ref{c5s4l15} and
\ref{c5s4l16} $|\limsup\seq{a_n}| \le M$ and $|\liminf\seq{a_n}| \le M$.
Since $-M \le \limsup\seq{a_n} \le M$, $\limsup\seq{a_n} \in \sor$. 
Therefore, by lemma \ref{c5s4l9}, $\limsup\seq{a_n}$ is a limit point of
$\seq{a_n}$. By proposition \ref{c5s6p2}, $\seq{a_n}$ has a subseqeuence
whose limit point is $\limsup\seq{a_n}$.
\end{proof}

\subsection{Exercises}
\begin{enumerate}
\item[1:] Can you find two sequences $\seq{a_n}$ and $\seq{b_n}$ which are
not the same sequence but that each is a subsequence of the other?
\item[Solution:] $a_n = (-1)^n, b_n = (-1)^{n+1}$.

\item[2:] Let $\seq{a_n}$ be an unbounded sequence. Show that it has a 
subsequence $\seq{b_n}$ such that $\lim b_n^{-1} = 0$.
\item[Solution:] $\seq{a_n}$ is unbounded means that for every $m \ge 1$
there is at least one $n \in \son$ such that $|a_n| > m$. Define $f: \son
\rightarrow \son$ such that $f(m)$ is the smallest $n > f(m-1)$ such that
$|a_{f(m)}| > m$ and $f(0)$ is the smallest $n$ for which $|a_n| > 0$.
$f$ is clearly a strictly increasing function. Therefore $b_m = a_{f(m)}$ 
is a subsequence such that $|b(m)| > m$ for all $m \in \son$. Therefore,
$|b_m^{-1}| < m^{-1}$ for all $m \in \son$. Therefore 
$\inf(\seq{|b_m|^{-1}}) \le \inf(\seq{m^{-1}}) = 0$. But $\seq{|b_m|^{-1}}$
is also a sequence of non-negative terms. Therefore, $0 \le \inf(\seq{|b_m|
^{-1}})$ and hence $\inf(\seq{|b_m|^{-1}}) = 0$. Since $\seq{|b_m|}$ is
monotonically decreasing, the infimum is a limit. Thus, $\lim\seq{|b_m|^{
-1}} = 0$.

Now $-|b_m| \le b_m \le |b_m| \Rightarrow -|b_m|^{-1} \le b_m^{-1} \le
|b_m|^{-1} \Rightarrow \lim\seq{b_m^{-1}} = 0$.
\end{enumerate}

\section{Real exponentiation}\label{c5s7}
We will now define the quantity $x^\alpha$ where $x \ge 0$ and $\alpha \in
\sor$ as a limit of the sequence $\seq{x^{q_n}}$ where $\seq{q_n}$ is a
sequence of rationals whose limit is $\alpha$. Before we do that, we need a 
few lemmas.
\begin{lem}\label{c5s7l1}
Let $x > 1$ be a real number, $\seq{p_n}$ and $\seq{q_n}$ be sequences of
rational numbers such that both converge to $\alpha$. Then 
$\lim\seq{x^{p_n}} = \lim\seq{x^{q_n}}$.
\end{lem}
\begin{proof}
We will first show that if $\seq{p_n}$ converges then so does 
$\seq{x^{p_n}}$. To do that it suffices to show that it is a Cauchy 
sequence. Fix an integer $K > 1$. As $\seq{p_n}$ is a convergent sequence
there exists $N_p \in \son$ such that for all $n > m > N_p$, $|p_n - p_m|
< 1/K$. Consider
\[
|x^{p_n} - x^{p_m}| < x^{p_n}|x^{p_n - p_m} - 1| < x^{p_n}|x^{1/K} - 1|.
\]
Since $\seq{p_n}$ is a Cauchy sequence, it is also bounded. Therefore, there
is an $M > 0$ such that $|p_n| < M$ for all $n \in \son$. Therefore,
\[
|x^{p_n} - x^{p_m}| < x^M|x^{1/K} - 1|.
\]
Fix an $\epsilon > 0$. Choose $K$ such that
\[
x^{1/K} < \frac{\epsilon}{x^M} + 1.
\]
Such a $K$ can be found out because $\lim\seq{x^{1/n}} = 1$. For this choice
of $K$, there exists $N_p \in \son$ such that for all $n > m > N_p$, 
$|x^{p_n} - x^{p_m}| < \epsilon$. We next show that if $\lim\seq{p_n} = 
\lim\seq{q_n}$ then $\lim\seq{x^{p_n}} = \lim\seq{x^{q_n}}$. Consider
\[
|x^{p_n} - x^{q_n}| = x^{q_n}|x^{p_n - q_n} - 1|.
\]
Since $\lim\seq{p_n} = \lim\seq{q_n}$, $\lim\seq{p_n - q_n} = 0$. For a
fixed $K > 0$, there exists $N \in \son$ such that $|p_n - q_n| <
1/K$ for all $n \ge N$. For such an $n$,
\[
|x^{p_n} - x^{q_n}| < x^{q_n}|x^\delta - 1| < x^M|x^{1/K} - 1|,
\]
where $M > |q_n|$ is a bound of $\seq{q_n}$. For a given $\epsilon > 0$
choose a $\delta$ such that
\[
x^{1/K} < \frac{\epsilon}{x^M} + 1.
\]
\end{proof}

\begin{lem}\label{c5s7l2}
Let $x \le 1$ be a real number, $\seq{p_n}$ and $\seq{q_n}$ be sequences of
rational numbers such that both converge to $\alpha$. Then 
$\lim\seq{x^{p_n}} = \lim\seq{x^{q_n}}$.
\end{lem}
\begin{proof}
If $x = 1$, $x^{p_n}$ and $x^{q_n}$ are constant sequences. The proof for
$x < 1$ is similar to the one for lemma \ref{c5s7l1}.
\end{proof}

\begin{defn}\label{c5s7d1}
Let $x > 0$ be a real number and $\alpha \in \sor$. If $\seq{q_n}$ be
any sequence of rationals such that $\lim\seq{q_n} = \alpha$ then $x^\alpha
:= \lim x^{q_n}$.
\end{defn}

Finally, we prove the following properties of exponentials. In the lemmas
below, $x, y, z, q, r \in \sor$ and $x, y, z$ are positive.
\begin{lem}\label{c5s7l3}
$x^q > 0$.
\end{lem}

\begin{lem}\label{c5s7l4}
$x^{q+r} = x^q x^r$.
\end{lem}

\begin{lem}\label{c5s7l5}
$x^{-q} = 1/x^q$.
\end{lem}

\begin{lem}\label{c5s7l6}
If $q > 0$ then $x > y \Rightarrow x^q > y^q$.
\end{lem}

\begin{lem}\label{c5s7l7}
If $x > 1$ then $x^q > x^r$ iff $q > r$.
\end{lem}

\begin{lem}\label{c5s7l8}
If $x < 1$ then $x^q < x^r$ iff $q > r$.
\end{lem}

\begin{lem}\label{c5s7l9}
$(xy)^q = x^q y^q$.
\end{lem}

