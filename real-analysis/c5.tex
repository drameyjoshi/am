\chapter{Limits of sequences}\label{c5}
\section{Convergence and limit laws}\label{c5s1}
\begin{defn}\label{c5s1d1}
Given two real numbers $x$ and $y$, we define the distance between them as
$d(x, y) = |x - y|$.
\end{defn}
\begin{defn}\label{c5s1d2}
Let $\epsilon > 0$ be a real number. We say that two real numbers $x$ and $y$
are $\epsilon$-close iff $d(x, y) \le \epsilon$.
\end{defn}
\begin{defn}\label{c5s1d3}
A sequence $\seq{a_n}$ of real number is a Cauchy sequence if for every
real $\epsilon > 0$ there exists a natural number $N$ such that for all
$n, m \ge N$, $d(a_n, a_m) \le \epsilon$.
\end{defn}
\begin{defn}\label{c5s1d4}
A sequence $\seq{a_n}$ of real numbers converges to $L \in \sor$ iff for
every $\epsilon > 0$ we can find $N \in \son$ such that for all $n \ge N$,
$d(a_n, L) \le \epsilon$. A sequence that converges to a real number is 
called convergent. Otherwise it is called divergent.
\end{defn}

\begin{prop}\label{c5s1p1}
If a sequence is convergent then it converges to a unique limit.
\end{prop}
\begin{proof}
Let, if possible, $\seq{a_n}$ converge to $L_1$ and $L_2$. Hence, for any 
$\epsilon > 0$, we can find $N_1, N_2 \in \son$ such that for all $n \ge 
N_1$, $d(L_1, a_n) < \epsilon/2$ and for all $n \ge N_2$, $d(L_2, a_n) <
\epsilon/2$. If $N = \max(N_1, N_2)$, then both inequalities are true for
all $n \ge N$. Now consider $|L_1 - L_2| = |L_1 - a_n + a_n - L_2| \le
|L_1 - a_n| + |L_2 - a_n| \le \epsilon$. Therefore, $L_1 = L_2$.
\end{proof}
\begin{rem}
If $x, y \in \sor$ are such that for any $\epsilon > 0$, $|x - y| < \epsilon$
then $x = y$.
\end{rem}

When a sequence $\seq{a_n}$ converges to a limit $L$ we write it as
\[
\lim_{n \rightarrow \infty} a_n = L.
\]

\begin{prop}\label{c5s1p2}
\[
\lim_{n \rightarrow \infty} \frac{1}{n} = 0.
\]
\end{prop}
\begin{proof}
Let $\epsilon > 0$. Then if $N > \epsilon^{-1}$, $n > N$, we have 
\[
\frac{1}{n} < \frac{1}{N} < \epsilon \Rightarrow \left|\frac{1}{n} - 0\right|
< \epsilon.
\]
\end{proof}

\begin{prop}\label{c5s1p3}
A convergent sequence is a Cauchy sequence.
\end{prop}
\begin{proof}
Let 
\[
\lim_{n \rightarrow \infty} a_n = L.
\]
Then for any $\epsilon > 0$ we can find $N \in \son$ such that for all
$n > N$, $|a_n - L| < \epsilon/2$. If $m, n > N$, then $|a_n - a_m| <
|a_n - L + L - a_m| \le |a_n - L| + |a_m - L| > \epsilon$.
\end{proof}

\begin{prop}\label{c5s1p4}
If a sequence of rationals is a Cauchy sequence in the sense of chapter 
\ref{c4} then it also a Cauchy sequence in the sense of definition 
\ref{c5s1d3}. The converse is also true.
\end{prop}
\begin{proof}
If a sequence is Cauchy in the sense of chapter \ref{c4} then for any real
$\epsilon > 0$ we find a rational $\epsilon_r > 0$ such that $\epsilon_r <
\epsilon$. For this choice of $\epsilon_r$ we can find $N \in \son$ such that
for all $n, m > N$, $d(a_n, a_m) < \epsilon_r < \epsilon$.

To prove the converse, interchange the role of $\epsilon$ and $\epsilon_r$.
\end{proof}

\begin{prop}\label{c5s1p5}
Let $\seq{a_n}$ be a Cauchy sequence of rational and $L = 
\overline{\seq{a_n}}$ then 
\[
\lim_{n \rightarrow \infty} a_n = L.
\]
\end{prop}
\begin{proof}
Let $L$ be represented by a sequence $\seq{b_n}$ in the equivalance class
$\overline{\seq{a_n}}$. Therefore, $\overline{\seq{a_n}} = 
\overline{\seq{b_n}}$. Let $\epsilon > 0$ be a given real number. Let 
$\epsilon > \epsilon_r > 0$ be a positive rational number. Then
\[
|L - a_N| = |\overline{\seq{b_n}} - a_N|
\]
For a fixed $N$, $a_N = \overline{\seq{a_N}}$ where all terms of the sequence
are $a_N$. Therefore,
\[
|L - a_N| = |\overline{\seq{b_n}} - \seq{a_N}| = |\overline{b_n - a_N}| <
\epsilon
\]
for $n, N > M$, where the number $M$ depends on $\epsilon_r$. Since 
$\seq{a_N}$ is also a member of $\overline{\seq{b_n}}$.
\end{proof}

\begin{defn}\label{c5s1d5}
A sequence $\seq{a_n}$ is said to be bounded if there is a real number $M>0$
such that $|a_n| < M$ for all $n$. The number $M$ is called a bound of the
sequence.
\end{defn}

\begin{prop}\label{c5s1p6}
A convergent sequence is bounded.
\end{prop}
\begin{proof}
Let $\seq{a_n}$ converge to $L$. For a fixed $\epsilon > 0$, we can find
$N \in \son$ such that, for all $n > N$, $|a_n - L| < \epsilon$. Therefore,
for all $n > N$, $L - \epsilon < a_n < L + \epsilon$. Let $M_1$ be the 
maximum number in the finite set $|a_1|, \ldots, |a_n|$. Then $M = M_1 + 
L + \epsilon$ is such that $|a_n| < M$ for all $n$.
\end{proof}
