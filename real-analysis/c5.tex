\chapter{Limits of sequences}\label{c5}
\section{Convergence and limit laws}\label{c5s1}
\begin{defn}\label{c5s1d1}
Given two real numbers $x$ and $y$, we define the distance between them as
$d(x, y) = |x - y|$.
\end{defn}
\begin{defn}\label{c5s1d2}
Let $\epsilon > 0$ be a real number. We say that two real numbers $x$ and $y$
are $\epsilon$-close iff $d(x, y) \le \epsilon$.
\end{defn}
\begin{defn}\label{c5s1d3}
A sequence $\seq{a_n}$ of real number is a Cauchy sequence if for every
real $\epsilon > 0$ there exists a natural number $N$ such that for all
$n, m \ge N$, $d(a_n, a_m) \le \epsilon$.
\end{defn}
\begin{defn}\label{c5s1d4}
A sequence $\seq{a_n}$ of real numbers converges to $L \in \sor$ iff for
every $\epsilon > 0$ we can find $N \in \son$ such that for all $n \ge N$,
$d(a_n, L) \le \epsilon$. A sequence that converges to a real number is 
called convergent. Otherwise it is called divergent.
\end{defn}

\begin{prop}\label{c5s1p1}
If a sequence is convergent then it converges to a unique limit.
\end{prop}
\begin{proof}
Let, if possible, $\seq{a_n}$ converge to $L_1$ and $L_2$. Hence, for any 
$\epsilon > 0$, we can find $N_1, N_2 \in \son$ such that for all $n \ge 
N_1$, $d(L_1, a_n) < \epsilon/2$ and for all $n \ge N_2$, $d(L_2, a_n) <
\epsilon/2$. If $N = \max(N_1, N_2)$, then both inequalities are true for
all $n \ge N$. Now consider $|L_1 - L_2| = |L_1 - a_n + a_n - L_2| \le
|L_1 - a_n| + |L_2 - a_n| \le \epsilon$. Therefore, $L_1 = L_2$.
\end{proof}
\begin{rem}
If $x, y \in \sor$ are such that for any $\epsilon > 0$, $|x - y| < \epsilon$
then $x = y$.
\end{rem}

When a sequence $\seq{a_n}$ converges to a limit $L$ we write it as
\[
\lim_{n \rightarrow \infty} a_n = L.
\]

\begin{prop}\label{c5s1p2}
\[
\lim_{n \rightarrow \infty} \frac{1}{n} = 0.
\]
\end{prop}
\begin{proof}
Let $\epsilon > 0$. Then if $N > \epsilon^{-1}$, $n > N$, we have 
\[
\frac{1}{n} < \frac{1}{N} < \epsilon \Rightarrow \left|\frac{1}{n} - 0\right|
< \epsilon.
\]
\end{proof}

\begin{prop}\label{c5s1p3}
A convergent sequence is a Cauchy sequence.
\end{prop}
\begin{proof}
Let 
\[
\lim_{n \rightarrow \infty} a_n = L.
\]
Then for any $\epsilon > 0$ we can find $N \in \son$ such that for all
$n > N$, $|a_n - L| < \epsilon/2$. If $m, n > N$, then $|a_n - a_m| <
|a_n - L + L - a_m| \le |a_n - L| + |a_m - L| > \epsilon$.
\end{proof}

\begin{prop}\label{c5s1p4}
If a sequence of rationals is a Cauchy sequence in the sense of chapter 
\ref{c4} then it also a Cauchy sequence in the sense of definition 
\ref{c5s1d3}. The converse is also true.
\end{prop}
\begin{proof}
If a sequence is Cauchy in the sense of chapter \ref{c4} then for any real
$\epsilon > 0$ we find a rational $\epsilon_r > 0$ such that $\epsilon_r <
\epsilon$. For this choice of $\epsilon_r$ we can find $N \in \son$ such that
for all $n, m > N$, $d(a_n, a_m) < \epsilon_r < \epsilon$.

To prove the converse, interchange the role of $\epsilon$ and $\epsilon_r$.
\end{proof}

\begin{prop}\label{c5s1p5}
Let $\seq{a_n}$ be a Cauchy sequence of rational and $L = 
\overline{\seq{a_n}}$ then 
\[
\lim_{n \rightarrow \infty} a_n = L.
\]
\end{prop}
\begin{proof}
Let $L$ be represented by a sequence $\seq{b_n}$ in the equivalance class
$\overline{\seq{a_n}}$. Therefore, $\overline{\seq{a_n}} = 
\overline{\seq{b_n}}$. Let $\epsilon > 0$ be a given real number. Let 
$\epsilon > \epsilon_r > 0$ be a positive rational number. Then
\[
|L - a_N| = |\overline{\seq{b_n}} - a_N|
\]
For a fixed $N$, $a_N = \overline{\seq{a_N}}$ where all terms of the sequence
are $a_N$. Therefore,
\[
|L - a_N| = |\overline{\seq{b_n}} - \seq{a_N}| = |\overline{b_n - a_N}| <
\epsilon
\]
for $n, N > M$, where the number $M$ depends on $\epsilon_r$. Since 
$\seq{a_N}$ is also a member of $\overline{\seq{b_n}}$.
\end{proof}

\begin{defn}\label{c5s1d5}
A sequence $\seq{a_n}$ is said to be bounded if there is a real number $M>0$
such that $|a_n| < M$ for all $n$. The number $M$ is called a bound of the
sequence.
\end{defn}

\begin{prop}\label{c5s1p6}
A convergent sequence is bounded.
\end{prop}
\begin{proof}
Let $\seq{a_n}$ converge to $L$. For a fixed $\epsilon > 0$, we can find
$N \in \son$ such that, for all $n > N$, $|a_n - L| < \epsilon$. Therefore,
for all $n > N$, $L - \epsilon < a_n < L + \epsilon$. Let $M_1$ be the 
maximum number in the finite set $|a_1|, \ldots, |a_n|$. Then $M = M_1 + 
L + \epsilon$ is such that $|a_n| < M$ for all $n$.
\end{proof}

We will now prove a few elementary properties of limits of Cauchy sequences.
In each of these we assume that the sequences $\seq{a_n}$ and $\seq{b_n}$
converge to $x$ and $y$ respectively.
\begin{lem}\label{c5s1l1}
$\seq{a_n + b_n}$ converges to $x + y$.
\end{lem}
\begin{proof}
Fix an $\epsilon > 0$. Then we can find $N \in \son$ such that $|a_n - x| <
\epsilon/2$ and $|b_n - y| < \epsilon/2$ for all $n \ge N$. Therefore,
$|a_n + b_n - (x + y)| \le |a_n - x| + |b_n - y| < \epsilon$.
\end{proof}

\begin{lem}\label{c5s1l2}
$\seq{a_n - b_n}$ converges to $x - y$.
\end{lem}
\begin{proof}
$|a_n - b_n - (x - y)| = |a_n - x - (b_n - y)| \le |a_n - x| + |b_n - y| 
< \epsilon$.
\end{proof}

\begin{lem}\label{c5s1l3}
If $c \in sor$ then $\seq{ca_n}$ converges to $cx$.
\end{lem}
\begin{proof}
Fix an $\epsilon > 0$. If $c = 0$, the sequence is a constant sequence of all
zeros and it converges to $0$. Let us assume that $c \ne 0$. Then there 
exists $N \in \son$ such that for all $n \ge N$, $|a_n - x| < \epsilon/|c|$. 
Then $|ca_n - cx| = |c||a_n - x| < \epsilon$.
\end{proof}

\begin{lem}\label{c5s1l4}
$\seq{a_nb_n}$ converges to $xy$.
\end{lem}
\begin{proof}
Let us first consider the case $x \ne 0$.
Since $\seq{b_n}$ is convergent, by proposition \ref{c5s1p6}, it is bounded.
Let $M$ be its upper bound. Fix an $\epsilon > 0$. We can find an $N \in 
\son$ such that for all $n \ge N$,
\[
|a_n - x| \le \frac{\epsilon}{M} \;\land\; |b_n - y| \le \frac{\epsilon}{|x|}
\]
Therefore,
\[
|a_nb_n - xy| = |a_nb_n - xb_n + xb_n - xy| \le |a_n - x||b_n| + |x||b_n - y|
< \epsilon.
\]

If $x = 0$ then we will show that $\seq{a_nb_n}$ converges to $0$. Fix an
$\epsilon > 0$. Then $|a_nb_n - 0| = |a_n||b_n| < |a_n|M$. If we choose an
$N \in \son$ such that $|a_n - 0| < \epsilon/M$ for all $n \ge N$ then we
also have $|a_nb_n - 0| < \epsilon$ for all $n \ge N$.
\end{proof}

\begin{lem}\label{c5s1l5}
If $y \in 0$ and $b_n \ne 0$ for all $n \in \son$ then $\seq{b_n^{-1}}$
converges to $y^{-1}$.
\end{lem}
\begin{proof}
Since $b_n \ne 0$ for all $m$, the sequence $\seq{|b_n|}$ is bounded away
from zero and there exists $0 < m$ such that $m < |b_n|$ for all $n \in
\son$. Fix an $\epsilon > 0$. Then we can find an $N \in \son$ such that
for all $n \ge N$, 
\[
|b_n - y| < \epsilon|y|m.
\]
Now consider
\[
\left|\frac{1}{b_n} - \frac{1}{y}\right| = \frac{|b_n - y|}{|y||b_n|}.
\]
Since $m < |b_n|$, $|b_n|^{-1} < m^{-1}$. Therefore,
\[
\left|\frac{1}{b_n} - \frac{1}{y}\right| < \frac{|b_n - y|}{|y|m} <
\epsilon.
\]
\end{proof}

\begin{lem}\label{c5s1l6}
If $y \ne 0$ and $b_n \ne 0$ for all $n$ then $\seq{a_n/b_n}$ converges
to $x/y$.
\end{lem}
\begin{proof}
Follows immediately from lemmas \ref{c5s1l5} and \ref{c5s1l4}.
\end{proof}

\begin{lem}\label{c5s1l7}
$\seq{\max(a_n, b_n)}$ converges to $\max(x, y)$.
\end{lem}
\begin{proof}
Without loss of generality assume that $\max(x, y) = x$. Let $d = |x - y|$. 
Fix an $\epsilon > 0$ and let $N \in \son$ be such that for all $n \ge N$,
$|a_n - x| < \epsilon/2, |b_n - y| < \epsilon/2$. Thus $a_n - \epsilon/2 < 
x < a_n + \epsilon/2$ and $b_n - \epsilon/2 < y < b_n + \epsilon/2$. From
these two equations we get $x - y - \epsilon < a_n - b_n < x - y + \epsilon$.
If we choose $\epsilon > d$ then we have $a_n - b_n > 0$ for all $n \in 
\son$ or that $\max(a_n, b_n) = a_n$ for all $n \ge N$. Therefore, for 
all $n \ge N$, $|\max(a_n, b_n) - x| = |a_n - x| < \epsilon/2 < \epsilon$.
\end{proof}

\begin{lem}\label{c5s1l8}
$\seq{\min(a_n, b_n)}$ converges to $\min(x, y)$.
\end{lem}
\begin{proof}
Without loss of generality assume that $\min(x, y) = x$. Let $d = |y - x|$. 
Fix an $\epsilon > 0$ and let $N \in \son$ be such that for all $n \ge N$,
$|a_n - x| < \epsilon/2, |b_n - y| < \epsilon/2$. Thus $a_n - \epsilon/2 < 
x < a_n + \epsilon/2$ and $b_n - \epsilon/2 < y < b_n + \epsilon/2$. From
these two equations we get $x - y - \epsilon < a_n - b_n < x - y + \epsilon$.
If we choose $\epsilon > d$ then we have $a_n - b_n < 0$ for all $n \in 
\son$ or that $\min(a_n, b_n) = a_n$ for all $n \ge N$. Therefore, for 
all $n \ge N$, $|\min(a_n, b_n) - x| = |a_n - x| < \epsilon/2 < \epsilon$.
\end{proof}

\subsection{Exercices}
\begin{enumerate}
\item[1:] Explain why lemma \ref{c5s1l6} fails when $\seq{b_n}$ converges 
to $0$.
\item[Solution:] We will show that when
\[
\lim_{n \rightarrow \infty}b_n = 0
\]
then $\seq{b_n^{-1}}$ does not converge. Fix an $\epsilon > 0$, then we can
find $n \in \son$ such that for all $n \ge N$, $|b_n| < \epsilon$. Therefore,
\[
\frac{1}{|b_n|} = \left|\frac{1}{b_n}\right| > \frac{1}{\epsilon}
\]
Thus, as $n \rightarrow \infty$, the terms of $\seq{b_n^{-1}}$ rise without
limit. It is not a bounded sequence and therefore not convergent.

\begin{rem}
For any $M > 0$, we can find $N \in \son$ such that $|b_n| < M^{-1}$ or that
$|b_n|^{-1} > M$.
\end{rem}
\end{enumerate}

\section{The extended real number system}\label{c5s2}
\begin{defn}\label{c5s2d1}
The extended real number system $\soe$ is the set $\sor$ with two additional
symbols $+\infty$ and $-\infty$. They are distinct from each other and from
every real numbers. $x \in \soe$ is called finite iff it is a real number
and infinite iff it is $\pm\infty$.
\end{defn}

\begin{defn}\label{c5s2d2}
The operation of negation $x \mapsto -x$ on $\sor$ is extended to $\soe$ by
letting $-(+\infty) := -\infty$ and $-(-\infty) = \infty$.
\end{defn}

\begin{defn}\label{c5s2d3}
Let $x, y \in \soe$. Then $x \le y$ iff one of the following three statements
is true
\begin{enumerate}
\item $x, y \in \sor$ and $x \le y$,
\item $y = +\infty$.
\item $x = -\infty$.
\end{enumerate}
\end{defn}

\begin{lem}\label{c5s2l1}
Let $x, y, z \in \soe$. Then the following statements are true.
\begin{enumerate}
\item $x \le x$.
\item Exactly one of the statements $x < y, x = y$ of $x > y$ is true.
\item $x \le y, y \le z \Rightarrow x \le z$.
\item $x \le y \Rightarrow -x \le -y$.
\end{enumerate}
\end{lem}
\begin{proof}
If $x, y, z \in \sor$ then we have already proved these statements. If $x = 
\infty$ then $x \le x$ because of statement 2 in definition \ref{c5s2d3}. If
$x = -\infty$ then $x \le x$ because of statement 3 in definition 
\ref{c5s2d3}

If $x = \infty$ then $x > y$ is true for all $y \in \soe - \{\infty\}$. 
Likewise, if $x = -\infty$ then $x < y$ is true for all $y \in \soe - 
\{-\infty\}$. $x = y$ is true of $x = \infty, y = \infty$ or $x = -\infty,
y = -\infty$ or $x = y \in \sor$.

$x \le y$ if $x \le y \;\land\; x, y \in \sor$ or $y = \infty$ or $x = 
-\infty$. Likewise, $y \le z$ if $y \le z \;\land\; y, z \in \sor$ or $z =
\infty$ of $y = -\infty$. If $x = -\infty$ then $x \le z$ for all $z \in 
\soe$. If $z = \infty$, $x \le z$ for all $x \in \infty$. If $y = \infty$
then $z = \infty$ and hence $x \le z$ is true. If $y = -\infty$ then $x =
-\infty$ and hence $x \le z$ is true. 

Let $x \le y$. Then either $x, y \in \sor \;\land\; x \le y$ of $y = \infty$
or $x = -\infty$. In the first case, we have already proved that $x \le y
\Rightarrow -x \ge -y$. If $y = \infty$ then by definition \ref{c5s2d2},
$-y = -\infty$ and hence $-y \le -x$ for all $x \in \soe$. If $x = -\infty$
then by definition \ref{c5s2d2}, $-x = \infty$ and hence $-x \ge -y$ for all
$y \in \soe$.
\end{proof}

\begin{defn}\label{c5s2d4}
Let $E \subset \soe$. Then we define the supremum of $E$ using the following
rule.
\begin{enumerate}
\item If $E \subset \sor$ then its supremum is as defined as in definition 
\ref{c4s5d2}.
\item If $\infty \in E$ then $\sup(E) := \infty$.
\item If $\infty \notin E$ but $-\infty \in E$ then $\sup(E) := 
\sup(E - \{-\infty\})$.
\end{enumerate}
\end{defn}

\begin{defn}\label{c5s2d5}
If $E \subset \soe$ then $\inf(E) = -\sup(-E)$, where $-E = \{-x \;|\; x
\in E\}$.
\end{defn}

\begin{thm}\label{c5s2t1}
Let $E \subset \soe$ then the following statements are true.
\begin{enumerate}
\item $\forall\; x \in E$, $x \le \sup(E)$ and $x \ge \inf(E)$.
\item If $M \in \soe$ is an upper bound for $E$ then $\sup(E) \le M$.
\item If $m \in \soe$ is an lower bound for $E$ then $\inf(E) \ge m$.
\end{enumerate}
\end{thm}
\begin{proof}
If $\sup(E) \in \sor$ then $\sup(E)$ is an upper bound of $E$ so that
$x \le \sup(E)$ is true. Further, it is a least upper bound so that if
$M \in \sor$ is any other upper bound then $\sup(E) \le M$. If $M = \infty$
then $\sup((E) \le M$ continues to be true.

If $\sup(E) = \infty$ then $x \le \sup(E)$ is true by definition 
\ref{c5s2d3}. Further, $M \in \soe$ can be an upper bound of $E$ iff $M =
\infty$. Therefore, $\sup(E) \le M$ by lemma \ref{c5s2l1}.

The statements about infimum can be proved analogously.
\end{proof}

It is important to note that the supremum or the infimum of a set need not
be a member of a set.

\section{Suprema and infima of sequences}\label{c5s3}
\begin{defn}\label{c5s3d1}
Let $\seq{a_n}$ be a sequence of real numbers. Then $\sup\seq{a_n} :=
\sup\{a_n \;|\; n \in \son\}$ and $\inf\seq{a_n} := \inf\{a_n \;|\; n 
\in \son\}$.
\end{defn}

\begin{prop}\label{c5s3p1}
Let $\seq{a_n}$ be a sequence of real numbers and let $x = \sup\seq{a_n}
\in \soe$. Then $a_n \le x$ for all $n \in \son$. If $M \in \soe$ is an
upper bound for $\{a_n \;|\; n \in \son\}$ then we have $x \le M$. Finally, 
for every extended
real number $y$ for which $y < x$ there exists at least one $n \in \son$
such that $y < a_n \le x$.
\end{prop}
\begin{proof}
Let $E = \{a_n \;|\; n \in \son\}$. Then $x = \sup\seq{a_n} \Rightarrow
x = \sup(E)$. By theorem \ref{c5s2t1} $a_n \le x$ for all $n \in \son$.
and if $M \in \soe$ is an upper bound of $E$ then $x \le M$.

Now let $y < x$. Then $y \in \sor$ and $y \notin \soe$. If there was no $a_n$
greater than $y$ then $y$ will be an upper bound of $E$ and we would have $
x \ge y$, a contradiction.
\end{proof}

The corresponding proposition for infima is
\begin{prop}\label{c5s3p2}
Let $\seq{a_n}$ be a sequence of real numbers and let $x = \inf\seq{a_n}
\in \soe$. Then $a_n \ge x$ for all $n \in \son$. If $m \in \soe$ is a
lower bound for $\{a_n \;|\; n \in \son\}$ then we have $x \ge m$. Finally, 
for every extended real number $y$ for which $y > x$ there exists at least 
one $n \in \son$ such that $y > a_n \ge x$.
\end{prop}
The proof is similar to that of proposition \ref{c5s3p1}.

In proposition \ref{c5s1p6} we proved that convergent sequences are bounded.
However, the converse is not true. The sequence $\{1, -1, 1, -1, \ldots\}$
is bounded but not convergent. However, if a sequence is bounded and monotone
then we show that it converges. 
\begin{prop}\label{c5s3p3}
Let $\seq{a_n}$ be a sequence of real numbers and let $M \in \sor$ be its
upper bound. If $a_{n+1} \ge a_n$ for all $n \ge N$ for some $N \in \son$
then $\seq{a_n}$ converges and
\[
\lim_{n \rightarrow \infty}a_n = \sup\seq{a_n} \le M.
\]
\end{prop}
\begin{proof}
Let us first assume that the sequence is positive after some index $m$. 
We will show that the sequence has to be Cauchy. For if were not then there
will be an $\epsilon_0 > 0$ such that $a_{n+1} - a_n \ge \epsilon_0$ for
all $n \ge N > m$ and some $N \in \son$. Therefore $a_{n + k} \ge a_n + 
k\epsilon_0$. By 
Archimedean property of real numbers (\ref{c4s4p3}) there will be a positive 
integer $K$ such that $M < K\epsilon$ or that $a_{n + K} \ge a_n + 
K\epsilon > a_n + M > M$, contradicting the requirement that $M$ is an upper
bound of $\seq{a_n}$.

Now let us consider the case where $a_n$ are all negative. For some $N \in
\son$ we can find $K \in \son$ such that $-a_N + M < K\epsilon_0$. Therefore
$a_{N + K} > a_N + K\epsilon>0 > -a_N + M > M$, again a contradiction. Note
that $-a_N > 0$.

Since $\seq{a_n}$ is Cauchy, it converges and as a limit, say $L$. We will
now show that $L = \sup\seq{a_n}$. We first show that $L$ is an upper bound
of $E = \{a_n \;|\; n \in \son$. If it was not then there will be some $N
\in \son$ for which $a_N > L$ and if $\epsilon = a_N - L$. Then, for $n > N$,
$|a_n - L| = a_n - L > a_N - L$, as the sequence is monotonic. Therefore,
$|a_n - L| > \epsilon$ so that $L$ is not a limit of the sequence, a
contradiction. 

We will next show that $L$ is the least upper bound of $E$. If it were
not then there would be a real number $K < L$ such that $a_n \le K < L$ for
all $n \in \son$. If $0 < \epsilon < L - K$ then $|a_n - L| = L - a_n =
L - K + K - a_n > \epsilon + (K - a_n) > \epsilon$. Therefore $L$ cannot be
a limit of the sequence.

Thus, we have shown that $L = \sup(E)$. Since $M$ is an upper bound of $E$
we surely have $L \le M$.
\end{proof}

We can analogously show that
\begin{prop}\label{c5s3p4}
Let $\seq{a_n}$ be a sequence of real numbers and let $m \in \sor$ be its
lower bound. If $a_{n+1} \le a_n$ for all $n \ge N$ for some $N \in \son$
then $\seq{a_n}$ converges and
\[
\lim_{n \rightarrow \infty}a_n = \inf\seq{a_n} \le m.
\]
\end{prop}

\begin{prop}\label{c5s3p5}
Let $0 < x < 1$ be a real number. Then we have
\[
\lim_{n \rightarrow \infty}x^n = 0,
\]
\end{prop}
\begin{proof}
We first note that $x^n > 0$ and $x^{n+1} < x^n$ for all $n \in \son$. 
Therefore the sequence $\seq{x^n}$ is monotonically decreasing and bounded
below. Therefore, by proposition \ref{c5s3p5} it converges to a limit $L$.
We will show that $L = 0$. The sequence $\seq{x^{n+1}}$ is also monotonically
decreasing and bounded below and its limit will be $Lx$. However both the
sequences are the same so that $Lx = L$, Since $0 < x < 1$, this is possible
only if $L = 0$.
\end{proof}
