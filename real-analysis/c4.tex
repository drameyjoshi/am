\chapter{The real numbers}\label{c4}
There are several ways to construct the reals from the rationals. We will
follow the one that uses Cauchy sequences. It is profitable because it 
introduces a familiar motif in analysis, that of `completion' of a space
(that is, a set with certain properties).

\section{Cauchy sequences}\label{c4s1}
\begin{defn}\label{c4s1d1}
A sequence of rational numbers is a mapping $\son \rightarrow \soq$. It is
customary to denote it be $\seq{a_n}$ its range rather than the functional
notation $a(n)$.
\end{defn}

\begin{rem}
We will always mean an infinite sequence when we mention `sequence'. If we
want to describe the set $\{a_0, \ldots, a_n\}$ of rational numbers we will
call it a \emph{finite} sequence.
\end{rem}

\begin{defn}\label{c4s1d2}
A sequence $\seq{a_n}$ is called a Cauchy sequence if for any rational 
$\epsilon > 0$ there exists $N \in \son$ such that for all $m, n > N$,
$d(a_m, a_n) < \epsilon$.
\end{defn}

\begin{rem}
We could have written the Cauchy condition as $|a_m - a_n| < \epsilon$. 
However, we chose to write it in terms of a distance function $d$ because
it makes it easy for us to take this definition to metric spaces.
\end{rem}

\begin{defn}\label{c4s1d3}
A sequence $\seq{a_n}$ of rationals is bounded if we can find a rational 
number $M \ge 0$ such that $|a_n| \le M$ for all $n \in \son$. The number
$M$ is called a bound of the sequence.
\end{defn}

\begin{lem}\label{c4s1l1}
A finite sequence is always bounded.
\end{lem}
\begin{proof}
If $\seq{a_n}$ is a Cauchy sequence then for any $\epsilon > 0$ there exists
an $N \in \son$ such that for all $n \ge N$, $|a_n - a_N| \le \epsilon$. Let
$a = \min\{a_1, \ldots, a_N\}$ and $A = \max\{a_1, \ldots, a_N\}$. Then
$-|a| \le a \le a_n \le A \le |A|$ for all $n \in \{1, \ldots, N\}$. For 
$n \ge N$, we have $a_N - \epsilon \le a_n \le a_N + \epsilon \le |a_N| + 
\epsilon$. Therefore, $-|a| - |a_N| - \epsilon \le a_n \le |A| + |a_N| + 
\epsilon$.
\end{proof}

\begin{lem}\label{c4s1l2}
A Cauchy sequence is bounded.
\end{lem}
\begin{proof}
Let $\seq{a_n}$ be a Cauchy sequence. Fix an $\epsilon > 0$. Then we can 
find a natural number $N$ such that for all $m, n > N$, $d(a_m, a_n) < 
\epsilon$. By lemma \ref{c4s1l1}, the finite sequence $\{a_0, \ldots,
a_N\}$ is bounded. Let $M_0$ be its bound. For all $n \ge N + 1$, $d(a_n,
a_{N+1}) < \epsilon \Rightarrow |a_n - a_{N+1}| < \epsilon \Rightarrow
a_{N+1} - \epsilon < a_n < a_{N+1} + \epsilon$. Thus, the rest of the
terms are bounded by $a_{N+1} + \epsilon$. If we let $M = M_0 + |a_{N+1}|
+ \epsilon$ then $|a_n| < M$ for all $n \in \son$.
\end{proof}

\subsection{Properties of Cauchy sequences}
\begin{lem}\label{c4s1l3}
If $\seq{a_n}$ and $\seq{b_n}$ are Cauchy sequences then so is $\seq{a_n
+ b_n}$.
\end{lem}
\begin{proof}
Fix an $\epsilon > 0$. Then there exist natural numbers $N_1, N_2$ such
that for all $m, n > N_1$, $d(a_m, a_n) < \epsilon/2$ and for all $m, n >
N_2$, $d(b_m, b_n) < \epsilon/2$. Let $N = \max(N_1, N_2)$. For $n, m > N$,
consider 
\begin{eqnarray*}
d(a_m + b_m, a_n + b_n) &=& |a_m + b_m - a_n - b_n| \\
 &\le& |a_m - a_n| + |b_m - b_n| \\
 &<& \epsilon.
\end{eqnarray*}
\end{proof}

\begin{lem}\label{c4s1l4}
If $\seq{a_n}$ and $\seq{b_n}$ are Cauchy sequences then so is $\seq{a_n
b_n}$.
\end{lem}
\begin{proof}
By lemma \ref{c4s1l2} $\seq{a_n}$ and $\seq{b_n}$ are bounded. Let $M_a$
and $M_b$ be their bounds. Fix an $\epsilon > 0$. Let $N_a \in \son$ be
such that for all $n, m > N_a$, $d(a_m, a_n) < \epsilon/(2M_b)$. Let $N_b 
\in \son$ be such that for all $n, m > N_b$, $d(b_m, b_n) < \epsilon/
(2M_a)$.  Let $N = \max(N_a, N_b)$. Then, for $n, m > N$, consider
\begin{eqnarray*}
d(a_nb_n, a_mb_m) &=& |a_nb_n - a_mb_m| \\
 &=& |a_nb_n - b_na_m + b_na_m - a_mb_m| \\
 &\le& |b_n||a_n - a_m| + |a_m||b_n - b_m| \\
 &<& M_b\frac{\epsilon}{2M_b} + M_a\frac{\epsilon}{2M_a} \\
 &<& \epsilon.
\end{eqnarray*}
\end{proof}

\begin{lem}\label{c4s1l5}
If $\seq{a_n}$ is a Cauchy sequence then so is $\seq{ka_n}$ for a constant
$k \in \soq$. In particular, $\seq{-a_n}$ is a Cauchy sequence.
\end{lem}
\begin{proof}
Fix an $\epsilon > 0$ and let $N \in \son$ be such that for all $n, m > N$,
$d(a_m, a_n) < \epsilon/|k|$. If $b_n = ka_n$ then we clearly have $d(b_n,
b_m) < \epsilon$ for all $m, n > N$.
\end{proof}

\section{Equivalent Cauchy sequences}\label{c4s2}
\begin{defn}\label{c4s2d1}
Sequences $\seq{a_n}$ and $\seq{b_n}$ are equivalent if for a given
rational $\epsilon > 0$ we can find a natural number $N$ such that for
all $n > N$, $d(a_n, b_n) < \epsilon$.
\end{defn}

\begin{lem}\label{c4s2l2}
Let $\seq{a_n}$ and $\seq{b_n}$ be equivalent sequences. Then $\seq{a_n}$ 
is a Cauchy sequence iff $\seq{b_n}$ is a Cauchy sequence.
\end{lem}
\begin{proof}
Fix an $\epsilon > 0$. Let $\seq{a_n}$ be a Cauchy sequence. Then we can
find $N_1 \in \son$ such that for all $m, n > N_1$, $d(a_m, a_n) < 
\epsilon/3$. Since $\seq{a_n}$ is equivalent to $\seq{b_n}$ we can find $N_2
\in \son$ such that for all $n > N_2$, $d(a_n, b_n) < \epsilon/3$. Choose
$N = \max(N_1, N_2)$ and let $m, n > N$. Then
\begin{eqnarray*}
d(b_m, b_n) &=& |b_m - b_n| \\
 &=& |b_m - a_m + a_m + a_n - a_n - b_n| \\
 &\le& |b_m - a_m| + |a_m - a_n| + |a_n - b_n| \\
 &<& \epsilon.
\end{eqnarray*}
is Cauchy implies that so is $\seq{b_n}$.

Now let $\seq{b_n}$ be a Cauchy sequence and $\seq{a_n}$ be equivalent to 
it.  Fix an $\epsilon$ and let $N_1 \in \son$ be such that for all $m, n > 
N_1$ $d(b_m, b_n) < \epsilon/3$ and let $N_2 \in \son$ be such that for 
all $n > N_2$, $d(a_n, b_n) < \epsilon/3$. Choose $N = \max(N_1, N_2)$ 
and let $m, n > N$. Then,
\begin{eqnarray*}
d(a_m, a_n) &=& |a_m - a_n| \\
 &=& |a_m - b_m + b_n - a_n + b_m - b_n| \\
 &\le& |a_m - b_m| + |a_n - b_n| + |b_m - b_n| \\
 &<& \epsilon.
\end{eqnarray*}
\end{proof}

Note that two sequences can be equivalent without being bounded or Cauchy.
The set of rationals is dense. Therefore two sequences can arbitrarily 
close to each other, term by term, without being identical. However, for
all practical purposes they are the same. Equivalence means that for a
given $\epsilon > 0$ we can find a natural number $N$ such that all terms
of $\seq{b_n}$ are within a distance $\epsilon$ from the corresponding 
terms of $\seq{a_n}$.

Let us denote the equivalance of Cauchy sequences $\seq{a_n}$ and 
$\seq{b_n}$ as $\seq{a_n} \sim \seq{b_n}$. We next show that
\begin{prop}\label{c4s2p1}
The relation $\sim$ between Cauchy sequences is an equivalence relation.
\end{prop}
\begin{proof}
For any $\epsilon > 0$, $d(a_n, a_n) < \epsilon$ for all $n \in \son$.
Therefore, $\seq{a_n} \sim \seq{a_n}$. Since $d(a_n, b_n) = d(b_n, a_n)$, 
$\seq{a_n} \sim \seq{b_n} \Rightarrow \seq{b_n} \sim \seq{a_n}$.

Let $\seq{a_n} \sim \seq{b_n}$ and $\seq{b_n} \sim \seq{c_n}$. Then for a
given $\epsilon > 0$, there exists $N_1, N_2 \in \son$ such that for all
$n > N_1$, $d(a_n, b_n) < \epsilon/2$ and for all $n > N_2$, $d(b_n, c_n)
< \epsilon/2$. Let $N = \max(N_1, N_2)$ and let $n > N$. Then,
\begin{eqnarray*}
d(a_n, c_n) &=& |a_n - c_n| \\
 &=& |a_n - b_n + b_n - c_n| \\
 &\le& |a_n - b_n| + |b_n - c_n| \\
 &<& \epsilon.
\end{eqnarray*}
Thus, $\seq{a_n} \sim \seq{c_n}$.
\end{proof}

\section{The construction of the real numbers}\label{c4s3}
\begin{defn}\label{c4s3d1}
The equivalence class of a Cauchy sequence of rational numbers is called 
a real number. We will denote the equivalence class of a Cauchy sequence
$\seq{a_n}$ by $\overline{\seq{a_n}}$. The set of all real number is 
denoted by $\sor$.
\end{defn}

\begin{rem}
If $q \in \soq$ then the sequence $\seq{a_n}$ with $a_n = q$ for all $n$
is a Cauchy sequence and it represents the real number $q$. We can thus
identify rational numbers with real numbers of same `value'.
\end{rem}

\begin{defn}\label{c4s3d2}
The equivalence class of Cauchy sequences containing the sequence 
$\seq{a_n}$ with the property $|a_n| < \epsilon$ for all $n > N$, where
$N$ may depend on $\epsilon$ is called the zero real number. Sequences
in the corresponding equivalence class are called zero sequences.
\end{defn}

\begin{defn}\label{c4s3d3}
If $x = \overline{\seq{a_n}}$ and $y = \overline{\seq{b_n}}$ are two
real number then their sum is defined as $x + y := \overline{\seq{a_n + 
b_n}}$.
\end{defn}

\begin{lem}\label{c4s3l1}
Addition of real numbers is a well-defined operation.
\end{lem}
\begin{proof}
Let $\seq{a_n} \sim \seq{b_n}$. Then we will show that $\seq{a_n + c_n}
\sim \seq{b_n + c_n}$. Fix an $\epsilon > 0$. Then we can find $N \in 
\son$ such that for all $n > N$. $d(a_n, b_n) < \epsilon/2$. Consider
$d(a_n + c_n - b_n - c_n) = d(a_n - b_n) < \epsilon$ for all $n > N$.
\end{proof}

\begin{defn}\label{c4s3d4}
If $x = \overline{\seq{a_n}}$ and $y = \overline{\seq{b_n}}$ are two
real number then their product is defined as $xy := 
\overline{\seq{a_nb_n}}$.
\end{defn}

\begin{lem}\label{c4s3l2}
Multiplication of real numbers is a well-defined operation.
\end{lem}
\begin{proof}
Let $\seq{a_n} \sim \seq{b_n}$. Since $\seq{c_n}$ is a Cauchy sequence, it
is bounded. Let $M$ be its bound. Fix an $\epsilon > 0$. We can find $N 
\in \son$ such that for all $n > N$, $d(a_n - b_n) < \epsilon/M$. Consider 
$d(c_na_n - c_nb_n) = |c_na_n - c_nb_n| = |c_n||a_n - b_n| < \epsilon$.
\end{proof}

\begin{defn}\label{c4s3d5}
If $x = \overline{\seq{a_n}}$ is a real number then its negation is 
defined as $-x := \overline{\seq{-a_n}}$.
\end{defn}

\begin{lem}\label{c4s3l3}
Negation of real numbers is a well-defined operation.
\end{lem}
\begin{proof}
Let $\seq{a_n} \sim \seq{b_n}$. Then $d(a_n, b_n) = d(-a_n, -b_n)$. 
Therefore, $\seq{-a_n} \sim \seq{-b_n}$.
\end{proof}

\begin{defn}\label{c4s3d6}
A sequence $\seq{a_n}$ of rational numbers is said to be bounded away 
from zero iff there exists a rational number $c > 0$ such that $|a_n| > c$
for all $n \in \son$.
\end{defn}

\begin{lem}\label{c4s3l4}
For all $r, s \in \soq$, $|r + s| \ge |r| - |s|$.
\end{lem}
\begin{proof}
$|r| = |r + s - s| \le |r + s| + |-s| \Rightarrow |r| - |s| \le |r + s$.
\end{proof}

\begin{lem}\label{c4s3l5}
Let $x$ be a non-zero real number. Then its equivalence class contains 
a sequence that is bounded away from zero.
\end{lem}
\begin{proof}
Let $\seq{a_n}$ belong to the equivalence class of $x$ which may not be
bounded away from zero. That is, some of its terms may be zero although
all terms beyond a certain $n$ will be non-zero. 

Since $\seq{a_n}$ is a Cauchy sequence, for a given $\epsilon > 0$, we
can find $N_0 \in \son$ such that for all $n, m > N_0$, $d(a_m, a_n) <
\epsilon/2$.

For all $n > N_0$, it will be \emph{not} be the case that $|a_n| < 
\epsilon$. If that were the case then $\seq{a_n}$ will be a zero sequence.
Therefore, there is at least one $N > N_0$ such that $|a_N| > \epsilon$.

Now, $|a_m| = |a_m - a_N + a_N| \ge |a_N| - |a_m - a_N| > \epsilon/2$,
where we used lemma \ref{c4s3l4}. Here $m > N_0$. Now define a sequence
$\seq{b_n}$ as
\[
b_n = \begin{cases}
 a_{N_0} & \;\forall\; 0 \le n \le N_0 \\
 a_n     & \;\forall\; n > N_0.
\end{cases}
\]
Clearly $\seq{b_n} \sim \seq{a_n}$ but $\seq{b_n}$ is bounded away from 
zero.
\end{proof}

\begin{lem}\label{c4s3l6}
If $\seq{a_n}$ is a Cauchy sequence of rationals bounded away from zero 
then $\seq{a_n^{-1}}$ is also a Cauchy sequence.
\end{lem}
\begin{proof}
Since $\seq{a_n}$ is bounded away from zero there exists a rational number
$c > 0$ such that $|a_n| > c$ for all $n \in \son$. Fix an $\epsilon > 0$.
Choose an $N \in \son$ such that for all $m, n > N$, $d(a_m, a_n) < c^2
\epsilon$. Now consider
\begin{eqnarray*}
d(a_n^{-1}, a_m^{-1}) &=& \left|\frac{1}{a_n} - \frac{1}{a_m}\right| \\
 &=& \frac{|a_n - a_m|}{|a_na_m|} \\
 &<& \frac{c^2\epsilon}{c^2} \\
 &<& \epsilon.
\end{eqnarray*}
\end{proof}

\begin{defn}\label{c4s3d7}
Let $x$ be a non-zero real number. Then its equivalence class will have
at least one sequence $\seq{a_n}$ that is bounded away from zero. The
reciprocal of $x$ is defined as $x^{-1} := \overline{\seq{a_n^{-1}}}$.
\end{defn}

\begin{lem}\label{c4s3l7}
Reciprocation of real numbers is a well-defined operation.
\end{lem}
\begin{proof}
Let $\seq{a_n} \sim \seq{b_n}$. If their reciprocal exists then they
are bounded away from zero. Let $c_1, c_2 > 0$ be such that $|a_n| > c_1$
and $|b_n| > c_2$ for all $n \in \son$. Let $c = \min(c_1, c_2)$. Then
we have 
\[
\frac{1}{|a_n|} < c; \frac{1}{|b_n|} < c
\]
so that
\[
\frac{1}{|a_nb_n|} < c^2.
\]
Fix an $\epsilon > 0$. Then there exists $N \in \son$ such that $d(a_n,
b_n) < \epsilon c^2$ for all $n > N$. Now consider,
\[
\left|\frac{1}{a_n} - \frac{1}{b_n}\right| = \left|\frac{b_n - a_n}{a_nb_n}
\right| = \frac{|a_n - b_n|}{a_nb_n} < \frac{c^2\epsilon}{c^2} = 
\epsilon.
\]
\end{proof}

From lemma \ref{c4s3l1} and \ref{c4s3l3} we infer
\begin{lem}\label{c4s3l8}
Subraction is a well-defined operation.
\end{lem}

Similarly, from \ref{c4s3l2} and \ref{c4s3l7} we infer
\begin{lem}\label{c4s3l9}
Division is a well-defined operation.
\end{lem}

We will now prove the field theoretic properties of real numbers.
\begin{lem}\label{c4s3l10}
Addition is commutative.
\end{lem}
\begin{proof}
Let $x = \overline{\seq{a_n}}$ and $y = \overline{\seq{b_n}}$ then $x + y
= \overline{\seq{a_n}} + \overline{\seq{b_n}} = \overline{\seq{a_n+b_n}}
= \overline{\seq{b_n + a_n}} = \overline{\seq{b_n}} + \overline{\seq{a_n}}
= y + x$.
\end{proof}

\begin{lem}\label{c4s3l11}
Addition is associative.
\end{lem}
\begin{proof}
Very similar to above and depends on the associativity of rationals.
\end{proof}

\begin{lem}\label{c4s3l12}
For all real numbers $x$, $x + 0 = x$.
\end{lem}
\begin{proof}
By definition \ref{c4s3d2}, the real number $0$ is the equivalence class of
a sequence $\seq{b_n}$ with the property $|b_n| < \epsilon$ got all $n
> N$. Let $x$ be the equivalence class of a sequence $\seq{a_n}$. Then
$x + 0$ is the equivalence class of $\seq{c_n} = \seq{a_n} + \seq{b_n}$. 
For $n > N$, $|c_n - a_n| = |b_n| < \epsilon$ so that $\seq{c_n} \sim
\seq{a_n}$ and hence $x + 0 = x$.
\end{proof}

\begin{lem}\label{c4s3l13}
For all real numbers $x$, $x + (-x) = 0$.
\end{lem}
\begin{proof}
Follows immediately from the definition of negation and the fact that for
any rational $a_n$, $a_n + (-a_n) = 0$.
\end{proof}

\begin{lem}\label{c4s3l14}
Multiplication is commutative.
\end{lem}
\begin{proof}
Similar to proof of lemma \ref{c4s3l10}.
\end{proof}

\begin{lem}\label{c4s3l15}
Multiplication is associative.
\end{lem}
\begin{proof}
Similar to proof of lemma \ref{c4s3l11}.
\end{proof}

\begin{lem}\label{c4s3l16}
For all real numbers $x$, $x \times 1 = x$.
\end{lem}
\begin{proof}
The real number $1$ is the equivalence class $\overline{\seq{b_n}}$
where $b_n = 1$ for $n \in \son$. If $x = \overline{\seq{a_n}}$ then $x
\times = \overline{\seq{a_nb_n}} = \overline{\seq{a_n}} = 1$. We used the
property of rationals that $a_n \times 1 = a_n$.
\end{proof}

\begin{lem}\label{c4s3l17}
For all real numbers $x \ne 0$, $x \times x^{-1} = 1$.
\end{lem}
\begin{proof}
Follows immediately from the definition of reciprocal of a non-zero real
number.
\end{proof}

\begin{lem}\label{c4s3l18}
Multiplication distributes over addition.
\end{lem}
\begin{proof}
Follows immediately from the corresponding property of rationals.
\end{proof}

\begin{lem}\label{c4s3l19}
For all real numbers $x, y$ $x(-y) = -(xy) = (-x)y$.
\end{lem}
\begin{proof}
Follows immediately from the corresponding property of rationals.
\end{proof}

\begin{lem}\label{c4s3l20}
For all real numbers $x$, $x \times 0 = 0$.
\end{lem}
\begin{proof}
From lemma \ref{c4s3l13}, $1 + (-1) = 0$. Therefore, $x \times 0 =
x \times (1 + (-1)) = x \times 1 + x \times (-1)$ using \ref{c4s3l18}.
Using \ref{c4s3l19}, we have $x \times 0 = x \times 1 - x \times 1 = 0$.
\end{proof}

\section{Ordering the reals}\label{c4s4}
\begin{defn}\label{c4s4d1}
Let $\{a_n\}$ be a sequence of rationals. We say it is positively 
(negatively) bounded away from zero iff we have a positive (negative)
rational number $c > 0$ ($c < 0$) such that $a_n \ge c$ ($a_n \le c$)
for all $n \in \son$.
\end{defn}

\begin{defn}\label{c4s4d2}
A real number $x$ is said to be positive (negative) if $x = 
\overline{\seq{a_n}}$ for some positively (negatively) bounded away from
zero sequence $\seq{a_n}$ of rationals.
\end{defn}

\begin{lem}\label{c4s4l1}
For every real number $x$ exactly one of the following three statements
is true: (1) $x$ is zero, (2) $x$ is positive, (3) $x$ is negative. 
\end{lem}
\begin{proof}
Let $x = \overline{\seq{a_n}}$. $x$ is positive of $\seq{a_n}$ is 
positively bounded away from zero. Let, if possible, it be negative as 
well. That is, $\seq{b_n} \in \overline{\seq{a_n}}$ but $\seq{b_n}$ is
negatively bounded away from zero. If $c_b < 0$ is the bound of $\seq{b_n}$
and $c_a > 0$ is the bound of $\seq{a_n}$ then $d(a_n, b_n) \ge c_a + c_b
$ and they will not be equivalent. Therefore, $\seq{b_n}$ will not be a 
member of $\overline{\seq{a_n}}$. Now let us examine the possibility
of it being zero simultaneously. In that case, there will be a $\seq{b_n}$
in $\overline{\seq{a_n}}$ such that $|b_n| < \epsilon$ for any $\epsilon > 
0$ for all $n$ greater than some $N$. In particular, $|b_n| < c_a/2$ for 
all $n$ beyond some $N$. If that were to be the case then $|a_n - b_n| \ge
|a_n| - |b_n| \ge c_a/2$ for all large enough $n$. Once again it means 
that $\seq{b_n}$ cannot be equivalent to $\seq{a_n}$.
\end{proof}

\begin{lem}\label{c4s4l2}
A real number $x$ is negative iff $-x$ is positive.
\end{lem}
\begin{proof}
$x = \overline{\seq{a_n}}$ is negative implies that $\seq{a_n}$ is 
negatively bounded away from zero. Therefore, $\seq{-a_n}$ is positively
bounded away from zero. By definition \ref{c4s3d5}, the equivalence class
of this sequence is $-x$. Therefore $-x$ is positive.

If $-x = \overline{\seq{b_n}}$ then $\seq{b_n}$ is negatively bounded away
from zero. Therefore, $\seq{-b_n}$ is positively bounded away from zero 
and its equivalence class will be $-(-x) = x$, a positive real.
\end{proof}

\begin{rem}
$-(-x) = x$ follow immediately from the corresponding property of 
rationals.
\end{rem}

\begin{lem}\label{c4s4l3}
If $x$ and $y$ are positive then so are $x + y$ and $xy$.
\end{lem}
\begin{proof}
Follows immediately from the corresponding properties of rationals.
\end{proof}

\begin{defn}\label{c4s4d3}
The absolute value of a real number $x$ is defined as 
\[
|x| = \begin{cases}
 x & \text{ if } x \ge 0 \\
 -x & \text{ if } x < 0.
\end{cases}
\]
\end{defn}

\begin{defn}\label{c4s4d4}
Let $x, y \in \sor$. We say $x > y$ iff $x - y$ is a positive real number
and $x < y$ iff $x - y$ is a negative real number. We define $x \ge y$ iff
$x > y$ or $x = y$ and $x \le y$ iff $x < y$ or $x = y$.
\end{defn}

\begin{prop}\label{c4s4p1}
Let $x, y, z \in \sor$. Then:
\begin{enumerate}
\item Exactly one of the three statements $x = y, x < y, x > y$ is true.
\item $x > y$ iff $y < x$.
\item $x < y$ and $y < z$ implies $x < z$.
\item If $x < y$ then $x + z < y + z$.
\item If $z > 0$ and $x < y$ then $xz < yz$.
\end{enumerate}
\end{prop}
\begin{proof}
\begin{enumerate}
\item Let $d = x - y$. By lemma \ref{c4s4l1} exactly one of the statements
$d > 0, d < 0, d = 0$ is true. Each of these statements corresponds to 
$x > y, x < y, x = y$ respectively.
\item If $x > y$ then $d > 0$, so that $-d < 0$, that is $-x + y < 0$ or 
$y - x < 0$ or $y < x$. The converse can be proved similarly.
\item Let $d = y - x$ and $e = z - y$ then $d + e = z - x$. Since $d > 0,
e > 0$, by lemma \ref{c4s4l3} $d + e > 0$. Therefore $z > x$.
\item $y + z - (x + z) = y - x > 0 \Rightarrow y + z > x + z$.
\item $yz - xz = (y - x)z$. Since $x < y$, $y - z > 0$. We are given that
$z > 0$. By lemma \ref{c4s4l3}, $(y - x)z > 0$ so that $yz > xz$.
\end{enumerate}
\end{proof}

\begin{lem}\label{c4s4l4}
Let $x > 0$ be a real number then $x^{-1} > 0$.
\end{lem}
\begin{proof}
By lemma \ref{c4s3l19}, if $y = -x^{-1} > 0$ then $x(-y) = -(xy)$. Since
$x > 0, y > 0$, by lemma \ref{c4s4l3}, $xy$ is positive, so that $-(xy)$
is negative. Therefore $x(-y) = xx^{-1} = 1$ is negative.
\end{proof}

\begin{lem}\label{c4s4l5}
Let $x, y \in \sor$, $x > 0, y > 0$ and $x < y$. Then $x^{-1} > 
y^{-1}$.
\end{lem}
\begin{proof}
$x < y \Rightarrow x^{-1}x < x^{-1}y$ because $x^{-1} > 0$. Therefore,
$1 < x^{-1}y \Rightarrow y^{-1} < y^{-1}x^{-1}y = x^{-1}$.
\end{proof}

\begin{lem}\label{c4s4l6}
Let $\seq{a_n}$ be a Cauchy sequence of non-negative rational numbers.
Then $\overline{\seq{a_n}}$ is a non-negative real number.
\end{lem}
\begin{proof}
Let, if possible, $\overline{\seq{a_n}} = x < 0$. Therefore, there is a
sequence $\seq{b_n}$ which is negatively bounded away from zero such that
$x = \overline{\seq{b_n}}$. Therefore, $\seq{a_n} \sim \seq{b_n} 
\Rightarrow d(a_n, b_n) < \epsilon$ for all $n > N$. Let $c > 0$ be 
such that $b_n < -c$ for all $n \in \son$. Then $-b_n > c > 0$ or $a_n
- b_n > a_n + c \ge c$ because $a_n \ge 0$. Therefore, $|a_n - b_n| \ge c
> 0$. Therefore, it is not possible for $\seq{b_n}$ to be in the 
equivalence class of $\seq{a_n}$. This contradicts our assumption that
$\seq{a_n} \sim \seq{b_n}$.
\end{proof}

An immediate corollary of this lemma is
\begin{cor}\label{c4s4c1}
If $\seq{a_n}$ and $\seq{b_n}$ are Cauchy sequences of rationals such
that $a_n \ge b_n$ for all $n \in \son$ then $\overline{\seq{a_n}} \ge
\overline{\seq{b_n}}$.
\end{cor}
\begin{proof}
If $c_n = a_n - b_n$ then $\seq{c_n}$ is a sequence of non-negative
rationals. Therefore, $\overline{\seq{c_n}} \ge 0 \Rightarrow \overline{
\seq{a_n - b_n}} \ge 0 \Rightarrow \overline{\seq{a_n}} - \overline{\seq{
b_n}} \ge 0 \Rightarrow \overline{\seq{a_n}} \ge \overline{\seq{b_n}}$.
\end{proof}

\begin{prop}\label{c4s4p2}
Let $x > 0$ be a real number. Then there exists a rational number $q > 0$
such that $q < x$ and a natural number $M$ such that $x \le M$.
\end{prop}
\begin{proof}
$x > 0 \Rightarrow x = \overline{\seq{a_n}}$ for some $\seq{a_n}$ 
positively bounded away from zero. Therefore, there is a positive rational
$q > 0$ such that $a_n > q > 0$ for all $n \in \son$. Therefore $x > q$.

Claim: If $x, z \in \soq$ such that $x < z$ then we can find $y \in \soq$
such that $x < z < y$.
Proof: $y = 2z - z$ has the desired property.

Since $\seq{a_n}$ is a Cauchy sequence, there exists $N \in \son$ such 
that for all $n, m \ge N$, $|a_n - a_m| < 1$. In particular, $|a_n - a_N|
< 1 \Rightarrow -1 < a_n - a_N < 1 \Rightarrow a_N - 1 < a_n < a_N + 1$
Therefore $x = \overline{\seq{a_n}} < a_N + 1$. By proposition 
\ref{c3s4p1} there is a natural number $M$ greater than the rational number
$a_N + 1$. Therefore $x < M \Rightarrow x \le M$.
\end{proof}

\begin{prop}[Archimedean property]\label{c4s4p3}
Let $x \in \sor$ and $\epsilon > 0$ be another real number. Then there 
exists a positive natural number $M$ such that $x < M\epsilon$.
\end{prop}
\begin{proof}
If $x \le 0$, choose $M = 1$. If $x > 0$, then we know that $x/\epsilon > 
0$. By proposotion \ref{c4s4p2} there exists a positive natural number 
$N$ such that $x/\epsilon \le N < N + 1$. If $M = N + 1$, we have $x < 
\epsilon M$.
\end{proof}

\begin{lem}\label{c4s4l7}
Let $x$ be a positive real number. Then there exists a positive natural
number $N$ such that $x > N^{-1}$.
\end{lem}
\begin{proof}
If $x > 1$ then $N = 1$. If $0 < x < 1$, $1 < x^{-1}$. By proposition 
\ref{c4s4p2} there exists a natural number $N$ such that $x^{-1} \le M <
M + 1$ so that $x > (M + 1)^{-1}$. Choose $N = M + 1$.
\end{proof}

\begin{prop}\label{c4s4p4}
Given two real numbers $x < y$ we can find a rational number $q$ such
that $x < q < y$.
\end{prop}
\begin{proof}
Since $y - x > 0$, using lemma \ref{c4s4l7}, there exists a positive
natural number $N$ such that $y - x > N^{-1}$. Therefore, $N(y - x) > 1$.
Therefore, there exists an integer $M$ such that $Nx < M < Ny$ or that
$x < M/N < y$.
\end{proof}

\subsection{Exercises}
\begin{enumerate}
\item[1:] Let $a \le x < b$ and $c \le x < d$ be such that $|d - c| \ge
|b - a|$. Then either $c \le a \le d$ or $c \le b \le d$ or both.
\item[Solution:] We have $a \le x < b$ and $c \le x < d$. Let, if 
possible, neither $c \le a \le d$ nor $c \le b \le d$. $a$ cannot be 
greater than $d$ for that would mean $x \ge d$. Similarly, $b$ cannot be
smaller than $c$ for that would mean $x \le c$. Therefore $a < c$ and $b 
> d$ so that $b - a < d - c$ contradicting the assumption that $|d - c| 
> |b - a|$. Therefore, one of the three possibilities must be true.

\item[2:] Show that for every real number $x$ there is exactly one integer
$N$ such that $N \le x < N + 1$.
\item[Solution:] Let $x = \overline{\seq{a_n}}$ and let, if possible,
there be two integers $N$ and $M$ such that $N \le x < N + 1$ and $M \le
x < M + 1$. Since $\seq{a_n}$ is a Cauchy sequence, for a given $\epsilon
> 0$ we can find $N_a$ such that for all $n, m \ge N_a$, $|a_n - a_m| 
< \epsilon$. Choose $\epsilon = 1$ so that $|a_n - N| < 1$. Similarly,
$|a_n - M| < 1$. From these two inequalities, we have
\begin{eqnarray}
N - 1 \le a_n \le N + 1 \\
M - 1 \le a_n \le M + 1
\end{eqnarray}
From the previous problem we conclude that either $N$ lies between $M - 1$
and $M + 1$ or $M$ lies between $N - 1$ and $N + 1$ either of which mean 
that $N = M$.

\item[3:] Let $x, y \in \sor$ and $\epsilon > 0$ be another real. Show that
$|x - y| < \epsilon$ iff $y - \epsilon < x < y + \epsilon$ and that 
$|x - y| \le \epsilon$ iff $y - \epsilon \le x \le y + \epsilon$.
\item[Solution:] $|x - y| \le \epsilon$ implies $x - y \le \epsilon$ if 
$x \ge y$ or $y - x \le \epsilon$ if $x < y$. Therefore $x \le y + 
\epsilon$ if $x \ge y$ or $x - y \le -\epsilon$ if $x < y$. Combining the 
two, we get $y - \epsilon \le x \le y + \epsilon$. The case of strict 
inequality is similarly proved.

\item[4:] Let $x, y \in \sor$. Show that $x \le y + \epsilon$ for all
$\epsilon > 0$ iff $x \le y$. Show that $|x - y| \le \epsilon$ for all
$\epsilon > 0$ iff $x = y$.
\item[Solution:] If $x > y$ then $x - y = c$ for some positive real number 
$c$. Therefore, $x = y + c > y + c/2$ contradicting the assumption that
$x \le y + \epsilon$ for all $\epsilon > 0$.

If $x = y$ then $|x - y| = 0 < \epsilon$ for any positive $\epsilon$. If
$|x - y| \le \epsilon$ then $y - \epsilon \le x \le y + \epsilon$. Thus,
$y - \epsilon \le x$ and $x \le y + \epsilon$, that is, $y \le x + 
\epsilon$ and $x \le y + \epsilon$. Therefore, $y \le x$ and $x \le y$

\item[5:] Let $\seq{a_n}$ be a Cauchy sequence of rationals and let $x$ be
a real number. Show that if $a_n \le x$ ($a_n \ge x$) for all $n$ then 
$\overline{\seq{a_n}} \le x$ ($\overline{\seq{a_n}} \ge x$).
\item[Solution:] Let, if possible, $\overline{\seq{a_n}} > x$. By 
proposition \ref{c3s4p4} there is a rational number $q$ such that 
$\overline{\seq{a_n}} > q > x$. Therefore the sequence $\seq{a_n - q}$ is
positively bounded away from zero. That is, for all $n \in \son$, $a_n -
q > c$ for some positive rational $c > 0$. Therefore, $a_n > q$ but we
also have $q > x$. Therefore we get $a_n > x$, a contradiction.

\item[6:] If $x, y \in \sor$, define 
\[
\max(x, y) = \begin{cases}
x & \text{ if } x \ge y \\
y & \text{ if } x < y
\end{cases}
\]
and
\[
\min(x, y) = \begin{cases}
x & \text{ if } x \le y \\
y & \text{ if } x > y.
\end{cases}
\] 
Show that
\begin{enumerate}
\item[(a)] $\max(x, y) = -\min(-x, -y)$ and $\min(x, y) = -\max(-x, -y)$.
\item[Solution:] Let $z = -\min(-x, -y)$. Therefore, $-z = -x$ if $-x \le 
-y$ or $-z = -y$ if $-x > -y$. That is $z = x$ if $x \ge y$ of $z = y$ is
$x < y$. But this is exactly the definition of the $\max$ function.

The other equality can be similarly proved.

\item[(b)] $\max(x, y) = \max(y, x)$. 
\item[Solution:] Let $z = \max(x, y)$. Then $z = x$
if $x \ge y$ and $z = y$ if $x < y$. Equivalently, $z = x$ if $y < x$ and 
$z = y$ if $y \ge x$. But this makes $z = \max(y, x)$.

\item[(c)] $\max(x, x) = x$.
\item[Solution:] Let $\max(x, x) = z$. Then $z = x$ if $x \ge x$, which is
always true.

\item[(d)] $\max(x + z, y + z) = \max(x, y) + z$.
\item[Solution:] Let $u = \max(x + z, y + z)$, Then $u = x + z$ if
$x + z \ge y + z$ or $x \ge y$. Else $u = y + z$ if $x + z < y + z$ or
$x < y$. Thus,
\[
u - z = \begin{cases}
x & \text{ if } x \ge y \\
y & \text{ if } x < y
\end{cases}
\]
or $u - z = \max(x, y)$.

\item[(e)] $\max(xz, yz) = z\max(x, y)$ if $z \ge 0$.
\item[Solution:] Let $u = \max(xz, yz)$. Therefore, $u = xz$ if $xz \ge
yz$ and $yz$ if $xz < yz$. Let us assume $z > 0$. Then we have $u/z = x$
if $x \ge y$ and $u/z = y$ if $x < y$. Thus, we have $u/z = \max(x, y)$.
If $z = 0$ then both sides are identically zero.

If $z < 0$ then $\max(xz, yz) = z\min(x, y)$.

\item[(f)] Let $x, y > 0$. Then $\max(x, y)^{-1} = \min(x^{-1}, y^{-1})$
and $\min(x, y)^{-1} = \max(x^{-1}, y^{-1})$.
\item[Solution:] Let $z = \max(x, y)^{-1}$ so that $z^{-1} = \max(x, y)$
Thus, $z^{-1} = x$ if $x \ge y$ and $z^{-1} = y$ if $x < y$. Equivalently,
$z^{-1} = x$ if $x^{-1} \le y^{-1}$ and $z^{-1} = y$ if $x^{-1} > y^{-1}$
or $z = x^{-1}$ if $y^{-1} \ge x^{-1}$ and $z = y^{-1}$ if $y^{-1} <
x^{-1}$. Thus, $x = \min(x^{-1}, y^{-1})$.
\end{enumerate}
\end{enumerate}

\section{The least upper bound property}\label{c4s5}
\begin{defn}\label{c4s5d1}
Let $E$ be a subset of $\sor$ and $M \in \sor$ be such that for all $x \in
E, x \le M$. Then $M$ is called an upper bound of the set $E$.
\end{defn}

\begin{defn}\label{c4s5d2}
Let $E \subset \sor$ and $M \in \sor$. $M$ is a least upper bound of $E$
iff (a) $M$ is an upper bound of $E$ and (b) if $M^\op$ is another upper
bound of $E$ then $M \le M^\op$.
\end{defn}

\begin{lem}\label{c4s5l1}
Let $E \subset R$ and let $L, M \in \son$ be such that $L < M$ and for
some integer $n > 0$, $L/n \in E$ and $M/n \notin E$. Then there exists
an integer $m$ such that $L < m \le M$, $m/n$ is an upper bound of $E$
but $(m - 1)/n$ is not. Further, $m$ is unique.
\end{lem}
\begin{proof}
Suppose there is no such integer $m$. Then either all $m/n$ are members of
$E$ or all of them are upper bounds of $E$. If they are members of $E$ 
then by induction on integers starting with base case of $L = m$, we can
show that $m/n \in E$ for all $m \le L$. In particular, $M/N \in E$, a
contradiction.

If all are upper bounds then we have $(m - 1)/n$ is an upper bound if 
$m/n$ is. However, carrying out this logic $M - L$ times starting with
$m = M$ we infer that $L/n$ is an upper bound of $E$, which is also not
true.

Let, if possible, there be two integers $m_1$ and $m_2$ with this 
property. Without loss of generality, let $m_1 < m_2$. If $m_1/n$ is an 
upper bound then so is $m_2/n$. However, $(m_2 - 1)/n$ is also an upper
bound because it is either equal to $m_1$ or is greater than $m_1$. In
either case we have $m_2/n$ and $(m_2 - 1)/n$ both being upper bounds of
$E$.
\end{proof}

\begin{prop}\label{c4s5p1}
Let a non-empty set $E \subset \sor$ have an upper bound. Then it has a 
least upper bound.
\end{prop}
\begin{proof}
Let $M$ be an upper bound of $E$. Since $E$ is not empty it has an element
$x_0$ Let $n \ge 1$ be an integer. By Archimedean property, proposition
\ref{c4s4p3}, there are integers $L$ and $H$ such that $L/n \le x_0$ and
$H/n \ge M$. Since $x_0 \in E$. $L/n$ not an upper bound of $E$. Since
$H/n \ge M$, $H/n$ is an upper bound of $E$. By lemma \ref{c4s5l1} there is
a unique integer $m_n$ such that $L < m_n \le H$ such that $m_n/n$ is 
an upper bound of $E$ but $(m_n - 1)/n$ is not. Note that the integer
$m_n$ depends on $n$.

Let $l, k \in \son$ such that $l > k \le N$ for some $N \in \son$ be
integers with the property that $m_l/l$ is an upper bound of $E$ and
$(m_l - 1)/l$ is not.

Since $m_l/l$ is an upper bound but $(m_k - 1)/k$ is not, we have
\[
\frac{m_l}{l} > \frac{m_k - 1}{k} = \frac{m_k}{k} - \frac{1}{k}.
\]
Since $k \ge N$, $k^{-1} \le N^{-1}$ and $-k^{-1} \ge -N^{-1}$. Thus,
\begin{equation}\label{c4s5e1}
\frac{m_l}{l} - \frac{m_k}{k} > -\frac{1}{N}.
\end{equation}

Since $m_k/k$ is an upper bound but $(m_l - 1)/l$ is not, we have
\[
\frac{m_k}{k} > \frac{m_l - 1}{l}
\]
from which we get
\begin{equation}\label{c4s5e2}
\frac{m_k}{k} - \frac{m_l}{l} > -\frac{1}{N} \Rightarrow 
\frac{m_l}{l} - \frac{m_k}{k} < \frac{1}{N}.
\end{equation}
From equations \eqref{c4s5e1} and \eqref{c4s5e2} we have
\begin{equation}\label{c4s5e3}
-\frac{1}{N} < \frac{m_l}{l} - \frac{m_k}{k} < \frac{1}{N}
\Rightarrow \left|\frac{m_l}{l} - \frac{m_k}{k}\right| < \frac{1}{N}
\end{equation}
making $\seq{m_k/k}$ a Cauchy sequence. Let 
\begin{equation}\label{c4s5e4}
S = \overline{\seq{m_k/k}}.
\end{equation}
$\overline{\seq{1/n}} = 0$, we also have 
\begin{equation}\label{c4s5e5}
S = \overline{\seq{(m_k - 1)/k}}.
\end{equation}
Let $x \in E$ be an arbitrary element. Then $x \le m_k/k$ for all $k$.
From problem (5) in the set of exercises in the previous section, we
conclude that $x \le S$. Therefore, $S$ is an upper bound of $E$. Let
$T$ be another upper bound of $E$. Then $T \ge (m_k - 1)/k$ for all $k$
so that $S \ge T$. Therefore, $S$ is indeed the least upper bound of $E$.
\end{proof}

It is easy to prove
\begin{prop}\label{c4s5p2}
Let $E \subset \sor$ then $E$ can have at most one least upper bound.
\end{prop}
\begin{proof}
Let $M_1$ and $M_2$ be two upper bounds of $E$. Since $M_1$ is a least 
upper bound and $M_2$ is an upper bound we have $M_1 \le M_2$. We also
have $M_2 \le M_1$.
\end{proof}

Combining propositions \ref{c4s5p1} and \ref{c4s5p2} we have
\begin{thm}\label{c4s5t1}
If $E \subset \sor$ is non-empty and has an upper bound then it has a 
unique least upper bound.
\end{thm}

The least upper bound of a set $E \subset \sor$ is also called
the supremum of $E$ and is denoted by $\sup(E)$. If $E$ has no upper bound
then we write $\sup(E) = +\infty$. If $E$ is an empty set then we write
$\sup(E) - -\infty$.

A glaring shortcoming of the set $Q$ was that none of its members had
$2$ as its square. The set $\sor$ does not have it.
\begin{prop}\label{c4s5p3}
There exists a positive real number $x$ such that $x^2 = 2$.
\end{prop}
\begin{proof}
Let $E = \{y \in \sor \;|\; y \ge 0, y^2 < 2\}$. $E \ne \varnothing$
because $1 \in E$. $E$ is bounded above because $y = 2$ is not a member
of $E$. Therefore by theorem \ref{c4s5t1} $\sup(E)$ exists. If we denote it
by $x$, we will show that $x^2 = 2$.

If $x^2 < 2$, consider $(x + \epsilon)^2 = x^2 + 2\epsilon x + \epsilon^2
< 2$. Since $x^2 < 2$, $x < 2$ so that $2\epsilon x < 4\epsilon$. If we 
choose $\epsilon < 1$ then $\epsilon^2 < \epsilon$ so that
$x^2 + 5\epsilon < 2$ Therefore $x + \epsilon \in E$. But this contradicts
the assumption that $x$ is an upper bound of $E$.

If $x^2 > 2$ then we will show that we can find $\epsilon > 0$ such that
$(x - \epsilon)^2$ is also greater than $2$. For $(x - \epsilon)^2 =
x^2 - 2\epsilon x + \epsilon^2 \ge x^2 - 2\epsilon x$ for $\epsilon \le 1$.
We also have $x < 2$ because a number like $1.6$ is smaller than $2$ and is
yet an upper bound of $E$. $x < 2 \Rightarrow -x > -2 \Rightarrow -2x
\epsilon > -4\epsilon \Rightarrow x^2 - 2\epsilon x > x^2 - 4\epsilon$.
Thus, $(x - \epsilon)^2 > x^2 - 4\epsilon$. Since $x^2 > 2$, there is a
positive real $c$ such that $x^2 - c = 2$. If we choose $\epsilon < c/4$
then $4\epsilon < c \Rightarrow -4\epsilon > -c \Rightarrow x^2 - 4\epsilon
> x^2 - c = 2$. Thus, we have $(x - \epsilon)^2 > 2$ or that $(x-\epsilon)$
is also an upper bound of $E$. This too contradicts the fact that $x$ is
the least upper bound of $E$.
\end{proof}

\begin{defn}\label{c4s5d3}
A real number $m$ is said to be a lower bound of a set $E$ if $m \le x$
for all $x \in E$. A set is said to be bounded below if it has a lower 
bound.
\end{defn}

\begin{rem}
Every real number is a lower bound of an empty set.
\end{rem}

\begin{defn}\label{c4s5d4}
Let $E$ be a set that is bounded below. A real number $m$ is said to be
the greatest lower bound of $E$ is $m$ is a lower bound of $E$ and every
other lower bound is less than or equal to $m$. The greatest lower bound
of a set $E$ is also called the infimum of $E$ and is denoted by $\inf(E)$.
\end{defn}

\subsection{Exercises}
\begin{enumerate}
\item[1:] Let $E \subset \sor$ be such that $M = \sup(E)$ is a real 
number. If $-E = \{-x \;|\; \forall x  \in E\}$. Then $-M = \inf(-E)$.
\item[Solution:] $M = \sup(E) \Rightarrow x \le M \;\forall\; x \in E
\Rightarrow -M \le -x \;\forall\; x \in E \Rightarrow -M \le y \;\forall\;
y \in -E$. Therefore, $-M$ is a lower bound of $-E$. Let $-m$ be another
lower bound of $-E$. That is $-m \le y \;\forall\; y \in -E \Rightarrow
m \ge -y \;\forall\; y \in -E \Rightarrow m \ge x \;\forall\; x \in E$.
Thus, $m$ is an upper bound of $E$. Therefore $m \ge \sup(E) = M$ so that
$-m \le -M$. Therefore $-M = \inf(-E)$.

\item[2:] Let $q_1, q_2, \ldots$ be a sequence of rational numbers with
the property that $|q_n - q_m| < M^{-1}$ whenever $M \ge 1$ and $n, m > M$.
Show that $\seq{q_n}$ is a Cauchy sequence. Further more, if $S = 
\overline{\seq{q_n}}$ then show that $|q_m - S| \le M^{-1}$ for all
$M \ge 1$.
\item[Solution:] Fix a rational $\epsilon > 0$. By proposition \ref{c3s4p1}
there is a natural number, $N$ such that $N + 1 > \epsilon^{-1} > N$ or 
that $(N+1)^{-1} < \epsilon < N^{-1}$. Let $M = N+1$. Therfore, if 
$|q_n - q_m| < M^{-1}$ then $|q_n - q_m| < \epsilon$ for all $n, m > M$,
making $\seq{q_n}$ a Cauchy sequence.

Let $S = \overline{\seq{q_n}}$. Therefore, given an $\epsilon > 0$, $|
S - q_n| < \epsilon$. By proposition \ref{c3s4p1} we can find $N$ such that
$N < \epsilon^{-1} < N + 1$ of $\epsilon < N^{-1}$.
\end{enumerate}

\section{Real exponentiation - I}\label{c4s6}
\begin{defn}\label{c4s6d1}
Let $x \in \sor$. Then $x^0 := 1$ and $x^{n+1} = x \times x^n$ for all
$n \in \son$.
\end{defn}

\begin{defn}\label{c4s6d2}
Let $x \in \sor, x \ne 0$. Then $x^{-n} := 1/x^n$ for all $n \in \son$.
\end{defn}

None of the lemmas about exponentiation of rational numbers depend on
the least upper bound property. Therefore, they are true for real numbers
as well. However, we recall that in the case of rational numbers we 
defined exponentiation only in the case where the exponent is an integer.
We did not define $p^q$, where $p, q \in \sor$ because we knew that for
$p = 2, q = 1/2$ the result is not a rational number.

Real numbers are constructed to get over this problem. We can, therefore,
define
\begin{defn}\label{c4s6d3}
Let $x \ge 0$ be a real number and $n \ge 1$ be a positive integer. Then
$x^{1/n} := \sup\{y \in \sor \;|\; y^n \le x\}$. The number $x^{1/n}$ is
called the $n$-th root of $x$.
\end{defn}

\begin{lem}\label{c4s6l1}
The $n$-th root of a positive real number as defined in \ref{c4s6d3} 
exists.
\end{lem}
\begin{proof}
The $n$-root is defined as the supremum of a set. Therefore, the existence
of the $n$-th root is equivalent to the existence of the supremum. If
$E = \sup\{y \in \sor \;|\; y^n \le x\}$ then $0 \in E$ so that $E$ is
not empty. Next we show that $E$ is bounded above. We consider two cases
\begin{itemize}
\item $x < 1$. In this case $1$ is an upper bound of $E$. For if there was
a $y \in E$ greater than $1$ then $y > 1 \Rightarrow y^n > 1$ but such a
$y$ cannot be a member of $E$.
\item $x > 1$. In this case $x$ is an upper bound of $E$. For if there was
$y > x$ in $E$ the we would have $y^n > x^n$. However $x > 1 \Rightarrow
x^n > x$. This implies $y^n > x^n > x$, again a contradiction.
\end{itemize}
Thus in all cases $E$ is a non-empty set that is bounded above.
\end{proof}

We will now prove basic properties of roots of positive reals.
\begin{lem}\label{c4s6l2}
$y = x^{1/n} \Rightarrow y^n = x$.
\end{lem}
\begin{proof}
$y^n$ cannot be greater than $x$ because we defined $y$ to be the supremum
of $E = \{y \in \sor \;|\; y^n \le x\}$. Let, if possible, $y^n < x$. Let
$\delta > 0$. Consider 
\[
(y + \delta)^n = y^n + ny^{n-1}\delta + \frac{n(n-1)}{2}y^{n-2}\delta^2
+ \cdots + \delta^n > y^n,
\]
because all terms in $\delta$ are positive. This implies that $y$ is not
an upper bound of $E$, a contradiction.
\end{proof}

\begin{lem}\label{c4s6l3}
$y^n = x \Rightarrow y = x^{1/n}$.
\end{lem}
\begin{proof}
If $y^n = x$ then $y \in E$ where $E = \{y \in \sor \;|\; y \ge 0, y \le
x^n\}$. $y$ is also $\sup(E)$ because any number greater than it will have
its $n$th power greater than $x$. $\sup(E)$ is defined to be the $n$-th
root of $x$.
\end{proof}

The previous two lemmas are not silly. We defined $x^{1/n}$ as a superemum
of a set not as a certain value. This definition has a different flavour 
from the definitions of binary operations.

\begin{lem}\label{c4s6l4}
$x^{1/n}$ is a non-negative real number. It is positive iff $x > 0$.
\end{lem}
\begin{proof}
$x^{1/n} := \sup(E)$ where $E = \{y \in \sor \;|\; y \ge 0, y^n \le x\}$.
If $y < 0$, it will be less than all members of $E$ and cannot be the
supremum of $E$.

If $y > 0$ then $y^n > 0$ therefore $x > 0$. Conversely, if $x > 0$ then
$y > 0$. For if not $x = y^n = 0$.
\end{proof}

\begin{lem}\label{c4s6l5}
$x > y$ iff $x^{1/n} > y^{1/n}$.
\end{lem}
\begin{proof}
Let $E_x = \{u \in \sor \;|\; u^n \le x\}$ and $E_y = \{v \in \sor \;|\; 
v^n \le y\}$. We want to show that $x > y \Rightarrow \sup(E_x) > 
\sup(E_y)$. Now $(\sup(E_x))^n = x > y$. Therefore, $\sup(E_x)$ is an
upper bound of $E_y$ or $\sup(E_x) \ge \sup(E_y)$. If $\sup(E_x) = 
\sup(E_y)$ then $(\sup(E_x))^n = (\sup(E_y))^n$ or $x = y$, a 
contradiction. Therefore, $\sup(E_x) > \sup(E_y)$.

To prove the converse, we use lemmas \ref{c3s3l25} and \ref{c3s3l26},
for they are applicable to real numbers as well. $x^{1/n} > y^{1/n} 
\Rightarrow (x^{1/n})^n > (y^{1/n})^n \Rightarrow x > y$.
\end{proof}

\begin{lem}\label{c4s6l6}
If $x > 1$ then $\sup(E) > \sup(F)$ where $E = \{y \in \sor \;|\; y \ge 0,
y^n \le x\}$ and $F = \sup\{y\in \sor \;|\; y \ge 1, y^n \le x\}$.
\end{lem}
\begin{proof}
$x > 1 \Rightarrow x^{1/n} > 1$, by lemma \ref{c4s6l5}. Therefore, $\sup
(E) > 1$. Therefore, we can as well restrict the values of $y$ to be 
greater than $1$.
\end{proof}

\begin{lem}\label{c4s6l7}
If $x > 1$ and $k > l > 0$ then $x^{1/k} < x^{1/l}$.
\end{lem}
\begin{proof}
Using lemma \ref{c4s6l6} we can as well define $x^{1/k} := \sup(F_k)$ and
$x^{1/l} := \sup(F_l)$ where
\begin{eqnarray}
F_k &:=& \{y \in \sor \;|\; y > 1, y^k \le x\} \\
F_l &:=& \{y \in \sor \;|\; y > 1, y^l \le x\} 
\end{eqnarray}
If $k > l$ and $y > 1$ then $y^k > y^l$. Therefore, $\sup(F_k) > 
\sup(F_l)$.
\end{proof}

\begin{lem}\label{c4s6l8}
If $0 < x < 1$ and $k > l > 0$ then $x^{1/k} > x^{1/l}$.
\end{lem}
\begin{proof}
By lemma \ref{c4s6l5} $x < 1 \Rightarrow x^{1/k} < 1, x^{1/l} < 1$. 
Therefore, the suprema of the sets
\begin{eqnarray}
E_k &:=& \{y \in \sor \;|\; y \ge 0, y^k \le x\} \\
E_l &:=& \{y \in \sor \;|\; y \ge 0, y^l \le x\} 
\end{eqnarray}
Since $y < 1$ and $k > l > 0$, $y^k < y^l$. Therefore, $\sup(E_l) >
\sup(E_k)$.
\end{proof}

\begin{lem}\label{c4s6l9}
$(xy)^{1/n} = x^{1/n}y^{1/n}$.
\end{lem}
\begin{proof}
Let
\begin{eqnarray}
E_x &:=& \{u \in \sor \;|\; u \ge 0, u^n \le x\} \\
E_y &:=& \{v \in \sor \;|\; v \ge 0, v^n \le y\} \\
E_{xy} &:=& \{w \in \sor \;|\; w \ge 0, w^n \le xy\}.
\end{eqnarray}
We want to show that $\sup(E_{xy}) = \sup(E_x)\sup(E_y)$. If we define
$w = uv$ for $u, v \ge 0$ then we have $w^n = (uv)^n = u^nv^n$. Further,
$u^n \le x, v^n \le y \Rightarrow u^nv^n \le xy \Rightarrow w^n \le xy$.
Therefore, we can as well define $E_{xy}$ as
\[
E_{xy} = \{uv \;|\; u, v \in \sor u, v \ge 0, (uv)^n = u^nv^n\le xy\}.
\]
Claim: $\sup(E_{xy}) = \sup(E_x)\sup(E_y)$.
Proof: If $\sup(E_{xy}) < \sup(E_x)\sup(E_y)$ then we have $u_1^nv_1^n <
u^nv^n$, where $u_1v_1 \in E_{xy}$. That is $u_1^nv^n < xy$. This means 
that $u_1^nv_1^n$ is not an upper bound of $E_{xy}$, a contradiction.

If $\sup(E_{xy}) > \sup(E_x)\sup(E_y)$ then we have $u_1^nv_1^n > u^nv^n
= xy$. Therefore $u_1^nv_1^n$ is not the lowest upper bound of $E_{xy}$,
a contradiction.
\end{proof}

\begin{lem}\label{c4s6l10}
$(x^{1/n})^{1/m} = x^{1/nm}$.
\end{lem}
\begin{proof}
Let $u = (x^{1/n})^{1/m}$, then by lemmas \ref{c4s6l2}, $u^m = x^{1/n}$. 
Using the same lemma once again, $(u^m)^n = x$. Using the analogue of 
lemma \ref{c3s3l25} for real numbers, we have $u^{nm} = x$. Using lemma
\ref{c4s6l3} we have $u = x^{1/nm}$.
\end{proof}

\begin{defn}\label{c4s6d4}
Let $x > 0$ be a real number and $q = a/b \in \sor$. Then we define $x^q
:= (x^{1/b})^a$.
\end{defn}

\begin{lem}\label{c4s6l11}
Let $q = a/b = c/d$ then for any real $x > 0$, $(x^{1/b})^a = (x^{1/d})^c$.
\end{lem}
\begin{proof}
$(x^{1/b})^a = (x^{1/bd})^{ad} = (x^{1/bcd})^{acd}$. $a/b = c/d \Rightarrow
ad = bc$ so that $(x^{1/b})^a = (x^{1/d})^c$.
\end{proof}

This lemma shows that the exponentiation with rational numbers is a 
well-defined operation. We next prove a few properties of $x^q$.

\begin{lem}\label{c4s6l12}
If $x > 0$ and $q \in \sor$, $x^q > 0$.
\end{lem}
\begin{proof}
Let $q = a/b$. By lemma \ref{c4s6l4}, if $x > 0$ then $x^{1/b} > 0$. By the
analogue of lemma \ref{c3s3l26} for real numbers, we have $x^{1/b} > 0
\Rightarrow (x^{1/b})^a > 0$.
\end{proof}

\begin{lem}\label{c4s6l13}
If $x > 0$, $q, r \in \sor$, $x^{q+r} = x^q x^r$.
\end{lem}
\begin{proof}
Let $q = a/b, r = c/d$ then 
\[
x^{q+r} = (x^{1/bd})^{ad + bc}.
\]
Using the analogue of lemma \ref{c3s3l25} for real numbers,
\[
x^{q+r} = (x^{1/bd})^{ad} (x^{1/bd})^{bc} = x^{a/b} x^{c/d} = x^q x^r.
\]
\end{proof}

\begin{lem}\label{c4s6l14}
If $x > 0$, $q, r \in \sor$ then $(x^q)^r = x^{qr}$.
\end{lem}
\begin{proof}
Let $q = a/b, r = c/d$. Then, 
\[
x^q = x^{a/b} = (x^{1/b})^a = x^{1/b} \cdots x^{1/b}, a \text{ times.}
\]
The last equation follows from the definition of exponentiation with 
integers. Therefore,
\[
(x^q)^r = ((x^q)^{1/d})^c = ((x^{1/b} \cdots x^{1/b})^{1/d})^c.
\]
From lemma \ref{c4s6l9},
\[
(x^q)^r = ((x^{1/b})^{1/d} \cdots (x^{1/b})^{1/d})^c.
\]
From lemma \ref{c4s6l10},
\[
(x^q)^r = (x^{1/bd} \cdots x^{1/bd})^c = (x^{a/bd})^c = x^{ac/bd} = x^{qr}.
\]
In the second equality we used the definition of exponentiation with 
integer, in the third we used the definition of exponentiation with 
rationals and in the last one the definition of product of rationals.
\end{proof}

\begin{lem}\label{c4s6l15}
$x^{-q} = 1/x^q$.
\end{lem}
\begin{proof}
Let $q = a/b$ then 
\[
x^{-q} = x^{-a/b} = (x^{1/b})^{-a} = \frac{1}{(x^{1/b})^a} = \frac{1}{x^q}.
\]
\end{proof}

\begin{lem}\label{c4s6l16}
If $q > 0$ then $x > y$ iff $x^q > y^q$.
\end{lem}
\begin{proof}
Let $q = a/b$ then using lemma \ref{c4s6l5}, $x > y \Leftrightarrow
x^{1/b} > y^{1/b}$. Since $q > 0$, we have $a > 0$. Therefore, by the 
analogue of lemma \ref{c3s3l26} we get $x^{1/b} > y^{1/b} \Leftrightarrow
(x^{1/b})^a > (y^{1/b})^a \Leftrightarrow x^{a/b} > y^{a/b}$.
\end{proof}

\begin{lem}\label{c4s6l17}
If $x > 1$ then $x^q > x^r$ iff $q > r$. If $x < 1$ then $x^q > x^r$ iff 
$q < r$.
\end{lem}
\begin{proof}
Let $q = a/b, r = c/d$. $q > r \Rightarrow ad > bc$. If $x > 1$ then
$x^{ad} > x^{bc}$ iff $ad > bc$ iff $q > r$. Using lemma \ref{c4s6l5},
taking the $d$-th root, we have $x^a > x^{bc/d} \Leftrightarrow q > r$.
Taking the $b$-th root, we have $x^{a/b} > x^{c/d} \Leftrightarrow q > r$.

If $x < 1$ then $x^{ad} > x^{bc}$ iff $ad < bc$. The rest of the proof is
similar.
\end{proof}

\begin{lem}\label{c4s6l18}
$(xy)^q = x^q y^q$.
\end{lem}
\begin{proof}
Let $q = a/b$. Then $(xy)^q = ((xy)^{1/b})^q$. Using lemma \ref{c4s6l9}, we
have $(xy)^q = (x^{1/b} y^{1/b})^a = (x^{1/b})^a (y^{1/b})^a = 
x^{a/b}y^{a/b} = x^qy^q$.
\end{proof}

\subsection{Exercises}
\begin{enumerate}
\item[1:] If $x \in \sor$ and $n$ is an even natural number then show that
$x^n \ge 0$.
\item[Solution:] $n$ is even allows us to write $n = 2m$. Therefore, $x^n
= x^{2m} = (x^2)^m$. For all $x \in \sor$, $x^2 \ge 0$. Therefore, $x^n
= (x^2)^m \ge 0$.

\item[2:] If $x \in \sor$ then $|x| = (x^2)^{1/2}$.
\item[Solution:] The $n$-th root is always defined to be positive. 
Therefore, 
\[
(x^2)^{1/2} = \begin{cases}
x & \text{ if } x \ge 0 \\
-x & \text{ if } x < 0
\end{cases}
\]
This is the same as the definition of $|x|$.

\item[3:] Let $x, y$ by positive reals and $q \ge 1$ be a rational. Then
show that $\max(x^q, y^q) = \max(x, y)^q$ and $\min(x^q, y^q) = 
\min(x, y)^q$. What happens if $q < 1$?
\item[Solution:] If $q \ge 1$ then $x > y \Rightarrow x^q > y^q$. Thus
$max(x, y) = x$ and $\max(x^q, y^q) = x^q$. Also $\min(x, y) = y$ and $\min(
x^q, y^q) = y^q$. The case $x < y$ is similarly analysed.

If $q < 1$ then the situation is the same because lemma \ref{c4s6l16} is
applicable to $q \ge 0$.
\end{enumerate}
