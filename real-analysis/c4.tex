\chapter{The real numbers}\label{c4}
There are several ways to construct the reals from the rationals. We will
follow the one that uses Cauchy sequences. It is profitable because it 
introduces a familiar motif in analysis, that of `completion' of a space
(that is, a set with certain properties).

\section{Cauchy sequences}\label{c4s1}
\begin{defn}\label{c4s1d1}
A sequence of rational numbers is a mapping $\son \rightarrow \soq$. It is
customary to denote it be $\seq{a_n}$ its range rather than the functional
notation $a(n)$.
\end{defn}

\begin{rem}
We will always mean an infinite sequence when we mention `sequence'. If we
want to describe the set $\{a_0, \ldots, a_n\}$ of rational numbers we will
call it a \emph{finite} sequence.
\end{rem}

\begin{defn}\label{c4s1d2}
A sequence $\seq{a_n}$ is called a Cauchy sequence if for any rational 
$\epsilon > 0$ there exists $N \in \son$ such that for all $m, n > N$,
$d(a_m, a_n) < \epsilon$.
\end{defn}

\begin{rem}
We could have written the Cauchy condition as $|a_m - a_n| < \epsilon$. 
However, we chose to write it in terms of a distance function $d$ because
it makes it easy for us to take this definition to metric spaces.
\end{rem}

\begin{defn}\label{c4s1d3}
A sequence $\seq{a_n}$ of rationals is bounded if we can find a rational 
number $M \ge 0$ such that $|a_n| \le M$ for all $n \in \son$. The number
$M$ is called a bound of the sequence.
\end{defn}

\begin{lem}\label{c4s1l1}
A finite sequence is always bounded.
\end{lem}
\begin{proof}
Consider the finite sequence $\{a_0, \ldots, a_n\}$. Then we will show 
that it is bounded for any $n \in \son$. We will use induction on $\son$.
The base case is the sequence $\{a_0\}$. It is clearly bounded if we choose
$M = |a_0|$. Let us assume that the claim is true for some $n$ and consider
the sequence $\{a_0, \ldots, a_n, a_{n+1}\}$. By induction hypothesis, 
there exists $M \ge 0$ such that $|a_i| \le M$ for all $0 \le i \le n$. 
Then the constant $M^\op = M + |a_{n+1}|$ is such that $|a_i| \le M^\op$
for all $0 \le i \le n + 1$, thus closing the induction.
\end{proof}

\begin{lem}\label{c4s1l2}
A Cauchy sequence is bounded.
\end{lem}
\begin{proof}
Let $\seq{a_n}$ be a Cauchy sequence. Fix an $\epsilon > 0$. Then we can 
find a natural number $N$ such that for all $m, n > N$, $d(a_m, a_n) < 
\epsilon$. By lemma \ref{c4s1l1}, the finite sequence $\{a_0, \ldots,
a_N\}$ is bounded. Let $M_0$ be its bound. For all $n \ge N + 1$, $d(a_n,
a_{N+1}) < \epsilon \Rightarrow |a_n - a_{N+1}| < \epsilon \Rightarrow
a_{N+1} - \epsilon < a_n < a_{N+1} + \epsilon$. Thus, the rest of the
terms are bounded by $a_{N+1} + \epsilon$. If we let $M = M_0 + |a_{N+1}|
+ \epsilon$ then $|a_n| < M$ for all $n \in \son$.
\end{proof}

\subsection{Properties of Cauchy sequences}
\begin{lem}\label{c4s1l3}
If $\seq{a_n}$ and $\seq{b_n}$ are Cauchy sequences then so is $\seq{a_n
+ b_n}$.
\end{lem}
\begin{proof}
Fix an $\epsilon > 0$. Then there exist natural numbers $N_1, N_2$ such
that for all $m, n > N_1$, $d(a_m, a_n) < \epsilon/2$ and for all $m, n >
N_2$, $d(b_m, b_n) < \epsilon/2$. Let $N = \max(N_1, N_2)$. For $n, m > N$,
consider 
\begin{eqnarray*}
d(a_m + b_m, a_n + b_n) &=& |a_m + b_m - a_n - b_n| \\
 &\le& |a_m - a_n| + |b_m - b_n| \\
 &<& \epsilon.
\end{eqnarray*}
\end{proof}

\begin{lem}\label{c4s1l4}
If $\seq{a_n}$ and $\seq{b_n}$ are Cauchy sequences then so is $\seq{a_n
b_n}$.
\end{lem}
\begin{proof}
By lemma \ref{c4s1l2} $\seq{a_n}$ and $\seq{b_n}$ are bounded. Let $M_a$
and $M_b$ be their bounds. Fix an $\epsilon > 0$. Let $N_a \in \son$ be
such that for all $n, m > N_a$, $d(a_m, a_n) < \epsilon/(2M_b)$. Let $N_b 
\in \son$ be such that for all $n, m > N_b$, $d(b_m, b_n) < \epsilon/
(2M_a)$.  Let $N = \max(N_a, N_b)$. Then, for $n, m > N$, consider
\begin{eqnarray*}
d(a_nb_n, a_mb_m) &=& |a_nb_n - a_mb_m| \\
 &=& |a_nb_n - b_na_m + b_na_m - a_mb_m| \\
 &\le& |b_n||a_n - a_m| + |a_m||b_n - b_m| \\
 &<& M_b\frac{\epsilon}{2M_b} + M_a\frac{\epsilon}{2M_a} \\
 &<& \epsilon.
\end{eqnarray*}
\end{proof}

\begin{lem}\label{c4s1l5}
If $\seq{a_n}$ is a Cauchy sequence then so is $\seq{ka_n}$ for a constant
$k \in \soq$. In particular, $\seq{-a_n}$ is a Cauchy sequence.
\end{lem}
\begin{proof}
Fix an $\epsilon > 0$ and let $N \in \son$ be such that for all $n, m > N$,
$d(a_m, a_n) < \epsilon/|k|$. If $b_n = ka_n$ then we clearly have $d(b_n,
b_m) < \epsilon$ for all $m, n > N$.
\end{proof}

\section{Equivalent Cauchy sequences}\label{c4s2}
\begin{defn}\label{c4s2d1}
Sequences $\seq{a_n}$ and $\seq{b_n}$ are equivalent if for a given
rational $\epsilon > 0$ we can find a natural number $N$ such that for
all $n > N$, $d(a_n, b_n) < \epsilon$.
\end{defn}

\begin{lem}\label{c4s2l2}
Let $\seq{a_n}$ and $\seq{b_n}$ be equivalent sequences. Then $\seq{a_n}$ 
is a Cauchy sequence iff $\seq{b_n}$ is a Cauchy sequence.
\end{lem}
\begin{proof}
Fix an $\epsilon > 0$. Let $\seq{a_n}$ be a Cauchy sequence. Then we can
find $N_1 \in \son$ such that for all $m, n > N_1$, $d(a_m, a_n) < 
\epsilon/3$. Since $\seq{a_n}$ is equivalent to $\seq{b_n}$ we can find $N_2
\in \son$ such that for all $n > N_2$, $d(a_n, b_n) < \epsilon/3$. Choose
$N = \max(N_1, N_2)$ and let $m, n > N$. Then
\begin{eqnarray*}
d(b_m, b_n) &=& |b_m - b_n| \\
 &=& |b_m - a_m + a_m + a_n - a_n - b_n| \\
 &\le& |b_m - a_m| + |a_m - a_n| + |a_n - b_n| \\
 &<& \epsilon.
\end{eqnarray*}
is Cauchy implies that so is $\seq{b_n}$.

Now let $\seq{b_n}$ be a Cauchy sequence and $\seq{a_n}$ be equivalent to 
it.  Fix an $\epsilon$ and let $N_1 \in \son$ be such that for all $m, n > 
N_1$ $d(b_m, b_n) < \epsilon/3$ and let $N_2 \in \son$ be such that for 
all $n > N_2$, $d(a_n, b_n) < \epsilon/3$. Choose $N = \max(N_1, N_2)$ 
and let $m, n > N$. Then,
\begin{eqnarray*}
d(a_m, a_n) &=& |a_m - a_n| \\
 &=& |a_m - b_m + b_n - a_n + b_m - b_n| \\
 &\le& |a_m - b_m| + |a_n - b_n| + |b_m - b_n| \\
 &<& \epsilon.
\end{eqnarray*}
\end{proof}

Note that two sequences can be equivalent without being bounded or Cauchy.
The set of rationals is dense. Therefore two sequences can arbitrarily 
close to each other, term by term, without being identical. However, for
all practical purposes they are the same. Equivalence means that for a
given $\epsilon > 0$ we can find a natural number $N$ such that all terms
of $\seq{b_n}$ are within a distance $\epsilon$ from the corresponding 
terms of $\seq{a_n}$.

Let us denote the equivalance of Cauchy sequences $\seq{a_n}$ and 
$\seq{b_n}$ as $\seq{a_n} \sim \seq{b_n}$. We next show that
\begin{prop}\label{c4s2p1}
The relation $\sim$ between Cauchy sequences is an equivalence relation.
\end{prop}
\begin{proof}
For any $\epsilon > 0$, $d(a_n, a_n) < \epsilon$ for all $n \in \son$.
Therefore, $\seq{a_n} \sim \seq{a_n}$. Since $d(a_n, b_n) = d(b_n, a_n)$, 
$\seq{a_n} \sim \seq{b_n} \Rightarrow \seq{b_n} \sim \seq{a_n}$.

Let $\seq{a_n} \sim \seq{b_n}$ and $\seq{b_n} \sim \seq{c_n}$. Then for a
given $\epsilon > 0$, there exists $N_1, N_2 \in \son$ such that for all
$n > N_1$, $d(a_n, b_n) < \epsilon/2$ and for all $n > N_2$, $d(b_n, c_n)
< \epsilon/2$. Let $N = \max(N_1, N_2)$ and let $n > N$. Then,
\begin{eqnarray*}
d(a_n, c_n) &=& |a_n - c_n| \\
 &=& |a_n - b_n + b_n - c_n| \\
 &\le& |a_n - b_n| + |b_n - c_n| \\
 &<& \epsilon.
\end{eqnarray*}
Thus, $\seq{a_n} \sim \seq{c_n}$.
\end{proof}

\section{The construction of the real numbers}\label{c4s3}
\begin{defn}\label{c4s3d1}
The equivalence class of a Cauchy sequence of rational numbers is called 
a real number. We will denote the equivalence class of a Cauchy sequence
$\seq{a_n}$ by $\overline{\seq{a_n}}$.
\end{defn}

\begin{rem}
If $q \in \soq$ then the sequence $\seq{a_n}$ with $a_n = q$ for all $n$
is a Cauchy sequence and it represents the real number $q$. We can thus
identify rational numbers with real numbers of same `value'.
\end{rem}

\begin{defn}\label{c4s3d2}
The equivalence class of Cauchy sequences containing the sequence 
$\seq{a_n}$ with the property $|a_n| < \epsilon$ for all $n > N$, where
$N$ may depend on $\epsilon$ is called the zero real number. Sequences
in the corresponding equivalence class are called zero sequences.
\end{defn}

\begin{defn}\label{c4s3d3}
If $x = \overline{\seq{a_n}}$ and $y = \overline{\seq{b_n}}$ are two
real number then their sum is defined as $x + y := \overline{\seq{a_n + 
b_n}}$.
\end{defn}

\begin{lem}\label{c4s3l1}
Addition of real numbers is a well-defined operation.
\end{lem}

\begin{defn}\label{c4s3d4}
If $x = \overline{\seq{a_n}}$ and $y = \overline{\seq{b_n}}$ are two
real number then their product is defined as $xy := 
\overline{\seq{a_nb_n}}$.
\end{defn}

\begin{lem}\label{c4s3l2}
Multiplication of real numbers is a well-defined operation.
\end{lem}

\begin{defn}\label{c4s3d5}
If $x = \overline{\seq{a_n}}$ is a real number then its negation is 
defined as $-x := \overline{\seq{-a_n}}$.
\end{defn}

\begin{lem}\label{c4s3l3}
Negation of real numbers is a well-defined operation.
\end{lem}

\begin{defn}\label{c4s3d6}
A sequence $\seq{a_n}$ of rational numbers is said to be bounded away 
from zero iff there exists a rational number $c > 0$ such that $|a_n| > c$
for all $n \in \son$.
\end{defn}

\begin{lem}\label{c4s3l4}
For all $r, s \in \soq$, $|r + s| \ge |r| - |s|$.
\end{lem}
\begin{proof}
$|r| = |r + s - s| \le |r + s| + |-s| \Rightarrow |r| - |s| \le |r + s$.
\end{proof}

\begin{lem}\label{c4s3l5}
Let $x$ be a non-zero real number. Then its equivalence class contains 
a sequence that is bounded away from zero.
\end{lem}
\begin{proof}
Let $\seq{a_n}$ belong to the equivalence class of $x$ which may not be
bounded away from zero. That is, some of its terms may be zero although
all terms beyond a certain $n$ will be non-zero. 

Since $\seq{a_n}$ is a Cauchy sequence, for a given $\epsilon > 0$, we
can find $N_0 \in \son$ such that for all $n, m > N_0$, $d(a_m, a_n) <
\epsilon/2$.

For all $n > N_0$, it will be \emph{not} be the case that $|a_n| < 
\epsilon$. If that were the case then $\seq{a_n}$ will be a zero sequence.
Therefore, there is at least one $N > N_0$ such that $|a_N| > \epsilon$.

Now, $|a_m| = |a_m - a_N + a_N| \ge |a_N| - |a_m - a_N| > \epsilon/2$,
where we used lemma \ref{c4s3l4}. Here $m > N_0$. Now define a sequence
$\seq{b_n}$ as
\[
b_n = \begin{cases}
 a_{N_0} & \;\forall\; 0 \le n \le N_0 \\
 a_n     & \;\forall\; n > N_0.
\end{cases}
\]
Clearly $\seq{b_n} \sim \seq{a_n}$ but $\seq{b_n}$ is bounded away from 
zero.
\end{proof}

\begin{lem}\label{c4s3l6}
If $\seq{a_n}$ is a Cauchy sequence of rationals bounded away from zero 
then $\seq{a_n^{-1}}$ is also a Cauchy sequence.
\end{lem}
\begin{proof}
Since $\seq{a_n}$ is bounded away from zero there exists a rational number
$c > 0$ such that $|a_n| > c$ for all $n \in \son$. Fix an $\epsilon > 0$.
Choose an $N \in \son$ such that for all $m, n > N$, $d(a_m, a_n) < c^2
\epsilon$. Now consider
\begin{eqnarray*}
d(a_n^{-1}, a_m^{-1}) &=& \left|\frac{1}{a_n} - \frac{1}{a_m}\right| \\
 &=& \frac{|a_n - a_m|}{|a_na_m|} \\
 &<& \frac{c^2\epsilon}{c^2} \\
 &<& \epsilon.
\end{eqnarray*}
\end{proof}

\begin{defn}\label{c4s3d7}
Let $x$ be a non-zero real number. Then its equivalence class will have
at least one sequence $\seq{a_n}$ that is bounded away from zero. The
reciprocal of $x$ is defined as $x^{-1} := \overline{\seq{a_n^{-1}}}$.
\end{defn}

\begin{lem}\label{c4s3l7}
Reciprocation of real numbers is a well-defined operation.
\end{lem}
