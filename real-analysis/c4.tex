\chapter{The real numbers}\label{c4}
There are several ways to construct the reals from the rationals. We will
follow the one that uses Cauchy sequences. It is profitable because it 
introduces a familiar motif in analysis, that of `completion' of a space
(that is, a set with certain properties).

\section{Cauchy sequences}\label{c4s1}
\begin{defn}\label{c4s1d1}
A sequence of rational numbers is a mapping $\son \rightarrow \soq$. It is
customary to denote it be $\seq{a_n}$ its range rather than the functional
notation $a(n)$.
\end{defn}

\begin{rem}
We will always mean an infinite sequence when we mention `sequence'. If we
want to describe the set $\{a_0, \ldots, a_n\}$ of rational numbers we will
call it a \emph{finite} sequence.
\end{rem}

\begin{defn}\label{c4s1d2}
A sequence $\seq{a_n}$ is called a Cauchy sequence if for any rational 
$\epsilon > 0$ there exists $N \in \son$ such that for all $m, n > N$,
$d(a_m, a_n) < \epsilon$.
\end{defn}

\begin{rem}
We could have written the Cauchy condition as $|a_m - a_n| < \epsilon$. 
However, we chose to write it in terms of a distance function $d$ because
it makes it easy for us to take this definition to metric spaces.
\end{rem}

\begin{defn}\label{c4s1d3}
A sequence $\seq{a_n}$ of rationals is bounded if we can find a rational 
number $M \ge 0$ such that $|a_n| \le M$ for all $n \in \son$. The number
$M$ is called a bound of the sequence.
\end{defn}

\begin{lem}\label{c4s1l1}
A finite sequence is always bounded.
\end{lem}
\begin{proof}
Consider the finite sequence $\{a_0, \ldots, a_n\}$. Then we will show 
that it is bounded for any $n \in \son$. We will use induction on $\son$.
The base case is the sequence $\{a_0\}$. It is clearly bounded if we choose
$M = |a_0|$. Let us assume that the claim is true for some $n$ and consider
the sequence $\{a_0, \ldots, a_n, a_{n+1}\}$. By induction hypothesis, 
there exists $M \ge 0$ such that $|a_i| \le M$ for all $0 \le i \le n$. 
Then the constant $M^\op = M + |a_{n+1}|$ is such that $|a_i| \le M^\op$
for all $0 \le i \le n + 1$, thus closing the induction.
\end{proof}

\begin{lem}\label{c4s1l2}
A Cauchy sequence is bounded.
\end{lem}
\begin{proof}
Let $\seq{a_n}$ be a Cauchy sequence. Fix an $\epsilon > 0$. Then we can 
find a natural number $N$ such that for all $m, n > N$, $d(a_m, a_n) < 
\epsilon$. By lemma \ref{c4s1l1}, the finite sequence $\{a_0, \ldots,
a_N\}$ is bounded. Let $M_0$ be its bound. For all $n \ge N + 1$, $d(a_n,
a_{N+1}) < \epsilon \Rightarrow |a_n - a_{N+1}| < \epsilon \Rightarrow
a_{N+1} - \epsilon < a_n < a_{N+1} + \epsilon$. Thus, the rest of the
terms are bounded by $a_{N+1} + \epsilon$. If we let $M = M_0 + |a_{N+1}|
+ \epsilon$ then $|a_n| < M$ for all $n \in \son$.
\end{proof}

\subsection{Properties of Cauchy sequences}
\begin{lem}\label{c4s1l3}
If $\seq{a_n}$ and $\seq{b_n}$ are Cauchy sequences then so is $\seq{a_n
+ b_n}$.
\end{lem}
\begin{proof}
Fix an $\epsilon > 0$. Then there exist natural numbers $N_1, N_2$ such
that for all $m, n > N_1$, $d(a_m, a_n) < \epsilon/2$ and for all $m, n >
N_2$, $d(b_m, b_n) < \epsilon/2$. Let $N = \max(N_1, N_2)$. For $n, m > N$,
consider 
\begin{eqnarray*}
d(a_m + b_m, a_n + b_n) &=& |a_m + b_m - a_n - b_n| \\
 &\le& |a_m - a_n| + |b_m - b_n| \\
 &<& \epsilon.
\end{eqnarray*}
\end{proof}

\begin{lem}\label{c4s1l4}
If $\seq{a_n}$ and $\seq{b_n}$ are Cauchy sequences then so is $\seq{a_n
b_n}$.
\end{lem}
\begin{proof}
By lemma \ref{c4s1l2} $\seq{a_n}$ and $\seq{b_n}$ are bounded. Let $M_a$
and $M_b$ be their bounds. Fix an $\epsilon > 0$. Let $N_a \in \son$ be
such that for all $n, m > N_a$, $d(a_m, a_n) < \epsilon/(2M_b)$. Let $N_b 
\in \son$ be such that for all $n, m > N_b$, $d(b_m, b_n) < \epsilon/
(2M_a)$.  Let $N = \max(N_a, N_b)$. Then, for $n, m > N$, consider
\begin{eqnarray*}
d(a_nb_n, a_mb_m) &=& |a_nb_n - a_mb_m| \\
 &=& |a_nb_n - b_na_m + b_na_m - a_mb_m| \\
 &\le& |b_n||a_n - a_m| + |a_m||b_n - b_m| \\
 &<& M_b\frac{\epsilon}{2M_b} + M_a\frac{\epsilon}{2M_a} \\
 &<& \epsilon.
\end{eqnarray*}
\end{proof}

\begin{lem}\label{c4s1l5}
If $\seq{a_n}$ is a Cauchy sequence then so is $\seq{ka_n}$ for a constant
$k \in \soq$. In particular, $\seq{-a_n}$ is a Cauchy sequence.
\end{lem}
\begin{proof}
Fix an $\epsilon > 0$ and let $N \in \son$ be such that for all $n, m > N$,
$d(a_m, a_n) < \epsilon/|k|$. If $b_n = ka_n$ then we clearly have $d(b_n,
b_m) < \epsilon$ for all $m, n > N$.
\end{proof}

\section{Equivalent Cauchy sequences}\label{c4s2}
\begin{defn}\label{c4s2d1}
Sequences $\seq{a_n}$ and $\seq{b_n}$ are equivalent if for a given
rational $\epsilon > 0$ we can find a natural number $N$ such that for
all $n > N$, $d(a_n, b_n) < \epsilon$.
\end{defn}

\begin{lem}\label{c4s2l2}
Let $\seq{a_n}$ and $\seq{b_n}$ be equivalent sequences. Then $\seq{a_n}$ 
is a Cauchy sequence iff $\seq{b_n}$ is a Cauchy sequence.
\end{lem}
\begin{proof}
Fix an $\epsilon > 0$. Let $\seq{a_n}$ be a Cauchy sequence. Then we can
find $N_1 \in \son$ such that for all $m, n > N_1$, $d(a_m, a_n) < 
\epsilon/3$. Since $\seq{a_n}$ is equivalent to $\seq{b_n}$ we can find $N_2
\in \son$ such that for all $n > N_2$, $d(a_n, b_n) < \epsilon/3$. Choose
$N = \max(N_1, N_2)$ and let $m, n > N$. Then
\begin{eqnarray*}
d(b_m, b_n) &=& |b_m - b_n| \\
 &=& |b_m - a_m + a_m + a_n - a_n - b_n| \\
 &\le& |b_m - a_m| + |a_m - a_n| + |a_n - b_n| \\
 &<& \epsilon.
\end{eqnarray*}
is Cauchy implies that so is $\seq{b_n}$.

Now let $\seq{b_n}$ be a Cauchy sequence and $\seq{a_n}$ be equivalent to 
it.  Fix an $\epsilon$ and let $N_1 \in \son$ be such that for all $m, n > 
N_1$ $d(b_m, b_n) < \epsilon/3$ and let $N_2 \in \son$ be such that for 
all $n > N_2$, $d(a_n, b_n) < \epsilon/3$. Choose $N = \max(N_1, N_2)$ 
and let $m, n > N$. Then,
\begin{eqnarray*}
d(a_m, a_n) &=& |a_m - a_n| \\
 &=& |a_m - b_m + b_n - a_n + b_m - b_n| \\
 &\le& |a_m - b_m| + |a_n - b_n| + |b_m - b_n| \\
 &<& \epsilon.
\end{eqnarray*}
\end{proof}

Note that two sequences can be equivalent without being bounded or Cauchy.
The set of rationals is dense. Therefore two sequences can arbitrarily 
close to each other, term by term, without being identical. However, for
all practical purposes they are the same. Equivalence means that for a
given $\epsilon > 0$ we can find a natural number $N$ such that all terms
of $\seq{b_n}$ are within a distance $\epsilon$ from the corresponding 
terms of $\seq{a_n}$.

Let us denote the equivalance of Cauchy sequences $\seq{a_n}$ and 
$\seq{b_n}$ as $\seq{a_n} \sim \seq{b_n}$. We next show that
\begin{prop}\label{c4s2p1}
The relation $\sim$ between Cauchy sequences is an equivalence relation.
\end{prop}
\begin{proof}
For any $\epsilon > 0$, $d(a_n, a_n) < \epsilon$ for all $n \in \son$.
Therefore, $\seq{a_n} \sim \seq{a_n}$. Since $d(a_n, b_n) = d(b_n, a_n)$, 
$\seq{a_n} \sim \seq{b_n} \Rightarrow \seq{b_n} \sim \seq{a_n}$.

Let $\seq{a_n} \sim \seq{b_n}$ and $\seq{b_n} \sim \seq{c_n}$. Then for a
given $\epsilon > 0$, there exists $N_1, N_2 \in \son$ such that for all
$n > N_1$, $d(a_n, b_n) < \epsilon/2$ and for all $n > N_2$, $d(b_n, c_n)
< \epsilon/2$. Let $N = \max(N_1, N_2)$ and let $n > N$. Then,
\begin{eqnarray*}
d(a_n, c_n) &=& |a_n - c_n| \\
 &=& |a_n - b_n + b_n - c_n| \\
 &\le& |a_n - b_n| + |b_n - c_n| \\
 &<& \epsilon.
\end{eqnarray*}
Thus, $\seq{a_n} \sim \seq{c_n}$.
\end{proof}

\section{The construction of the real numbers}\label{c4s3}
\begin{defn}\label{c4s3d1}
The equivalence class of a Cauchy sequence of rational numbers is called 
a real number. We will denote the equivalence class of a Cauchy sequence
$\seq{a_n}$ by $\overline{\seq{a_n}}$. The set of all real number is 
denoted by $\sor$.
\end{defn}

\begin{rem}
If $q \in \soq$ then the sequence $\seq{a_n}$ with $a_n = q$ for all $n$
is a Cauchy sequence and it represents the real number $q$. We can thus
identify rational numbers with real numbers of same `value'.
\end{rem}

\begin{defn}\label{c4s3d2}
The equivalence class of Cauchy sequences containing the sequence 
$\seq{a_n}$ with the property $|a_n| < \epsilon$ for all $n > N$, where
$N$ may depend on $\epsilon$ is called the zero real number. Sequences
in the corresponding equivalence class are called zero sequences.
\end{defn}

\begin{defn}\label{c4s3d3}
If $x = \overline{\seq{a_n}}$ and $y = \overline{\seq{b_n}}$ are two
real number then their sum is defined as $x + y := \overline{\seq{a_n + 
b_n}}$.
\end{defn}

\begin{lem}\label{c4s3l1}
Addition of real numbers is a well-defined operation.
\end{lem}
\begin{proof}
Let $\seq{a_n} \sim \seq{b_n}$. Then we will show that $\seq{a_n + c_n}
\sim \seq{b_n + c_n}$. Fix an $\epsilon > 0$. Then we can find $N \in 
\son$ such that for all $n > N$. $d(a_n, b_n) < \epsilon/2$. Consider
$d(a_n + c_n - b_n - c_n) = d(a_n - b_n) < \epsilon$ for all $n > N$.
\end{proof}

\begin{defn}\label{c4s3d4}
If $x = \overline{\seq{a_n}}$ and $y = \overline{\seq{b_n}}$ are two
real number then their product is defined as $xy := 
\overline{\seq{a_nb_n}}$.
\end{defn}

\begin{lem}\label{c4s3l2}
Multiplication of real numbers is a well-defined operation.
\end{lem}
\begin{proof}
Let $\seq{a_n} \sim \seq{b_n}$. Since $\seq{c_n}$ is a Cauchy sequence, it
is bounded. Let $M$ be its bound. Fix an $\epsilon > 0$. We can find $N 
\in \son$ such that for all $n > N$, $d(a_n - b_n) < \epsilon/M$. Consider 
$d(c_na_n - c_nb_n) = |c_na_n - c_nb_n| = |c_n||a_n - b_n| < \epsilon$.
\end{proof}

\begin{defn}\label{c4s3d5}
If $x = \overline{\seq{a_n}}$ is a real number then its negation is 
defined as $-x := \overline{\seq{-a_n}}$.
\end{defn}

\begin{lem}\label{c4s3l3}
Negation of real numbers is a well-defined operation.
\end{lem}
\begin{proof}
Let $\seq{a_n} \sim \seq{b_n}$. Then $d(a_n, b_n) = d(-a_n, -b_n)$. 
Therefore, $\seq{-a_n} \sim \seq{-b_n}$.
\end{proof}

\begin{defn}\label{c4s3d6}
A sequence $\seq{a_n}$ of rational numbers is said to be bounded away 
from zero iff there exists a rational number $c > 0$ such that $|a_n| > c$
for all $n \in \son$.
\end{defn}

\begin{lem}\label{c4s3l4}
For all $r, s \in \soq$, $|r + s| \ge |r| - |s|$.
\end{lem}
\begin{proof}
$|r| = |r + s - s| \le |r + s| + |-s| \Rightarrow |r| - |s| \le |r + s$.
\end{proof}

\begin{lem}\label{c4s3l5}
Let $x$ be a non-zero real number. Then its equivalence class contains 
a sequence that is bounded away from zero.
\end{lem}
\begin{proof}
Let $\seq{a_n}$ belong to the equivalence class of $x$ which may not be
bounded away from zero. That is, some of its terms may be zero although
all terms beyond a certain $n$ will be non-zero. 

Since $\seq{a_n}$ is a Cauchy sequence, for a given $\epsilon > 0$, we
can find $N_0 \in \son$ such that for all $n, m > N_0$, $d(a_m, a_n) <
\epsilon/2$.

For all $n > N_0$, it will be \emph{not} be the case that $|a_n| < 
\epsilon$. If that were the case then $\seq{a_n}$ will be a zero sequence.
Therefore, there is at least one $N > N_0$ such that $|a_N| > \epsilon$.

Now, $|a_m| = |a_m - a_N + a_N| \ge |a_N| - |a_m - a_N| > \epsilon/2$,
where we used lemma \ref{c4s3l4}. Here $m > N_0$. Now define a sequence
$\seq{b_n}$ as
\[
b_n = \begin{cases}
 a_{N_0} & \;\forall\; 0 \le n \le N_0 \\
 a_n     & \;\forall\; n > N_0.
\end{cases}
\]
Clearly $\seq{b_n} \sim \seq{a_n}$ but $\seq{b_n}$ is bounded away from 
zero.
\end{proof}

\begin{lem}\label{c4s3l6}
If $\seq{a_n}$ is a Cauchy sequence of rationals bounded away from zero 
then $\seq{a_n^{-1}}$ is also a Cauchy sequence.
\end{lem}
\begin{proof}
Since $\seq{a_n}$ is bounded away from zero there exists a rational number
$c > 0$ such that $|a_n| > c$ for all $n \in \son$. Fix an $\epsilon > 0$.
Choose an $N \in \son$ such that for all $m, n > N$, $d(a_m, a_n) < c^2
\epsilon$. Now consider
\begin{eqnarray*}
d(a_n^{-1}, a_m^{-1}) &=& \left|\frac{1}{a_n} - \frac{1}{a_m}\right| \\
 &=& \frac{|a_n - a_m|}{|a_na_m|} \\
 &<& \frac{c^2\epsilon}{c^2} \\
 &<& \epsilon.
\end{eqnarray*}
\end{proof}

\begin{defn}\label{c4s3d7}
Let $x$ be a non-zero real number. Then its equivalence class will have
at least one sequence $\seq{a_n}$ that is bounded away from zero. The
reciprocal of $x$ is defined as $x^{-1} := \overline{\seq{a_n^{-1}}}$.
\end{defn}

\begin{lem}\label{c4s3l7}
Reciprocation of real numbers is a well-defined operation.
\end{lem}
\begin{proof}
Let $\seq{a_n} \sim \seq{b_n}$. If their reciprocal exists then they
are bounded away from zero. Let $c_1, c_2 > 0$ be such that $|a_n| > c_1$
and $|b_n| > c_2$ for all $n \in \son$. Let $c = \min(c_1, c_2)$. Then
we have 
\[
\frac{1}{|a_n|} < c; \frac{1}{|b_n|} < c
\]
so that
\[
\frac{1}{|a_nb_n|} < c^2.
\]
Fix an $\epsilon > 0$. Then there exists $N \in \son$ such that $d(a_n,
b_n) < \epsilon c^2$ for all $n > N$. Now consider,
\[
\left|\frac{1}{a_n} - \frac{1}{b_n}\right| = \left|\frac{b_n - a_n}{a_nb_n}
\right| = \frac{|a_n - b_n|}{a_nb_n} < \frac{c^2\epsilon}{c^2} = 
\epsilon.
\]
\end{proof}

From lemma \ref{c4s3l1} and \ref{c4s3l3} we infer
\begin{lem}\label{c4s3l8}
Subraction is a well-defined operation.
\end{lem}

Similarly, from \ref{c4s3l2} and \ref{c4s3l7} we infer
\begin{lem}\label{c4s3l9}
Division is a well-defined operation.
\end{lem}

We will now prove the field theoretic properties of real numbers.
\begin{lem}\label{c4s3l10}
Addition is commutative.
\end{lem}
\begin{proof}
Let $x = \overline{\seq{a_n}}$ and $y = \overline{\seq{b_n}}$ then $x + y
= \overline{\seq{a_n}} + \overline{\seq{b_n}} = \overline{\seq{a_n+b_n}}
= \overline{\seq{b_n + a_n}} = \overline{\seq{b_n}} + \overline{\seq{a_n}}
= y + x$.
\end{proof}

\begin{lem}\label{c4s3l11}
Addition is associative.
\end{lem}
\begin{proof}
Very similar to above and depends on the associativity of rationals.
\end{proof}

\begin{lem}\label{c4s3l12}
For all real numbers $x$, $x + 0 = x$.
\end{lem}
\begin{proof}
By definition \ref{c4s3d2}, the real number $0$ is the equivalence class of
a sequence $\seq{b_n}$ with the property $|b_n| < \epsilon$ got all $n
> N$. Let $x$ be the equivalence class of a sequence $\seq{a_n}$. Then
$x + 0$ is the equivalence class of $\seq{c_n} = \seq{a_n} + \seq{b_n}$. 
For $n > N$, $|c_n - a_n| = |b_n| < \epsilon$ so that $\seq{c_n} \sim
\seq{a_n}$ and hence $x + 0 = x$.
\end{proof}

\begin{lem}\label{c4s3l13}
For all real numbers $x$, $x + (-x) = 0$.
\end{lem}
\begin{proof}
Follows immediately from the definition of negation and the fact that for
any rational $a_n$, $a_n + (-a_n) = 0$.
\end{proof}

\begin{lem}\label{c4s3l14}
Multiplication is commutative.
\end{lem}
\begin{proof}
Similar to proof of lemma \ref{c4s3l10}.
\end{proof}

\begin{lem}\label{c4s3l15}
Multiplication is associative.
\end{lem}
\begin{proof}
Similar to proof of lemma \ref{c4s3l11}.
\end{proof}

\begin{lem}\label{c4s3l16}
For all real numbers $x$, $x \times 1 = x$.
\end{lem}
\begin{proof}
The real number $1$ is the equivalence class $\overline{\seq{b_n}}$
where $b_n = 1$ for $n \in \son$. If $x = \overline{\seq{a_n}}$ then $x
\times = \overline{\seq{a_nb_n}} = \overline{\seq{a_n}} = 1$. We used the
property of rationals that $a_n \times 1 = a_n$.
\end{proof}

\begin{lem}\label{c4s3l17}
For all real numbers $x \ne 0$, $x \times x^{-1} = 1$.
\end{lem}
\begin{proof}
Follows immediately from the definition of reciprocal of a non-zero real
number.
\end{proof}

\begin{lem}\label{c4s3l18}
Multiplication distributes over addition.
\end{lem}
\begin{proof}
Follows immediately from the corresponding property of rationals.
\end{proof}

\begin{lem}\label{c4s3l19}
For all real numbers $x, y$ $x(-y) = -(xy) = (-x)y$.
\end{lem}
\begin{proof}
Follows immediately from the corresponding property of rationals.
\end{proof}

\begin{lem}\label{c4s3l20}
For all real numbers $x$, $x \times 0 = 0$.
\end{lem}
\begin{proof}
From lemma \ref{c4s3l13}, $1 + (-1) = 0$. Therefore, $x \times 0 =
x \times (1 + (-1)) = x \times 1 + x \times (-1)$ using \ref{c4s3l18}.
Using \ref{c4s3l19}, we have $x \times 0 = x \times 1 - x \times 1 = 0$.
\end{proof}

\section{Ordering the reals}\label{c4s4}
\begin{defn}\label{c4s4d1}
Let $\{a_n\}$ be a sequence of rationals. We say it is positively 
(negatively) bounded away from zero iff we have a positive (negative)
rational number $c > 0$ ($c < 0$) such that $a_n \ge c$ ($a_n \le c$)
for all $n \in \son$.
\end{defn}

\begin{defn}\label{c4s4d2}
A real number $x$ is said to be positive (negative) if $x = 
\overline{\seq{a_n}}$ for some positively (negatively) bounded away from
zero sequence $\seq{a_n}$ of rationals.
\end{defn}

\begin{lem}\label{c4s4l1}
For every real number $x$ exactly one of the following three statements
is true: (1) $x$ is zero, (2) $x$ is positive, (3) $x$ is negative. 
\end{lem}
\begin{proof}
Let $x = \overline{\seq{a_n}}$. $x$ is positive of $\seq{a_n}$ is 
positively bounded away from zero. Let, if possible, it be negative as 
well. That is, $\seq{b_n} \in \overline{\seq{a_n}}$ but $\seq{b_n}$ is
negatively bounded away from zero. If $c_b < 0$ is the bound of $\seq{b_n}$
and $c_a > 0$ is the bound of $\seq{a_n}$ then $d(a_n, b_n) \ge c_a + c_b
$ and they will not be equivalent. Therefore, $\seq{b_n}$ will not be a 
member of $\overline{\seq{a_n}}$. Now let us examine the possibility
of it being zero simultaneously. In that case, there will be a $\seq{b_n}$
in $\overline{\seq{a_n}}$ such that $|b_n| < \epsilon$ for any $\epsilon > 
0$ for all $n$ greater than some $N$. In particular, $|b_n| < c_a/2$ for 
all $n$ beyond some $N$. If that were to be the case then $|a_n - b_n| \ge
|a_n| - |b_n| \ge c_a/2$ for all large enough $n$. Once again it means 
that $\seq{b_n}$ cannot be equivalent to $\seq{a_n}$.
\end{proof}

\begin{lem}\label{c4s4l2}
A real number $x$ is negative iff $-x$ is positive.
\end{lem}
\begin{proof}
$x = \overline{\seq{a_n}}$ is negative implies that $\seq{a_n}$ is 
negatively bounded away from zero. Therefore, $\seq{-a_n}$ is positively
bounded away from zero. By definition \ref{c4s3d5}, the equivalence class
of this sequence is $-x$. Therefore $-x$ is positive.

If $-x = \overline{\seq{b_n}}$ then $\seq{b_n}$ is negatively bounded away
from zero. Therefore, $\seq{-b_n}$ is positively bounded away from zero 
and its equivalence class will be $-(-x) = x$, a positive real.
\end{proof}

\begin{rem}
$-(-x) = x$ follow immediately from the corresponding property of 
rationals.
\end{rem}

\begin{lem}\label{c4s4l3}
If $x$ and $y$ are positive then so are $x + y$ and $xy$.
\end{lem}
\begin{proof}
Follows immediately from the corresponding properties of rationals.
\end{proof}

\begin{defn}\label{c4s4d3}
The absolute value of a real number $x$ is defined as 
\[
|x| = \begin{cases}
 x & \text{ if } x \ge 0 \\
 -x & \text{ if } x < 0.
\end{cases}
\]
\end{defn}

\begin{defn}\label{c4s4d4}
Let $x, y \in \sor$. We say $x > y$ iff $x - y$ is a positive real number
and $x < y$ iff $x - y$ is a negative real number. We define $x \ge y$ iff
$x > y$ or $x = y$ and $x \le y$ iff $x < y$ or $x = y$.
\end{defn}

\begin{prop}\label{c4s4p1}
Let $x, y, z \in \sor$. Then:
\begin{enumerate}
\item Exactly one of the three statements $x = y, x < y, x > y$ is true.
\item $x > y$ iff $y < x$.
\item $x < y$ and $y < z$ implies $x < z$.
\item If $x < y$ then $x + z < y + z$.
\item If $z > 0$ and $x < y$ then $xz < yz$.
\end{enumerate}
\end{prop}
\begin{proof}
\begin{enumerate}
\item Let $d = x - y$. By lemma \ref{c4s4l1} exactly one of the statements
$d > 0, d < 0, d = 0$ is true. Each of these statements corresponds to 
$x > y, x < y, x = y$ respectively.
\item If $x > y$ then $d > 0$, so that $-d < 0$, that is $-x + y < 0$ or 
$y - x < 0$ or $y < x$. The converse can be proved similarly.
\item Let $d = y - x$ and $e = z - y$ then $d + e = z - x$. Since $d > 0,
e > 0$, by lemma \ref{c4s4l3} $d + e > 0$. Therefore $z > x$.
\item $y + z - (x + z) = y - x > 0 \Rightarrow y + z > x + z$.
\item $yz - xz = (y - x)z$. Since $x < y$, $y - z > 0$. We are given that
$z > 0$. By lemma \ref{c4s4l3}, $(y - x)z > 0$ so that $yz > xz$.
\end{enumerate}
\end{proof}

\begin{lem}\label{c4s4l4}
Let $x > 0$ be a real number then $x^{-1} > 0$.
\end{lem}
\begin{proof}
By lemma \ref{c4s3l19}, if $y = -x^{-1} > 0$ then $x(-y) = -(xy)$. Since
$x > 0, y > 0$, by lemma \ref{c4s4l3}, $xy$ is positive, so that $-(xy)$
is negative. Therefore $x(-y) = xx^{-1} = 1$ is negative.
\end{proof}

\begin{lem}\label{c4s4l5}
Let $x, y \in \sor$, $x > 0, y > 0$ and $x < y$. Then $x^{-1} > 
y^{-1}$.
\end{lem}
\begin{proof}
$x < y \Rightarrow x^{-1}x < x^{-1}y$ because $x^{-1} > 0$. Therefore,
$1 < x^{-1}y \Rightarrow y^{-1} < y^{-1}x^{-1}y = x^{-1}$.
\end{proof}

\begin{lem}\label{c4s4l6}
Let $\seq{a_n}$ be a Cauchy sequence of non-negative rational numbers.
Then $\overline{\seq{a_n}}$ is a non-negative real number.
\end{lem}
\begin{proof}
Let, if possible, $\overline{\seq{a_n}} = x < 0$. Therefore, there is a
sequence $\seq{b_n}$ which is negatively bounded away from zero such that
$x = \overline{\seq{b_n}}$. Therefore, $\seq{a_n} \sim \seq{b_n} 
\Rightarrow d(a_n, b_n) < \epsilon$ for all $n > N$. Let $c > 0$ be 
such that $b_n < -c$ for all $n \in \son$. Then $-b_n > c > 0$ or $a_n
- b_n > a_n + c \ge c$ because $a_n \ge 0$. Therefore, $|a_n - b_n| \ge c
> 0$. Therefore, it is not possible for $\seq{b_n}$ to be in the 
equivalence class of $\seq{a_n}$. This contradicts our assumption that
$\seq{a_n} \sim \seq{b_n}$.
\end{proof}

An immediate corollary of this lemma is
\begin{cor}
If $\seq{a_n}$ and $\seq{b_n}$ are Cauchy sequences of rationals such
that $a_n \ge b_n$ for all $n \in \son$ then $\overline{\seq{a_n}} \ge
\overline{\seq{b_n}}$.
\end{cor}
\begin{proof}
If $c_n = a_n - b_n$ then $\seq{c_n}$ is a sequence of non-negative
rationals. Therefore, $\overline{\seq{c_n}} \ge 0 \Rightarrow \overline{
\seq{a_n - b_n}} \ge 0 \Rightarrow \overline{\seq{a_n}} - \overline{\seq{
b_n}} \ge 0 \Rightarrow \overline{\seq{a_n}} \ge \overline{\seq{b_n}}$.
\end{proof}

\begin{prop}\label{c4s4p2}
Let $x > 0$ be a real number. Then there exists a rational number $q > 0$
such that $q < x$ and a natural number $M$ such that $x \le M$.
\end{prop}
\begin{proof}
$x > 0 \Rightarrow x = \overline{\seq{a_n}}$ for some $\seq{a_n}$ 
positively bounded away from zero. Therefore, there is a positive rational
$q > 0$ such that $a_n > q > 0$ for all $n \in \son$. Therefore $x > q$.

Claim: If $x, z \in \soq$ such that $x < z$ then we can find $y \in \soq$
such that $x < z < y$.
Proof: $y = 2z - z$ has the desired property.

Since $\seq{a_n}$ is a Cauchy sequence, there exists $N \in \son$ such 
that for all $n, m \ge N$, $|a_n - a_m| < 1$. In particular, $|a_n - a_N|
< 1 \Rightarrow -1 < a_n - a_N < 1 \Rightarrow a_N - 1 < a_n < a_N + 1$
Therefore $x = \overline{\seq{a_n}} < a_N + 1$. By proposition 
\ref{c3s4p1} there is a natural number $M$ greater than the rational number
$a_N + 1$. Therefore $x < M \Rightarrow x \le M$.
\end{proof}

\begin{prop}[Archimedean property]\label{c4s4p3}
Let $x \in \sor$ and $\epsilon > 0$ be another real number. Then there 
exists a positive natural number $M$ such that $x < M\epsilon$.
\end{prop}
\begin{proof}
If $x \le 0$, choose $M = 1$. If $x > 0$, then we know that $x/\epsilon > 
0$. By proposotion \ref{c4s4p2} there exists a positive natural number 
$N$ such that $x/\epsilon \le N < N + 1$. If $M = N + 1$, we have $x < 
\epsilon M$.
\end{proof}

\begin{lem}\label{c4s4l7}
Let $x$ be a positive real number. Then there exists a positive natural
number $N$ such that $x > N^{-1}$.
\end{lem}
\begin{proof}
If $x > 1$ then $N = 1$. If $0 < x < 1$, $1 < x^{-1}$. By proposition 
\ref{c4s4p2} there exists a natural number $N$ such that $x^{-1} \le M <
M + 1$ so that $x > (M + 1)^{-1}$. Choose $N = M + 1$.
\end{proof}

\begin{prop}\label{c4s4p4}
Given two real numbers $x < y$ we can find a rational number $q$ such
that $x < q < y$.
\end{prop}
\begin{proof}
Since $y - x > 0$, using lemma \ref{c4s4l7}, there exists a positive
natural number $N$ such that $y - x > N^{-1}$. Therefore, $N(y - x) > 1$.
Therefore, there exists an integer $M$ such that $Nx < M < Ny$ or that
$x < M/N < y$.
\end{proof}
