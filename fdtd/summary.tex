\documentclass[11pt]{article}
\usepackage{amsmath, amsfonts, amssymb, amsthm}
\usepackage{physics, mathtools, graphicx}
\newcommand{\opr}{\prime}
\linespread{1.2}
\title{Young's experiment}
\author{TBD}
\date{12-Nov-2020}
\begin{document}
\maketitle
Young's experiment is a landmark in the history of physics. It provided a firm
evidence in favor of the wave theory of light. The experiment consists of
allowing light to pass through two narrow slits, a distance $d$ apart, in a 
screen and letting it on another screen a distance $D$ from the first screen.
Young observed a pattern of alternating bright and dark bands on the second 
screen and concluded that it was because of constructive and destructive
interference of waves. The nature of the waves was not known in Young's time.
It was only in the later part of the nineteenth century that light was
understood as an electromagnetic wave. This article describes Young's experiment
, derives the electromagnetic wave equation from Maxwell's equations, describes
their nature in material media and solves them numerically to demonstrate the
formation of interference fringes in a numerical simulation.

The numerical simulation of the interference phenomenon is carried out using
the finite difference time domain technique. The derivatives in each one of 
Maxwell's equations are written as central differences on a carefully selected
grid of points spanning the domain of integration. The electric and the magnetic
field are evaluated at alternating points. The finite size of the domain of
integration is taken into account by the absorbing boundary conditions. We
first develop them in the simpler case of one-dimensional problem and extend
it to two-dimensional problems using the perfectly matched layers. We
demonstrate the finite difference time domain technique first in a simple 
situation where we consider the propagation of a gaussian light pulse. We stop
the simulation well before the light reaches the boundary so that we do not
have to worry about the boundary conditions. We build on these simple examples
to simulate Young's experiment where we take into account the boundary
conditions by implementing a perfectly matched layer.

Numerical algorithms, especially the ones solving hyperbolic partial
differential equations, can be unstable. We explain the 
Courant-Friedrichs-Lewy condition that ensures that the time-step of simulation
is chosen in such a way that it is shorter than the time it takes for the 
wave to reach a neighboring grid point.
\end{document}
