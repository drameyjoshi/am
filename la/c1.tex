\chapter{Spaces}\label{c1}
\section{Fields}\label{c1s1}
\begin{defn}\label{c1s1d1}
A set $F$ with binary operations $+$ and $\cdot$ defined on it is a
field is
\begin{enumerate}
\item (A) Properties of the operation `$+$':
\begin{enumerate}
\item For every $\alpha, \beta \in F$, $\alpha + \beta \in F$,
\item For every $\alpha, \beta \in F$, $\alpha + \beta = \beta +
\alpha$,
\item For all $\alpha, \beta, \gamma \in F$, $\alpha + (\beta + 
\gamma) = (\alpha + \beta) + \gamma$,
\item There exists a member $0$ such that $\alpha + 0 = \alpha$ for
all $\alpha \in F$,
\item For every $\alpha \in F$, there exists a member $-\alpha$
such that $\alpha + (-\alpha) = 0$.
\end{enumerate}
\item (B) Properties of the operation `$\cdot$':
\begin{enumerate}
\item For every $\alpha, \beta \in F$, $\alpha \cdot \beta \in F$,
\item For every $\alpha, \beta \in F$, $\alpha \cdot \beta = \beta 
\cdot \alpha$,
\item For all $\alpha, \beta, \gamma \in F$, $\alpha \cdot (\beta 
\cdot \gamma) = (\alpha \cdot \beta) \cdot \gamma$,
\item There exists a member $0$ such that $\alpha \cdot 1 = \alpha$ for
all $\alpha \in F$,
\item For every $\alpha \ne 0 \in F$, there exists a member 
$\alpha^{-1}$ such that $\alpha \cdot \alpha^{-1} = 1$.
\end{enumerate}
\item The operation $\cdot$ is distributive over $+$. That is $\alpha
\cdot (\beta + \gamma) = \alpha \cdot \beta + \alpha \cdot \gamma$
for all $\alpha, \beta, \gamma \in F$.
\end{enumerate}
\end{defn}

\begin{rem}
Members of a field are called scalars.
\end{rem}

\subsection{Exercises}
\begin{enumerate}
\item We will prove the following assertions.
\begin{enumerate}
\item $0 + \alpha$ by commutativity of `+' is the same as $\alpha + 0
= \alpha$.
\item $\alpha + \beta = \alpha + \gamma \Rightarrow -\alpha + (\alpha
+ \beta) = -\alpha + (\alpha + \gamma) \Rightarrow (-\alpha + \alpha)
+ \beta = (-\alpha + \alpha) + \gamma$. Now, by commutativity of `+',
$\alpha + (-\alpha) = 0 \Rightarrow -\alpha + \alpha = 0$. Therefore,
$0 + \beta = 0 + \gamma \Rightarrow \beta = \gamma$.
\item $\alpha + (\beta - \alpha) = \alpha + (-\alpha + \beta) = 
\alpha + (-\alpha) + \beta = 0 + \beta = \beta$.
\item Since $0$ is an additive identity, $0 + 0 = 0 \Rightarrow
\alpha\cdot(0 + 0) = \alpha\cdot 0 \Rightarrow \alpha\cdot 0 + 
\alpha\cdot 0 = \alpha\cdot 0$. Since $\alpha\cdot 0$ is a member of
$F$, it has an additive inverse $-\alpha\cdot 0$. Thus,
$-\alpha\cdot 0 + (\alpha\cdot 0 + \alpha\cdot 0) = 
-\alpha\cdot 0 + \alpha\cdot 0 \Rightarrow
(-\alpha\cdot 0 + \alpha\cdot 0) + \alpha\cdot 0 = 0 \Rightarrow
0 + \alpha\cdot 0 = 0 \Rightarrow \alpha\cdot 0 = 0.$
The claim $0\cdot\alpha = 0$ follows from commutativity of $\cdot$.
\item We observe that $(-1)\alpha + \alpha = \alpha(-1) + \alpha(1)
= \alpha(-1 + 1) = \alpha 0 = 0$. Thus, $(-1)\alpha$ is the additive
inverse of $\alpha$. $(-1)\alpha = -\alpha$ follows from the
uniqueness of the additive inverse.
\item $(-\alpha)(-\beta) + (-\alpha)(\beta) = -\alpha(-\beta + \beta)
= -\alpha \cdot 0 = 0$. Thus, $(-\alpha)(-\beta)$ is the additive 
inverse of $(-\alpha)(\beta)$. We also observe that $\alpha\beta + 
(-\alpha)\beta = (\alpha - \alpha)\beta = 0\cdot\beta = 0$. Thus,
$\alpha\beta$ is also an additive inverse of $(-\alpha)(\beta)$. The
claim follows from the uniqueness of additive inverse. 

This claim used the fact that $(\alpha + \beta)\gamma = \alpha\gamma
+ \beta\gamma$. It follows from commutativity of the two operations.

\item Let $\alpha\beta = 0$. If $\alpha \ne 0$ then $\alpha^{-1}$
exists. Multiply both sides of the equation by $\alpha^{-1}$ to get
$\beta = 0$. Similarly, $\beta \ne 0$ gives $\alpha = 0$. The equality
is also true when both $\alpha$ and $\beta$ are zero.
\end{enumerate}

\item
\begin{enumerate}
\item The set of all integers is not a field because the inverse
under addition does not exist.
\item The set of all integers is a group. It is not a field because
multiplicative inverse does not exist.
\end{enumerate}

\item Given $m \in \soi$ and $Z_m = \{0, 1, \ldots, m - 1\}$, the 
field operations are defined as $\alpha \oplus \beta = (\alpha + 
\beta)\mod n$ and $\alpha \odot \beta = (\alpha\cdot\beta)\mod n$.
The set $(Z_m, \oplus)$ is an abelian group. However, $(Z_m, \odot)$
has an inverse if and only if $m$ is prime. If $m$ is prime then the 
remainders when $\alpha\dot\beta$ are divided by $m$ are in the range
$0, \ldots, m- 1$. If it is not then the remainders may not include $1$
when one of $\alpha$ of $\beta$ is a divisor of $m$. As a result, in
this latter case, some elements may not have an inverse.
\begin{enumerate}
\item The table of additions modulo $5$ is
\[
\begin{array}{c|c c c c c}
+ & 0 & 1 & 2 & 3 & 4 \\
\hline 
0 & 0 & 1 & 2 & 4 & 4 \\
1 & 1 & 2 & 3 & 4 & 0 \\
2 & 2 & 3 & 4 & 0 & 1 \\
3 & 3 & 4 & 0 & 1 & 2 \\
4 & 4 & 0 & 1 & 2 & 3
\end{array}
\]
From this table it we observe that $1 + 4 = 0$ so that $-1 = 4$.
\item The table of multiplication modulo $7$ is
\[
\begin{array}{c|c c c c c c c c}
\cdot & 0 & 1 & 2 & 3 & 4 & 5 & 6 \\
\hline
0 & 0 & 0 & 0 & 0 & 0 & 0 & 0 \\
1 & 0 & 1 & 2 & 3 & 4 & 5 & 6 \\
2 & 0 & 2 & 4 & 6 & 1 & 3 & 5 \\
3 & 0 & 3 & 6 & 2 & 5 & 1 & 4 \\
4 & 0 & 4 & 1 & 5 & 2 & 6 & 3 \\
5 & 0 & 5 & 3 & 1 & 6 & 4 & 2 \\
6 & 0 & 6 & 5 & 4 & 3 & 2 & 1 
\end{array}
\]
From this table we read $3 \times 5 = 1$ so that $1/3 = 5$.
\end{enumerate}

\item Let
\[
x_m = \begin{cases}
1 & \text{ when } m = 0 \\
x_{m-1} + 1 & \text{ otherwise}
\end{cases}
\]
Suppose that $x_m = 0$ and the field is not finite. Then $x_{km} = 0$ for all
$k = 1, 2, \ldots$ or that $x_{km - 1} + 1 = 0$ for all $k = 1, 2, \ldots$ or
that $x_{km - 1} = -1$ for all $k = 1, 2, \ldots$. However, the additive
inverse has to be unique.

If the field is finite then we proved in the previous exercise that $m$ must
be prime.

\item The set $Q(\sqrt{2})$ is a field but $Z(\sqrt{2})$ is not. The latter
set does not have a multiplicative inverse.

\item The set of all polynomials with integer coefficients is not a field.
The reciprocal of $x^2 + x - 2$ is calculated as
\begin{eqnarray*}
\frac{1}{x^2+x-2} &=& -\frac{1}{3}\left(\frac{1}{1-x} - 
                      \frac{1}{2}\frac{1}{1+x/2}\right) \\
 &=& -\frac{1}{6} + \frac{x}{4} - \frac{7}{8}x^2 + \frac{5}{16}x^3 + \cdots.
\end{eqnarray*}
Clearly, the reciprocal is not a polynomial of integer coefficients.

The set of all polynomials with real coefficients is a field. By the 
fundamental theorem of algebra, any polynomial of degree $n$ has $n$ roots,
some of which may be complex. Further, the complex roots always come in pairs
of conjugates. If all roots are real then the polynomial definitely has a
reciprocal with real coefficients. We will show that even if the polynomial has
complex roots, the resulting reciprocal is real.

Since the complex roots come in pairs, the reciprocal will have factors like
\[
\frac{1}{(x - a)(x - \bar{a})} = \frac{i}{2\Im(a)}\left(\frac{1}{x - \bar{a}}
- \frac{1}{x - a}\right).
\]
We can write this as
\begin{eqnarray*}
\frac{1}{(x - a)(x - \bar{a})} &=& \frac{i}{2\Im(a)}\left[\left(\frac{1}{a} + 
\frac{1}{\bar{a}}\right) - x\left(\frac{1}{a^2} + \frac{1}{\bar{a}^2}\right) 
+ x^2\left(\frac{1}{a^3} + \frac{1}{\bar{a}^3}\right) + \cdots\right] \\
 &=& \frac{i}{2\Im(a)}\sum_{k \ge 0}(-x)^k
     \left(\frac{1}{a^k} + \frac{1}{\bar{a}^k}\right) \\
 &=& \frac{i}{2\Im(a)}\sum_{k \ge 0}(-x)^k\frac{a^k + \bar{a}^k}{|a|^{2k}}
\end{eqnarray*}
If $a = x + iy$ then
\[
a^k + \bar{a}^k = \sum_{l=0}^k\binom{k}{l}x^{l}y^{k-l}(i^{k-1} + (-i)^{k-l}),
\]
which is always real.

\item Consider the pair $(\alpha, \beta)$ of real numbers.
\begin{enumerate}
\item Let addition and multiplication be defined as
\begin{eqnarray*}
(\alpha, \beta) + (\gamma, \delta) &=& (\alpha + \gamma, \beta + \delta) \\
(\alpha, \beta)(\gamma, \delta) &=& (\alpha\gamma, \beta\delta)
\end{eqnarray*}
Then $(0, 0)$ is the additive identity and $(1, 1)$ the multiplicative. 
Therefore, the multiplicative inverse of $(\alpha, \beta)$ is defined with the
equation
\[
(\alpha, \beta)(\gamma, \delta) = (1, 1).
\]
Clearly, the inverse is not defined for $(\alpha, 0)$ and $(0, \beta)$.

\item If the addition and multiplication are defined as
\begin{eqnarray*}
(\alpha, \beta) + (\gamma, \delta) &=& (\alpha + \gamma, \beta + \delta) \\
(\alpha, \beta)(\gamma, \delta) &=& (\alpha\gamma - \beta\delta,
                                     \alpha\delta + \beta\gamma)
\end{eqnarray*}
then the operations are same as that for complex numbers, a field.

\item If $\alpha$ and $\beta$ are complex numbers then the first scheme fails
to form a field for the same reasons. In the second scheme, consider the
product $(1, i)(1, -i) = (0, 0)$. Thus, a product of two non-zero numbers is
zero. This is the same problem that we had in $Z_m$ when $m$ was not a prime.
However, if non-zero elements $a, b$ belong to a field then $ab \ne 0$.
\end{enumerate}

\end{enumerate}
