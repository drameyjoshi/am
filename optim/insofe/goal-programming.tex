\documentclass{beamer}
\usefonttheme[onlymath]{serif}
\usepackage{graphicx, bm, amsmath}
\usetheme{Boadilla}
\DeclareMathOperator*{\argmin}{arg\,min}

\title{Goal Programming}
\author{Amey Joshi}
\date{\today}
\begin{document}
\begin{frame}
\titlepage
\end{frame}

\begin{frame}
\frametitle{Multi-objective optimisation}
\begin{enumerate}
\item Few business problems are as simple as optimising only one objective 
function.
\item Practical problems involve balancing several objectives/goals. Sometimes
they are mutually antagonistic.
\item Think of the goals in personal tasks like
\begin{enumerate}
\item Buying a house;
\item Selecting a job offer;
\item Choosing a life partner.
\end{enumerate}
\item Some complex business tasks could be 
\begin{enumerate}
\item Setting up an offshore delivery centre;
\item Choosing a location for a new factory;
\item Allocating funds to departments.
\end{enumerate}
\end{enumerate}
\end{frame}

\begin{frame}
\frametitle{Multi-objective optimisation}
\begin{enumerate}
\item A multi-objective optimisation problem defined as 
\begin{equation}\label{e1}
\min\{f_1(\bm{x}, \bm{y}), \ldots, f_k(\bm{x}, \bm{y})\},
\end{equation}
subject to
\begin{eqnarray}
g(\bm{x}, \bm{y}) &=& 0 \label{e2} \\
h(\bm{x}, \bm{y}) &\le& 0 \label{e3}
\end{eqnarray}
where $\bm{x} \in \mathbb{R}^n$ is a vector of continuous decision variables and
$\bm{y} \in \mathbb{R}^m$ is a vector of discrete decision variables. The functions
$g$ and $h$ map $\mathbb{R}^{n+m}$ to $\mathbb{R}^p$ and $\mathbb{R}^s$.
\item The objective functions $f_1, \ldots, f_k$ and the constraints $g$ and 
$h$ are arbitrary.
\end{enumerate}
\end{frame}

\begin{frame}
\frametitle{Dominance}
\begin{enumerate}
\item In a single-objective optimisation problem, a solution $\bm{x}_1$ is 
better than $\bm{x}_2$ if $f(\bm{x}_1) \le f(\bm{x}_2)$. There is an easy way 
to compare one feasible solution with another.
\item Situation is much different in the case of a multi-objective optimisation
problem. We compare two feasible solutions by checking if one of them dominates
the other.
\item A feasible solution $\bm{x}_1$ dominates another feasible solution 
$\bm{x}_2$ if
\begin{enumerate}
\item $f_i(\bm{x}_1) \le f_i(\bm{x}_2)$ for all $i = 1, \ldots, k$. Thus, 
$\bm{x}_1$ is no worse than $\bm{x}_2$ in all objective functions and
\item $f_i(\bm{x}_1) < f_i(\bm{x}_2)$ for at least one $i = 1, \ldots, k$.
That is, $\bm{x}_1$ is strictly better than $\bm{x}_2$ in at least one 
objective.
\end{enumerate}
\item The relation `dominates' is a bit different than $>$. If $\bm{x}_1$ and
$\bm{x}_3$ dominate $\bm{x}_2$ we may not be able to tell if $\bm{x}_1$ 
dominates $\bm{x}_3$ or the other way round.
\end{enumerate}
\end{frame}

\begin{frame}
\frametitle{Pareto Optimality}
\begin{enumerate}
\item In general, there is no solution $\bm{x}^\ast$ that simultaneously 
minimises all objective functions $f_1, \ldots, f_k$.
\item Sometimes the goals may work against each other.
\item A solution to a multi-objective optimisation problem is Pareto optimal if
there is no other feasible solution for which one or more objective functions 
take a lesser value. You cannot improve an objective function without spoiling 
the rest.
\item A feasible solution can be Pareto improved if some objective functions can
be improved without adversely affecting the others.
\item A feasible solution is called Pareto dominated if there exists a Pareto
improvement.
\item A Pareto front is the locus of all Pareto optimal solutions.
\end{enumerate}
\end{frame}

\begin{frame}
\frametitle{Solution techniques}
\begin{enumerate}
\item The existence of a Pareto front indicates that there is no unique Pareto
optimal solution to the multi-objective optimisation problem. Often times, a 
human decision maker decides which solution to select.
\item There are four ways to solve the problem:
\begin{itemize}
\item In \emph{No preference methods} a solution of the vector valued objective
function $\bm{f}(\bm{x}) = (f_1(\bm{x}), \ldots, f_k(\bm{x}))^T$ is
found without the decision maker influencing it.
\item In \emph{a priori methods} the decision maker indicates his preferences of
objective functions that must be minimised at a priority.
\item In \emph{a posteriori methods} the decision maker selects one solution 
from the Pareto front.
\item In \emph{interactive methods} the decision maker actively intervenes in
selecting solutions from the Pareto by deciding which objective functions must
be prioritised over others.
\end{itemize}
\item Goal programming is an example of a priori methods.
\end{enumerate}
\end{frame}

\begin{frame}
\frametitle{Weighted sum method}
\begin{enumerate}
\item One can easily convert a multi-objective program into a single-objective
one using `scalarisation'.
\item The multiple objective functions are considered as components of a vector.
\item They can be converted into a scalar in many ways. Although the most 
obvious one is to consider the norm it is more convenient to use the weighted
sum
\[
f(\bm{x}) = \sum_{i=1}^k w_if_i(\bm{x}),
\]
where $w_1, \ldots, w_k$ are \emph{given} weights. The constraints remain the 
same.
\item If the objective functions $f_1, \ldots, f_k$ are linear then so it $f$.
\item The idea is easy to implement but it is not so easy to prove that the
solution is Pareto optimal.
\end{enumerate}
\end{frame}

\begin{frame}
\frametitle{$\epsilon$-constraint method}
\begin{enumerate}
\item Identify the most important objective function, say $f_j$, $1\le j\le k$,
and relegate the rest of them as constraints.
\item For the other objective functions, set upper limits $\epsilon_1, \ldots, 
\epsilon_k$ and turn them into constraints
\begin{eqnarray*}
f_1(\bm{x}) &\le& \epsilon_1 \\
\vdots &\le& \vdots \\
f_k(\bm{x}) &\le& \epsilon_k,
\end{eqnarray*}
where the function $f_j$ is excluded.
\item The new constraints may make the feasible region non-convex.
\item It is not easy to come up with the `best' values of $\epsilon_i$s. We may
not get a Pareto optimal solution.
\end{enumerate}
\end{frame}

\begin{frame}
\frametitle{Goal programming}
\begin{enumerate}
\item We allow the solution to deviate from the constraints. The deviations
could be in either directions.
\item The objective function is to minimise the total deviations.
\item It is possible to prioritise the goals by assigning weights to them. The
same weights are assigned to the deviations and when we minimise the aggregate
deviation we tend to minimise the deviations of the goals with highest weights.
\item Some deviations are not important or they do not matter as much. We can
assign a zero weight to their deviations.
\item Some deviations may be subject to a limit. For example, we may want the
deviation in cost price to be within a certain limit of the objective.
\item Deviations in integral decision variables are themselves integers. This 
makes goal programming an instance of MILP. A greater number of integral 
variables makes the problem harder to solve.
\end{enumerate}
\end{frame}

\begin{frame}
\frametitle{Illustration of a goal program - the problem statement}
\begin{enumerate}
\item A hotel owner must decide the number of small, medium and large conference
rooms to be constructed. A small room has an area of $400$ square feet, a medium
one $750$ and a large one $1050$ square feet. It takes $18000$ dollars to 
construct a small conference room, $33000$ to construct a medium one and $45150$ 
to construct a large one. The owner has the following goals
\begin{enumerate}
\item Have aroung $5$ small, $10$ medium and $15$ large conference rooms;
\item The total built up area should be around $25,000$ square feet and
\item The cost of the project should be around $1 M$ dollars. 
\end{enumerate}
\item Notice the phrase ``should be around''.
\item The decision maker has a rough idea of what he is looking for and wants to 
know if that is feasible. He wants to examine several ways in which he can 
achieve his goals.
\end{enumerate}
\end{frame}

\begin{frame}
\frametitle{Illustration of a goal program - the mathematical formulation}
\begin{enumerate}
\item Let $x_1, x_2, x_3$ be the number of small, medium and large rooms to be
constructed. They are our decision variables. 
\item The first constraint is expressed as
\begin{eqnarray}
x_1 + d_1^- - d_1^+ &=& 5 \label{e4} \\
x_2 + d_2^- - d_2^+ &=& 10 \label{e5} \\
x_3 + d_3^- - d_3^+ &=& 15 \label{e6}
\end{eqnarray}
\item $d_i^-$ and $d_i^+$ are called deviational variables. $d_i^-$ is a measure
of underachieving the target and $d_i^+$ that of exceeding it.
\item The second constraint is
\begin{equation}\label{e7}
400x_1 + 750 x_2 + 1050x_3 + d_4^- - d_4^+ = 25000.
\end{equation}
\end{enumerate}
\end{frame}

\begin{frame}
\frametitle{Illustration of a goal program - the mathematical formulation}
\begin{enumerate}
\item The third constraint is
\begin{equation}\label{e8}
18000x_1 + 33000 x_2 + 45150x_3 + d_5^- - d_5^+ = 1000000.
\end{equation}
\item We want to minimize the deviations from the stated goals.
\item The first candidate for the objective function could be
\begin{equation}\label{e9}
D = \sum_{i=1}^5 (d_i^+ + d_i^-).
\end{equation}
\item The deviations $d_i^+$ and $d_i^-$ have different units for different 
constraints. The sum does not make sense. Therefore, we modifiy it to
\begin{equation}\label{e10}
D = \sum_{i=1}^5 \frac{d_i^+ + d_i^-}{t_i},
\end{equation}
where $t_i$ is the approximate target to be achieved.
\end{enumerate}
\end{frame}

\begin{frame}
\frametitle{Illustration of a goal program - the mathematical formulation}
\begin{enumerate}
\item Some goals are more important than others. This aspect is expressed 
mathematically by assigning weights to the deviations.
\item Higher weights are assigned to deviations we want to have the least. The
resulting objective function is
\begin{equation}\label{e11}
D = \sum_{i=1}^5 \frac{w_i^+d_i^+ + w_i^-d_i^-}{t_i}.
\end{equation}
\item Note that $w_i$ are constants supplied by the decision maker. The 
remaining constraints are
\begin{eqnarray}
x_i &\ge& 0 \label{e12} \\
d_i^+ &\ge& 0 \label{e13} \\
d_i^- &\ge& 0 \label{e14}
\end{eqnarray}
\end{enumerate}
\end{frame}

\begin{frame}
\frametitle{Illustration of a goal program - the mathematical formulation}
\begin{enumerate}
\item If a certain deviation is unimportant or irrelevant, we assign it a
zero weight.
\item Sometimes we may want to express the limits on certain deviations. For 
example we do not want the cost overruns to exceed $50000$ dollars. We express 
it as
\begin{equation}\label{e15}
d_5^+ \le 50000.
\end{equation}
\item Finally, the decision variables $x_1, x_2, x_3$ must be integers. The
deviational variables $d_i^+$ and $d_i^-$ should also be integers for the
first three constraints related to the number of rooms. The remaining ones may
be permitted to take any real values.
\item As usual, we try to minimize the number of integer constraints.
\end{enumerate}
\end{frame}

\begin{frame}
\frametitle{Example: portfolio optimisation}
\begin{enumerate}
\item A portfolio consists of and amount $X$ to be invested in $n$ assets. The
allocation is expressed in terms of weights $w_i$, $i = 1, \ldots, n$ such that
\begin{equation}\label{e16}
\sum_{i=1}^n w_iX = X \text{ or } \sum_{i=1}^n w_i = 1.
\end{equation}
\item If no short-selling is allowed then $w_i \ge 0$.
\item Each asset has a certain expected return and volatility.
\item The goal is to give at least $\mu$ percent return with a volatality at
most $\sigma$. 
\item There could be many other constraints like a limit on the number of weight
in an asset class, the minimum and maximum number of asset classes etc.
\item The Pareto frontier is called \emph{efficient frontier} in Finance.
\end{enumerate}
\end{frame}

\begin{frame}
\frametitle{Example: transport of materials}
\begin{enumerate}
\item You want to transport $n$ goods from a source to a destination. There
are several modes of transporting them, each one characterised by a cost 
per unit and time.
\item Some of your goods are perishable. 
\item Some goods are valuable enough that you cannot send all of them by a
single more.
\item Each mode of transport is limited by a certain capacity.
\item There could also be a certain amount of pilferage of goods with each mode.
\item You want to transport the goods so that
\begin{itemize}
\item they reach the destination in the fastest possible way
\item at the least cost and
\item with least pilferage.
\end{itemize}
\end{enumerate}
\end{frame}
\end{document}