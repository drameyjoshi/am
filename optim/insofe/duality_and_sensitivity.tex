\documentclass{beamer}
\usefonttheme[onlymath]{serif}
\usepackage{graphicx, bm, amsmath}
\usetheme{Boadilla}
\DeclareMathOperator*{\argmin}{arg\,min}

\title{Duality and Sensitivity Analysis}
\author{Amey Joshi}
\date{\today}
\begin{document}
\begin{frame}
\titlepage
\end{frame}

\begin{frame}
\frametitle{Notation and mathematical preliminaries}
\begin{enumerate}
\item $\mathbb{R}$ is the set of real numbers. The Cartesian product $\mathbb{R}
\times \mathbb{R}$ is denoted by $\mathbb{R}^2$ and $\mathbb{R}^n \times 
\mathbb{R}$ by $\mathbb{R}^{n+1}$ for all non-negative integers $n$.
\item Members of $\mathbb{R}^n$ are called vectors and they are denoted by bold
small letters like $\bm{a}, \bm{x}$ etc. In component form $\bm{x} = (x_1, 
\ldots, x_n)$.
\item The notation $\bm{x} \ge 0$ means $x_1 \ge 0, \ldots, x_n \ge 0$.
\item Matrices are denoted by bold capital letters $\bm{A}, \bm{X}$ etc. Their 
elements are denoted by $a_{ij}, x_{ij}$ etc.
\item Transposes are denoted by $\bm{x}^T, \bm{A}^T$ etc.
\item Inner product of $\bm{x}, \bm{y} \in \mathbb{R}^n$ is denoted by 
$(\bm{x}, \bm{y})$ and defined as $\sum_{i=1}^n x_iy_i$.
\item Any linear program can be written in the form
\begin{equation}\label{e1}
\text{Minimize } (\bm{c}, \bm{x}) \text{ subject to } \bm{A}\bm{x} \ge \bm{b}
\text{ and } \bm{x} \ge 0.
\end{equation}
\end{enumerate}
\end{frame}

\begin{frame}
\frametitle{Diet problem - the primal form}
\begin{enumerate}
\item We are given the daily needs $b_1, 
\ldots b_m$ of $m$ nutrients available in $n$ foods. The availability of $i$th
nutrient in $j$th food is denoted by $a_{ij}$ and the cost of food $i$th food
is given by $c_1, \ldots, c_n$. If $x_1, \ldots, x_n$ is the quantity of food
to be consumed daily then we want to
\begin{equation}\label{e2}
\text{Minimize } c_1x_1 + \cdots + c_nx_n = (\bm{c}, \bm{x}) \text{ subject to,}
\end{equation}
\begin{eqnarray}
a_{11}x_1 + \cdots + a_{1n}x_n &\ge& b_1 \label{e3} \\
\vdots & & \vdots \nonumber \\
a_{m1}x_1 + \cdots + a_{mn}x_n &\ge& b_m \label{e4} \\
x_1, \ldots, x_n &\ge& 0 \label{e5}
\end{eqnarray}
\item Equations \eqref{e3} to \eqref{e5} can be summarised as $\bm{A}\bm{x} \ge
\bm{b}$ and $\bm{x} \ge 0$.
\end{enumerate}
\end{frame}

\begin{frame}
\frametitle{Diet problem - the dual form}
\begin{enumerate}
\item The primal form of the diet problem was stated from the buyer's point of
view. How much of each food type should he buy to meet his nutritional needs?
\item There is a related problem from the seller's side as well. Let $y_1, 
\ldots, y_m$ be the prices of nutrients. A buyer needs $b_1, \ldots, b_m$ of 
them so that the seller's revenue is $y_1b_1 + \cdots + y_mb_m$. The buyer's 
cost preferences are
\begin{eqnarray}
y_1a_{11} + \cdots + y_ma_{m1} &\le& c_1 \label{e6} \\
\vdots & & \vdots \nonumber \\
y_ma_{1n} + \cdots + y_ma_{mn} &\le& c_n \label{e7}
\end{eqnarray}
\item In matrix form the objective function is $(\bm{y}, \bm{b})$ and the 
constraints of equations \eqref{e6} to \eqref{e7} are
\begin{equation}\label{e8}
\bm{y}^T\bm{A} \le \bm{c}.
\end{equation}
\end{enumerate}
\end{frame}

\begin{frame}
\frametitle{Diet problem - primal and dual forms}
\begin{enumerate}
\item Primal problem is to minimise $(\bm{c}, \bm{x})$ subject to 
$\bm{A}\bm{x} \ge \bm{b}$ and $\bm{x} \ge 0$.
\item Dual problem is to maximise $(\bm{y}, \bm{b})$ subject to 
$\bm{y}^T\bm{A} \le \bm{c}$ and $\bm{y} \ge 0$.
\item The primal form seeks to minimise total cost as long as the nutritional
needs are met.
\item The dual form seeks to maximise the profit as long as the prices do not 
become unaffordable to the buyer.
\item The solution of the primal problem is the quantity of food.
\item The solution of the dual problem is the price of food.
\item Quantity and price of food are constrained to be non-negative.
\end{enumerate}
\end{frame}

\begin{frame}
\frametitle{The primal and dual forms}
\begin{enumerate}
\item Every linear program has an associated dual program.
\item The two are not identical but they are closely related to each other.
They have the same set of parameters $\bm{A}, \bm{b}, \bm{c}$ but their 
decision variables, objective functions and goals are different. If the primal
is a minimisation problem then the dual is a maximisation problem.
\item The Simplex algorithm that solves the primal problem also gives the
solution to the dual problem.
\item The dual of the dual problem is the primal problem. Therefore, it is
sometimes simpler to solve the dual problem instead of the primal one.
\item The idea of duality is not confined to linear programs. It is applicable 
to a wide class of optimisation problems. However, it is simplest (and perhaps
most profitable) in the case of linear programs.
\end{enumerate}
\end{frame}

\begin{frame}
\frametitle{Duals of LPs in other forms}
\begin{enumerate}
\item Consider the LP minimising $(\bm{c}, \bm{x})$ subject to $\bm{A}\bm{x} =
\bm{b}$ and $\bm{x} \ge 0$. The equality constraint can be written as 
\begin{eqnarray}
\bm{A}\bm{x} &\ge& \bm{b} \label{e9} \\
-\bm{A}\bm{x} &\ge& -\bm{b} \label{e10}
\end{eqnarray}
which is same as
\begin{equation}\label{e11}
\begin{bmatrix}\bm{A} \\ -\bm{A}\end{bmatrix}\bm{x} \ge 
\begin{bmatrix}\bm{b} \\ -\bm{b}\end{bmatrix}
\end{equation}
\item Its dual is to maximise $(\bm{y}, \bm{b}^\prime)$ where $\bm{b}^\prime =
[\bm{b}\;-\bm{b}]^T$ and $\bm{y} \in \mathbb{R}^{2n}$ if $\bm{x} \in 
\mathbb{R}^n$. The constraints are
\begin{equation}\label{e12}
\bm{y}^T\begin{bmatrix}\bm{A} \\ -\bm{A}\end{bmatrix} \le \bm{c}
\end{equation}
and $\bm{y} \ge 0$.
\end{enumerate}
\end{frame}

\begin{frame}
\frametitle{Mathematical properties of duality - 1}
\begin{enumerate}
\item {[Weak duality lemma]} If $\bm{x}$ and $\bm{y}$ are feasible solutions to 
the primal and dual LPs then $(\bm{c}, \bm{x}) \ge (\bm{y}, \bm{b})$.
\item It states that cost of a feasible solution of the primal program is at 
least as large as the `revenue' of a feasible solution. If you get the solution
of the primal program then you know that the solution of the dual program
cannot be greater than it. On the other hand, if you find the solution of the
dual problem then you are guaranteed that the solution of the primal problem
will not be smaller than it. (By greater and smaller solutions we mean the 
values of the respective objective functions.)
\item If one of the problems is unbounded then the other has no feasible 
solution.
\item The solution of the primal problem inches towards that of the dual problem
as they iterate towards the optimal values.
\end{enumerate}
\end{frame}

\begin{frame}
\frametitle{Mathematical properties of duality - 2}
\begin{enumerate}
\item It can also be shown that if $\bm{x}^\ast$ and $\bm{y}^\ast$ are feasible
solutions of the primal and the dual theorems such that $(\bm{c}, \bm{x}^\ast) =
(\bm{y}^\ast, \bm{b})$ then $\bm{x}^\ast$ and $\bm{y}^\ast$ are the optimal 
solutions of the respective problems.
\item The converse of the above statement is also true. If the primal problem
has an optimal value then so does the dual problem. Moreover, their objective
functions take the same value.
\item An immediate application of this theorem is to the Simplex algorithm 
itself. It is computationally cheaper to solve whichever problem has fewer
constraints. This is true for the straightforward implementation of the Simplex
algorithm. 
\item The close relationship between the primal and the dual form also permit an
easy interpretation of the coefficients of the objective function. They appear 
in the constraint equations of the dual problem.
\end{enumerate}
\end{frame}

\begin{frame}
\frametitle{Sensitivity analysis}
\begin{enumerate}
\item Practical optimisation problems emerge from a complicated business setting.
\item The objective function and the constraints are known only approximately.
Often times, even their statistical properties are not known.
\item It is important to understand how good is our solution if our knowledge of
the problem's parameters is imperfect.
\item The idea of `stable solutions' is an important concept in applied 
mathematics with very deep ramifications. A solution is said to be stable if it
changes only slightly if the parameters of the problem are changed slightly. 
\item Stable solutions are `useful' and stable solutions are `observed in 
nature'.
\end{enumerate}
\end{frame}

\begin{frame}
\frametitle{Sensitivity analysis}
\begin{enumerate}
\item Yet, sensitivity analysis is not confined to stability analysis alone.
\item Practical optimisation problems involve thousands of decision variable and
hundreds of constraints. It is important to understand which of these are most
important.
\item The most important parameters are those for which a slight change in their
values results in a solution being sub-optimal or even infeasible. Even a change 
in the solution is not to be discounted.
\item A solution to an optimisation problem is often a business plan. For 
example, it may determine how much of a product must be produced or how should 
be the schedule of flights. One cannot change these plans just because you
discoverd a newer solution.
\end{enumerate}
\end{frame}

\begin{frame}
\frametitle{Sensitivity analysis}
\begin{enumerate}
\item Let us consider situation where we want to carry our sensitivity analysis
for the all the coefficients of the decision variables in the objective function.
Suppose further that for each one of them we want to consider a value lower and
a value higher then the one we started with. 
\item Each coefficient has $3$ values. If there are $100$ decision variables then
there are $3^{100}$ possibilities. This is an extremely large number.
\item One must rely on domain knowledge and a deeper understanding of the problem
space to identify the variables that are either important or about which our 
knowledge is shaky. Otherwise, a systematic examination of all variables is 
possible only in the simplest problems.
\item Sensitivity analysis is also called `what-if analysis' in Finance. For 
several optimsation problems in finance the practioners know which one 
parameters are the most uncertain and `what-if' analysis is restricted to only
these.
\end{enumerate}
\end{frame}
\end{document}