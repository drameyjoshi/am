\documentclass{article}
\usepackage{amsmath, amsthm, amssymb, amsfonts, graphicx, color}
\usepackage{mathtools}
\usepackage{hyperref}
\usepackage{mathabx}

\newcommand{\td}[2]{\frac{d{#1}}{d{#2}}}
\newcommand{\soc}{\mathbb{C}}
\newcommand{\sor}{\mathbb{R}}
\newcommand{\son}{\mathbb{N}}
\newcommand{\op}{\prime}

\theoremstyle{plain}
\newtheorem{thm}{Theorem}
\numberwithin{thm}{section}

\theoremstyle{plain}
\newtheorem{prop}{Proposition}
\numberwithin{prop}{section}

\theoremstyle{definition}
\newtheorem{defn}{Definition}
\numberwithin{defn}{section}

\theoremstyle{remark}
\newtheorem*{rem}{Remark}

\theoremstyle{plain}
\newtheorem{cor}{Corollary}
\numberwithin{cor}{section}

\numberwithin{equation}{section}
\title{Measure Theory}
\author{Amey Joshi}
\date{26-Sep-2024}
\begin{document}
\maketitle
\section{$\sigma$-algebra}\label{s1}
\begin{defn}\label{s1d1}
Let $X$ be a non-empty set. A $\sigma$-algebra, $\Sigma$, on $X$ is a family
of subsets of $X$ such that:
\begin{enumerate}
\item $\Sigma$ is non-empty,
\item If $E \in \Sigma$ then $E^c \in \Sigma$ and
\item If $E_1, E_2, \ldots \in \Sigma$ then $\cup_{n \in \mathbb{N}}E_n \in \Sigma$. 
\end{enumerate}
\end{defn}
\begin{prop}\label{s1p1}
If $\Sigma$ is a $\sigma$-algebra on $X$ then $X, \varnothing \in \Sigma$.
\end{prop}
\begin{proof}
Since $\Sigma$ is non-empty, it has at least one member, say $A \subset X$. 
Therefore, $A^c \in \Sigma$ and hence $A \cup A^c = X \in \Sigma$. 
Therefore, $X^c = \varnothing \in \Sigma$.
\end{proof}

\begin{prop}\label{s1p2}
A $\sigma$-algebra $\Sigma$ on $X$ is closed under countable intersections.
\end{prop}
\begin{proof}
Let $E_1, E_2, \ldots \in \Sigma$. Then,
\[
\bigcap_{n \in \mathbb{N}}E_n = \left(\bigcup E_n^c\right)^c.
\]
Since $E_n \in \Sigma$, $E_n^c$ are also in $\Sigma$, so is their
countable union and its complement.
\end{proof}

\begin{defn}\label{s1d2}
If $\Sigma$ is a $\sigma$-algebra on $X$ then the pair $(X, \Sigma)$
is called a measurable space and the members of $\Sigma$ are called measurable
sets.
\end{defn}

A topology is also a family of subsets of a set. It is defined as
\begin{defn}\label{s1d3}
Let $X$ be a non-empty set. A collection $\tau$ of subsets of $X$ is a topology
if:
\begin{enumerate}
\item $\varnothing, X \in \tau$,
\item An arbitrary union of members of $\tau$ is a member of $\tau$.
\item An intersection of a finite number of members of $\tau$ is a member of $\tau$.
\end{enumerate}
The pair $(X, \tau)$ is called a topological space and members of $\tau$ are
called open sets.
\end{defn}
The differences between the two families are:
\begin{enumerate}
\item A $\sigma$-algebra is closed under complements, a topology is not.
\item A $\sigma$-algebra is closed under countable unions, a topology is closed
under arbitrary unions.
\item A $\sigma$-algebra is closed under countable intersections while a topology
is closed under finite intersections.
\end{enumerate}

\subsection{Examples and non-examples of $\sigma$-algebras}
\begin{enumerate}
\item For any non-empty set $X$, $2^X$ and $\{\varnothing, X\}$ are $\sigma$-
algebras. They are called trivial $\sigma$-algebras.
\item If $X = \{1, 2, 3\}$ then $\{\varnothing, \{1\}, \{2, 3\}, \{1, 2, 3\}\}$ is
a $\sigma$-algebra. It is said to be generated by $\{1\}$ because it is the 
smallest $\sigma$-algebra containing $\{1\}$. 

Note that there are $2^3 = 8$ subsets of $X$. Therefore, there are $2^8 = 256$
subsets of $2^X$.
\item If $\bar{\sor} = [-\infty, \infty]$ then $\{(a, b)\;|\; a, b\in\bar{\sor}\}$
is not a $\sigma$-algebra. This is because the complement of $(a, b)$ is not an open
interval. For the same reason, the set of all closed intervals is also not a
$\sigma$-algebra. (The set of all open intervals is not even closed under unions.)
\end{enumerate}

\begin{prop}\label{s1p3}
Let $\Sigma_1$ and $\Sigma_2$ be $\sigma$-algebras on $X$. Then their
intersection is also a $\sigma$-algebra on $X$.
\end{prop}
\begin{proof}
Let $\Sigma = \Sigma_1 \cap \Sigma_2$. Since both $\Sigma_1$ and 
$\Sigma_2$ contain $X$ and $\sigma$. $\Sigma$ is non-empty. If $A \in
\Sigma$ then $A \in \Sigma_1$ and $A \in \Sigma_2$, so that $A^c 
\in \Sigma_1$ and $A^c \in \Sigma_2$, that is $A^c \in \Sigma$. 
Similarly, if $E_1, E_2, \ldots \in \Sigma$ then these sets are also in 
$\Sigma_1$ and $\Sigma_2$ so that their countable union is in both
$\sigma$-algebras and therefore also in their intersection.
\end{proof}

\begin{defn}\label{s1d4}
Let $A \subset 2^X$. Then an intersection of all $\sigma$-algebras containing $A$
is called the $\sigma$-algebra generated by $A$. It is also the smallest $\sigma$-
algebra containing $A$. It is denoted by $\langle A \rangle$.
\end{defn}

\begin{defn}\label{s1d5}
Let $(X, \tau)$ be a topological space. Then $\langle\tau\rangle$ is called the
Borel $\sigma$-algebra. It is denoted by $\mathcal{B}(X)$.
\end{defn}

$\mathcal{B}(X)$ contains all open sets, closed sets and their countable unions
and intersections.

\begin{defn}\label{s1d6}
Let $(X, \Sigma)$ be a measurable space. A measure on it is a mapping
$\mu: \Sigma \rightarrow [0, \infty]$ such that
\begin{enumerate}
\item $\mu(\varnothing) = 0$,
\item If $E_1, E_2, \ldots \in \Sigma$ are mutually disjoint then
\[
\mu\left(\bigcup_{n \in \mathbb{N}}E_n\right) = \sum_{n \in \mathbb{N}}\mu(E_n).
\]
The triple $(X, \Sigma, \mu)$ is called a measure space.
\end{enumerate}
\end{defn}

Examples of measure:
\begin{itemize}
\item Let $X$ be an arbitrary set with a $\sigma$-algebra $\Sigma$. Then,
\[
\mu(A) = \begin{cases} |A| & \text{ if } $A$ \text{ is a finite set} \\
\infty & \text{ otherwise.}
\end{cases}
\]
is called a counting measure.

\item One the measure space $(\sor, \mathcal{B}(\sor))$ define a 
measure $\mu$ such that
\begin{eqnarray*}
\mu([a, b]) &=& b - a \\
\mu((a, b)) &=& b - a \\
\mu((a, b]) &=& b - a \\
\mu([a, b)) &=& b - a.
\end{eqnarray*}
It is called the Lebesgue measure. Note that $(a, b] = [a, b] \cap (a, \infty)$
and $[a, b) = [a, b] \cap (-\infty, b)$ so that semi-open intervals are members
of the Borel $\sigma$-algebra $\mathcal{B}(\son)$.

\item One the measure space $(\sor, \mathcal{B}(\sor))$ define a 
measure $P$ such that
\[
P(B) = \frac{1}{\sqrt{2\pi}}\int_B e^{-x^2/2}dx,
\]
is called the probability measure.
\end{itemize}

\begin{prop}\label{s1p4}Let $(X, \Sigma, \mu)$ be a measure space. Then,
\begin{enumerate}
\item $\mu$ is finitely additive.
\item $\mu$ is monotonic.
\item If $A \subseteq B$ has a finite measure then $\mu(B - A) = \mu(B) - \mu(A)$.
\item If $A$ or $B$ has a finite measure then so does $A \cap B$ and
\item If $A$ or $B$ has a finite measure then $\mu(A \cup B) = \mu(A) + \mu(B) -
\mu(A \cap B)$.
\end{enumerate}
\end{prop}
\begin{proof}
Let $A_1, \ldots A_n \in \Sigma$ be mutually disjoint. Then, 
\[
\mu\left(\bigcup_{i=1}^n A_i\right) = 
\mu\left( A_1 \cup \left(\bigcup_{i=2}^n A_i\right)\right) = 
\mu(A_1) + \mu\left(\bigcup_{i=2}^n A_i\right)
\]
Carrying out this step $n - 2$ times more, we get
\[
\mu\left(\bigcup_{i=1}^n A_i\right) = \sum_{i=1}^n \mu(A_i).
\]
This is called the finite additivity of $\mu$.

Let $A, B \in \Sigma$ be such that $A \subseteq B$. Then $B = A \cup (B - A)$
and the sets $A$ and $B - A$ are disjoint. Therefore, $\mu(B) = \mu(A) + 
\mu(B - A) \ge \mu(A)$.
This is called the monotonicity of $\mu$. Further, when $A \subset B$, and 
$\mu(A)$ is finite, we can subtract it from both sides of the equation to get
$\mu(B - A) = \mu(B) - \mu(A)$.

Without loss of generality, let $A$ have finite measure. Then $A = (A \cap B^c)
\cup (A \cap B)$ where the union is disjoint. Therefore, $\mu(A) = \mu(A \cap B^c)
+ \mu(A \cap B)$. Since measure is a non-negative quantity and lhs is finite, it
immediately follows that $A \cap B$ (and $A \cap B^c$) has finite measure.

Note the following set identities,
\begin{eqnarray*}
A &=& (A \cap B^c) \cup (A \cap B) \\
B &=& (A^c \cap B) \cup (A \cap B) \\
A \cup B &=& (A \cap B^c) \cup (A \cap B) \cup (A^c \cap B),
\end{eqnarray*}
where all the unions are among disjoint sets. Therefore,
\begin{eqnarray*}
\mu(A) &=& \mu(A \cap B^c) \cup \mu(A \cap B) \\
\mu(B) &=& \mu(A^c \cap B) \cup \mu(A \cap B) \\
\mu(A \cup B) &=& \mu(A \cap B^c) \cup \mu(A \cap B) \cup \mu(A^c \cap B),
\end{eqnarray*}
If one of $A$ of $B$ has a finite measure then $A \cap B$ will have a finite
measure permitting us to subtract $\mu(A \cap B)$ from the equation obtained
by summing the first two equations to get
\[
\mu(A \cup B) = \mu(A) + \mu(B) - \mu(A \cap B).
\]
\end{proof}

\begin{prop}\label{s1p5}Let $(X, \Sigma, \mu)$ be a measure space and let
$E_1 \subseteq E_2 \subseteq \ldots$ be such that
\[
\bigcup_{n \ge 1}E_n = E.
\]
Then,
\[
\mu(E) = \lim_{n \rightarrow \infty}\mu(E_n).
\]
\end{prop}
\begin{proof}
If $E_0 = \varnothing$ then for a finite $n$, $E_n = (E_n - E_{n-1}) \cup 
(E_{n-1} - E_{n-2}) \cup \cdots (E_2 - E_1) \cap (E_1 - E_0)$ while for $E$ 
itself,
\[
E = \bigcup_{n >= 0} (E_{n} - E_{n-1}).
\]
The sets $E_{n} - E_{n-1}$ are disjoint for every $n$ so that
\begin{eqnarray*}
\mu(E) &=& \sum_{n \ge 1} \mu(E_n - E_{n-1}) \\
  &=& \lim_{n \rightarrow \infty}\sum_{k=1}^n \mu(E_k - E_{k-1}) \\
  &=& \lim_{n \rightarrow \infty}\mu\left(\bigcup_{k=1}^n (E_k - E_{k-1})\right) \\
  &=& \lim_{n \rightarrow \infty}\mu(E_n)
\end{eqnarray*}
\end{proof}

\begin{prop}\label{s1p6}Let $(X, \Sigma, \mu)$ be a measure space and
$E_1, E_2, \ldots \in \Sigma$. Then
\[
\mu\left(\bigcup_{n \ge 1}E_n\right) \le \sum_{n \ge 1}\mu(E_n).
\]
\end{prop}
\begin{proof}
We will prove by induction on the number of sets $E_n$. If there are two sets
then, using item 5 of proposition \ref{s1p4},
\[
\mu(E_1 \cup E_2) = \mu(E_1) + \mu(E_2) - \mu(E_1 \cap E_2) \le \sum_{k=1}^2\mu(E_k).
\]
Let the induction hypothesis be true for $p$ sets. Consider
\[
\mu\left(\bigcup_{k \ge 1}^{p+1}E_k\right) = 
\mu\left(\bigcup_{k \ge 1}^{p}E_k \cup E_{p+1}\right) \le 
\mu\left(\bigcup_{k \ge 1}^{p}E_k\right) + \mu(E_{p+1})
\]
by induction hypothesis. Use the induction hypothesis once again for the first
term on rhs to get
\[
\mu\left(\bigcup_{k \ge 1}^{p+1}E_k\right) \le \sum_{k=1}^{p+1}\mu(E_k).
\]
Therefore, the hypothesis is true for all $p \in \mathbb{N}$.
\end{proof}

\begin{prop}\label{s1p7}Let $(X, \Sigma, \mu)$ be a measure space and let
$E_1 \supseteq E_2 \supseteq \ldots$ be such that 
\[
\bigcap_{n \ge 1}E_n = E
\]
and $\mu(E_1) < \infty$ then
\[
\lim_{n \rightarrow \infty}\mu(E_n) = \mu(E).
\]
\end{prop}
\begin{proof}
In this proof $E_k^c = E_1 - E_k$ for all $k \ge 1$, that is, complementation is
with respect to $E_1$. Since $E_1 \supseteq E_2 \supseteq \ldots$ and $\cap E_k 
= E$, we also have $E_1^c \subseteq E_2^c \subseteq \ldots$ and $\cup E_k^c = E^c$.
Therefore, by proposition \ref{s1p5},
\[
\mu(E^c) = \lim_{n \rightarrow \infty}\mu(E_n^c).
\]
Now, $\mu(E) + \mu(E^c) = \mu(E_1)$ and $\mu(E_n) + \mu(E_n^c) = \mu(E_1)$ so that
\[
\mu(E_1) - \mu(E) = \lim_{n \rightarrow \infty}(\mu(E_1) - \mu(E_n)).
\]
Since $\mu(E_1) < \infty$, the proposition follows immediately.
\end{proof}

\section{Measurable maps}\label{s2}
\begin{defn}\label{s2d1}
Let $(X, \Sigma_X)$ and $(Y, \Sigma_Y)$ be measurable spaces. A map $f: X
\rightarrow Y$ is said to be measurable if for all $B \in \Sigma_Y$, the set
$f^{-1}(B)$ is a member of $\Sigma_X$.
\end{defn}

A trivial example of a measurable map is $f(x) = c$ for all $x \in X$ where $c$
is a constant. For all $B$ which contain $c$, $f^{-1}(B) = X$ while for those 
which do not containt $c$, $f^{-1}(B) = \varnothing$.

\begin{prop}\label{s2p1}
A composition of two measurable maps is measurable.
\end{prop}
\begin{proof}
Let $(X, \Sigma_X), (Y, \Sigma_Y)$ and $(Z, \Sigma_Z)$ be measurable spaces and
$f:X \rightarrow Y, g: Y \rightarrow Z$ be measurable maps. Let $h = g \circ f$
and $C$ be an arbitrary member of $\Sigma_Z$. Then $h^{-1}(C) = f^{-1}(g^{-1}(C))$.
Since $g$ is measurable, $B = g^{-1}(C)$ is a member of $\Sigma_Y$. Since $f$ is
measurable, $f^{-1}(B)$ is a member of $\Sigma_X$.
\end{proof}

\begin{prop}\label{s2p2}
Let $(X, \Sigma_X)$ and $(Y, \Sigma_Y)$ be measurable spaces and let 
$\mathcal{H}$ generate $\Sigma_Y$. A map $f: X \rightarrow Y$ is measurable iff 
for all $V \in \mathcal{H}$, $f^{-1}(V) \in \Sigma_X$.
\end{prop}
\begin{proof}
If $V \in \mathcal{H}$ then $V \in \Sigma_Y$. Since $f$ is measurable, 
$f^{-1}(V) \in \Sigma_X$.

Let $\mathcal{G} = \{f^{-1}(V) \;\forall\; V \in \mathcal{H}\}$ and let 
$\Sigma_Y^\prime = \{V \in \Sigma_Y \;|\; f^{-1}(V) \in \mathcal{G}\}$. By
hypothesis, $\mathcal{G} \subseteq \Sigma_X$. We will show that:
\begin{enumerate}
\item $\Sigma_Y^\prime$ is a $\sigma$-algebra: Since $\Sigma_Y$ is a $\sigma$-
algebra, it has at least one member so that $\Sigma_Y^\prime$ is non-empty.
Let $V \in \Sigma_Y^\prime$. Then $V^c$ is a member of every $\sigma$-algebra
containing $\mathcal{H}$. In particular, $V^c \in \Sigma_Y^\prime$. If $E_1, E_2,
\ldots \in \mathcal{H}$ then their countable union is in every $\sigma$-algebra
containing $\mathcal{H}$ and in particular in $\Sigma_Y^\prime$. Thus, 
$\Sigma_Y^\prime$ is also a $\sigma$-algebra containing $\mathcal{H}$. 

\item Since $f^{-1}(V) \in \mathcal{G}$ for all $V \in \Sigma_Y^\prime$, 
$\mathcal{H}$ is a member of $\Sigma_Y^\prime$.
\end{enumerate}

Thus $\Sigma_Y^\prime$ is a $\sigma$-algebra containing $\mathcal{H}$. Therefore,
$\Sigma_Y^\prime \supseteq \Sigma_Y$ such that the pre-image of every member of
$\Sigma_Y^\prime$ is in $\Sigma_X$. Therefore, the pre-image of every member of 
$\Sigma_Y$ also lies in $\Sigma_X$, making $f$ a measurable map.
\end{proof}

\begin{defn}\label{s2d2}
Let $(X, \tau_X)$ and $(Y, \tau_Y)$ be topological spaces and $f: X \rightarrow Y$
be such that $f^{-1}(V) \in \tau_X$ for all $V \in \tau_Y$. Then $f$ is called a
continuous map.
\end{defn}

\begin{cor}\label{s2c1}
If $(X, \tau_X)$ and $(Y, \tau_Y)$ are topological spaces and $f: X \rightarrow Y$
is continuous then $f$ is measurable on $(X, \langle\tau_X\rangle)$ and $(Y, \langle
\tau_Y\rangle)$.
\end{cor}
\begin{proof}
Follows immediately from proposition \ref{s2p2} after identifying $\mathcal{H}$
with $\tau_Y$.
\end{proof}

\begin{cor}\label{s2c2}
Let $(X, \Sigma_X)$ be a measurable space. A function $f: X \rightarrow \sor$
is measurable iff $f^{-1}((a, \infty]) \in \Sigma_X$ for all $a \in \sor$.
\end{cor}
\begin{proof}
The $\sigma$-algebra on $\sor$ is $\mathcal{B}$. The corollary follows immediately
from proposition \ref{s2p2} if we can show that the set of all open intervals
can be generated from the set of all semi-open intervals by set operations used
to define a $\sigma$-algebra. If $a < b$, then $(a, b) = (a, \infty] \cap (b,
\infty]$.
\end{proof}

\begin{cor}\label{s2c3}
Let $(X, \Sigma_X)$ be a measurable space. A function $f: X \rightarrow \sor$
is measurable iff $f^{-1}([-\infty, a) \in \Sigma_X$ for all $a \in \sor$.
\end{cor}
\begin{proof}
If $a < b$, $(a, b) = [-\infty, a)^c \cap [-\infty, b)^c$. Therefore, all open
intervals can be generated from semi-open intervals by the operations used to
define a $\sigma$-algebra.
\end{proof}

\begin{prop}\label{s2p3}
Let $f: X \rightarrow \sor$ be a measuable function. Then so is $cf$, where $c$
is a non-zero real number.
\end{prop}
\begin{proof}
If $g = cf$ then $g^{-1}(A) = \{x/c \;|\; x \in f^{-1}(A)\}$ for any semi-open 
interval $A$ in $\mathbb{R}$. The proposition follows from \ref{s2c2}.\
\end{proof}

\begin{rem}
I don't think it is easy to show that if $A$ is an arbitrary measurable subset 
of $\sor$ then so is $\{x/c \;|\; x \in f^{-1}(A)\}$. I can easily show it only 
for an interval of any kind.
\end{rem}

\begin{prop}\label{s2p4}
Let $\{f_n\}$ be a sequence of measurable real valued functions. Then the functions
$\sup f_n, \inf f_n, \min f_n, \max f_n, \limsup f_n, \liminf f_n$ are also
measurable.
\end{prop}
\begin{proof}
Let $f_n: X \rightarrow \sor$ be a real-valued measurable functions for $n \in
\son$. If $\Sigma_X$ is a $\sigma$-algebra on $X$ then for all $(a, \infty]$,
$f_n^{-1}((a, \infty]) \in \Sigma_X$.
\begin{itemize}
\item Let $f = \sup f_n$. Then 
\[
f^{-1}((c, \infty]) = \bigcup_{n \ge 1}f_n^{-1}((c, \infty]).
\]
Since $f_n$ are measurable, $f_n^{-1}((c, \infty]) \in \Sigma_X$ for all $n$
and so is their countable union. Therefore, by corollary \ref{s2c2}, $f$ is 
measurable.

\item Let $f = \inf f_n$. Then 
\[
f^{-1}([-\infty, c)) = \bigcup_{n \ge 1}f_n^{-1}([-\infty, c)).
\]
Since $f_n$ are measurable, $f_n^{-1}([\infty, c)) \in \Sigma_X$ for all $n$
by corollary \ref{s2c3} and so is their countable union. Therefore, by corollary 
\ref{s2c3}, $f$ is measurable.

\item Let $f = \max f_n$, then $f$ is also the supremum.

\item Let $f = \inf f_n$ then $f$ is also the infimum.

\item Let $f = \limsup f_n$ then $f = \inf\{\sup\{f_m : m \ge n\} n \ge 1\}$.
Since $\sup\{f_m : m \ge n\}$ is measurable for all $n$, and the infimum is
also measurable, $f$ is measurable.

\item Let $f = \liminf f_n$ then $f = \sup\{\inf\{f_m : m \ge n\} n \ge 1\}$.
For the same reasons as mentioned in the previous case, $f$ is measurable.

\end{itemize}
\end{proof}

For sake of reference, we prove
\begin{prop}\label{s2p5}
Let $A \subseteq \sor$. If $a \in A$ is the maximum element of $A$ then it is also
the supremum of $A$.
\end{prop}
\begin{proof}
Let $b$ be a supremum of $A$ then $a \le b$. Since $a$ is a maximum, $a \ge b$.
Therefore, $a = b$.
\end{proof}

\begin{prop}\label{s2p6}
Let $f, g: \sor \rightarrow \sor$ be measurable functions. Then so are $f + g,
f - g, fg$ and $f/g$.
\end{prop}
\begin{proof}
We will prove each case separately.
\begin{itemize}
\item Let $s = f + g$ be the sum of two measurable functions. Then $s(x) > c$ 
iff $f(x) > r$ and $g(x) > c - r$ for some rational number $r$. The set
\[
s^{-1}((c, \infty]) = \bigcup_r f^{-1}((r, \infty]) \cap g^{-1}((c - r, \infty])
\]
Note that the set of rational numbers is countable. Since $f^{-1}((r, \infty])$
and $g^{-1}((c - r, \infty])$ are members of $\Sigma_\sor$, so is their 
intersection and the countable union of their intersection. Therefore, by 
corollary \ref{s2c2}, $s$ is measuable.

\item By proposition \ref{s2p3}, $-g$ is measurable. Using the previous point,
$f + (-g) = f - g$ is measuable.

\item Given a function $f: \sor \rightarrow \sor$, define,
\begin{eqnarray*}
f^+(x) &=& \max\{+f(x), 0\} \\
f^-(x) &=& \max\{-f(x), 0\}
\end{eqnarray*}
as the positive and negative parts of a function. In terms of these, $f = f^+
- f^-$ and $fg = f^+g^+ - f^+g^- - f^-g^+ + f^-g^-$. Each of these terms is a
product of positive valued function. Therefore, to prove the measurability of
$fg$, we can as well restrict ourselves to non-negative valued functions and use
the previous parts of this proposition.

Let $p = fg$ be the product of non-negative valued functions$f, g$. Now , $p(x)
> c$ iff $f(x) > r$ and $g(x) > c/r$ for every positive $r$. Furthermore,
\[
p^{-1}((c, \infty]) = \bigcup_r f^{-1}((r, \infty]) \cap g^{-1}((c/r, \infty])
\]

\item Let $q = 1/g$ be the reciprocal of a positive valued function $g$. Then
$q(x) > c$ iff $g(x) < 1/c$ for any positive real $c$. Thus,
\[
q^{-1}((c, \infty]) = g^{-1}([-\infty, c^{-1})).
\]
Since the right hand side is a measurable set, so is the left hand side, making
$q$ a measurable function.

Measurability of $f/g$ follows from the fact that $f/g = fq$.
\end{itemize}
\end{proof}

\begin{thm}\label{s1t1}
A real-valued function $f$ is measurable iff $f^+$ and $f^-$ are measurable. 
Moreover, if $f$ is measuable then so is $|f| = f^+ + f^-$.
\end{thm}
\begin{proof}
The `if' part follows from the previous proposition \ref{s2p6}. Assume that
$f$ is measurable. Then,
\[
{f^+}^{-1}((c, \infty]) = \begin{cases}
f^{-1}((c, \infty]) \text{ for } c \ge 0. \\
\varnothing \text{ for } c < 0
\end{cases}
\]
Both these sets are measurable making $f^+$ a measurable function. Since $f^- =
f^+ - f$, by the previous proposition, $f^-$ is also measurable. From this, the
measuarability of $f^+ + f^- = |f|$ also follows.
\end{proof}

\begin{defn}\label{s2d3}
The characteristic function of a subset $A$ of a set $X$ is defined as
$\chi_A: X \rightarrow \{0, 1\}$ and
\[
\chi_A(x) = \begin{cases} 1 & \text{ if } x \in A \\
0 & \text{ if } x \notin A
\end{cases}
\]
\end{defn}

\begin{prop}\label{s2p7}
Let $A \subseteq X$. $\chi_A$ is a measurable function iff $A$ is a measurable
set.
\end{prop}
\begin{proof}
$\chi_A: X \rightarrow \{0, 1\}$ is measurable iff $\chi_A^{-1}(1)$ and 
$\chi_A^{-1}(0)$ are measurable. But these sets are precisely $A$ and $A^c$.
\end{proof}

\section{Lebesgue integral}\label{s3}
\begin{defn}\label{s3d1}
A real valued simple function is of the form
\[
s(x) = \sum_{k=1}^n a_k \chi_{E_k}(x).
\]
\end{defn}

\begin{defn}\label{s3d2}
Let $(X, \Sigma_X, \mu)$ be a measure space and $s: X \rightarrow \sor$ be
a simple function. Then its Lebesuge integral over $X$ is defined as
\[
\int_Xs d\mu = \sum_{k=1}^na_k\mu(E_k),
\]
where $E_k \in \Sigma_X$ partition $X$.
\end{defn}

\begin{rem}
Note the way the integral is written. The variable $x$ over which the integration
is carried out is not mentioned. Instead the measure $\mu$ is emphasised.
\end{rem}

\begin{rem}
A simple function is expressed as a finite sum. Therefore, the integral too is
expressed as a finite sum.
\end{rem}

\begin{prop}\label{s3p1}
Let $A_1, \ldots, A_n$ and $B_1, \ldots, B_m$ be partitions of $X$. Then the sets
$\{C_{ij} = A_i \cap B_j \;|\; i = 1, \ldots, n; j = 1, \ldots m\}$ also partition
$X$.
\end{prop}
\begin{proof}
Any $x \in X$ belongs to exactly one $A_i$ and exactly one $B_j$ and therefore 
exactly one $C_{ij}$. Therefore,
\[
X \subseteq \bigcup_{i, j} C_{ij}.
\]
Conversely, if $x \in \cup_{ij}C_{ij}$ then $x$ is a member of at least one $C_{ij}$
and hence at least one $A_i$. Therefore,
\[
\bigcup_{i, j} C_{ij} \subseteq X.
\]
Therefore,
\[
\bigcup_{i, j} C_{ij} = X.
\]
That an $x \in X$ belongs to exactly one $C_{ij}$ follows from the definition of
$C_{ij}$ as an intersection of elements of two partitions of $X$.
\end{proof}

\begin{prop}\label{s3p2}
Let $A_1, \ldots, A_n$ and $B_1, \ldots, B_m$ be partitions of $X$ and
$\{C_{ij} = A_i \cap B_j \;|\; i = 1, \ldots, n; j = 1, \ldots m\}$. Then
\[
s = \sum_{i=1}^n\sum_{j=1}^m a_i \chi_{C_{ij}} = 
\sum_{i=1}^n\sum_{j=1}^m b_j \chi_{C_{ij}}.
\]
\end{prop}
\begin{proof}
Since $A_i = \cup_j C_{ij}$,
\[
s = \sum_{i=1}^na_i\chi_{A_i} = \sum_{i=1}^n\sum_{j=1}^m a_i\chi_{C_{ij}}.
\]
Similarly,
\[
B_j = \cup_i C_{ij} \Rightarrow s = \sum_{i=1}^n\sum_{j=1}^m b_j\chi_{C_{ij}}.
\]
\end{proof}

\begin{prop}\label{s3p3}
Let $X$ be partitioned as $X = \cup_k A_k$ and $X = \cup_k B_k$. If
\[
s(x) = \sum_{k=1}^na_k\chi_{A_k} = \sum_{k=1}^mb_k\chi_{B_k}
\]
then
\[
\int_Xs d\mu = \sum_{k=1}^na_k\mu(A_k) = \sum_{k=1}^mb_k\mu(B_k).
\]
That is, the Lebesuge integral is independent of the way $X$ is partitioned.
\end{prop}
\begin{proof}
Let $C_{ij} = A_i \cap B_j$. We showed in proposition \ref{s3p1} that $\{C_{ij}\}$
partition $X$. Further,
\[
A_i = \cup_{j=1}^m C_{ij} \Rightarrow \mu(A_i) = \sum_{j=1}^m C_{ij}.
\]
Similarly,
\[
B_j = \cup_{i=1}^n C_{ij} \Rightarrow \mu(B_j) = \sum_{i=1}^n C_{ij}.
\]
Therefore,
\[
\int s d\mu = \sum_{i=1}^na_i\mu(A_i) = \sum_{i=1}^n a_i\sum_{j=1}^m C_{ij}.
\]
Using proposition \ref{s3p2}, the last term on the right hand side can be written
as
\[
\int s d\mu = \sum_{i=1}^na_i\mu(A_i) = \sum_{i=1}^n\sum_{j=1}^m b_j\chi_{C_{ij}}
= \sum_{j=1}^m b_j\chi_{B_j}.
\]
\end{proof}

\begin{prop}\label{s3p4}
Let $s$ and $ t$ be simple functions over a measure space $(X, \Sigma_X, \mu)$
such that $s \le  t$ then
\[
\int_Xs d\mu \le \int_X t d\mu.
\]
\end{prop}
\begin{proof}
If the two functions are not defined on the same partition we can redefine them
on the intersection of the two partitions. That is, let,
\[
s = \sum_{k=1}^n a_k\chi_{E_k} \text{ and }  t = \sum_{k=1}^n b_k\chi_{E_k}.
\]
Then $s \le  t \Rightarrow a_k \le b_k$ and hence, using the definition
\ref{s3d2} of the Lebesgue integral we conclude that
\[
\int_Xs d\mu \le \int_X t d\mu.
\]
\end{proof}

\begin{prop}\label{s3p5}
If $s_1, s_2$ are simple functions of a measure space $(X, \Sigma_X, \mu)$
and $c_1, c_2 \in \sor$ then
\[
\int_X cs_1 d\mu = c\int_Xs_1 d\mu
\]
and
\[
\int_X(s_1 + s_2)d\mu = \int_Xs_1 d\mu + \int_Xs_2 d\mu.
\]
\end{prop}
\begin{proof}
Let the collection of subsets $\{E_k\}$ of $X$ partition $X$ and let them be the
common partition over which $s_1$ and $s_2$ are defined. Let
\begin{eqnarray*}
s_1 &=& \sum_{k=1}^na_k\chi_{E_k} \\
s_1 &=& \sum_{k=1}^nb_k\chi_{E_k} 
\end{eqnarray*}
so that
\begin{eqnarray*}
cs_1 &=& \sum_{k=1}^n ca_k\chi_{E_k} \\
s_1 + s_2 &=& \sum_{k=1}^n(a_k + b_k)\chi_{E_k}
\end{eqnarray*}
so that
\[
\int_X cs_1 d\mu = \sum_{k=1}^n ca_k\mu(E_k) = c\sum_{k=1}^na_k\mu(E_k)
= c\int_Xs_1 d\mu.
\]
and
\[
\int_X(s_1 + s_2)d\mu = \sum_{k=1}^n(a_k + b_k)\mu(E_k) = 
\int_Xs_1 d\mu + \int_Xs_1 d\mu.
\]
\end{proof}

\begin{defn}
Let $(X, \Sigma_X, \mu)$ be a measure space and $f:X \rightarrow [0, \infty]$ be
a measurable function. Then
\[
\int_X f d\mu = 
\sup\left\{\int_X sd\mu \;|\; 0 \le s \le f \text{ and } s \text{ is simple.}\right\}.
\]
\end{defn}

We will now show that for any $f:X \rightarrow \sor$ where $(X, \Sigma_X, \mu)$
is a measure space, there exists a sequence of simple functions which approximate
$f$ as closely as we wish. For $n = 0, 1, 2, \ldots$ and $0 \le k \le 2^{2n} - 1$,
let
\begin{eqnarray*}
E_n^k &=& f^{-1}\left(\left(\frac{k}{2^n}, \frac{k+1}{2^n}\right]\right) \\
F_n &=& f^{-1}((2^n, \infty]).
\end{eqnarray*}
$F_n$ is the subset of $X$ which is mapped to $(2^n, \infty]$ while $E_n^k$ is
the subset of $X$ which is mapped to the interval 
\[
\left(\frac{k}{2^n}, \frac{k+1}{2^n}\right].
\]
We define $s_n$ as 
\begin{equation}\label{s3e1}
s_n = \sum_{k=0}^{2^{2n}-1}\frac{k}{2^n}\chi_{E_n^k} + 2^n\chi_{F_n}.
\end{equation}
The first few members of this sequence are
\begin{equation}\label{s3e2}
s_0 = \chi_{F_0},
\end{equation}
where $F_1 = f^{-1}((1, \infty])$,
\begin{equation}\label{c3e3}
s_1 = \frac{1}{2}\chi_{E_1^1} + \chi_{E_1^2} + \frac{3}{2}\chi_{E_1^3} + 2\chi_{F_1},
\end{equation}
where $F_2 = f^{-1}((2, \infty])$ and
\begin{eqnarray*}
E_1^1 &=& f^{-1}\left(\left(\frac{1}{2}, 1\right]\right) \\
E_1^2 &=& f^{-1}\left(\left(1, \frac{3}{2}\right]\right) \\
E_1^3 &=& f^{-1}\left(\left(\frac{3}{2}, 2\right]\right)
\end{eqnarray*}
Likewise,
\begin{equation}\label{s3e3}
s_2 = \sum_{k=0}^{15}\frac{k}{4}\chi_{E_2^k} + 4\chi_{F_2},
\end{equation}
where
\begin{eqnarray*}
F_2 &=& 4f^{-1}((4, \infty]) \\
E_2^k &=& f^{-1}\left(\frac{k}{4}, \frac{k+1}{4}\right)
\end{eqnarray*}
In a set like $s_2$, the $15$ terms of the sum are the $15$ rectangles which
approximate the area under $f$. Each rectangle has `width' $\mu(E_2^k)$ and
`height' $k/4$ while the last term has `width' $\mu(F_2)$ and `height' $4$.
We prove two properties of the sequence $\{s_n\}$ of simple functions.
\begin{prop}\label{s3p6}
$\{s_n\}$ is an monotone increasing and non-negative.
\end{prop}
\begin{proof}
Consider
\begin{eqnarray*}
s_n &=& \sum_{k=0}^{2^{2n}-1}\frac{k}{2^n}\chi_{E_n^k} + 2^n\chi_{F_n} \\
s_{n+1} &=& \sum_{k=0}^{2^{2n+2}-1}\frac{k}{2^{n+1}}\chi_{E_{n+1}^k} + 2^{n+1}\chi_{F_{n+1}}
\end{eqnarray*}
If $x \in X$ such that $f(x) > 2^{n+1}$, then $x \in F_{n+1} \subset F_n$ and 
$s_{n+1}(x) - s_n(x) = 2^{n+1} - 2^n > 0$.
If $x$ is such that $2^n < x < 2^{n+1}$ then $x \in F_n$ but $x \notin F_{n+1}$.
The smallest such $x$ will lie in
\[
f^{-1}\left(\left(\frac{2^{2n+1}}{2^{n+1}}, \frac{2^{2n+1} + 1}{2^{n+1}}\right]\right)
\]
which corresponds to $k = 2^{2n+1}$. Therefore,
\[
s_{n+1}(x) - s_n(x) = \frac{2^{2n+1}}{2^{n+1}} = 2^n + \frac{1}{2^{n+1}}
\]
and hence
\[
s_{n+1}(x) - s_n(x) = \frac{1}{2^{n+1}} > 0.
\]
We can follow the same steps for the entire range of $x$. Thus, $\{s_n\}$ is a
point-wise monotone increasing sequence of simple functions.

That $s_n$ is non-negative follows from the fact that the coefficients of terms
in the defining equation are positive.
\end{proof}

\begin{prop}\label{s3p7}
$s_n \rightarrow f$.
\end{prop}
\begin{proof}
Let $f(x) = (k + \delta)/2^n$ where $k$ is a positive integer and $0 < \delta < 1$
is a real. Then 
\[
x \in f^{-1}\left(\left(\frac{k}{2^n}, \frac{k+1}{2^n}\right]\right)
\]
so that
\[
s_n(x) = \frac{k}{2^n}
\]
and hence
\[
f(x) - s_n(x) = \frac{\delta}{2^n}.
\]
\end{proof}

\begin{defn}\label{s3d4}
Let $f:X \rightarrow \sor$ and $(X, \Sigma_X, \mu)$ be a measure space. If $f^{\pm}$
are defined as in the proof of proposition \ref{s2p6} then
\[
\int_X f d\mu = \int_X f^+ d\mu - \int_X f^-d\mu
\]
\end{defn}

\begin{prop}\label{s3p8}
Let $a = \sup\{x_n\}, b = \sup\{y_n\}$ then $\sup\{x_n + y_n\} = a + b$.
\end{prop}
\begin{proof}
Let $c = \sup\{x_n + y_n\}$. Then $a \ge x_n, b \ge y_n \Rightarrow a+ b \ge x_n + y_n
\Rightarrow a + b \ge c$. 

Let, if possible, $c > a + b$ or $c - b > a$. Since $a = \sup\{x_n\}$, there 
exists $\delta > 0$ such that $c - b > x_n + \delta$ for all $n \in \son$.
Therefore, $c - b + y_n > y_n + x_n + \delta \Rightarrow c - b + \sup\{y_n\}
> \sup\{x_n + y_n\} + \delta \Rightarrow c > \sup\{x_n + y_n\} + \delta$, or
that $c$ is not a supremum of $\{x_n + y_n\}$, a contradiction.
\end{proof}

\begin{prop}\label{s3p9}
Let $(X, \Sigma_X, \mu)$ be a measure space and $f, g:X \rightarrow [0, \infty]$ 
be measurable functions. Then,
\[
\int_X(f + g)d\mu = \int_X f d\mu + \int_X g d\mu.
\]
\end{prop}
\begin{proof}
Let $\{s_n\}$ and $\{t_n\}$ be simple functions with $X$ as the domain such that
\begin{eqnarray*}
\int_X f d\mu &=& \sup\left\{\int_X s_n d\mu \;|\; s_n \le f\right\} \\
\int_X g d\mu &=& \sup\left\{\int_X t_n d\mu \;|\; t_n \le f\right\} 
\end{eqnarray*}
If $s_n \le f$ and $t_n \le g$ then $s_n + t_n \le f + g$ and by proposition
\ref{s3p8}, $\sup\{s_n + t_n\} = \sup\{s_n\} + \sup\{t_n\}$. Now,
\[
\int_X fd\mu + \int_X gd\mu = \sup\left\{\int_X s_n d\mu + \int_X t_nd\mu\right\}
= \sup\left\{\int_X(s_n + t_n)d\mu\right\}
\]
where we used proposition \ref{s3p5}. The last term, by definition is the integral
of $f+g$.
\end{proof}

\begin{cor}\label{s3c1}
Proposition \ref{s3p9} holds good for $f, g: X \rightarrow \sor$.
\end{cor}
\begin{proof}
Let $f = f^+ - f^-$ and $g = g^+ - g^-$ then $f + g = (f^+ + g^+) - (f^- + g^-)
= (f+g)^+ - (f+g)^-$. From proposition \ref{s3p9},
\begin{eqnarray*}
\int_X(f^+ + g^+)d\mu &=& \int_X f^+ d\mu + \int_X g^+d\mu \\
\int_X(f^- + g^-)d\mu &=& \int_X f^- d\mu + \int_X g^-d\mu
\end{eqnarray*}
so that
\[
\int_X(f^+ + g^+)d\mu - \int_X(f^- + g^-)d\mu = \int_X f^+ d\mu - \int_X f^-d\mu
+ \int_X g^+d\mu - \int_X g^-d\mu
\]
from which the corollary follows.
\end{proof}

\begin{prop}\label{s3p10}
If $f \le g$, where $f$ and $g$ are non-negative valued measurable functions on 
$(X, \Sigma_X, \mu)$ then
\[
\int_X fd\mu \le \int_X gd\mu.
\]
\end{prop}
\begin{proof}
Let $\{s_n\}$ and $\{t_n\}$ be sequences of simple functions which approximate
$f$ and $g$. Then $s_n \le t_n$ for all but a finite number of $n$. For if not,
then their limits will not be related as $f \le g$. Therefore, by proposition
\ref{s3p4},
\[
\int_X s_n d\mu \le \int_X t_n d\mu
\]
for all but a finite number of $n$ so that
\[
\sup\left\{\int_X s_n d\mu\right\} \le \sup\left\{\int_X t_n d\mu\right\}
\]
which proves the proposition.
\end{proof}

\begin{cor}\label{s3c2}
Proposition \ref{s3p10} is true for measurable $f, g: X \rightarrow \sor$.
\end{cor}
\begin{proof}
If $f = f^+ - f^-$ and $g = g^+ - g^-$ then $f \le g \Rightarrow f^+ \le
g^+ \text{ and } g^- \le f^-$ from which the corollary follows.
\end{proof}

\begin{prop}\label{s3p11}
If $f: X \rightarrow \sor$ is a measurable function on the measure space $(X,
\Sigma_X, \mu)$ then
\[
\left|\int_X fd\mu\right| \le \int_X |f|d\mu.
\]
\end{prop}
\begin{proof}
\[
|f(x)| = \begin{cases}
f(x) & \text{ if } f(x) \ge 0 \\
-f(x) & \text{ if } f(x) < 0
\end{cases}
\]
so that
\[
|f(x)| = \begin{cases}
f^+(x) & \text{ if } f(x) \ge 0 \\
f^-(x) & \text{ if } f(x) < 0
\end{cases}
\]
where $f^+(x) = \max\{f(x), 0\}$ and $f^-(x) = -\min\{f(x), 0\}$, making
both functions non-negative. From this it is clear that $|f| = |f^+| + |f^-|
= f^+ + f^-$.
Now,
\[
\left|\int_X f d\mu\right| = \left|\int_X f^+ d\mu - \int_X f^- d\mu\right|.
\le \left|\int_X f^+ d\mu\right| + \left|\int_X f^- d\mu\right|.
\]
Since $f^+$ and $f^-$ are positive valued,
\[
\left|\int_X fd\mu\right| \le \int_X (f^+ + f^-)d\mu = \int_X|f| d\mu.
\]
\end{proof}

\begin{defn}\label{s3d5}
Let $(X, \Sigma_X, \mu)$ be a measure space. A set $E \in \Sigma_X$ is 
said to be a measure-zero set if $\mu(E) = 0$.
\end{defn}

\begin{defn}\label{s3d6}
A property $P$ over a set $X$ is set to be true almost everywhere if it is 
false on a measure-zero set. That is, if $E = \{x \in X \;|\; P(x) \text{ is
false}\}$ then $\mu(E) = 0$.
\end{defn}

\begin{prop}\label{s3p12}
Let $(X, \Sigma_X, \mu)$ be a measure space and $E \in \Sigma_X$ be a measure-zero
set. Then, $\int_E fd\mu = 0$.
\end{prop}
\begin{proof}
Let $f$ be approximated by the sequence $\{s_n\}$ of simple functions. Then the
restriction of $f$ to $E$ too will be approximated the restriction of $s_n$ to
$E$. If $E_1, \ldots, E_k$ partition $E$ then
\[
\int_E s_n d\mu = \sum_{k=1}^n a_k \mu(E_k).
\]
Since $E_1, \ldots, E_k$ partition $E$, $\mu(E) = \mu(E_1) + \cdots \mu(E_n)$.
As $\mu(E) = 0$, so are $\mu(E_k)$ for $k=1, \ldots, n$ making
\[
\int_E s_n d\mu = 0.
\]
Since,
\[
\int_E f d\mu = \sup\left\{\int s_n d\mu \;|\; s_n \le f\right\}
\]
the integral of $f$ over $E$ vanishes.
\end{proof}

\begin{prop}\label{s3p13}
Let $(X, \Sigma_X, \mu)$ be a measure space and the sets $A$ and $B$ partition
$X$. Then,
\[
\int_X fd\mu = \int_A f d\mu + \int_B fd\mu.
\]
\end{prop}
\begin{proof}
Let $s$ be a simple function on the measure space and $E_1, \ldots, E_m, E_{m+1},
\ldots, E_n$ partition $X$ in such a way that $E_1, \ldots, E_m$ partition $A$
and the remaining ones partition $B$. Then
\[
\int_X sd\mu = \sum_{k=1}^n a_k \mu(E_k) = \sum_{k=1}^m a_k \mu(E_k) + 
\sum_{k=m+1}^n a_k\mu(E_k) = \int_A sd\mu + \int_B sd\mu.
\]
Therefore,
\begin{eqnarray*}
\int_X fd\mu &=& \sup\left\{\int_X s_n d\mu\right\} \\
 &=& \sup\left\{\int_A s_n d\mu + \int_B s_n d\mu\right\} \\
 &=& \sup\left\{\int_A s_n d\mu\right\} + \sup\left\{\int_B s_n d\mu\right\} \\
 &=& \int_A fd\mu + \int_B fd\mu.
\end{eqnarray*} 
\end{proof}

\begin{prop}\label{s3p14}
Let $(X, \Sigma_X, \mu)$ be a measure space. A measurable function $f:X 
\rightarrow [0, \infty]$ vanishes almost everywhere if $\int_X fd\mu = 0$.
\end{prop}
\begin{proof}
Let $A$ be the set of points on which $f$ does not vanish. Then
\[
\int_X f d\mu = \int_A fd\mu + \int_{A^c}fd\mu.
\]
Since $\mu(A) = 0$, the first integral vanishes. Since $f = 0$ on $A^c$ the
second one vanishes.

Now let $\int_X f d\mu = 0$. Let $A$ be the subset of $X$ on which $f$ does
not vanish. Then $f$ vanished on $A^c$. Further,
\[
\int_X fd\mu = \int_A fd\mu = 0.
\]
Now,
\[
\int_A fd\mu = \sup\left\{\int_A s_n d\mu\right\} =
\sup\left\{\sum_{k=1}^n a_k\mu(E_k)\right\} = 0,
\]
for a suitable partition $E_1, \ldots E_n$ of $A$ and $a_k > 0$. The only way
a supremum of non-negative points will be zero is when all points are zero,
which can happen only if $\mu(E_k) = 0$ for all $k = 1, \ldots, n$, which in
turn implies that $\mu(A) = 0$.
\end{proof}

\begin{prop}\label{s3p15}
Let $(X, \Sigma_X, \mu)$ be a measure space and $f: X \rightarrow [0, \infty]$
be a measurable function. If $c \in \sor$ then,
\[
\int_X (cf) d\mu = x\int_X fd\mu.
\]
\end{prop}
\begin{proof}
Let $\{s_n\}$ be a sequence of simple functions which converge to $f$. Then
$\{c s_n\}$ converges to $cf$. Using proposition \ref{s3p5},
\[
\int_X cs_n d\mu = c\int_X s_n d\mu
\]
so that
\begin{eqnarray*}
c\int_X f d\mu &=& 
    c\sup\left\{\int_X s_n d\mu\;|\; 0 \le s_n \le f\right\} \\
&=& \sup\left\{c\int_X s_n d\mu\;|\; 0 \le s_n \le f\right\} \\
&=& \sup\left\{\int_X cs_n d\mu\;|\; 0 \le s_n \le f\right\}
\end{eqnarray*}
where we used proposition \ref{s3p5}. If $\{s_n\} \rightarrow f$ then
$\{cs_n\} \rightarrow cf$  so that the supremum in the last step is
$\int_X (cf)d\mu$.
\end{proof}

From propositions \ref{s3p9} and \ref{s3p15} it follows that 
\begin{thm}\label{s3t1}
If $(X, \Sigma_X, \mu)$ is a measure space and $V$ is a set of measurable
functions on $X$ then $V$ is a vector space.
\end{thm}

\section{Convergence theorems}\label{c4}
Before proving the monotone convergence theorem for Lebesgue integrals of a
sequence of measurable functions, we will prove a similar theorem for sequences
of numbers.
\begin{thm}\label{c4t1}
Let $\{a_n\}$ be a non-decreasing sequence of real numbers which is bounded 
above. Then $\{a_n\}$ converges to its supremum.
\end{thm}
\begin{proof}
Since $\{a_n\}$ is a sequence of real numbers that is bounded above, it has a
supremum, say $A$. We will show that $\lim a_n = A$. Fix an $\epsilon > 0$.
Since $A$ is the supremum of $\{a_n\}$, there exists an $N \in \son$ for which
$|A - a_N| < \epsilon$. Since $\{a_n\}$ is a non-decreasing sequence, we also
have $|A - a_n| < \epsilon$ for all $n \ge N$ so that $A$ is also the limit of
the sequence.
\end{proof}

\begin{prop}\label{c4p1}
Let $(X, \Sigma_X, \mu)$ be a measure space and $\{f_n\}$ be a sequence of
measurable functions on $X$. If $\lim f_n = f$ point-wise then $f$ is
measurable.
\end{prop}
\begin{proof}
By proposition \ref{s2p4}, $\limsup f_n$ and $\liminf f_n$ are measurable.
Since $\lim f_n = f$, $\limsup f_n = \liminf f_n = f$.
\end{proof}

Analogous to this theorem, we have
\begin{thm}[Monotone Convergence Theorem]\label{s4t2}Let $(X, \Sigma_X, \mu)$ be a 
measure space and $\{f_n\}$ be a sequence of measurable functions which 
converges to $f$. Then 
\[
\lim_{n\rightarrow\infty}\int_X f_n d\mu = \int_X fd\mu.
\]
\end{thm}
\begin{proof}
Given that $\lim f_n = f$, therefore, $\limsup f_n = \liminf f_n = \lim f_n = f$.
Since $f_n$ are measurable, by proposition \ref{s2p4}, $f$ is measurable and
we can define its Lebegue integral over $X$. As $\{f_n\}$ are a monotone
increasing sequence,
\begin{equation}\label{s4e1}
\int_X f_n d\mu \le \int_X fd\mu.
\end{equation}
Let 
\[
\int_X fd\mu = \sup\left\{\int_X s d\mu \;|\; s \in S_f\right\},
\]
where $S_f$ is the set of simple functions $X \rightarrow \sor$ such that
$0 \le s \le f$. Fix a function $s \in S_f$ and let $\lambda \in (0, 1)$. Define
\[
A_n = \{x \in X \;|\; f_n(x) - \lambda s(x) \ge 0\}.
\]
As $n$ increases, and since $\{f_n\}$ are an increasing sequence of functions
whoe limit is $f$, the domain of $A_n$ increases with $n$, that is $\{A_n\}$ 
is a sequence of sets such that $A_n \subseteq A_{n+1}$. Clearly,
\begin{equation}\label{s4e2}
\bigcup_n A_n \subseteq X.
\end{equation}
Let $x_0$ be any element of $X$. There will be an $m \in \son$ such that for
all $n \ge m$, $f_n(x_0) - \lambda s(x_0) \ge 0$. This is because $f_n \rightarrow
f$ and $s \in S_f$. Therefore, $x_0 \in A_n \forall n \ge m$ or that
\[
\bigcup_n A_n \subseteq X.
\]
\begin{equation}\label{s4e3}
X \subseteq \bigcup_n A_n.
\end{equation}
From equations \eqref{s4e2} and \eqref{s4e3} we have
\[
X = \bigcup_n A_n.
\]

For $E \in \Sigma_X$, define
\[
\nu(E) = \int_E \lambda s d\mu.
\]
Clearly, $\nu$ is a measure on $(X, \Sigma_X)$. Now,
\[
\int_X \lambda s d\mu = \nu(X) = \nu\left(\bigcup_n A_n\right).
\]
As $\{A_n\}$ are in increasing sequnce of sets the last term in the above equation
can be written as a limit and we get
\[
\int_X \lambda s d\mu = \lim_{n \rightarrow \infty}\nu(A_n)
\]
or
\[
\int_X \lambda s d\mu = \lim_{n \rightarrow \infty}\int_{A_n}\lambda s d\mu.
\]
Since $\lambda \in (0, 1)$ and $s \in S_f$, $\lambda s \le f_n$ so that
\[
\int_X \lambda s d\mu \le \lim_{n \rightarrow \infty}\int_{A_n}f_n d\mu
\le \lim_{n \rightarrow \infty}\int_X f_n d\mu.
\]
In the limit $\lambda \rightarrow 1$, this equation is
\[
\int_X \lambda s d\mu \le \lim_{n \rightarrow \infty}\int_X f_n d\mu.
\]
Taking supremum over all $s \in S_f$, we see that
\begin{equation}\label{s4e4}
\int_X f d\mu \le \lim_{n \rightarrow \infty}\int_X f_n d\mu.
\end{equation}
From equations \eqref{s4e1} and \eqref{s4e4}, we get
\[
\int_X f d\mu = \lim_{n \rightarrow \infty}\int_X f_n d\mu.
\]
\end{proof}

\begin{prop}\label{s4p2}
Let $(X, \Sigma_X, \mu)$ be a measure space and $f, g: X \rightarrow [0, \infty]$
be measurable functions. Then
\[
\int_X(f + g)d\mu = \int_X f d\mu + \int_X g d\mu.
\]
\end{prop}
\begin{proof}
Using propositions \ref{s3p6} and \ref{s3p7} we know that for any measurable
function we can find a monotonically increasing sequence of simple functions
which converges to it. Let $\{s_n\}$ and $\{t_n\}$ be monotonically increasing
functions such that $s_n \rightarrow f$ and $t_n \rightarrow g$. By monotone
convergence theorem,
\begin{eqnarray*}
\int_X f d\mu &=& \lim_{n \rightarrow \infty} \int_X f_n \mu \\
\int_X g d\mu &=& \lim_{n \rightarrow \infty} \int_X g_n \mu
\end{eqnarray*}
so that
\[
\int_X fd\mu + \int_X gd\mu = \lim_{n \rightarrow \infty} \int_X s_n d\mu +
\int_X t_n d\mu.
\]
Using the linearity property of Lebesgue integral of simple functions proved in
proposition \ref{s3p5}, we can write the above equation as
\[
\int_X fd\mu + \int_X gd\mu = \lim_{n \rightarrow \infty} \int_X (s_n + t_n) d\mu
\]
If $s_n \rightarrow f$ and $t_n \rightarrow g$ then, as the monotone convergence
theorem permits us to interchange the order of the limit and the integral, the
right hand side of the previous equation can be changed as
\[
\int_X fd\mu + \int_X gd\mu = \int_X (f + g)d\mu.
\]
\end{proof}

\begin{cor}\label{s4c1}
Let $f, g: X \rightarrow \sor$ be measurable on $(X, \Sigma_X, \mu)$ then
\[
\int_X(f + g)d\mu = \int_X f d\mu + \int_X g d\mu.
\]
\end{cor}
\begin{proof}
Follows immediately from proposition \ref{s2p1} after writing $f = f^+ - f^-$
and $g = g^+ - g^-$.
\end{proof}

\begin{thm}[Beppo Levi]\label{s4t3}
Let $\{f_n\}$ be a sequence of measurable, non-negative functions on $(X, 
\Sigma_X, \mu)$. Then
\[
\int_X \lim_{n \rightarrow \infty}f_n d\mu = 
\lim_{n \rightarrow \infty}\int_X f_n d\mu.
\]
Note that $\{f_n\}$ need not be monotone increasing.
\end{thm}
\begin{proof}
Define 
\[
g_n(x) = \sum_{k=1}^n f_n(x)
\]
so that $\{g_n\}$ is a monotone increasing sequence. Then,
\[
\lim_{n \rightarrow \infty}\sum_{k=1}^n f_nx(x) = \lim_{n \rightarrow \infty}g_n(x)
\]
so that by the monotone convergence theorem,
\[
\int_X \lim_{n \rightarrow \infty} g_n d\mu = \lim_{n \rightarrow \infty}\int_X
g_n d\mu.
\]
Substituting for $g_n$ proves the theorem.
\end{proof}

\begin{prop}[Fatou's lemma]\label{s4p3} Let $f_n: X \rightarrow [0, \infty]$ 
be measurable on $(X, \Sigma_X, \mu)$. Then,
\[
\liminf\int_X f_n d\mu \ge \int_X\liminf f_n d\mu.
\]
\end{prop}
\begin{proof}
Since $\liminf f_n = \sup_n\inf{k \ge n} f_k(x)$, define $g_n(x) = \inf_{k \ge n}
f_k(x)$ so that $\{g_n\}$ is a monotonically increasing sequence of functions. 
Furthermore, from its definition, it immediately follows that $g_n \le f_n$ and
$\liminf f_n = \sup_n g_n$. For a monotone bounded sequence (all monotone sequences
if we are working with extended real numbers), the supremum is also the limit.
Therefore, $\liminf f_n = \lim g_n$ and hence,
\[
\int_X\liminf f_n d\mu = \int_X\lim g_n d\mu.
\]
Since $\{g_n\}$ is monotone increasing, we can use the monotone convergence theorem
to interchange the order of the integral and the limit so that
\[
\int_X\liminf f_n d\mu = \lim\int_X g_n d\mu.
\]
When a limit exists it is equal to $\liminf$ (and $\limsup$) so that
\[
\int_X\liminf f_n d\mu = \liminf\int_X g_n d\mu \le \liminf\int_X f_n d\mu.
\]
\end{proof}

\begin{defn}\label{s4d1}
A measurable function $f: X \rightarrow \sor$ on a measure space $(X, \Sigma_X, \mu)$
is said to be integrable if
\[
\int_X|f|d\mu < \infty.
\]
\end{defn}

\begin{cor}\label{s4c2}
If $f$ is measurable and $|f| < K$, $K > 0$, then it is integrable.
\end{cor}
\begin{proof}
If $f$ is measurable then $\int_X fd\mu$ exists, although it is permitted to have
an infinite value. If $f$ is bounded then $\int_X |f| d\mu \le K\mu(X) < \infty$.
\end{proof}
\begin{rem}
I think there is an implicit assumption that $\mu(X) < \infty$.
\end{rem}

\begin{thm}[Dominated Convergence Theorem]\label{s4t4} Let $(X, \Sigma_X, \mu)$
be a measure space and $f_n:X \rightarrow \sor$ be measurable functions converging
to $f$. If there is an integrable function $g: X \rightarrow \sor$ such that $|f_n|
\le g$ then $f_n$ and $f$ are also integrable and 
\[
\lim_{n \rightarrow \infty}\int_X |f_n - f|d\mu = 0.
\]
\end{thm}
\begin{proof}
Since $f_n \rightarrow f$ and $|f_n| \le g$, by corollary \ref{s4c2}, $f_n$ and
$f$ are integrable. Since $g$ too is integrable, so is $2g - |f_n - f|$. 
Applying Fatou's lemma \ref{s4p3} to $\{2g - |f_n - f|\}$ we get
\[
\int_X\liminf (2g - |f_n - f|) d\mu \le \liminf\int_X (2g - |f_n - f|)d\mu.
\]
Since $f_n \rightarrow f$, $\liminf(2g - |f_n - f|) = 2g$ and
\begin{eqnarray*}
\int_X 2g d\mu &\le& \liminf\left(\int_X 2g d\mu - \int_X |f_n - f|d\mu\right) \\
 &\le& \int_X 2g d\mu + \liminf\left(-\int_X|f_n - f|d\mu\right) \\
 &\le& \int_X 2g d\mu - \limsup\int_X|f_n - f|d\mu.
\end{eqnarray*}
Since $g$ is integrable, $\int_X 2g d\mu$ is finite and hence, we have,
\[
0 \le -\limsup\int_X|f_n - f|d\mu \Rightarrow \limsup\int_X|f_n - f|d\mu <= 0.
\]
If $\limsup$ of a sequence of non-negative reals is less than or equal to $0$
then the sequence converges to $0$. Therefore,
\[
\lim_{n \rightarrow \infty}\int_X |f_n - f|d\mu = 0.
\]
\end{proof}

\begin{cor}\label{s4c3}
Under the hypotheses of the dominated convergence theorem \ref{s4t4},
\[
\lim_{n \rightarrow \infty}\int_X f_n d\mu = \int_X fd\mu.
\]
\end{cor}
\begin{proof}
The dominated convergence theorem gives
\[
\lim_{n \rightarrow 0}\int_X |f_n - f|d\mu = 0.
\]
so that
\[
\lim_{n \rightarrow \infty}\left|\int_X (f_n - f)d\mu\right| \le 0.
\]
As the limit involves a sequence of non-negative numbers, we can as well write,
\[
\lim_{n \rightarrow \infty}\left|\int_X (f_n - f)d\mu\right| = 0.
\]
Therefore,
\[
\lim_{n \rightarrow \infty}\int_X (f_n - f)d\mu = 0 \text{ and }
\lim_{n \rightarrow \infty}\int_X (f - f_n)d\mu = 0 
\]
both of which lead to
\[
\lim_{n \rightarrow \infty}\int_X f_n d\mu = \int_X fd\mu.
\]
\end{proof}

We will now prove some additional theorems whose validity depends on the convergence
theorems.
\begin{thm}\label{s4t5}
Let $(X, \Sigma, \mu)$ be a measure space and $f_n: X \rightarrow \sor$ be measurable
functions such that
\[
\int_X\sum|f_n| d\mu= \sum\int_X |f_n| d\mu < \infty.
\]
Then
\[
\int_X\sum f_n d\mu= \sum\int_X f_n d\mu.
\]
\end{thm}
\begin{proof}
Let 
\[
g_n = \sum_{k=1}^n f_n, g = \limsup g_n, h = \sum_{k=1}^\infty |f_k|.
\]
Since $\int h d\mu < \infty$ we must also have $|h| < \infty$ almost everywhere.
Therefore, the series $\sum_k f_k$ is absolutely convergent almost everywhere so
that the sequence $\{g_n\}$ converges to $g$ almost everywhere.

We have an integrable function $h$ that dominates the sequence $\{g_n\}$ so that
by corollary \ref{s4c3},
\[
\lim_{n \rightarrow \infty}\int_X g_n d\mu = \int_X gd\mu.
\]
Substituting for $g_n$ proves the theorem.
\end{proof}

We need a few propositions to prove the next theorem.
\begin{prop}\label{s4p4}
Let $(X, \Sigma, \mu)$ be a measure space and $s$ be a real-valued simple function
on $X$. If $E_1$ and $E_2$ are disjoint subsets of $X$ and if $E = E_1 \cup E_2$
then
\[
\int_E sd\mu = \int_{E_1} sd\mu + \int_{E_2}sd\mu.
\]
\end{prop}
\begin{proof}
\[
\int_E sd\mu = \sum_{k=1}^n a_k\mu(F_k),
\]
where the sets $F_k \subset E$, partition $E$. Without loss of generality, choose
the partition $\{F_k\}$ such that $F_1, \ldots, F_m$ partition $E_1$ and $F_{m+1},
\ldots, F_n$ partition $E_2$. Then,
\[
\int_E sd\mu = \sum_{k=1}^m a_k\mu(F_k) + \sum_{k=m+1}^n a_k\mu(F_k) =
\int_{E_1}sd\mu + \int_{E_2}sd\mu.
\]
\end{proof}

\begin{prop}\label{s4p5}
Let $(X, \Sigma, \mu)$ be a measure space and $f: X \rightarrow [0, \infty]$ be
a function on $X$. If $E_1$ and $E_2$ are disjoint subsets of $X$ and if $E = 
E_1 \cup E_2$ then
\[
\int_E fd\mu = \int_{E_1} fd\mu + \int_{E_2}fd\mu.
\]
\end{prop}
\begin{proof}
Let
\[
\int_E fd\mu = \sup\left\{\int_E sd\mu\;\Big|\; s \le f, s:X \rightarrow [0, \infty] 
\text{ is simple.}\right\}
\]
By proposition \ref{s4p4},
\begin{eqnarray*}
\int_E fd\mu &=& \sup\left\{\int_{E_1}sd\mu + \int_{E_2}sd\mu\right\} \\
 &=& \sup\left\{\int_{E_1}sd\mu\right\} + \sup\left\{\int_{E_2}sd\mu\right\} \\
 &=& \int_{E_1} fd\mu + \int_{E_2}fd\mu.
\end{eqnarray*}
We used proposition \ref{s3p8} to get to the second step.
\end{proof}

\begin{cor}\label{s4c4}
Proposition \ref{s4p5} is also true for $f: X \rightarrow \sor$.
\end{cor}
\begin{proof}
Follows immediately after writing $f = f^+ - f^-$ where $f^{\pm}$ are defined as
in the proof of proposition \ref{s2p6}.
\end{proof}

\begin{prop}\label{s4p6}
Let $(X, \Sigma, \mu)$ be a measure space, $s$ be a simple function on $X$
and $f: X \rightarrow [0, \infty]$ be a measurable function. If $E$ is an empty
set then $\int_E sd\mu = 0$ and $\int_E fd\mu = 0$.
\end{prop}
\begin{proof}
\[
\int_E sd\mu = a\mu(E) = a \times 0,
\]
as $\mu(\varnothing) = 0$.
For the same reason, $\int_E fd\mu = 0$.
\end{proof}

\begin{cor}\label{s4c5}
Proposition \ref{s4p6} is also true for $f: X \rightarrow \sor$.
\end{cor}
\begin{proof}
Follows immediately after writing $f = f^+ - f^-$ where $f^{\pm}$ are defined as
in the proof of proposition \ref{s2p6}.
\end{proof}

\begin{thm}\label{s4t6}
Let $(X, \Sigma, \mu)$ be a measure space and $g: X \rightarrow [0, \infty]$ be an
integrable function. Define,
\[
\nu(E) = \int_E gd\mu,
\]
for $E \in \Sigma$. Then $\nu$ is a measure on $(X, \Sigma)$ and for any
measurable function $f$ on $X$,
\[
\int_X fd\nu = \int_X fgd\mu.
\]
\end{thm}
\begin{proof}
By proposition \ref{s4p6}, $\nu(\varnothing) = 0$. Let $E_1, E_2, \ldots$ be
pairwise disjoint. Then by extending proposition \ref{s4p5} to countable unions,
we get 
\[
\nu(E) = \sum_{k=1}^\infty \mu(E_k).
\]
Thus, $\nu: \Sigma \rightarrow [0, \infty]$ is indeed a measure. Let $s:X \rightarrow
[0, \infty]$ be a simple function
\[
s = \sum_{k=1}^\infty a_k\chi_{E_k},
\]
and
\begin{eqnarray*}
\int_X sd\nu &=& \sum_{k=1}^n a_k\nu(E_k) \\
 &=& \sum_{k=1}^n a_k\int_{E_k}gd\mu \\
 &=& \sum_{k=1}^n a_k\int_X \chi_{E_k}gd\mu \\
 &=& \int_X\sum_{k=1}^na_k\chi_{E_k}gd\mu
\end{eqnarray*}
that is,
\[
\int_X sd\mu = \int_X sgd\mu.
\]
Let $S_f$ be the set of simple functions $X \rightarrow \sor$ such that
$0 \le s \le f$. 
\begin{eqnarray*}
\int_X fd\nu &=& \sup\left\{\int_X s d\nu \;\Big|\; s \in S_f\right\} \\
 &=& \sup\left\{\int_X s gd\mu \;\Big|\; s \in S_f\right\} \\
 &=& \int_X fg d\mu.
\end{eqnarray*}
\end{proof}

\begin{thm}(Change of variables)\label{s4t7}
Let $(X, \Sigma_X, \mu)$ and $(Y, \Sigma_Y, \nu)$ be a measure spaces. If $g:X
\rightarrow Y$ and $f: Y \rightarrow \sor$ are measurable functions then
\[
\int_X (f \circ g)d\mu = \int_Y fd\nu
\]
where $\nu(E) = \mu(g^{-1}(E))$ for any $E \in \Sigma_Y$.
\end{thm}
\begin{proof}
Let $f = \chi_F$ for $F \subset Y$. Then $(f \circ g)(x) = \chi_F(g(x)) = 1$ iff
$g(x) \in F$. Therefore,
\begin{equation}\label{s4e5}
\int_X (f \circ g) d\mu = \int_X \chi_F(g(x))d\mu = \mu(g^{-1}(F))
\end{equation}
and
\begin{equation}\label{s4e6}
\int_Y fd\nu = \int_Y\chi_F d\nu = \nu(F).
\end{equation}
Since we are given that $\nu(F) = \mu(g^{-1}(F)$ for $F \in \Sigma_Y$, we have
\begin{equation}\label{s4e7}
\int_X (f \circ g) d\mu = \int_Y fd\nu,
\end{equation}
for $f = \chi_F, F \in \Sigma_Y$. Now consider a simple function,
\[
t = \sum_{k}b_k\chi_{F_k},
\]
where $\{F_k\}$ partition $Y$. Then,
\begin{equation}\label{s4e8}
\int_Y td\nu = \int_Y \sum_k b_k \chi_{F_k} = \sum_k b_k \nu(F_k) = 
\sum_k b_k\mu(g^{-1}(F_k)),
\end{equation}
and
\begin{equation}\label{s4e9}
\int_X (t \circ g)d\mu = \sum_k b_k\int_X (\chi_{F_k} \circ g)f\mu = 
\sum_k b_k\mu(g^{-1}(F_k))
\end{equation}
so that once again we have
\begin{equation}\label{s4e10}
\int_X (t \circ g) d\mu = \int_Y td\nu.
\end{equation}
Finally, let $\{t_n\}$ be a monotone increasing sequence of simple functions
on $Y$ that converges point-wise to $f$. Then by the monotone convergence
theorem \ref{s4t2}
\[
\int_Y fd\nu = \lim_{n \rightarrow \infty}\int_Y t_n d\nu = 
\lim_{n \rightarrow \infty}\int_X (t_n \circ g)d\mu,
\]
where we used \eqref{s4e10} in the second step. Using \ref{s4t2} again, we get
\[
\int_Y fd\nu = \int_X\lim_{n \rightarrow \infty}(t_n \circ g)d\mu.
\]
If $t_n \rightarrow f$ then $t_n \circ g \rightarrow f \circ g$ so that
\[
\int_Y fd\nu = \int_X (f \circ g)d\mu.
\]
When the range of $f$ is a subset of $\sor$, we can apply the previous equation
to $f^{\pm}$ separately and arrive at the same conclusion.
\end{proof}

\begin{prop}\label{s4p7}
Let $f: X \rightarrow \sor$ be continuous at $x \in X$. Then there is a sequence
$\{x_n\}$ such that $f(x_n) \rightarrow f(x)$.
\end{prop}
\begin{proof}
Fix $\epsilon > 0$. If $f$ is continuous at $x$ then there exists $\delta > 0$
such that $|x - x^\op| < \delta \Rightarrow |f(x) - f(x^\op)| < \epsilon$. Since
$\{x_n\} \rightarrow x$, for a given $\delta > 0$, there is an $n_0 \in \son$ 
such that for all $n \ge n_0$, $|x_n - x| < \delta$. By continuity of $f$ at $x$
this also means that $f(x_n) - f(x)| < \epsilon$ so that $\{f(x_n)\} \rightarrow
f(x)$.
\end{proof}

\begin{prop}\label{s4p8}
If $\{x_n\} \rightarrow x \Rightarrow \{f(x_n)\} \rightarrow f(x)$ then $f$ is 
continuous at $x$.
\end{prop}
\begin{proof}
$\{x_n\} \rightarrow x \Rightarrow \exists N \in \son$ such that for all $n \ge
N$, $|x_n - x| \rightarrow \delta$. $\{f(x_n)\} \rightarrow f(x) \Rightarrow 
\exists M \in \son$ such that for all $n \ge M$, $|f(x_n) - f(x)| < \epsilon$.
If $N_0 = \max\{N, M\}$ then for all $n \ge N_0$, $|x_n - x| < \delta 
\Rightarrow |f(x_n) - f(x)| < \epsilon$, that is $f$ is continuous at $x$.
\end{proof}

From propositions \ref{s4p7} and \ref{s4p8} it follows that
\begin{thm}\label{s4t8}
$\{x_n\} \rightarrow x \Rightarrow \{f(x_n)\} \rightarrow f(x)$ iff $f$ is 
continuous at $x$.
\end{thm}

\begin{thm}[First Fundamental theorem of Calculus]\label{s4t9} Let $I \subset 
\sor$ be an interval and $f: I \rightarrow \sor$ be integrable. If $I(x) = 
[a, x]$ then the function,
\[
F(x) = \int_{I(x)} fd\mu,
\]
is continuous and $F^\prime(x) = f(x)$. Here $\mu(I(x)) = x - a$. Moreover, if
$f$ is continuous at $x$ then $F^\op(x) = f(x)$.
\end{thm}
\begin{proof}
Let $\{x_n\} \rightarrow x$ and consider
\[
F(x_n) = \int_{I(x_n)}fd\mu = \int_\sor f\chi_{I(x_n)}d\mu.
\]
Since $|f\chi_{I(x_n)}| < f$, by the dominated convergence theorem \ref{s4t4},
\[
\lim_{n \rightarrow \infty}F(x_n) = \int_\sor\lim_{n \rightarrow \infty}
f\chi_{I(x_n)}d\mu = \int_\sor f\chi_{I(x)} d\mu = \int_{I(x)}fd\mu = F(x).
\]
This proves continuity of $F$ using theorem \ref{s4t8}.

Consider 
\[
\frac{F(x+h) - F(x)}{h} = \frac{1}{h}\int_{(x, x+h)}f d\mu,
\]
where we used proposition \ref{s4p5}. Since $f$ is continuous at $x$
$|f(x^\op) - f(x)| < \epsilon$ for all $x^\op \in (x, x + h)$. Therefore,
$-\epsilon < f(x^\op) - f(x) < \epsilon$ for all $x^\op \in (x, x+h)$ so that
\[
\frac{F(x+h) - F(x)}{h} < \frac{f(x) + \epsilon}{h}\int_{(x, x+h)} d\mu < f(x)
+ \epsilon.
\]
For a given $\epsilon > 0$. due to continuity of $f$, we can always choose $h>0$
so that this is true. Therefore $F^\op(x) = f(x)$.
\end{proof}

\begin{prop}\label{s4p9}
Let $f: X \rightarrow \sor$ be a measurable function and $(a, b) \subset X$. 
Then,
\[
\int_{(a, b)} f(x + h) d\mu = \int_{(a+h, b+h)}f(x) d\mu.
\]
\end{prop}
\begin{proof}
Define $t: X \rightarrow X$ such that $t(x) = (x + h)$. Then $f(x + h) = f(t(x))
= (f \circ t)(x)$. For any $(a, b) \in \Sigma_X$, $\mu(t^{-1}((a, b))) = \mu((
a-h, b-h)) = b - h - a + h = b - a = \mu((a, b))$. Therefore, by theorem 
\ref{s4t7},
\[
\int_{(a, b)} (f \circ t)d\mu = \int_{(a+h, b+h)} fd\nu = \int_{(a+h, b+h)}fd\mu
\]
as we just showed that $\mu$ is invariant under $t$.
\end{proof}

\begin{thm}[Second Fundamental Theorem of Calculus]\label{s4t10} Let $f:[a, b]
\rightarrow \sor$ be measurable and bounded. If $f = g^\op$ for some $g$ then
\[
\int_{[a, b]} fd\mu = g(b) - g(a).
\]
\end{thm}
\begin{proof}
Since $f = g^\op$,
\[
\int_{[a, b]} fd\mu = \int_{[a, b]} g^\op d\mu = 
\int_{[a, b]} \lim_{h \rightarrow 0}\frac{g(x+h)-g(x)}{h}d\mu,
\]
Over a finite interval $(g(x+h)-g(x))/h$ is bounded by the mean value theorem.
Therefore, the dominated convergence theorem allows us to interchange the 
integral and the limit so that.
\begin{eqnarray*}
\int_{[a, b]} fd\mu &=&
	\lim_{h \rightarrow 0}\int_{[a, b]}\frac{g(x+h)-g(x)}{h}d\mu \\
 &=& \lim_{h \rightarrow 0}\frac{1}{h}\left(\int_{[a, b]}g(x+h)d\mu
	- \int_{[a, b]}g()d\mu\right)
\end{eqnarray*}
Using proposition \ref{s4p9},
\[
\int_{[a, b]} fd\mu = \lim_{h \rightarrow 0}\frac{1}{h}
\left(\int_{[a+h, b+h]}g(x)d\mu - \int_{[a, b]}gd\mu\right)
\]
Using proposition \ref{s4p5},
\[
\int_{[a, b]} fd\mu = \lim_{h \rightarrow 0}\frac{1}{h}
\left(\int_{[b,b+h]}g d\mu - \int_{[a, a+h]}gd\mu\right).
\]
Since $g^\op$ exists on $[a, b]$, $g$ is continuous on $[a, b]$. Therefore,
for all $x \in [b, b+h]$, $|g(x) - g(b)| \le \epsilon/2$ or that $g(x) \le g(b) + 
\epsilon$ so that
\begin{equation}\label{c4e11}
\int_{[b, b+h]}gd\mu \le \left(g(b) + \frac{\epsilon}{2}\right)\int_{[b,b+h]}d\mu = 
\left(g(b)+\frac{\epsilon}{2}\right)h.
\end{equation}
Similarly,
\[
\int_{[a, a+h]}gd\mu \ge \left(g(a) - \frac{\epsilon}{2}\right)\int_{[a,a+h]}d\mu = 
\left(g(a)-\frac{\epsilon}{2}\right)h
\]
which is equivalent to
\begin{equation}\label{c4e12}
-\int_{[a, a+h]}gd\mu \le \left(-g(a) + \frac{\epsilon}{2}\right)\int_{[a,a+h]}d\mu = 
\left(-g(a)+\frac{\epsilon}{2}\right)h
\end{equation}
Adding \eqref{c4e11} and \eqref{c4e12}, we get
\[
\int_{[a+h, b+h]}g(x)d\mu - \int_{[a, b]}gd\mu \le (g(b)-g(a))h + \epsilon h
\]
so that
\[
\int_{[a, b]} fd\mu \le \lim_{h \rightarrow 0}\frac{1}{h}((g(b)-g(a))h + \epsilon h)
= g(b) - g(a) + \epsilon.
\]
Since $\epsilon$ can be made arbitrarily small, the theorem follows.
\end{proof}

\begin{thm}\label{c4t11} Let $(X, \Sigma_X, \mu)$ be a measure space, $T$ a 
metric space and $f:X \times T \rightarrow \sor$ be such that $f(\cdot, t)$ is
measurable for all $t \in T$. Let
\[
F(t) = \int_X f(x, t)d\mu.
\]
Then $F$ is continuous in $T$ if:
\begin{enumerate}
\item $f(x, \cdot)$ is continuous for each $x \in X$.
\item There exists an integrable function $g$ such that $|f(x, t)| \le g(x)$ 
for all $t \in T$.
\end{enumerate}
\end{thm}
\begin{proof}
Let $t_0 \in T$. Consider,
\[
\lim_{t \rightarrow t_0} F(t).
\]
Since $f$ is dominated by $g$, we can interchange the limit and the integral to
get
\[
\lim_{t \rightarrow t_0} F(t) = \int_X\lim_{t \rightarrow t_0}f(x, t)d\mu.
\]
Since for each $x$, $f$ is a continuous function of $t$,
\[
\lim_{t \rightarrow t_0} F(t) = \int_Xf(x, t_0)d\mu = F(t_0).
\]
\end{proof}

\begin{rem}
The continuity in $T$ requires $T$ to be at least a metric space. (Question: Can
it not be a topological space?)
\end{rem}

\begin{thm}\label{c4t12} Let $(X, \Sigma_X, \mu)$ be a measure space, $T$ a 
metric space and $f:X \times T \rightarrow \sor$ be such that $f(\cdot, t)$ is
measurable for all $t \in T$. Let
\[
F(t) = \int_X f(x, t)d\mu.
\]
Then $F$ is differentiable at $t \in T$ if:
\begin{enumerate}
\item $f(x, \cdot)$ is differentiable with respect to $t$ for each $x \in X$.
\item There exists an integrable function $g$ such that $|\partial_t f| \le g(x)$ 
for all $t \in T$.
\end{enumerate}
Furthermore,
\[
F^\op(t) = \int_X \partial_t f d\mu.
\]
\end{thm}
\begin{proof}
\begin{eqnarray*}
F^\op(t) &=& \lim_{h \rightarrow 0}\frac{1}{h}\left(\int_X f(x+h, t)d\mu -
\int_X f(x, t)d\mu\right) \\
&=& \lim_{h \rightarrow 0}\int_X \frac{f(x+h, t) - f(x, t)}{h} d\mu.
\end{eqnarray*}
Since $g$ dominates $\partial_t f$, is also dominated the integrand of the 
previous integral. (This follows from the mean value theorem.) Therefore, by the
dominated convergence theorem, we can interchange the limit and the integral, to
get
\[
F^\op(t) = \int_X\lim_{h \rightarrow 0}\frac{f(x+h, t) - f(x, t)}{h} d\mu = 
\int_X \partial_t f d\mu.
\]
\end{proof}

\section{$L^p$ spaces}\label{s5}
\begin{defn}\label{s5d1}Let $V$ be a vector space over the field of 
complex numbers. Then a norm on $V$ is a function $p: V \rightarrow 
[0, \infty)$ such that
\begin{enumerate}
\item $p(x + y) \le p(x) + p(y)$ for all $x, y \in V$,
\item $p(\alpha x) = |\alpha|p(x)$, for $\alpha \in \soc$,
\item $p(x) = 0$ iff $x = 0$.
\end{enumerate}
\end{defn}

\begin{defn}\label{s5d2}
Let $(X, \Sigma_X, \mu)$ be a measure space. Then the set $L^p(X)$, $p \ge
1$, is the set of measurable functions on $X \mapsto \sor$ for which
\[
\int_X |f|^p d\mu < \infty.
\]
\end{defn}

From theorem \ref{s3t1} it is easy to conclude that $L^p$ is a vector
space. Although the vector space can have $\soc$ as its field, we will, for
the moment, restrict the field to $\sor$. We now define a norm on it.
\begin{defn}\label{s2d2}
Let $(X, \Sigma_X, \mu)$ be a measure space and $f \in L^p(X)$. Then
\[
||f||_p = \left(\int_X |f|^p d\mu\right)^{1/p},
\]
is called the $L^p$ norm of $f$.
\end{defn}
The last property of the norm requires that $||f||_p = $ iff $f = 0$.
We will relax this property to
\begin{equation}\label{s5e1}
||f||_p = 0 \Leftrightarrow \int_X fd\mu = 0.
\end{equation}
The function $f$ need not be zero. A measurable non-zero function for 
which $\int_X fd\mu = 0$ is said to vanish almost everywhere. 

It is easy to see that
\[
||cf_p|| = \left(\int_X (cf)^p d\mu\right)^{1/p} =
\left(\int_X c^p f^p d\mu\right)^{1/p} = 
\left(c^p \int_X f^p d\mu\right)^{1/p} = c||f||_p.
\]

To show that $||f||_p$ is a norm we must also prove that it satisfies
the triangle inequality $||f + g||_p \le ||f||_p + ||g||_p$. In order
to prove it, we need a few more propositions.

\begin{defn}
Let $p, q \in (1, \infty)$ be such that $1/p + 1/q = 1$. Then they are
called conjugate exponents.
\end{defn}

\begin{prop}[H\"{o}lder inequality]\label{s5p1}
Let $(X, \Sigma_X, \mu)$ be a measure space and $f, g$ be real-valued
measurable functions on $X$. Then 
\[
\left|\int_X fgd\mu\right| \le \int_X |f||g| d\mu \le ||f||_p||g||_q
\]
where $p$ and $q$ are conjugate exponents.
\end{prop}
\begin{proof}
By proposition \ref{s3p11}
\[
\left|\int_X fg d\mu\right| \le \int_X |fg| d\mu.
\]
Since $|fg| = |f||g|$, the first equality follow immediately. If $||f||_p
= 0$ or $||g||_p = 0$ then the second inequality also follows immediately.
This is because $||f||_p = 0$ iff $f = 0$ almost everywhere. The
inequality is also true if any of the norms is infinite. Let us, therefore,
consider the case where the norms are non-zero and finite and define the
functions $F = |f|/||f||_p, G = |g|/||g||_q$. Clearly, $||F||_p = ||G||_q
= 1$.  A real-valued function $f$ is said to be concave if 
\[
f(cx + (1 - c)y) \ge cf(x) + (1 - c)(y).
\]
Since the logarithm is a concave function, using $c = 1/p$, we get
\[
\log\left(\frac{1}{p}x + \frac{1}{q}y\right) \ge \frac{1}{p}\log x +
\frac{1}{q}\log y.
\]
Let $x = F^p, y = G^q$ so that
\[
\log\left(\frac{F^p}{p} + \frac{G^q}{q}\right) \ge \log F + \log G
= \log (FG)
\]
that is 
\begin{equation}\label{s5e2}
FG \le F^p/p + G^q/q
\end{equation}
so that
\[
\int_X (FG) d\mu \le \frac{1}{p}\int_X F^p d\mu + \frac{1}{q}\int_X G^q
d\mu = \frac{1}{p}(||F||_p)^p + \frac{1}{q}(||G||_q)^q
\]
Since $||F||_p = ||G||_q = 1$ and $p, q$ are conjugate exponents,
\[
\int_X (FG) d\mu \le 1 \Rightarrow \int_X fg d\mu \le ||f||_p ||g||_q.
\]
\end{proof}

There is nothing in this proof that depends on measure theory and Lebesgue
integrals. Therefore the proposition seems to be valid in a wider arena. Let us
try to prove it for a finite dimenional inner product space with norm defined in
terms of the inner product. To begin with, let's start with the vector space 
over a real field. If $x, y \in \sor^n$, the inner product is $(x, y) = x_1y_1
+ \cdots + x_ny_n$ and the equivalent of $L^p$ norm is $||x||_p = (|x_1|^p + 
\cdots + |x_n|^p)^{1/p}$. We first note that
\[
|(x, y)| \le |x_1y_1| + \cdots + |x_ny_n| = |x_1||y_1| + \cdots + |x_n||y_n|.
\]
If we define the vector $x^a := (|x_1|, \ldots, |x_n|)$ then we can write the 
above equation as $|(x, y)| \le (x^a, y^a)$. This corresponds to the first 
inequality in proposition \ref{s5p1}. We now have to prove that $(x^a, y^a) \le
||x||_p||y||_q$ or equivalently,
\[
\left(\frac{x^a}{||x||_p}, \frac{y^a}{||y||_q}\right) \le 1.
\]
Define the vectors
\[
X = \frac{x^a}{||x||_p}, Y = \frac{y^a}{||y||_p}
\]
Using \eqref{s5e2},
\[
|X_i||Y_i| \le \frac{|X_i|^p}{p} + \frac{|X_i|^q}{q}, \forall i = 1, \ldots n
\]
so that
\[
|(X, Y)| \le \frac{1}{p}\sum_{i=1}^n |X_i|^p + \frac{1}{q}\sum_{i=1}^n |Y_i|^q
= \frac{(||X||_p)^p}{p} + \frac{(||Y||_q)^q}{q}.
\]
Since $||X||_p = ||Y||_q = 1$, we have
\[
|(X, Y)| \le \frac{1}{p} + \frac{1}{q},
\]
which, if $p$ and $q$ are conjugate exponents, becomes $|(X, Y)| \le 1$, which
is H\"{o}lder's inequality for for finite dimensional inner product spaces with
a norm.

\begin{rem}
Often times, the integral,
\[
\int_X fgd\mu
\]
is called the inner product of functions $f$ and $g$ when they belong to a 
suitably defined function space. It is then denoted by $(f, g)$. H\"{o}lder's
inequality then becomes
\begin{equation}\label{s5e3}
(f, g) \le ||f||_p||g||_q, \text{ for } \frac{1}{p} + \frac{1}{q} = 1.
\end{equation}
\end{rem}

A single variable function is convex in an interval if its second derivative is 
non-zero in it. The function $x^p$ has the second derivative $p(p-1)x^{p-2}$. It
is non-negative if $x \ge 0$ and $p > 1$. The function $|x|^p$ is convex for all
$x$ and $p > 1$. Using this fact, we show that
\begin{prop}\label{s5p2}
For any functions $f, g$, $|f + g|^p \le 2^{p-1}(|f|^p + |g|^p)$.
\end{prop}
\begin{proof}
Since the function $h(x) = |x|^p$ is convex, $h(cx_1 + (1 - c)x_2) \le ch(x_1) + 
(1-c)h(x_2)$, that is,
\[
|cx_1 + (1 - c)x_2|^p \le c|x_1|^p + (1-c)|x_2|^p.
\]
Choose $c = 1/2, x_1 = f, x_2 = g$ so that
\[
\frac{1}{2^p}(|f + g|)^p \le \frac{|f|^p}{2} + \frac{|g|^p}{2}
\Rightarrow |f + g|^p \le 2^{p-1}(|f|^p + |g|^p).
\]
\end{proof}

We next show that 
\begin{prop}\label{s5p3}
If $f$ and $g$ have finite $L^p$ norms then so does $f + g$.
\end{prop}
\begin{proof}
Using proposition \ref{s5p2}, $|f + g|^p \le 2^{p-1}(|f|^p + |g|^p)$, so that
\begin{eqnarray*}
(||f + g||_p)^p &=& \int_X |f + g|^p d\mu \\
 &\le& 2^{p-1}\left(\int_X |f|^p d\mu + \int_X |g|^p d\mu\right) \\
 &=& 2^{p-1}\left((||f||_p)^p + (||g||_p)^p\right)
\end{eqnarray*}
If the norms on the right hand side are finite then so is that on the left hand
side.
\end{proof}

We can now prove the triangle inequality for $L^p$ norm,
\begin{prop}[Minkowski inequality]\label{s5p4}
Let $(X, \Sigma_X, \mu)$ be a measure space and $f, g$ be real-valued
measurable functions on $X$. Then $||f + g||_p \le ||f||_p + ||g||_p$.
\end{prop}
\begin{proof}
We start with
\begin{eqnarray*}
(||f + g||_p)^p &=& \int_X |f + g|^p d\mu \\
 &=& \int_X |f + g||f + g|^{p-1} d\mu \\
 &\le& \int_X (|f| + |g|)|f + g|^{p-1} d\mu \\
 &\le& \int_X |f||f + g|^{p-1} d\mu + \int_X |g||f + g|^{p-1} d\mu.
\end{eqnarray*}
We now use H\"{o}lder's inequality to get
\[
\int_X f(f+g)^{p-1} d\mu \le \int_X |f||f + g|^{p-1} d\mu 
\le ||f||_p||(f + g)^{p-1}||_q,
\]
where $q = p/(p - 1)$.
Similarly,
\[
\int_X |g||f + g|^{p-1} d\mu \le ||g||_p||(f + g)^{p-1}||_q
\]
so that
\begin{eqnarray*}
(||f + g||_p)^p &\le& (|| f ||_p + || g ||_p||(f + g)^{p-1}||_q \\
 &\le& (||f||_p + ||g||_p||\left(\int_X |f + g|^{q(p-1)}d\mu\right)^{1/q} \\
 &\le& (||f||_p + ||g||_p||\left(\int_X |f + g|^{p}d\mu\right)^{1/q} \\
 &\le& (||f||_p + ||g||_p||\left(||f + g||_p\right)^{p/q}
\end{eqnarray*}
and hence
\[
(||f + g||_p)^{p - p/q} \le ||f||_p + ||g||_p \Rightarrow 
||f + g||_p \le ||f||_p + ||g||_p,
\]
where we used the fact that $p$ and $q$ are conjugate exponents.
\end{proof}

\begin{defn}\label{s5d5}
A measure space $(X, \Sigma_X, \mu)$ has a finite measure if $\mu(X)$ is finite.
\end{defn}

\begin{prop}\label{s5p5}
If $(X, \Sigma_X, \mu)$ has a finite measure then for $1 \le p < q < \infty$,
$L^p \supseteq L^q$.
\end{prop}
\begin{proof}
Since $1 \le p < q < \infty$, the exponent $r_1 = q/p$ is conjugate to $r_2$,
where
\[
\frac{1}{r_2} = 1 - \frac{1}{r_1} = 1 - \frac{p}{q} = \frac{q-p}{p}
\]
Using H\"{o}lder inequality, $|(g, h)| \le ||g||_{r_1}||h||_{r_2}$ for 
measurable functions $g, h$ on $X$. Choose $g = |f|^p$ and $h = \chi_X$, the
indicator function for $X$. Then $(g, h) = \int_X g d\mu$ and hence
\begin{equation}
\int_X g d\mu = \int_X |f|^p d\mu = (||f||_p)^p \le 
|| |f|^p ||_{r_1} || \chi_X ||_{r_2}.
\end{equation}
Now,
\[
|| |f|^p ||_{r_1} = \left(\int_X |f|^{pr_1}\right)^{1/r_1} = 
\left(\int_X |f|^q\right)^{1/r_1} = (|| f ||_q)^{q/r_1}
\]
and
\[
||\chi_X||_{r_2} = \left(\int_X |\chi_X|^{r_2} d\mu\right)^{1/r_2} = 
\left(\int_X d\mu\right)^{1/r_2} = \mu(X)^{1/r_2}.
\]
so that
\[
(||f||_p)^p \le (|| f ||_q)^{q/r_1}\mu(X)^{1/r_2} = 
(|| f ||_q)^p\mu(X)^{1/r_2}
\]
or
\[
||f||_p \le ||f||_q \mu(X)^{1/(pr_2)} < \infty,
\]
if $\mu(X)$ is finite. Thus, if $||f||_q$ is finite then so is $||f||_p$ so that
if $f \in L^q$ then $f \in L^p$ or that $L^q \supseteq L_p$.
\end{proof}

We next show 
\begin{prop}\label{s5p6}
Let $(X, \Sigma_X, \mu)$ be a measure space, $\{f_n\}$ be a sequence of 
real-valued, measurable functions on $X$ which converges to $f$ point wise,
almost everywhere. If $|f_n| \le g$ for some $g \in L^p(X)$ then $f, f_n \in
L^p(X)$ and $\{f_n\}$ converges to $f$ in $L^p(X)$ norm. That is,
\[
\lim_{n \rightarrow \infty}||f - f_n||_p = 0.
\]
\end{prop}
\begin{proof}
Since $\{f_n\} \rightarrow f$ and $|f_n| \le g$, $f_n$ and $f$ belong to $L^p$.
Furthermore, $|f - f_n| \le |f| + |f_n| \le 2g$. If $p \in [1, \infty)$, 
$|f - f_n|^p \le 2^p g^p$. The sequence of functions $\{|f - f_n|^p\}$ is 
dominated by $2^p g^p$ and its limit is zero. Therefore, by the dominated 
convergence theorem \ref{s4t4},
\[
\lim_{n \rightarrow \infty}\int_X \Big||f - f_n|^p - 0\Big| d\mu = 0
\Rightarrow
\lim_{n \rightarrow \infty}\int_X|f - f_n|^p d\mu = 0.
\]
Therefore, for a given $\epsilon > 0$, we can find $n_0 \in \son$ such that
\[
\int_X|f - f_n|^p d\mu < \epsilon^p, \forall n \ge n_0,
\]
or that
\[
\left(\int_X|f - f_n|^p d\mu\right)^{1/p} < \epsilon, \forall n \ge n_0,
\]
which means that
\[
\lim_{n \rightarrow \infty}\left(\int_X |f - f_n|^p d\mu\right)^{1/p} = 0.
\]
\end{proof}

\section{Product measures}\label{s6}
\begin{defn}\label{s6d1}
Let $(X, \Sigma_X)$ and $(Y, \Sigma_Y)$ be measurable spaces. A measurable 
rectangle in $X \times Y$ is a set $A \times B$ where $A \in \Sigma_X$ and $B
\in \Sigma_Y$.

A $\sigma$-algebra generated by all measurable rectangles will be denoted by
$\Sigma_X \otimes \Sigma_Y$. That is,
\[
\Sigma_X \otimes \Sigma_Y = \langle\Sigma_X\times\Sigma_Y\rangle
\]
\end{defn}

\begin{prop}\label{s6p1}
Let $(X, \Sigma_X)$ and $(Y, \Sigma_Y)$ be measurable spaces and $E$ be a member
of $\Sigma_X \otimes \Sigma_Y$. Let 
\begin{eqnarray*}
E_x &=& \{y \in Y \;|\; (x, y) \in E\}, x \in X \\
E_y &=& \{x \in X \;|\; (x, y) \in E\}, y \in Y 
\end{eqnarray*}
Then $E_x \in \Sigma_X$ and $E_y \in \Sigma_Y$.
\end{prop}
\begin{proof}
$E \in \Sigma_X \otimes \Sigma_Y$ implies that $E$ is a complement,
or a countable union of sets of the form $A \times B$ where $A \in \Sigma_X, B
\in \Sigma Y$. Suppose that $E$ is a countable union of $A_k \times B_k$, where
$k \ge 1$. Then $x$ is a member of at least on $A_k$ and $E_x$ is the union on
all $B_k$. Since $B_k \in \Sigma_Y$, $E_x \in \Sigma_Y$. Now suppose that $E$ is
a complement of $A_k \cup B_k$, that is $E = A_k^c \cap B_k^c$. $(x, y) \in E
\Rightarrow x \in A_k^c, y \in B_k^c$. The set $E_x$ will then be $B_k^c$. Since
$B_k \in \Sigma_Y, E_x = B_k^c \in \Sigma_Y$.

The proof for $E_y \in \Sigma_X$ runs along the same lines.
\end{proof}

\begin{prop}\label{s6p2}
Let $(X \times Y, \Sigma_X \otimes \Sigma_Y)$ and $(Z, \Sigma_Z$ be measurable
spaces. If $f: X \times Y \rightarrow Z$ is measurable then so are $f_x: Y 
\rightarrow Z$ and $f_y: X \rightarrow Z$, where $f_x(y) = f(x, y)$ and $f_y(x)
= f(x, y)$.
\end{prop}
\begin{proof}
Let $C \in \Sigma_Z$. Since $f$ is measurable, $E = f^{-1}(C) \in \Sigma_X 
\times \Sigma_Y$. Then $f_x^{-1}(C) = \{y \in Y \;|\; f(x, y) \in C\} = E_x$.
We showed in proposition \ref{s6p1} that $E_x \in \Sigma_Y$. Thus $f_x^{-1}(C)
\in \Sigma_Y$, making $f_x: Y \rightarrow C$ a measurable function.

We can similarly show that $f_y$ is a measurable function mapping $X$ into $C$.
\end{proof}

\begin{defn}\label{s6d2}
Let $X$ be a non-empty set. A family $\mathcal{M}$ of subsets of $X$ is called
a monotone class if:
\begin{enumerate}
\item If $A_1 \subseteq A_2 \subseteq \ldots \in \mathcal{M}$ then 
$\cup_{k \ge 1}A_k \in \mathcal{M}$.
\item If $B_1 \supseteq B_2 \supseteq \ldots \in \mathcal{M}$ then 
$\cap_{k \ge 1}B_k \in 
\mathcal{M}$.
\end{enumerate}
\end{defn}

\begin{defn}\label{s6d3}
Let $\mathcal{G} \subset 2^X$. The smallest monotone class containing 
$\mathcal{G}$ is the the intersection of all monotone classes which are supersets
of $\mathcal{G}$. It is also called the `monotone class generated by $\mathcal{G}$
and is denoted by $\mathcal{M}(\mathcal{G})$.
\end{defn}

\begin{defn}\label{s6d4}
Let $X$ be a non-empty set. A family $\mathcal{A} \in 2^X$ is called an algebra
if
\begin{enumerate}
\item $\varnothing \in \mathcal{A}$.
\item $A \in \mathcal{A} \Rightarrow A^c \in \mathcal{A}$,
\item $A, B \in \mathcal{A} \Rightarrow A \cup B \in \mathcal{A}$.
\end{enumerate}
\end{defn}

From the second and the third conditions in definition \ref{s6d4}, $\mathcal{A}$
contains finite unions and intersections of its members.

\begin{prop}\label{s6p3}
Let $X$ be a non-empty set and $\Sigma_X$ be a $\sigma$-algebra over $X$. Then 
$\sigma_X$ is also a monotone class over $X$.
\end{prop}
\begin{proof}
If $A_1 \subseteq A_2 \subseteq \ldots \in \Sigma_X$, then their union is a 
member of $\sigma_X$.

Let $B_1 \supseteq B_2 \ldots \in \Sigma_X$. Then $B_1^c \subseteq B_2^c \ldots 
\in \Sigma_X$ and hence
\[
\bigcup_{k \ge 1}B_k^c \in \Sigma_X.
\]
As $\Sigma_X$ is closed under complementation,
\[
\left(\bigcup_{k \ge 1}B_k^c\right)^c = \bigcap_{k \ge 1} B_k \in \Sigma_X.
\]
\end{proof}

\begin{prop}\label{s6p4}
Let $\mathcal{A}$ be an algebra of subsets of $X$ and $\mathcal{M}(\mathcal{A})$
be the monotone class generated by $\mathcal{A}$. If $S \notin \mathcal{A}$ and
$S \in \mathcal{M}(\mathcal{A})$ then $S$ is either a countable union or a 
countable intersection of members of $\mathcal{A}$.
\end{prop}
\begin{proof}
Suppose $S$ was neither. Then dropping $S$ would result in a smaller monotone 
class containing $\mathcal{A}$ contradicting the assumption that 
$\mathcal{M}(\mathcal{A})$ is the smallest such.
\end{proof}

\begin{prop}\label{s6p5}
Let $\mathcal{A}$ be an algebra of subsets of $X$. Then $\mathcal{M}(\mathcal{A})$
is closed under complementation.
\end{prop}
\begin{proof}
Let $S \in \mathcal{M}(\mathcal{A})$. If $S \in \mathcal{A}$, then $S^c \in
\mathcal{A}$ and hence $S \in \mathcal{M}(\mathcal{A})$.

If $S \in \mathcal{M}(\mathcal{A})$ is such that it is not a member of 
$\mathcal{A}$ then by proposition \ref{s6p4}, it is a countable union or 
intersection of member of $\mathcal{A}$.

Let $S = \cup_{k \ge 1}A_k$, where $A_1 \subseteq A_2, \ldots$ are members of
$\mathcal{A}$. Then $A_1^c \supseteq A_2^c \supseteq, \ldots$ are also members of
$\mathcal{A}$. Since $\mathcal{A}$ is a monotone class, their intersection is also
a member of $\mathcal{A}$. But it is precisely the complement of $S$.

We can similarly show that if $S = \cup_{k \ge 1}A_k \in \mathcal{M}(\mathcal{A})$
then $S^c$ is a member of $\mathcal{M}(\mathcal{A})$.
\end{proof}

\begin{prop}\label{s6p6}
Let $\mathcal{A}$ be an algebra of subsets of $X$. Define 
\[
K(F) = \{E \in \mathcal{M}(\mathcal{A})\;|\; E \cup F \in
\mathcal{M}(\mathcal{A})\},
\]
for $F \in \mathcal{M}(\mathcal{A})$
Then $K(G)$ is a monotone class.
\end{prop}
\begin{proof}
$K(F)$ is a collection of subsets of $X$ which themselves and their union
with the fixed set $F$ belong to the monotone class. Clearly, $K(F) \subseteq
\mathcal{M}(\mathcal{A})$.

We will next show that $K(F)$ is a monotone class. Consider a sequence $A_1
\subseteq A_2 \subseteq \ldots$ of classes in $K(F)$. For each of these, $A_k
\cup F \in K(F)$. Therefore, $\cup_{k \ge 1}(A_k \cup K) \in K(F)$ so that
$\cup_{k \ge 1}A_k \in K(F)$. Likewise, for a sequence $B_1 \supseteq B_2
\supseteq \ldots$ of classes in $K(F)$. For each of these $B_k \cup F \in K(F)$
so that $\cap_{k \ge 1}(B_k \cup F) \in K(F)$ or that $\cap_{k \ge 1}B_k \in 
K(F)$.
\end{proof}

\begin{prop}\label{s6p7}
Let $\mathcal{A}$ be an algebra of subsets of $X$ and $F \in \mathcal{A}$. 
Continuing the definition of $K(F)$ in proposition \ref{s6p6}, $\mathcal{A}
\subseteq K(F)$.
\end{prop}
\begin{proof}
Let $G \in \mathcal{A}$ then $G \cup F$ is a member of $\mathcal{A}$ and also a
member of $\mathcal{M}(\mathcal{A})$. Therefore, $G \in K(F)$ and hence 
$\mathcal{A} \subseteq K(F)$.
\end{proof}

From propositions \ref{s6p6} and \ref{s6p7} we conclude that if $F \in 
\mathcal{A}$ then $K(F)$ is a monotone class containing $\mathcal{A}$. However,
as $\mathcal{M}(\mathcal{A})$ is the smallest monotone class containing 
$\mathcal{A}$, we must have
\begin{prop}\label{s6p8}
$\mathcal{M}(\mathcal{A}) \subseteq K(F) \;\forall F \in \mathcal{A}$.
\end{prop}

However, by its definition, members of $K(F)$ also belong to 
$\mathcal{M}(\mathcal{A})$. Using this observation with proposition \ref{s6p8}
we get
\begin{prop}\label{s6p9}
$\mathcal{M}(\mathcal{A}) = K(F) \;\forall F \in \mathcal{A}$.
\end{prop}

Let $A_1, \ldots, A_k \in \mathcal{A}$. They need not form a monotonic, finite
sequence. Since $\mathcal{A}$ is an algebra, $F = \cup_{i=1}^k A_i \in 
\mathcal{A}$ and by proposition \ref{s6p9}, $F$ belongs to $\mathcal{M}(
\mathcal{A}$. Thus, an arbitrary finite union of members of $\mathcal{A}$
belongs to $\mathcal{M}(\mathcal{A})$. Taking the limit of the size of the union
we conclude that
\begin{prop}\label{s6p10}
If $A_1, A_2, \ldots \in \mathcal{A}$, is a sequence of not necessarily monotoic
sets then $\cup_{k \ge 1}A_k \in  \mathcal{M}(\mathcal{A})$.
\end{prop}

\begin{prop}\label{s6p11}
Let $\mathcal{A}$ be an algebra of subsets of $X$. Then $\varnothing \in
\mathcal{M}(\mathcal{A})$.
\end{prop}
\begin{proof}
Since $\mathcal{M}(\mathcal{A}) \supseteq \mathcal{A}$ and $\varnothing \in
\mathcal{A}$, it being an algebra, $\varnothing \in \mathcal{M}(\mathcal{A})$.
\end{proof}

Using propositions \ref{s6p5}, \ref{s6p10} and \ref{s6p11} we conclude that
\begin{prop}\label{s6p12}
$\mathcal{M}(\mathcal{A})$ is a $\sigma$-algebra containing $\mathcal{A}$.
\end{prop}

\begin{thm}[Monotone class theorem]\label{s6t1}
Let $\mathcal{A}$ be an algebra on $X$. Then $\mathcal{M}(\mathcal{A}) = \langle
\mathcal{A}\rangle$.
\end{thm}
\begin{proof}
A $\sigma$-algebra is also a monotone class. Therefore, $\mathcal{M}(\mathcal{A})
\subseteq \langle\mathcal{A}\rangle$. From proposition \ref{s6p12} we have $\langle
\mathcal{A}\rangle \subseteq \mathcal{M}(\mathcal{A})$.
\end{proof}


\end{document}
