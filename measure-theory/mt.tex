\documentclass{article}
\usepackage{amsmath, amsthm, amssymb, amsfonts, graphicx, color}
\usepackage{mathtools}
\usepackage{hyperref}

\newcommand{\td}[2]{\frac{d{#1}}{d{#2}}}

\theoremstyle{plain}
\newtheorem{thm}{Theorem}
\numberwithin{thm}{section}

\theoremstyle{plain}
\newtheorem{prop}{Proposition}
\numberwithin{prop}{section}

\theoremstyle{definition}
\newtheorem{defn}{Definition}
\numberwithin{defn}{section}

\theoremstyle{remark}
\newtheorem*{rem}{Remark}

\newtheorem*{cor}{Corollary}

\numberwithin{equation}{section}
\begin{document}
\section{$\sigma$-algebra}\label{s1}
\begin{defn}\label{s1d1}
Let $X$ be a non-empty set. A $\sigma$-algebra, $\mathcal{X}$, on $X$ is a family
of subsets of $X$ such that:
\begin{enumerate}
\item $\mathcal{X}$ is non-empty,
\item If $E \in \mathcal{X}$ then $E^c \in \mathcal{X}$ and
\item If $E_1, E_2, \ldots \in \mathcal{X}$ then $\cup_{n \in \mathbb{N}}E_n \in \mathcal{X}$. 
\end{enumerate}
\end{defn}
\begin{prop}\label{s1p1}
If $\mathcal{X}$ is a $\sigma$-algebra on $X$ then $X, \varnothing \in \mathcal{X}$.
\end{prop}
\begin{proof}
Since $\mathcal{X}$ is non-empty, it has at least one member, say $A \subset X$. 
Therefore, $A^c \in \mathcal{X}$ and hence $A \cup A^c = X \in \mathcal{X}$. 
Therefore, $X^c = \varnothing \in \mathcal{X}$.
\end{proof}

\begin{prop}\label{s1p2}
A $\sigma$-algebra $\mathcal{X}$ on $X$ is closed under countable intersections.
\end{prop}
\begin{proof}
Let $E_1, E_2, \ldots \in \mathcal{X}$. Then,
\[
\bigcap_{n \in \mathbb{N}}E_n = \left(\bigcup E_n^c\right)^c.
\]
Since $E_n \in \mathcal{X}$, $E_n^c$ are also in $\mathcal{X}$, so is their
countable union and its complement.
\end{proof}

\begin{defn}\label{s1d2}
If $\mathcal{X}$ is a $\sigma$-algebra on $X$ then the pair $(X, \mathcal{X})$
is called a measurable space and the members of $\mathcal{X}$ are called measurable
sets.
\end{defn}

A topology is also a family of subsets of a set. It is defined as
\begin{defn}\label{s1d3}
Let $X$ be a non-empty set. A collection $\tau$ of subsets of $X$ is a topology
if:
\begin{enumerate}
\item $\varnothing, X \in \tau$,
\item An arbitrary union of members of $\tau$ is a member of $\tau$.
\item An intersection of a finite number of members of $\tau$ is a member of $\tau$.
\end{enumerate}
The pair $(X, \tau)$ is called a topological space and members of $\tau$ are
called open sets.
\end{defn}
The differences between the two families are:
\begin{enumerate}
\item A $\sigma$-algebra is closed under complements, a topology is not.
\item A $\sigma$-algebra is closed under countable unions, a topology is closed
under arbitrary unions.
\item A $\sigma$-algebra is closed under countable intersections while a topology
is closed under finite intersections.
\end{enumerate}

\subsection{Examples and non-examples of $\sigma$-algebras}
\begin{enumerate}
\item For any non-empty set $X$, $2^X$ and $\{\varnothing, X\}$ are $\sigma$-
algebras. They are called trivial $\sigma$-algebras.
\item If $X = \{1, 2, 3\}$ then $\{\varnothing, \{1\}, \{2, 3\}, \{1, 2, 3\}$ is
a $\sigma$-algebra. It is said to be generated by $\{1\}$ because it is the 
smallest $\sigma$-algebra containing $\{1\}$. 

Note that there are $2^3 = 8$ subsets of $X$. Therefore, there are $2^8 = 256$
subsets of $2^X$.
\item If $X = \mathbb{R}$ then $\{(a, b)\;|\; a, b \in \bar{\mathbb{R}}$ is not
a $\sigma$-algebra. This is because the complement of $(a, b)$ is not an open
interval. For the same reason, the set of all closed intervals is also not a
$\sigma$-algebra. (The set of all open intervals is not even closed under unions.)
\item The set of all 
\end{enumerate}

\begin{prop}\label{s1p3}
Let $\mathcal{X}_1$ and $\mathcal{X}_2$ be $\sigma$-algebras on $X$. Then their
intersection is also a $\sigma$-algebra on $X$.
\end{prop}
\begin{proof}
Let $\mathcal{X} = \mathcal{X}_1 \cap \mathcal{X}_2$. Since both $\mathcal{X}_1$ and 
$\mathcal{X}_2$ contain $X$ and $\sigma$. $\mathcal{X}$ is non-empty. If $A \in
\mathcal{X}$ then $A \in \mathcal{X}_1$ and $A \in \mathcal{X}_2$, so that $A^c 
\in \mathcal{X}_1$ and $A^c \in \mathcal{X}_2$, that is $A^c \in \mathcal{X}$. 
Similarly, if $E_1, E_2, \ldots \in \mathcal{X}$ then these sets are also in 
$\mathcal{X}_1$ and $\mathcal{X}_2$ so that their countable union is in both
$\sigma$-algebras and therefore also in their intersection.
\end{proof}

\begin{defn}\label{s1d4}
Let $A \subset 2^X$. Then an intersection of all $\sigma$-algebras containing $A$
is called the $\sigma$-algebra generated by $A$. It is also the smallest $\sigma$-
algebra containing $X$. It is denoted by $\langle A \rangle$.
\end{defn}

\begin{defn}\label{s1d5}
Let $(X, \tau)$ be a topological space. Then $\langle\tau\rangle$ is called the
Borel $\sigma$-algebra. It is denoted by $\mathcal{B}(X)$.
\end{defn}

$\mathcal{B}(X)$ contains all open sets, closed sets and their countable unions
and intersections.

\begin{defn}\label{s1d6}
Let $(X, \mathcal{X})$ be a measurable space. A measure on it is a mapping
$\mu: \mathcal{X} \rightarrow [0, \infty]$ such that
\begin{enumerate}
\item $\mu(\varnothing) = 0$,
\item If $E_1, E_2, \ldots \in \mathcal{X}$ are mutually disjoint then
\[
\mu\left(\bigcup_{n \in \mathbb{N}}E_n\right) = \sum_{n \in \mathbb{N}}\mu(E_n).
\]
The triple $(X, \mathcal{X}, \mu)$ is called a measure space.
\end{enumerate}
\end{defn}

Examples of measure:
\begin{itemize}
\item Let $X$ be an arbitrary set with a $\sigma$-algebrao $\mathcal{X}$. Then,
\[
\mu(A) = \begin{cases} |A| & \text{ if } $A$ \text{ is a finite set} \\
\infty & \text{ otherwise.}
\end{cases}
\]
is called a counting measure.

\item One the measure space $(\mathbb{R}, \mathcal{B}(\mathbb{R}))$ define a 
measure $\mu$ such that
\begin{eqnarray*}
\mu([a, b]) &=& b - a \\
\mu((a, b)) &=& b - a \\
\mu((a, b]) &=& b - a \\
\mu([a, b)) &=& b - a.
\end{eqnarray*}
It is called the Lebesgue measure. Note that $(a, b] = [a, b] \cap (a, \infty)$
and $[a, b) = [a, b] \cap (-\infty, b)$ so that semi-open intervals are members
of the Borel $\sigma$-algebra $\mathcal{B}(\mathbb{R})$.

\item One the measure space $(\mathbb{R}, \mathcal{B}(\mathbb{R}))$ define a 
measure $P$ such that
\[
P(B) = \frac{1}{\sqrt{2\pi}}\int_B e^{-x^2/2}dx,
\]
is called the probability measure.
\end{itemize}

\begin{prop}\label{s1p4}Let $(X, \mathcal{X}, \mu)$ be a measure space. Then,
\begin{enumerate}
\item $\mu$ is finitely additive.
\item $\mu$ is monotonic.
\item If $A \subseteq B$ has a finite measure then $\mu(B - A) = \mu(B) - \mu(A)$.
\item If $A$ or $B$ has a finite measure then so does $A \cap B$ and
\item If $A$ or $B$ has a finite measure then $\mu(A \cup B) = \mu(A) + \mu(B) -
\mu(A \cap B)$.
\end{enumerate}
\end{prop}
\begin{proof}
Let $A_1, \ldots A_n \in \mathcal{X}$ be mutually disjoint. Then, 
\[
\mu\left(\bigcup_{i=1}^n A_i\right) = 
\mu\left( A_1 \cup \left(\bigcup_{i=2}^n A_i\right)\right) = 
\mu(A_1) + \mu\left(\bigcup_{i=2}^n A_i\right)
\]
Carrying out this step $n - 2$ times more, we get
\[
\mu\left(\bigcup_{i=1}^n A_i\right) = \sum_{i=1}^n \mu(A_i).
\]
This is called the finite additivity of $\mu$.

Let $A, B \in \mathcal{X}$ be such that $A \subseteq B$. Then $B = A \cup (B - A)$
and the sets $A$ and $B - A$ are disjoint. Therefore, $\mu(B) = \mu(A) + 
\mu(B - A) \ge \mu(A)$.
This is called the monotonicity of $\mu$. Further, when $A \subset B$, and 
$\mu(A)$ is finite, we can subtract it from both sides of the equation to get
$\mu(B - A) = \mu(B) - \mu(A)$.

Without loss of generality, let $A$ have finite measure. Then $A = (A \cap B^c)
\cup (A \cap B)$ where the union is disjoint. Therefore, $\mu(A) = \mu(A \cap B^c)
+ \mu(A \cap B)$. Since measure is a non-negative quantity and lhs is finite, it
immediately follows that $A \cap B$ (and $A \cap B^c$) has finite measure.

Note the following set identities,
\begin{eqnarray*}
A &=& (A \cap B^c) \cup (A \cap B) \\
B &=& (A^c \cap B) \cup (A \cap B) \\
A \cup B &=& (A \cap B^c) \cup (A \cap B) \cup (A^c \cap B),
\end{eqnarray*}
where all the unions are among disjoint sets. Therefore,
\begin{eqnarray*}
\mu(A) &=& \mu(A \cap B^c) \cup \mu(A \cap B) \\
\mu(B) &=& \mu(A^c \cap B) \cup \mu(A \cap B) \\
\mu(A \cup B) &=& \mu(A \cap B^c) \cup \mu(A \cap B) \cup \mu(A^c \cap B),
\end{eqnarray*}
If one of $A$ of $B$ has a finite measure then $A \cap B$ will have a finite
measure permitting us to subtract $\mu(A \cap B)$ from the equation obtained
by summing the first two equations to get
\[
\mu(A \cup B) = \mu(A) + \mu(B) - \mu(A \cap B).
\]
\end{proof}

\begin{prop}\label{s1p5}Let $(X, \mathcal{X}, \mu)$ be a measure space and let
$E_1 \subseteq E_2 \subseteq \ldots$ be such that
\[
\bigcup_{n \ge 1}E_n = E.
\]
Then,
\[
\mu(E) = \lim_{n \rightarrow \infty}\mu(E_n).
\]
\end{prop}
\begin{proof}
If $E_0 = \varnothing$ then for a finite $n$, $E_n = (E_n - E_{n-1}) \cup 
(E_{n-1} - E_{n-2}) \cup \cdots (E_2 - E_1) \cap (E_1 - E_0)$ while for $E$ 
itself,
\[
E = \bigcup_{n >= 0} (E_{n} - E_{n-1}).
\]
The sets $E_{n} - E_{n-1}$ are disjoint for every $n$ so that
\begin{eqnarray*}
\mu(E) &=& \sum_{n \ge 1} \mu(E_n - E_{n-1}) \\
  &=& \lim_{n \rightarrow \infty}\sum_{k=1}^n \mu(E_k - E_{k-1}) \\
  &=& \lim_{n \rightarrow \infty}\mu\left(\bigcup_{k=1}^n (E_k - E_{k-1})\right) \\
  &=& \lim_{n \rightarrow \infty}\mu(E_n)
\end{eqnarray*}
\end{proof}

\begin{prop}\label{s1p6}Let $(X, \mathcal{X}, \mu)$ be a measure space and
$E_1, E_2, \ldots \in \mathcal{X}$. Then
\[
\mu\left(\bigcup_{n \ge 1}E_n\right) \le \sum_{n \ge 1}\mu(E_n).
\]
\end{prop}
\begin{proof}
We will prove by induction on the number of sets $E_n$. If there are two sets
then, using item 5 of proposition \ref{s1p4},
\[
\mu(E_1 \cup E_2) = \mu(E_1) + \mu(E_2) - \mu(E_1 \cap E_2) \ge \sum_{k=1}^2\mu(E_k).
\]
Let the induction hypothesis be true for $p$ sets. Consider
\[
\mu\left(\bigcup_{k \ge 1}^{p+1}E_k\right) = 
\mu\left(\bigcup_{k \ge 1}^{p}E_k \cup E_{p+1}\right) \ge 
\mu\left(\bigcup_{k \ge 1}^{p}E_k\right) + \mu(E_{p+1})
\]
by induction hypothesis. Use the induction hypothesis once again for the first
term on rhs to get
\[
\mu\left(\bigcup_{k \ge 1}^{p+1}E_k\right) \ge \sum_{k=1}^{p+1}\mu(E_k).
\]
Therefore, the hypothesis is true for all $p \in \mathbb{N}$.
\end{proof}

\begin{prop}\label{s1p7}Let $(X, \mathcal{X}, \mu)$ be a measure space and let
$E_1 \supseteq E_2 \supseteq \ldots$ be such that 
\[
\bigcap_{n \ge 1}E_n = E
\]
and $\mu(E_1) < \infty$ then
\[
\lim_{n \rightarrow \infty}\mu(E_n) = \mu(E).
\]
\end{prop}
\begin{proof}
In this proof $E_k^c = E_1 - E_k$ for all $k \ge 1$, that is, complementation is
with respect to $E_1$. Since $E_1 \supseteq E_2 \supseteq \ldots$ and $\cap E_k 
= E$, we also have $E_1^c \subseteq E_2^c \subseteq \ldots$ and $\cup E_k^c = E^c$.
Therefore, by proposition \ref{s1p5},
\[
\mu(E^c) = \lim_{n \rightarrow \infty}\mu(E_n^c).
\]
Now, $\mu(E) + \mu(E^c) = \mu(E_1)$ and $\mu(E_n) + \mu(E_n^c) = \mu(E_1)$ so that
\[
\mu(E_1) - \mu(E) = \lim_{n \rightarrow \infty}(\mu(E_1) - \mu(E_n)).
\]
Since $\mu(E_1) < \infty$, the proposition follows immediately.
\end{proof}

\end{document}