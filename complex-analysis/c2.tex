\chapter{Continuity and Differentiability}\label{c2}
\section{Elementary topological properties}\label{c2s1}
The modulus of the complex numbers provides a natural way to measure distance
between them. If $z_1$ and $z_2$ are two complex numbers then $d(z_1, z_2)
= \abs{z_1 - z_2}$ has all the properties of a metric. In fact, this metric is
identical to the Euclidean metric of the space $\sor^2$. Therefore the two 
spaces are topologically identical. In particular, the Heine-Borel theorem that
guarantees the compactness of a closed and bounded subset of $\soc$ holds good.

For sake of completeness, we will mention the definitions of limits and 
continuity in $\soc$.

\begin{defn}\label{c2s1d1}
Let $f: A \rightarrow B$ where $A, B \subset \soc$. Then, for $z_0 \in A$,
\[
\lim_{z \rightarrow z_0}f(z) = L
\]
if for any $\epsilon > 0$ we can find a $\delta > 0$ such that $0 < \abs{z-z_0}
< \delta \Rightarrow \abs{f(z) - L} < \epsilon$.
\end{defn}
\begin{rem}
The limit of a function at $z_0$ does not require that the function be defined 
at that point.
\end{rem}

\begin{defn}\label{c2s1d2}
Let $f: A \rightarrow B$ where $A, B \subset \soc$. $f$ is said to be
continuous at $z_0 \in A$ if for any $\epsilon > 0$ we can find $\delta > 0$
such that $\abs{z - z_0} \le \delta \Rightarrow \abs{f(z) - f(z_0)} \le 
\epsilon$.
\end{defn}
\begin{rem}
The continuity of a function at $z_0$ requires that the function be defined at
that point.
\end{rem}

\section{Differentiability}\label{c2s2}
If $f: A \rightarrow B$ with $A, B \subset \sor$ and $x_0 \in A$ then $f$
is differentiable at $x_0$ if the limit 
\[
\lim_{x \rightarrow x_0}\frac{f(x) - f(x_0)}{x - x_0}
\]
exists and is unique. The uniqueness of limit means that the left and right
handed limits are identical. We can extend this definition to the case when
$A, B \in \soc$. However, the condition of uniqueness now requires that we
get the same limit irrespective of the direction in which we approach $z_0$
from its neighbourhood. Let us assume that $f: A \rightarrow B$ with $A, B
\subset \soc$ is differentiable at $z_0 \in A$. Therefore, the limit
\[
\lim_{z \rightarrow z_0} \frac{f(z) - f(z_0)}{z - z_0}
\]
exists and is unique. In particular, we get the same limit when we approach
$z_0$ along the real and the imaginary axes. Let us first write
\[
f(z) = u(x, y) + iv(x, y),
\]
where $u, v: A \rightarrow B$ and $A, B$ are considered subsets of $\sor^2$.
When we approach $z_0$ along the real axis, the limit is
\[
\lim_{x \rightarrow x_0} \frac{u(x,y_0)+iv(x,y_0) - u(x_0,y_0)-iv(x_0,y_0)}
{x-x_0} = u_x + iv_x.
\]
If, on the other hand, we approach $z_0$ along the imaginary axis, the limit
is
\[
\lim_{y \rightarrow y_0} \frac{u(x_0,y)+iv(x_0,y) - u(x_0,y_0)-iv(x_0,y_0)}
{i(y-y_0)} = -iu_y + v_y.
\]
The derivatives are evaluated at the point $(x_0, y_0)$. Since the two limits 
are equal, we have
\begin{eqnarray}
u_x &=& v_y \label{c2s2e1} \\
u_y &=& -v_x.\label{c2s2e2}
\end{eqnarray}
These equations are called the \emph{Cauchy-Riemann} conditions. We have
shown that
\begin{thm}\label{c2s2t1}
If $f:A \rightarrow B$, $A, B \subset \soc$ is differentiable at $z_0 = x_0 + 
iy_0$ and if $f(x, y) = u(x, y) + iv(x, y)$ then the functions $u$ and $v$
satisfy the Cauchy-Riemann conditions.
\end{thm}

We will next prove that the Cauchy-Riemann conditions are also sufficient for
a function to be differentiable. That is,
\begin{thm}\label{c2s2t2}
Let $f:A \rightarrow B$, $A, B \subset \soc$, $f(z) = u(x, y) + iv(x, y)$. If
the functions $u$ and $v$ satisfy the Cauchy-Riemann conditions at the point
$z_0 = x_0 + iy_0$ in $A$ then $f$ is differentiable at $z_0$ if the partial
derivatives $u_x, u_y, v_x, v_y$ are also continuous at $(x_0, y_0)$.
\end{thm}
\begin{proof}
Since the partial derivatives of $u$ and $v$ are continuous at $(x_0, y_0)$
we can find $\epsilon_1, \epsilon_2, \epsilon_3, \epsilon_4 > 0$ such that
\begin{eqnarray*}
u(x_0 + \delta x, y_0 + \delta y) &=& u(x_0, y_0) + u_x\delta x + u_y\delta_y + 
  \epsilon_1 \delta x + \epsilon_2 \delta y \\
v(x_0 + \delta x, y_0 + \delta y) &=& v(x_0, y_0) + v_x\delta x + v_y\delta y + 
  \epsilon_3 \delta x + \epsilon_4 \delta y
\end{eqnarray*}
Now,
\[
f(z_0 + \delta z) = u(x_0 + \delta x, y_0 + \delta y) + 
                   iv(x_0 + \delta x, y_0 + \delta y)
\]
or
\begin{eqnarray*}
f(z_0+\delta z) &=& u(x_0,y_0) + iv(x_0,y_0) + (u_x+iv_x)\delta x + 
     (-iu_y+v_y)i\delta y + \\
& &  (\epsilon_1 + i\epsilon_3)\delta x + (-i\epsilon_2 + \epsilon_4)i\delta y
\end{eqnarray*}
or
\begin{eqnarray*}
f(z_0+\delta z)-f(z_0) &=& (u_x+iv_x)\delta x + (-iu_y+v_y)i\delta y + \\
& &  (\epsilon_1 + i\epsilon_3)\delta x + (-i\epsilon_2 + \epsilon_4)i\delta y.
\end{eqnarray*}
Using Cauchy-Riemann conditions, this equation becomes
\begin{eqnarray*}
f(z_0+\delta z)-f(z_0) &=& (u_x+iv_x)\delta z + \\ 
& &  (\epsilon_1 + i\epsilon_3)\delta x + (-i\epsilon_2 + \epsilon_4)i\delta y.
\end{eqnarray*}
from which we get
\[
\abs{\frac{f(z_0+\delta z)-f(z_0)}{\delta z}-(u_x+iv_x)} = 
\abs{(\epsilon_1 + i\epsilon_3) \frac{\delta x}{\delta z} + 
(-i\epsilon_2 + \epsilon_4)\frac{i\delta y}{\delta z}}
\]
Since $\abs{\delta x/\delta z} \le 1$ and $\abs{\delta y/\delta z} \le 1$, we
have
\[
\abs{\frac{f(z_0+\delta z)-f(z_0)}{\delta z}-(u_x+iv_x)} \le
\abs{(\epsilon_1 + i\epsilon_3)} + \abs{(-i\epsilon_2 + \epsilon_4)}
\]
Thus, the derivative of $f$ at $z_0$ exists and is $u_x + iv_x$.
\end{proof}

\begin{defn}\label{c2s2d1}
A function $f:A \rightarrow B$, $A, B \subset \soc$ is said to be analytic at
$z_0 \in A$ if it is differentiable at $A$. 
\end{defn}

In the context of real functions, analyticity at a point $x_0$ means that the
function can be written as a power series in the neighbourhood of $x_0$. Thus,
the function has all derivatives at $x_0$. We defined a complex function to
be analytic based on the existence of just the first derivative. This is not
incorrect because we will show in the next chapter that when a complex function
differentiable once is differentiable any number of times. Anticipating this
result, we can assume that all partial derivatives of $u$ and $v$ exist.
Differentiating \eqref{c2s2e1} with respect to $x$ and \eqref{c2s2e2} with
respect to $y$ and adding the result we get
\begin{equation}\label{c2s2e3}
u_{xx} + u_{yy} = 0.
\end{equation}
Similarly, by differentiating \eqref{c2s2e1} with respect to $y$, \eqref{c2s2e2}
with respect to $x$ and adding the results we get
\begin{equation}\label{c2s2e4}
v_{xx} + v_{yy} = 0.
\end{equation}
Thus, the real and imaginary parts of an analytic complex function satisfy
Laplace's equation. Solutions of Laplace's equations are called \emph{harmonic
functions}. Therefore we say that the real and imaginary part of an analytic
function are harmonic functions.

The nature of an extremum of a function $u$ of two variables is determined by 
the value of the Hessian 
\[
\begin{vmatrix} u_{xx} & u_{xy} \\ u_{yx} & u_{yy} \end{vmatrix}
\]
at the critical point. If $u$ is a harmonic function then the value of
this determinant is $-(u_{xx})^2 - (u_{xy})^2 < 0$. Therefore we have,
\begin{thm}\label{c2s2t3}
The real and imaginary parts of an analytic complex function cannot have a 
maximum or a minimum in the domain of analyticity. They can at most have a 
saddle point.
\end{thm}

\begin{rem}
Theorem \ref{c2s2t3} is a consequence of the harmonic nature of the real and
imaginary part. Harmonic functions in any number of dimensions do not have
a maximum or a minumum in the domain in which they satisfy Laplace's equation.
\end{rem}

Another consequence of the Cauchy-Riemann conditions is 
\[
u_xv_x + u_yv_y = 0.
\]
In terms of gradients, we can write this as
\begin{equation}\label{c2s2e5}
\grad u \cdot \grad v = 0.
\end{equation}
Since the gradient at a point is normal to the level surface at the point,
\eqref{c2s2e5} implies that the level surfaces of the real and imaginary parts
of an analytic complex function are normal to each other.

If $f(z) = u(x, y) + iv(x, y)$ is an analytic function and if $u$ represents
the electric potential then the level sets of $u$ are the equipotential
surfaces. The level sets of $v$ are orthogonal to the equipotetial surfaces
and they represent the electric field line. Similarly, if $u$ represents the
velocity potential of an incompressible, inviscid flow then $v$ represents the
streamlines of the flow. A representation of two dimensional potentials as
one of the components of an analytic function will be used later on to solve
boundary value problems.

\section{Problems and examples}\label{c2s3}
\begin{enumerate}
\item Let $f_1(z) = u(x, y) + iv(x, y)$ and $f_2(z) = u(x, y) - iv(x, y)$ be
differentiable in a region $R$. Then, applying Cauchy-Riemann conditions
to $f_1$ we have
\begin{eqnarray*}
u_x &=& v_y \\
u_y &=& -v_x.
\end{eqnarray*}
Applying them to $f_2$ gives
\begin{eqnarray*}
u_x &=& -v_y \\
u_y &=& v_x.
\end{eqnarray*}
The four equations will be true only if all partial derivatives are zero.
Therefore, $f_1(z) = \bar{f}_2(z)$ must be a constant. \cite{aw}

\item Let $f(z) = u(x, y) + iv(x, y)$ be analytic. Then $\bar{f}(\bar{z}) =
u(x, -y) - iv(x, -y)$. We examine the Cauchy-Riemann conditions for $\bar{f}(
\bar{z})$,
\[
u_x = (-v)_{(-y)} = -(-v)_y = v_y
\]
and
\[
u_{(-y)} = -u_y = -(-v)_x = v_x.
\] 
Both these equations are true because $f$ is analytic \cite{aw}.

\item Let $f(z) = u(x, y) + iv(x, y)$ be analytic. Then
\[
\frac{df}{d\bar{z}} = 0
\]
unless $f$ is a constant \cite{aw}. Thus, an analytic function 
depends on $z = x + iy$. It is not an arbitrary function of $x$ and $y$.
\[
\frac{df}{d\bar{z}} = \frac{df}{dz}\frac{dz}{d\bar{z}} = \frac{df}{dz}\left(
\frac{dx}{d\bar{z}} + i\frac{dy}{d\bar{z}}\right).
\]
As
\begin{eqnarray*}
x &=& \frac{z + \bar{z}}{2} \\
y &=& \frac{z - \bar{z}}{2i} 
\end{eqnarray*}
we get
\[
\frac{df}{d\bar{z}} = \frac{df}{dz}\left(\frac{1}{2} + i\frac{-1}{2i}\right)
= 0.
\]
\end{enumerate}
