\chapter{Integration - 1}\label{c4}
\section{The definition}\label{c4s1}
It is easy to extend the definition of the Riemann integral of a real-valued
function of a single variable to the complex-valued function of a single
complex variable. Unlike the case of a single real variable in which there
is a single path to reach a point $b$ from a point $a$, two points in a complex
plane can be joined with an infinitely many curves. Therefore, when we define
an integral of a complex-valued function $f$ between two points $z_a$ and
$z_b$ we must specify the curve along which the integral is taken. Let us assume
that $C$ is a regular curve, that is, it can be expressed by the parametric
equation $x = x(t)$ and $y = y(t)$. The points $z_a$ and $z_b$ are characterized
with the values $t_a$ and $t_b$ of the parameter. We want to define
\begin{equation}\label{c4s1e1}
I = \int_C f(z)dz.
\end{equation}
Divide the curve $C$ into a mesh $z_0 = z_a, z_1, \ldots, z_n = z_b$. If $\xi_k$
is a point on the segment joining $z_{k-1}$ and $z_k$, consider the sum
\begin{equation}\label{c4s1e2}
S_n = \sum_{k=1}^n f(\xi_k)(z_{k} - z_{k-1}).
\end{equation}
If 
\[
\lim_{n \rightarrow \infty}S_n
\]
exists then it is defined as the integral in \eqref{c4s1e1}. Writing
$f(\xi_k) = u(\xi_k) + iv(\xi_k)$ and $z_k = x_k + iy_k$,
\begin{eqnarray*}
S_n &=& \sum_{k=1}^n u(\xi_k)(x_k - x_{k-1}) - v(\xi_k)(y_k - y_{k-1}) + \\
 & & i\sum_{k=1}^n u(\xi_k)(y_k - y_{k-1}) + v(\xi_k)(x_k - x_{k-1}).
\end{eqnarray*}
The points $\xi_k$ can always be written as the pair $\eta_k, \zeta_k$ so that
\begin{eqnarray}
S_n &=& \sum_{k=1}^n u(\eta_k, \zeta_k)(x_k - x_{k-1}) - v(\eta_k, \zeta_k)
(y_k - y_{k-1}) + \nonumber \\
 & & i\sum_{k=1}^n u(\eta_k, \zeta_k)(y_k - y_{k-1}) + v(\eta_k, \zeta_k)
     (x_k - x_{k-1}).\label{c4s1e3}
\end{eqnarray}
In the limit $n \rightarrow \infty$,
\begin{equation}\label{c4s1e4}
I = \int_{(x_a,y_a)}^{(x_b,y_b)}u(x,y)dx - v(x,y)dy + 
i\int_{(x_a,y_a)}^{(x_b,y_b)}u(x,y)dy + v(x,y)dx.
\end{equation}
Thus, the integral of a complex-valued function can be expressed in terms of
four real-valued integrals. One can express all of these in terms of the
parameter $t$ as
\begin{eqnarray}\label{c4s1e5}
I &=& \int_{t_a}^{t_b}\left[
u(x(t), y(t))\frac{dx}{dt} - v(x(t),y(t))\frac{dy}{dt}\right]dt \nonumber \\
 & &i\int_{t_a}^{t_b}\left[u(x(t),y(t))\frac{dy}{dt} + v(x(t),y(t))\frac{dx}{dt}
     \right]dt
\end{eqnarray}
or
\begin{eqnarray}\label{c4s1e6}
I &=& \int_{t_a}^{t_b}\left(u(x(t),y(t)) + iv(x(t),y(t))\right)\frac{dx}{dt}dt
+ \nonumber \\
 & & i\int_{t_a}^{t_b}\left(u(x(t),y(t)) + iv(x(t),y(t))\right)\frac{dy}{dt}dt
\end{eqnarray}
or
\begin{equation}\label{c4s1e7}
I = \int_{t_a}^{t_b}f(z(t))\frac{dz}{dt}dt.
\end{equation}
If we can find a function $g$ such that
\begin{equation}\label{c4s1e8}
f(z) = \frac{dg}{dz}
\end{equation}
then
\begin{equation}\label{c4s1e9}
\int_C f(z)dz = g(z(t_b)) - g(z(t_a)).
\end{equation}
Equation \eqref{c4s1e9} is the fundamental theorem of calculus for complex-
valued functions of complex variables \cite{dk}.

An immediate consequence of the Riemann integral of $f$ is
\begin{thm}[Darboux inequality]\label{c4s1t1}
\[
\abs{\int_C f(z)dz} \le \max\abs{f} L,
\]
where $L$ is the length of the curve $C$ and the maximum is taken over all
points on $C$.
\end{thm}
\begin{proof}
From equation \eqref{c4s1e2},
\[
\int_C f(z)dz = \lim_{n \rightarrow \infty}\sum_{k=1}^n f(\xi_k)(z_k - z_{k-1})
\]
so that
\begin{eqnarray*}
\abs{\int_C f(z)dz} &\le&
\abs{\lim_{n \rightarrow \infty}\sum_{k=1}^n f(\xi_k)(z_k - z_{k-1})} \\
 &\le& \max\abs{f}\abs{\lim_{n \rightarrow \infty}\sum_{k=1}^n(z_k - z_{k-1})}
= \max\abs{f}L.
\end{eqnarray*}
\end{proof}

From equation \eqref{c4s1e9} it appears that the value of the integral of a
complex-valued function along a curve $C$ depends only on the end points and
is therefore unique. But this is not always true. Several complex functions 
are multi-valued and the value of the primitive $g$ of the integral $f$ 
depends on the contour. However, if there are two curves $C_1$ and $C_2$ such
that they have the same end points and $f$ is analytic on the curves and in the
region bounded by the curves then the problem of multi-valuedness does not 
arise and the rhs of \eqref{c4s1e9} is unique. We will prove this statement as
\begin{thm}(Cauchy-Goursat)\label{c4s1t2}
Let $C$ be a piecewise, regular, closed curve in the complex plane and let a
function $f:\soc \rightarrow \soc$ be analytic on and within C. Then
\[
\oint_C f(z)dz = 0.
\]
\end{thm}
\begin{proof}
We first show that this statement is true if $f(z) = z^n$, $n \ge 0$. For this
choice of $f$,
\[
\oint_C z^n dz = \frac{1}{n+1} z^{n+1}\big|_{t_a}^{t_b},
\]
where $t_b$ is the value of the parameter when the curve closes on itself at
the initial point. Since both values of the parameter corresponding to the
same value of $z$ and therefore of $z^{n+1}$, the value of the integral is $0$.
Thus,
\begin{equation}\label{c4s1e10}
\oint_C z^n dz = 0.
\end{equation}
Define the function $g$ as
\begin{equation}\label{c4s1e11}
g(z, z_0) = f(z) - f(z_0) - (z - z_0)(Df)(z_0).
\end{equation}
The function $g$ is the remainder of the Taylor series of $f$ beyond the
linear term. Let the set of points on $C$ and in the interior of $C$ be denoted
by $A$. We divide $A$ into a a mesh of subsets $A_1, \ldots, A_n$ such that
\begin{equation}\label{c4s1e12}
A = \bigcup_{j=1}^n A_j
\end{equation}
such that for all $z, z_0 \in A_j$ and a fixed $\epsilon > 0$, there exists
$\delta_j(\epsilon) > 0$ such that
\begin{equation}\label{c4s1e13}
\abs{z - z_0} < \delta_j(\epsilon) \Rightarrow \abs{g(z, z_0)} < \epsilon
\abs{z - z_0}.
\end{equation}
Thus, in each $A_j$ the function $g(z, z_0)$ can be made as close to zero as
needed so that $f(z)$ can be approximated as the linear function
\[
f(z_0) + (z - z_0)(Df)(z_0).
\]
Since $f$ is analytic on and inside $C$, it is differentiable at all points
in the region $A$ and therefore such an approximation is always valid in a
small enough neighbourhood of every point $z_0 \in A$. If the neighbourhoods
are chosen to be squares $B_j$ of sides $l_j$ then the maximum distance
between any two points of $B_j$ is $\sqrt{2}l_j$ and hence we have
\begin{equation}\label{c4s1e14}
z, z_0 \in B_j \Rightarrow \abs{g(z,z_0)} < \epsilon \cdot \sqrt{2}l_j.
\end{equation}
We choose $A_j = B_j \cap A$. We can also write the integral
\[
\oint_C f(z)dz = \sum_{j=1}^n \oint_{C_j}f(z)dz,
\]
where $C_j$ is the boundary of $A_j$. This equation is justified because along
all the edges of $A_j$ in the interior of the curve $C$, the integration is
carried out in one direction along one of $C_j$ and the opposite direction
along the neighbouring $C_k$. The only surviving contributions come from the
boundaries of $A_j$ which lie on $C$. Using equation \eqref{c4s1e11} in the
previous equation we get
\begin{equation}\label{c4s1e15}
\oint_C f(z)dz = \sum_{j=1}^n \oint_{C_j}\left(g(z,z_0) + f(z_0) + (z - z_0)
(Df)(z_0)\right)dz.
\end{equation}
From equation \eqref{c4s1e10},
\[
\oint_{C_j}f(z_0)dz = f(z_0)\oint_{C_j}dz = 0
\]
and
\[
\oint_{C_j}(z - z_0)(Df)(z_0)dz = (Df)(z_0)\oint_{C_j}zdz - (Df)(z_0)(z_0)
\oint_{C_j}dz = 0 + 0.
\]
so that \eqref{c4s1e15} becomes
\begin{equation}\label{c4s1e16}
\oint_C f(z)dz = \sum_{j=1}^n \oint_{C_j}g(z,z_0)dz.
\end{equation}
Using Darboux inequality of theorem \ref{c4s1t1}
\[
\abs{\oint_{C_j} g(z,z_0)dz} \le \max{\abs{g(z,z_0)}}(4l_j) < \epsilon \cdot
4\sqrt{2}l_j^2.
\]
so that
\[
\abs{\oint_C f(z)dz} < 4\sqrt{2}\epsilon \sum_{j=1}^n l_j^2
\]
If the curve $C$ can be enclosed in a square of length $L$ then
\[
\sum_{j=1}^n l_j^2 < L^2
\] 
so that
\[
\abs{\oint_C f(z)dz} < 4\sqrt{2}\epsilon L^2.
\]
Thus, the lhs can be made as small as possible so that
\[
\oint_C f(z)dz = 0.
\]
\end{proof}

An immediate consequence of Cauchy-Goursat theorem is
\begin{thm}(Cauchy integral formula)\label{c4s1t3}
Let $f:\soc \rightarrow \soc$ be analytic in a simply-connected open subset
$U$ of the complex plane. If a curve $C$ lies entirely in $U$ then
\begin{equation}\label{c4s1e17}
\frac{1}{2\pi i}\oint_C \frac{f(z)}{z - z_0}dz = \begin{cases}
f(z_0) & \text{ if $z_0$ is in the interior of $C$} \\
0      & \text{ if $z_0$ is in the exterior of $C$}
\end{cases}
\end{equation}
\end{thm}
\begin{proof}
Let $z_0$ be a point in the interior of the curve $C$. Draw a circle $C_0$ of 
radius $r$ around it. Join $C_0$ with $C$ by two closely spaced lines $L_1$ and
$L_2$ so that the $C, L_0, C_0, L_1$ form a closed curve in which the function
\[
\frac{f(z)}{z - z_0}
\]
is analytic and
\[
\oint_{C + L_0 + C_0 + L_1}\frac{f(z)}{z - z_0}dz = 0.
\]
If $L_0$ and $L_1$ are very close to each other then
\[
\int_{L_0} \frac{f(z)}{z - z_0}dz = 
-\int_{L_1} \frac{f(z)}{z - z_0}dz
\]
so that
\[
\oint_{C + C_0}\frac{f(z)}{z - z_0}dz = 0.
\]
or
\[
\oint_C \frac{f(z)}{z - z_0}dz = -\oint_{C_0}\frac{f(z)}{z - z_0}dz.
\]
Choose $R$ so small that
\[
f(z) = f(z_0) + (z - z_0)(Df)(z_0)
\]
on $C_0$ and
\[
\oint_C \frac{f(z)}{z - z_0}dz = -f(z_0)\oint_{C_0}\frac{dz}{z - z_0} -
 (Df)(z_0)\oint_{C_0}dz.
\]
On $C_0$, $z - z_0 = Re^{i\theta}$ so that $dz = iRe^{i\theta}d\theta$ and
\[
\oint_C \frac{f(z)}{z - z_0}dz = -if(z_0)\int_{2\pi}^0 d\theta - iR(Df)(z_0)
\int_{2\pi}^0 e^{i\theta}d\theta.
\]
The limits of the $\theta$-integral are from $2\pi$ to $0$ because we traverse
$C$ anti-clockwise and hence $C$ clock-wise. Thus,
\[
\frac{1}{2\pi i}\oint_C \frac{f(z)}{z - z_0}dz = f(z_0).
\]
If $z_0$ is outside $C$ then the integrand
\[
\frac{f(z)}{z - z_0}
\]
is analytic on $C$ and inside it so that the integral is zero by Cauchy-Goursat
theorem \ref{c4s1e2}.
\end{proof}

Equation \eqref{c4s1e17} is of the form
\begin{equation}\label{c4s1e18}
f(z_0) = \int K(z, z_0)f(z)fz
\end{equation}
with
\[
K(z, z_0) = \frac{1}{z - z_0}.
\]
The value of the function $f$ at $z_0$ is expressed as an integral of a product
of terms, one of which is $f$ and the other is a function that depends on both
$z$ and $z_0$. Such a representation is called an integral representation of 
the function and $K$ is called the \emph{kernel} of the representation.

There is another notable observation about Cauchy integral formula of 
\eqref{c4s1e17}. The value of $f$ at $z_0$ is determined by the value of $f$
on a closed contour. The boundary conditions alone determine the values of the
function at all points inside the boundary. This is possible only because the
real and imaginary parts of $f$ individually satisfy Laplace's equation. We do
not have such a situation happening in the case of functions of real variable 
unless we also know the differential equation satisfied by the function.

We will now consider derivatives of functions having an integral representation.
\begin{thm}\label{c4s1t4}
Let $C$ be a piecewise regular curve of finite length, $g: \soc\rightarrow\soc$
be continuous on $C$ and $f:\soc\rightarrow\soc$ be defined as
\[
f(z) = \frac{1}{2\pi i}\int_C \frac{g(w)}{w - z}dw.
\]
Then $f$ is analytic at any point $z$ not on $C$.
\end{thm}
\begin{proof}
\[
f(z + \delta z) = \frac{1}{2\pi i}\int_C \frac{g(w)}{w - z - \delta z}dw.
\]
so that
\[
f(z + \delta z) - f(z) = \frac{1}{2\pi i}\int_C g(w)\frac{\delta z}{(w - z)^2
- (w - z)\delta z}dw
\]
and
\[
\frac{f(z+\delta z) - f(z)}{\delta z} = \frac{1}{2\pi i}
\int_C \frac{g(w)}{(w - z)^2 - (w - z)\delta z}dw.
\]
Taking the limit $\delta z \rightarrow 0$, one gets
\[
Df = \frac{1}{2\pi i}\int_C \frac{g(w)}{(w - z)^2}dw.
\]
Thus, $f$ is analytic at all points $z$ not on $C$.
\end{proof}

\begin{cor}\label{c4s1c1}
The function $f$ defined in theorem \eqref{c4s1t4} has all derivatives.
\end{cor}
\begin{proof}
One can easily prove by induction that
\[
D^nf = \frac{1}{2\pi i}\int_C \frac{g(w)dw}{(w - z)^{n+1}}.
\]
\end{proof}

\begin{cor}\label{c4s1c2}
All derivatives of analytic function are themselves analytic.
\end{cor}
\begin{proof}
Choose $g = f$ in theorem \ref{c4s1t4} and corollary \ref{c4s1c1}.
\end{proof}

