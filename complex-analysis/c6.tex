\chapter{Analytic Continuation}\label{c6}
\section{Beyond Cauchy integral formula}\label{c6s1}
Cauchy's integral formula (theorem \ref{c4s1t3} tells us that if we are given the
values of an analytic function $f$ on a closed curve $C$ such that $C$ lies 
entirely in the open subset $U$ in which $f$ is analytic then one can find the 
value of $f$ at any point inside $C$. We would now like to consider a more general
question. If we know the values of $f$ in a set $D$ can we infer the value of $f$
in a larger set, given that $f$ is analytic in either of them.

\begin{thm}\label{c6s1t1}
Let $f_1$ and $f_2$ be complex-valued functions analytic on a domain $U \subset \soc$.
If their values coincide in a set whose limit point lies in $U$ then $f_1 = f_2$ in 
$U$.
\end{thm}
\begin{proof}
Let $f_1$ and $f_2$ agree on a set $V$ whose limit point lies in $U$. The function 
$f = f_1 - f_2$ is analytic and has value zero in set $V$. By corollary \ref{c5s1c1}
$f = 0$ throughout $U$.
\end{proof}

This theorem assures us that two different analytic functions cannot coincide in a
neighbourhood. If they differ in a neighbourhood then they differ everywhere else.

When a function is analytic at a point $z_0$ it has a Taylor series expansion
\begin{equation}\label{c6s1e1}
f(z) = \sum_{k \ge 0}a_k^0(z - z_0)^k
\end{equation}
in a neighbourhood $\abs{z - z_0} < \gamma_0$ of $z_0$. Let $z_1$ be a point in
this neighbourhood. Equation \eqref{c6s1e1} allows us to compute $f(z_1)$. Since
the series in \eqref{c6s1e1} is uniformly continuous in the neighbourhood, we can
differentiate it term by term and get the values $D^k(f)(z_1)$ for all $k \ge 1$.
This allows us to construct the representation
\begin{equation}\label{c6s1e2}
f(z) = \sum_{k \ge 0}a_k^1(z - z_1)^k.
\end{equation}
Equation \eqref{c6s1e2} allows us to compute values of $f$ in the neighbourhood
$\gamma_1$ of $z_1$. The values of $f$ computed using \eqref{c6s1e1} and 
\eqref{c6s1e2} coincinde in $\gamma_0 \cap \gamma_1$. Therefore, by theorem 
\ref{c6s1t1} they coincide all throughout. 

We can continue this process all thoughout the region $U$ in which $f$ is analytic.
It allows us to use the Taylor series representation of an analytic function at
one point to compute that at another point in the domain of analyticity. It is 
called \emph{analytic continuation}. Note that one cannot carry out this procedure
beyond the domain in which $f$ is analytic. One can only use extend a power series
representation in one neighbourhood to another one in a different neighbourhood.

\section{Examples}\label{c6s2}
\begin{enumerate}
\item Consider the functions \cite{spiegel1964schaum}
\begin{equation}\label{c6s2e1}
f(z) = \sum_{k=1}^\infty \frac{z^k}{k}.
\end{equation}
If $f_k$ denotes its $k$-th term then,
\[
\frac{f_{k+1}}{f_k} = \frac{k}{k+1}z
\]
and 
\[
\abs{\frac{f_{k+1}}{f_k}} < 1 \text{ if } \abs{z} < 1 + \frac{1}{k}
\]
for all $k \in \soi^+$. Thus, the series converges for all $z$ with $\abs{z} < 1$.
Now consider the function
\begin{equation}\label{c6s2e2}
g(z) = \frac{\pi i}{4} - \frac{\log 2}{2} + 
\sum_{k=1}^\infty \frac{1}{k}\left(\frac{z-i}{1-i}\right)^k.
\end{equation}
The convergence of the series is unaffected by its first two terms. If $g_k$ denotes 
the $k$-th term of the sum then,
\[
\frac{g_{k+1}}{g_k} = \frac{k}{k+1}\frac{z-i}{1-i}.
\]
The series converges if
\[
\abs{\frac{g_{k+1}}{g_k}} < 1
\] 
that is if
\[
\abs{\frac{z-i}{1-i}} < 1 + \frac{1}{k}
\]
for all $k \in \soi^+$. If we write $z = x + iy$ then the left hand side is
\[
\sqrt{\frac{x^2 + (y - 1)^2}{2}} < 1.
\]
That is, the series in \eqref{c6s2e2} converges in the circle with centre $(0, 1)$
and radius $\sqrt{2}$. In terms of the complex variable, the region of convergence
is $\abs{z - i} < \sqrt{2}$.

Now consider $g(0)$. From \eqref{c6s2e2},
\begin{eqnarray*}
g(0) &=& \frac{\pi i}{4} - \frac{\log 2}{2} + 
\sum_{k=1}^\infty \frac{1}{k}\left(\frac{-i}{1-i}\right)^k \\
 &=& \frac{\pi i}{4} - \frac{\log 2}{2} + 
\sum_{k=1}^\infty \frac{1}{k}\left(\frac{1-i}{2}\right)^k \\
 &=& \frac{\pi i}{4} - \frac{\log 2}{2} + f\left(\frac{1-i}{2}\right)
\end{eqnarray*}

We know that $f$ is the Taylor expansion of $-\log(1 - z)$ around the origin and
\[
1 - \frac{1 + i}{2} = \frac{1 - i}{2}
\]
and that
\[
\log\left(\frac{1 + i}{2}\right) = \log\frac{1}{\sqrt{2}} + \frac{i\pi}{4}
= \frac{i\pi}{4} - \frac{\log 2}{2}.
\]
Therefore,
\[
g(0) = \log\left(\frac{1 + i}{2}\right) - \log\left(1 - \frac{1 + i}{2}\right) = 0 = f(0).
\]
Since the two functions agree at $z = 0$, $g$ is an analytic continuation of $f$.

\item Consider the two series \cite{dk}
\begin{eqnarray}
f(z) &=& \sum_{k \ge 0}z^k \label{c6s2e3} \\
g(z) &=& \sum_{k \ge 0}\left(\frac{2}{3}\right)^{k+1}\left(z + \frac{1}{2}\right)^k.
\label{c6s3e4}
\end{eqnarray}
$f$ converges in $\abs{z} < 1$. If $g_k$ denotes the $k$th term of $g$ then the series
converges in the region where
\[
\abs{\frac{g_{k+1}}{g_k}} < 1
\]
that is
\[
\frac{2}{3}\abs{z + \frac{1}{2}} < 1 \Rightarrow \abs{z + \frac{1}{2}} < \frac{3}{2}.
\]
Thus, $f$ converges on the open unit disc centred at the origin and $g$ converges on
the open disc of radius $3/2$ and centre $(-1/2, 0)$. The two discs overlap and if the
two functions are equal at any point in the common region then they are an analytic
continuation of each other.

We can readily conclude that
\begin{eqnarray*}
f\left(\frac{1}{2}\right) &=& 2 \\
g\left(\frac{1}{2}\right) &=& 2.
\end{eqnarray*}
\end{enumerate}

If $f$ and $g$ are analytic functions then they are different algebraic expressions of
the same function. The original, traditional view of a function was an algebraic/functional
expression. The modern view of a function is that of a mapping between two sets. The idea
of analytic continuation is probably the simplest illustration of the power and generality
of the modern concept of a function. It allows us to express a function using different
algebraic forms in different parts of the domain.

\section{Schwarz reflection principle}\label{c6s3}
In the examples considered in section \ref{c6s2} the functions that were analytic 
continuation of each other were defined on open sets with non-null intersections. The
domains of the functions were \emph{not} disjoint. We now relax that requirement.

\begin{prop}\label{c6s3p1}
Let $f: U \rightarrow \soc$ and $g: V \rightarrow \soc$ be analytic on their domains and 
let $R$ be the boundary between $U$ and $V$. If $f(z) = g(z)$ on $R$ then $f$ and $g$ are
analytic continuation of each other and they define a function $h: U \cup V \rightarrow
\soc$ such that
\[
h(z) = \begin{cases} f(z) & \text{ if } z \in U \cup R \\
g(z) & \text{ if } z \in V \cup R
\end{cases}
\]
\end{prop}
\begin{proof}
The open sets $U$ and $V$ has a common boundary $R$. We can construct a closed contour $C$
in $U + V$ such that a part $C_1$ lies in $U$ and $C_2$ in $V$. We can now construct curves
$B_1$ and $B_2$, arbitrarily close to $R$ such that the closed curve $C_1 + B_1$ is a closed
curve in $U$ and $C_2 + B_2$ is a closed curve in $V$. Since $f$ and $g$ are analytic in
$U$ and $V$,
\begin{eqnarray*}
\int_{C_1} f(z)dz + \int_{B_1} f(z)dz &=& 0 \\
\int_{C_2} g(z)dz + \int_{B_2} g(z)dz &=& 0 
\end{eqnarray*}
The curves $B_1$ and $B_2$ are traversed in an opposite sense. Since $f$ and $g$ agree on 
$R$ and are analytic in $U$ and $V$, they also agree on $B_1$ and $B_2$ which are 
arbitrarily close to $R$. Therefore,
\[
\int_{B_1} f(z)dz = -\int_{B_2} g(z)dz
\]
because of the opposite directions along the two curves. Therefore,
\[
\int_{C_1} f(z) dz + \int_{C_2} g(z)dz = 0
\]
and hence, by Morera's theorem (\ref{c4s2t3}) the function $h$ is analytic on $U \cup V$.
As a result, $f$ and $g$ are analytic continuation of each other.
\end{proof}

\begin{prop}\label{c6s3p2}
If $f$ is analytic in $U$ then $\overline{f(z)} = \bar{f}(\bar{z})$.
\end{prop}
\begin{proof}
Since $f$ is analytic in $U$, for any $z_0 \in U$, we can express
\[
f(z) = \sum_{k \ge 0}a_n(z - z_0)^k
\]
in the neighbourhood of $z_0$. Therefore,
\[
\overline{f(z)} = \sum_{k \ge 0}\bar{a}_n(\bar{z} - \bar{z}_0)^k = \bar{f}(\bar{z}).
\]
Since $z_0$ was chosen arbitrarily, this is true for all $U$.
\end{proof}

\begin{prop}\label{c6s3p3}
If $f: U \rightarrow \soc$ is analytic in $U$ and if $C$ is a curve entirely in $U$ then
\[
\overline{\int_C f(z)dz } = \int_{\overline{C}}\bar{f}(\bar{z})dz.
\]
\end{prop}
\begin{proof}
We start with the definition of the Riemann integral
\[
\int_C f(z)dz = \lim_{n \rightarrow \infty}\sum_{k=1}^n f(\xi_k)(z_k - z_{k-1})
\]
where the curve $C$ is divided into a mesh $z_0, z_1, \ldots, z_n$. Then
\[
\abs{\int_C f(z)dz} =  \lim_{n \rightarrow \infty}\sum_{k=1}^n \overline{f(\xi_k)}
\overline{(z_k - z_{k-1})}
\]
Since $f$ is analytic, by proposition \ref{c6s3p2}, $\overline{f(\xi_k)} = \bar{f}
(\bar{\xi}_k)$ and hence,
\[
\abs{\int_C f(z)dz} =  \lim_{n \rightarrow \infty}\sum_{k=1}^n \bar{f}(\bar{\xi}_k)
\overline{(z_k - z_{k-1})} = \int_{\overline{C}} \bar{f}(\bar{z})dz.
\]
Note that the grid $\bar{z}_0, \bar{z}_1, \ldots$ is a grid on the curve $\overline{C}$.
\end{proof}

This proposition allows us to find $g$ given $f$ if $f$ is real-valued on the real axis.
\begin{thm}[Schwarz reflection principle]\label{c6s3t1}
Let $f$ be analytic in the open set $U$ whose boundary consists of a portion $R$ of the
real axis. Then an analytic continuation of $f$ exists on the other side of the real axis
and is given by $g(z) = \bar{f}(\bar{z})$ on $\bar{U} = \{z \in \soc: \bar{z} \in U\}$.
\end{thm}
\begin{proof}
Construct a curve $C$ in $U$ such that the segment $R$ lies on it. Let $C_1$ be the part
of $C$ not on the real axis. Since $f$ is analytic in $U$, by Cauchy-Goursat theorem,
\ref{c4s1t2},
\[
\int_C f(z)dz = 0 \Rightarrow \int_{C_1} f(z)dz + \int_R f(x)dx = 0.
\]
Define $\bar{U} = \{z \in \soc: z \in U\}$. Define $g: \bar{U} \rightarrow \soc$ as
\[
g(z) = \bar{f}(\bar{z}).
\]
Then, by proposition \ref{c6s3p3},
\[
\int_{\overline{C}}\bar{f}(\bar{z})dz = 0
\]
and hence
\[
\int_{\bar{C_1}}g(z)dz - \int_R g(x)dx = 0.
\]
Clearly $f(x) = g(x)$ on $R$ and hence,
\[
\int_C f(z)dz + \int_{\overline{C}} g(z)dz = 0.
\]
Hence, the function $h : U \cup \bar{U} \rightarrow \soc$, defined as
\[
h(z) = \begin{cases}
f(z) & \text{ if } z \in U \\
g(z) & \text{ if } z \in \bar{U}
\end{cases}
\]
is analytic, by proposition \ref{c6s3p1}, throughout $U \cup \bar{U}$. That 
also makes $f$ and $g$ analytic continuations of each other.
\end{proof}

\begin{rem}
Schwarz reflection principles gives the algebraic form of the continued function,
albeit in the special case when the function is real-valued on the real axis.
\end{rem}

\begin{rem}
It is important to have $f$ real-valued on the real axis. Otherwise, $f(z)$
will not be the same as $g(z) = \bar{f}(\bar{z})$ and the integrals of $f$
and $g$ on $R$ will not have the same magnitude.
\end{rem}

\section{Problems}\label{c6s5}
\begin{enumerate}
\item Let
\[
f(z) = \int_0^\infty t^3 e^{-zt}dt.
\]
Show that $f$ is analytic in the half-plane $\abs{z} > 0$. Find a function which is
its analytic continuation in the left-half plane.

Successively integrating by parts, if $\re{z} > 0$,
\[
f(z) = \frac{6}{z^4}.
\]
Now, $6/z^4$ is analytic on the entire complex plane, except $z = 0$. Therefore, it
is the analytic continuation of $f$ in the left-half of the plane.

\item We showed in equation \eqref{c1s3e29} that $\sin^2(z) + \cos^2(z) = 1$ for
all $z \in \soc$ by writing $z = x + iy$. There is an easier way to show it. Since
$\sin^2(x) + \cos^2(x) = 1$ for all $x \in \sor$, the analytic function $\sin^2(z)
+ \cos^2(z) - 1 = 0$ on the real line. Therefore, by analytic continuation it is
zero throughout the complex plane.
\end{enumerate}
