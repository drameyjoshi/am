\chapter{Complex Numbers}\label{c1}
\section{Hierarchy of numbers}\label{c1s1}
The set of natural numbers is a monoid. When extended to the set of integers
it becomes a group under addition. In fact, it is also a ring. The set of
rational numbers is a field. But it is not a complete field. The set of real
numbers is a complete field which also has an order. However, the set $\sor[x]$
, of all polynomials with real coefficients, is not algebraically closed. That 
is, some polynomials do not have real roots. The set of complex numbers is an
algebraically closed, complete field. However, it does not have an order. More
on that later.

\section{Construction of complex numbers}\label{c1s2}
There are several ways to construct the set of complex numbers. We will consider
the one that is the simplest. A real number signifies a displacement along a 
line. A complex number does the same on a plane. Therefore, one expects the
complex number to be made up of two real numbers. Let us consider the set of 
all ordered pairs $(x, y)$ of real numbers with the following operations defined
on them.
\begin{itemize}
\item $(x_1, y_1) + (x_2, y_2) = (x_1 + x_2, y_1 + y_2)$,
\item $(x_1, y_1)(x_2, y_2) = (x_1x_2 - y_1y_2, x_1y_2 + x_2y_1)$.
\end{itemize}
It is easy to check that the set of ordered pairs with these definitions is a
field. It is also easy to convince oneself that $(x, 0)$ has all the properties
of the real numbers. If we denote $(0, 1)$ as $i$ and write $(x, y) = x + iy$
then all the properties above can be derived if we assume $i^2 = -1$. 

We will now show that there is no order relation in $\soc$. An order relation
$<$ on a field $F$ has the following properties.
\begin{itemize}
\item for any $x, y \in F$ exactly one of $x < y, x = y, x > y$ is true.
\item if $x < y$ and $y < z$ then $x < z$.
\item if $x < y$ and $u < v$ then $x + u < y + v$.
\item if $0 < x < u$ and $0 < y < v$ then $xy < uv$.
\end{itemize}
Let, if possible, $\soc$ be ordered. We know that $i \ne 0$. If $i > 0$ then
$i^2 = -1 > 0$ is a contradiction. If $i < 0$ then $-i > 0$ and $(-i)^2 = -1 
> 0$ another contradiction.

Since a complex number can be represented as a point on the plane, we can
as well represent it in the polar form $(r\cos\theta, r\sin\theta)$ or
$r\cos\theta + ir\sin\theta = r(\cos\theta + i\sin\theta)$. The number $r$
is easily seen to be the positive square root $\sqrt{x^2 + y^2}$ and it is
unique for a given complex number. It is called the \emph{modulus} of the
complex number. The angle $\theta$ is called the \emph{argument} and it is
clearly multivalued. All angles $\theta + 2\pi n$, $n \in \soi$ correspond to
the same complex number. The multi-valuedness of the argument has many 
interesting consequences to the nature of functions of complex variables. The
modulus of a complex number $z = x + iy$ is denoted by $\abs{z}$ and the
argument by $\Arg{z}$. The value of $\theta$ in the range $[0, 2\pi)$ is called
the \emph{principal value} of the argument. (An equivalent convention is to let
it be in the range $(-\pi, \pi]$).

We know the Taylor series expansions of the real functions $\sin$ and $\cos$,
\begin{eqnarray*}
\sin\theta &=& \theta - \frac{\theta^3}{3!} + \frac{\theta^5}{5!} - \ldots \\
\cos\theta &=& 1 + \frac{\theta^2}{2!} - \frac{\theta^4}{4!} + \ldots.
\end{eqnarray*}
From these expressions we can easily infer that
\begin{equation}\label{c1s2e1}
\cos\theta + i\sin\theta = e^{i\theta}.
\end{equation}
It is called \emph{Euler's formula}. It allows us to represent a complex number
as $z = (x, y) = x + iy = r\exp(i\theta)$. The last representation simplifies
many calculations and has some immediate interesting consequences. First of all
$e^{i\pi} = -1$ so that $\log(-1) = i\pi$. Logarithms of negative numbers are
imaginary. They not only exist but they are multi-valued. $\log(-1) = i\pi + 
2\pi in$ for $n \in \soi$. Since $e^{2\pi im} = 1$, it is easy to see 
that 
\[
1^{1/n} = \exp\left(i\frac{2\pi m}{n}\right).
\]
If we let $m$ range from $0$ to $n - 1$ we get $n$ different roots of unity.

The conjugate of a complex number $z = (x, y) = x + iy = re^{i\theta}$ is
defined as $\bar{z} = (x, -y) = x - iy = re^{-i\theta}$.

\section{A few inequalities}\label{c1s3}
Since $\abs{z} = r = \sqrt{x^2 + y^2}$, it follows that
\begin{eqnarray}
\abs{x} &\le& \abs{z} \label{c1s3e1} \\
\abs{y} &\le& \abs{z} \label{c1s3e2}
\end{eqnarray}
Now consider
\[
\abs{z_1 + z_2}^2 = (z_1 + z_2)\overline{(z_1 + z_2)} = \abs{z_1}^2 + \abs{z_2}^2
 + z_1\bar{z}_2 + \bar{z}_1z_2.
\]
We can combine the last two terms,
\[
\abs{z_1 + z_2}^2 = \abs{z_1}^2 + \abs{z_2} + 2\re(z_1z_2) \le 
\abs{z_1}^2 + \abs{z_2}^2 + 2\abs{z_1z_2},
\] 
where we used \eqref{c1s3e1} to get the last inequality. Since $\abs{z_1z_2}
= \abs{z_1}\abs{z_2}$, we get 
\begin{equation}\label{c1s3e3}
\abs{z_1 + z_2} \le \abs{z_1} + \abs{z_2}.
\end{equation}
We can use the same reasoning to conclude that
\begin{equation}\label{c1s3e4}
\abs{z_1 - z_2} \ge \abs{z_1} - \abs{z_2}.
\end{equation}
We next prove the Cauchy-Schwarz inequality,
\begin{equation}\label{c1s3e5}
\abs{z_1\bar{w}_1 + \cdots + z_n\bar{w}_n}^2 \le \sum_{i=1}^n\abs{z_i}^2
\sum_{i=1}^n\abs{w_i}^2.
\end{equation}
We can consider $z = (z_1, \ldots, z_n)$ and $w = (w_1, \ldots, w_n)$ to be 
elements of $\soc^n$. We can define the inner product in $\soc^n$ as
\begin{equation}\label{c1s3e6}
(z, w) = \sum_{i=1}^n z_i\bar{w}_i.
\end{equation}
We note the following properties of the inner product.
\begin{eqnarray}
(z, w) &=& \overline{(w, z)} \label{c1s3e7} \\
(\alpha z, w) &=& \alpha(z, w) \label{c1s3e8} \\
(z, \alpha w) &=& \bar{\alpha}(z, w), \label{c1s3e9}
\end{eqnarray} 
where $\alpha \in \soc$. 

In terms of the inner product, the Cauchy-Schwarz inequality is
\begin{equation}\label{c1s3e10}
(z, w)^2 \le (z, z)(w, w).
\end{equation}
In order to prove this, consider the vector
\[
u = z - \frac{(z, w)}{(w, w)}w.
\]
Then
\[
(u, u) = (z, z) - \frac{\overline{(z, w)}}{(w, w)}(z, w) - 
\frac{(z, w)}{(w, w)}(w, z) + \frac{\abs{(z, w)}^2}{(w, w)^2}(w, w).
\]
Using properties \eqref{c1s3e7} to \eqref{c1s3e9} we get
\[
(u, u) = (z, z) - \frac{\abs{(z, w)}^2}{(w, w)}
\]
Since $(u, u) \ge 0$ is always true, equation \eqref{c1s3e10} follows. Since
the modulus of a complex number and its conjugate are the same, we can as well
write equation \eqref{c1s3e5} as
\begin{equation}\label{c1s3e11}
\abs{z_1w_1 + \cdots + z_nw_n}^2 \le \sum_{i=1}^n\abs{z_i}^2
\sum_{i=1}^n\abs{w_i}^2.
\end{equation}

{\color{red}{Add Lagrange's identity}}.

Before leaving this section we note that the inner product in $\soc$ is just
the product $z_1\bar{z}_2$. In the representation using ordered pair of reals,
\[
(z_1, z_2) = (x_1, y_1)(x_2, -y_2) = (x_1x_2 + y_1y_2, x_1y_2 - x_2y_1)
\]
If $(x_1, y_1)$ and $(x_2, y_2)$ were 2-dimensional vectors then the real part
of $(z_1, z_2)$ is their dot product and the imaginary part is the magnitude of
the cross product. This idea is exploited to develop the exterior algebra and
differential forms.

\subsection{Problems and examples}
\begin{enumerate}
\item Let $K$ denote the complex conjugation operator. Consider $K(c_1z_1 + 
c_2z_2)$, where $c_1, c_2 \in \soc$. Then $K(c_1z_1 + c_2z_2) = \bar{c}_1K(z_1)
+ \bar{c}_2K(z_2)$. $K$ is \emph{not} a linear operator \cite{aw}.

\item Consider the sum \cite{aw}
\begin{eqnarray*}
\sum_{n=0}^N e^{in\theta} &=& \frac{1.(e^{iN\theta} - 1)}{e^{i\theta} - 1} \\
 &=& \frac{cos(N\theta)+i\sin(N\theta)-1}{\cos(\theta) + i\sin(\theta) - 1} \\
 &=& \frac{2i\sin(N\theta/2)\cos(N\theta/2) - 2\sin^2(N\theta/2)}
     {2i\sin(\theta/2)\cos(\theta/2) - 2\sin^2(\theta/2)} \\
 &=& \frac{\sin{N\theta/2}}{\sin(\theta/2)}
     \frac{i\cos(N\theta/2)-\sin(N\theta/2)}{i\cos(\theta/2)-\sin(\theta/2)} \\
 &=& \frac{\sin{(N\theta/2)}}{\sin(\theta/2)}
     \exp\left(\frac{N-1}{2}\theta\right) \\
 &=& \frac{\sin{(N\theta/2)}}{\sin(\theta/2)}\left(
     \cos\left(\frac{N-1}{2}\theta\right)+i\sin\left(\frac{N-1}{2}\theta\right)
     \right)
\end{eqnarray*}
from which we readily conclude that
\begin{eqnarray}
\sum_{n=0}^{N-1}\cos(n\theta) &=& \frac{\sin{(N\theta/2)}}{\sin(\theta/2)}
                                  \cos\left(\frac{N-1}{2}\theta\right)  
                                  \label{c1s3e12} \\
\sum_{n=0}^{N-1}\sin(n\theta) &=& \frac{\sin{(N\theta/2)}}{\sin(\theta/2)}
                                  \sin\left(\frac{N-1}{2}\theta\right).
								  \label{c1s3e13}
\end{eqnarray}
These identities would have been much harder to prove without the use of 
Euler's relation.

\item Deriving formulae for sine and cosine of multiple angles using 
trigonometry alone can be quite tedious \cite{aw}. However, using Euler 
relation,
\begin{equation}\label{c1s3e14}
e^{in\theta} = \cos(n\theta) + i\sin(n\theta)
\end{equation}
and 
\[
e^{in\theta} = \left(e^{i\theta}\right)^n = \sum_{j=0}^n\binom{n}{j}\cos^j
(\theta)\sin^{n-j}(\theta)i^{n-j}
\]
We can expand the binomial sum as
\begin{eqnarray}
e^{in\theta} &=& 
\left(\cos^n(\theta)-\binom{n}{2}\cos^{n-2}(\theta)\sin^2(\theta)+\ldots\right) 
 + \nonumber \\
 & &i\left(n\cos^{n-1}(\theta)\sin(\theta)-\binom{n}{3}\cos^{n-3}(\theta)
  \sin^3(\theta) + \ldots\right)  \nonumber \\
 \label{c1s3e15}
\end{eqnarray}
Equating the real and imaginary parts of equations \eqref{c1s3e14} and 
\eqref{c1s3e15} gives the necessary results.

\item Consider the sum \cite{aw}
\begin{eqnarray*}
\sum_{n=0}^\infty p^ne^{inx} &=&
 \lim_{N \rightarrow \infty}\sum_{n=0}^{N-1}p^n e^{inx} \\
 &=& \lim_{N \rightarrow \infty}\frac{p^Ne^{iNx}-1}{pe^{ix}-1} \\
 &=& \lim_{N \rightarrow \infty}\frac{(p^Ne^{iNx}-1)(pe^{-ix}-1)}
                                     {p^2 - 2p\cos(x) + 1}.
\end{eqnarray*}
Equating the real and imaginary parts we get
\begin{eqnarray}
\sum_{n=0}^\infty p^n\cos(nx) &=& \frac{1 - p\cos(x)}{p^2 - 2p\cos(x) + 1} 
\label{c1s3e16} \\
\sum_{n=0}^\infty p^n\sin(nx) &=& \frac{p\sin(x)}{p^2 - 2p\cos(x) + 1} 
\label{c1s3e17} 
\end{eqnarray}

\item Complex circular and hyperbolic functions are defined in a manner 
analogous to their real counterparts \cite{aw}. If $z \in \soc$,
\begin{eqnarray}
\cos(z) &=& \frac{e^{iz} + e^{-iz}}{2} \label{c1s3e18} \\
\sin(z) &=& \frac{e^{iz} - e^{-iz}}{2i} \label{c1s3e19} \\
\cosh(z) &=& \frac{e^{z} + e^{-z}}{2} \label{c1s3e20} \\
\sinh(z) &=& \frac{e^{z} + e^{-z}}{2} \label{c1s3e21}
\end{eqnarray}
from which is it evident that
\begin{eqnarray}
\cos(z) &=& \cosh(iz) \label{c1s3e22} \\
\sin(z) &=& -i\sinh(iz) \label{c1s3e23} \\
\cosh(z) &=& \cos(iz) \label{c1s3e24} \\
\sinh(z) &=& -i\sin(iz). \label{c1s3e25}
\end{eqnarray}
Since 
\[
2\cos(z) = e^{iz} + e^{-iz} = e^{-y}(\cos(x) + i\sin(x)) + e^{y}
(\cos(x) - i\sin(x))
\]
we have
\begin{equation}\label{c1s3e26}
\cos(z) = \cos(x)\cosh(y) - i\sin(x)\sinh(y).
\end{equation}
Thus, $\abs{\cos(z)}^2 = \cos^2(x)\cosh^2(y) + \sin^2(x)\sinh^2(y)$. It is
unbounded as $y \rightarrow \pm\infty$. Thus, unlike their real counterparts,
the complex circular functions do not have a bounded magnitude. We can also
conclude that
\begin{equation}\label{c1s3e27}
\abs{\cos(z)} \ge \abs{\cos(x)}.
\end{equation}

\item Analogous to \eqref{c1s3e26} we can also show that
\begin{equation}\label{c1s3e28}
\sin(z) = \sin(x)\cosh(y) + i\cos(x)\sinh(y).
\end{equation}
so that
\begin{eqnarray*}
\sin^2(z) + \cos^2(z) &=& 
 \sin^2(x)\cosh^2(y) + 2i\sin(x)\cos(x)\sinh(y)\cosh(y) \\
 & & - \cos^2(x)\sinh^2(y) + \\
 & & \cos^2(x)\cosh^2(y) - 2i\sin(x)\cos(x)\sinh(y)\cosh(y)- \\
 & & \sin^2(x)\sinh^2(y) \\
 &=& \cosh^2(y) - \sinh^2(y)
\end{eqnarray*}
or
\begin{equation}\label{c1s3e29}
\sin^2(z) + \cos^2(z) = 1
\end{equation}
is true even in the complex domain. Similarly,
\begin{equation}\label{c1s3e30}
\cosh^2(z) - \sinh^2(z) = 1.
\end{equation}

\item The inverses of the circular and the hyperbolic functions can be written 
in terms of the logarithm \cite{aw}. If $w = \arcsin(z)$ then $z = \sin(w),
\cos(w) = \pm\sqrt{1 - z^2}$ and
\[
e^{iw} = \pm\sqrt{1 - z^2} + iz
\]
or
\[
iw = \log(iz \pm\sqrt{1 - z^2})
\]
or
\begin{equation}\label{c1s3e31}
\arcsin(z) = -i\log(iz \pm\sqrt{1 - z^2})
\end{equation}
Similarly, we can show that
\begin{equation}\label{c1s3e32}
\arccos(z) = -i\log(z \pm \sqrt{z^2 - 1}).
\end{equation}
Now, if $z = \tan(w), \sec^2(w) = (1 + z^2)$ or
\[
\cos(w) = \frac{1}{\sqrt{1 + z^2}}
\]
and
\[
\sin(w) = \frac{z}{\sqrt{1 + z^2}}
\]
so that
\[
e^{iw} = \frac{1 + iz}{\sqrt{1 + z^2}} = \sqrt{\frac{1 + iz}{1 - iz}}
\]
or
\[
iw = \frac{1}{2}\log\left(\frac{1 + iz}{1 - iz}\right).
\]
Finally, we get
\begin{equation}\label{c1s3e33}
w = \arctan(z) = \frac{i}{2}\log\left(\frac{i + z}{i - z}\right).
\end{equation}
The analogous relations for hyperbolic functions are
\begin{eqnarray}
\arcsinh(z) &=& \log(z + \sqrt{z^2 + 1}) \label{c1s3e34} \\
\arccosh(z) &=& \log(z + \sqrt{z^2 - 1}) \label{c1s3e35} \\
\arctanh(z) &=& \frac{1}{2}\log\left(\frac{1+z}{1-z}\right) \label{c1s3e36} 
\end{eqnarray}

\end{enumerate}
