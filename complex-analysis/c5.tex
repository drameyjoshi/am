\chapter{Integration - 2}\label{c5}
\section{Series of complex-valued functions}\label{c5s1}
A sequence of functions $\{f_n\}$ is said to converge to a function $f$ if for
a given $\epsilon > 0$, we can find $n_0 \in \son$ such that for all $n > n_0$,
$\abs{f(z) - f_n(z)} < \epsilon$. In general, $n_0$ depends on $z$. If it does 
not then the convergence is said to be \emph{uniform}.

We will first show that an integral and a limit can be interchanged if the 
convergence of a sequence of functions is uniform.
\begin{thm}\label{c5s1t1}
If $\{f_n\} \rightarrow f$ uniformly in a region $U \subset \soc$ then for
any curve of finite length, entirely in $U$,
\[
\lim_{n \rightarrow \infty}\int_C f_n(z) dz = \int_C \lim_{n \rightarrow \infty}
f_n(z)dz.
\]
\end{thm}
\begin{proof}
Let $f_n \rightarrow f$ uniformly. Therefore, given an $\epsilon > 0$,
there exists $n_0 \in \son$ such that
\[
\abs{f_n(z) - f(z)} \le \frac{\epsilon}{L}
\]
for all $z \in U$ and for all $n > n_0$, $L$ being the length of the curve $C$. 
Thus,
\[
f(z) - \frac{\epsilon}{L} \le f_n(z) \le f(z) + \frac{\epsilon}{L}
\]
so that
\[
\int_C f(z)dz - \epsilon \le \int_C f(z)dz \le \int_C f(z)dz + \epsilon
\]
or
\[
\abs{\int_C f_n(z)dz - \int_C f(z)dz} \le \epsilon
\]
for all $n_0 \ge N$. Therefore,
\[
\lim_{n \rightarrow \infty}\int_C f_n(z)dz = \int_C f(z)dz = 
\int_C \lim_{n \rightarrow \infty} f_n(z)dz.
\]
\end{proof}

\begin{cor}\label{c5s1c1}
If $f_n \rightarrow f$ and the convergence if uniform then if all $f_n$ are
analytic then so is $f$.
\end{cor}
\begin{proof}
By theorem \ref{c5s1t1},
\[
\int_C f(z)dz = \lim_{n \rightarrow \infty} \int_C f_n(z)dz
\]
for any closed curve $C$. Since $f_n$ are analytic, their integral is $0$ so
that
\[
\int_C f(z) dz = 0
\]
for any closed curve. Analyticity of $f$ follows from Morera's theorem.
\end{proof}

We next show that the operations of differentiation and limit can also be
interchanged if the convergence of a sequence of functions is uniform. To 
prove it we need another theorem,
\begin{thm}\label{c5s1t2}
If $f_n \rightarrow f$ in a region $U \subset \soc$ and if a function $g$ is
bounded in $U$ then $gf_n \rightarrow gf$ uniformly.
\end{thm}
\begin{proof}
Since $g$ is bounded in $U$, there exists $M > 0$ such that $g(z) < M$ for all
$z \in U$. Fix $\epsilon > 0$. Then, there exists $n_0 \in \son$ such that
for all $n > n_0$,
\[
\abs{f_n(z) - f(z)} \le \frac{\epsilon}{M} \;\forall z \in U.
\]
Therefore,
\[
f(z) - \frac{\epsilon}{M} \le f_n(z) \le f(z) + \frac{\epsilon}{M}
\]
and 
\[
g(z)f(z) - \epsilon \le g(z)f(z) - \frac{\epsilon g(z)}{M} \le g(z)f_n(z) \le
g(z)f(z) + \frac{\epsilon g(z)}{M} \le g(z)f(z) + \epsilon
\]
so that
\[
\abs{g(z)f_n(z) - g(z)f(z)} \le \epsilon \;\forall n \ge n_0, \forall z \in U.
\]
\end{proof}

\begin{thm}[Weierstrass]\label{c5s1t3}
If $f_n \rightarrow f$ uniformly in any region $R \subset \soc$ then $Df_n 
\rightarrow Df$ uniformly in $R$.
\end{thm}
\begin{proof}
Let $z_0$ be an arbitrary point in $R$ and $C$ be a closed curve around
$z_0$ lying entirely in $R$. Then for all $z$ on $C$, the function
\[
\frac{1}{(z - z_0)^2}
\]
is bounded. Therefore, by theorem \ref{c5s1t2},
\[
\frac{f_n(z)}{(z - z_0)^2} \rightarrow \frac{f(z)}{(z - z_0)^2}
\]
uniformly for all $z$ on $C$. By theorem \ref{c5s1t1},
\[
\int_C \frac{f_n(z)}{(z - z_0)^2} dz \rightarrow 
\int_C \frac{f(z)}{(z - z_0)^2} dz
\]
From equation \eqref{c4s1e9},
\[
D_n f(z_0) \rightarrow Df(z_0).
\]
\end{proof}

A real-valued function $f$ is said to be analytic at a point $x_0 \in \sor$ if
it $f(x)$ can be expressed as a power series
\[
f(x) = \sum_{n \ge 0}a_n(x - x_0)^n
\]
in a small neighbourhood of $x_0$. On the other hand, we defined a complex-
valued function $f$ to be analytic at a point $z_0 \in \soc$ if it is 
differentiable at $z_0$. We will show that these two definitions are 
equivalent to each other. If a complex-valued function is differentiable once
it is differentiable any number of times and therefore it can be expressed
as a Taylor series in the neighbourhood of any point in the region of its
analyticity. So the equivalence of the two notions is not surprising. In order
to prove it, we need the lemma
\begin{lem}\label{c5s1l1}
A power series is uniformly convergent in the region of its convergence.
\end{lem}
\begin{proof}
Let the series 
\[
f(z) = \sum_{n \ge 0} a_n(z - z_0)^n
\]
be convergent in the region $|z - z_0| < R$. Choose $r$ such that $\abs{z - 
z_0} = r < R$. Since the series converges, every term of the series is bounded.
That is, there exists $M > 0$ such that 
\[
\abs{a_n(z - z_0)^n} = \abs{a_n}r^n < M.
\]
Now choose $z$ such that $\abs{z - z_0} < r$. Then,
\[
\abs{a_n (z - z_0)^n} = \abs{a_n (z - z_0)^n} = 
\abs{a_n}r^n\abs{\frac{z - z_0}{r}}^n < M\abs{\frac{z - z_0}{r}}^n.
\]
Now the series on the extreme rhs converges because $\abs{z - z_0} < r$. 
Therefore, by Weierstrass M-test,
\[
\sum_{n \ge 0} a_n(z -z_0)^n
\]
converges uniformly.
\end{proof}

We will now prove the equivalance of the two notions of analyticity of 
complex-valued functions. 

\begin{thm}\label{c5s1t4}
Let $f$ be analytic at all points on and inside a circle $\Gamma$ of a point 
$z_0 \in \soc$. Then for all points $z$ in the neighbourhood,
\[
f(z) = \sum_{n \ge 0} a_n(z - z_0)^n
\]
where 
\[
a_n=\frac{D^nf(z_0)}{n!}=
\frac{1}{2\pi i}\int_\Gamma\frac{f(z)dz}{(z-z_0)^{n+1}}.
\]
Conversely, if $f$ has a power series representation in a region $U$ then it
is analytic in $U$.
\end{thm}
\begin{proof}
By Cauchy integral formula, if $z$ is inside $\Gamma$,
\[
f(z) = \frac{1}{2\pi i}\int_\Gamma \frac{f(w)dw}{w - z}.
\]
Now $w - z = w - z_0 + z_0 - z = w - z_0 - (z - z_0)$ and
\[
\frac{1}{w-z} = \frac{1}{w-z_0}\frac{1}{1 - \frac{z - z_0}{w - z_0}}
\]
Since $w$ is a point on $\Gamma$ and $z$ in inside it, 
\begin{equation}\label{c5s1e2}
\abs{\frac{z - z_0}{w - z_0}} < 1
\end{equation}
so that 
\[
\frac{1}{w-z} = \frac{1}{w-z_0}\sum_{k \ge 0}\left(\frac{z-z_0}{w-z_0}\right)^k
= \sum_{k \ge 0}\frac{(z - z_0)^k}{(w - z_0)^{k+1}}.
\]
This series is uniformly convergent in the region defined by \eqref{c5s1e2}.
Therefore,
\[
f(z) = \frac{1}{2\pi i}\int_\Gamma\sum_{k \ge 0}\frac{f(w)}{(w - z_0)^{k+1}}
(z - z_0)^k dw.
\]
The uniform convergence of the series allows us to interchange the sum and the
integral to get
\[
f(z) = \sum_{k \ge 0}\frac{1}{2\pi i}\int_\Gamma \frac{f(w)dw}{(w-z_0)^{k+1}}
(z - z_0)^k.
\]
Using equation \eqref{c4s1e19}, we get
\[
f(z) = \sum_{k \ge 0}\frac{D^k f(z_0)}{k!}(z - z_0)^k.
\]

To prove the converse, consider
\[
\int_C f(z)dz = \int_C\sum_{n\ge 0} a_n(z - z_0)^n,
\]
where $C$ is a closed curve entirely in $U$. Since a power series is uniformly
convergent in its region of convergence, we can interchange the sum with the
integral to get
\[
\int_C f(z)dz = \sum_{n\ge 0} \int_C a_n(z - z_0)^n = 0
\]
because a polynomial is an entire function. The analyticity of $f$ follows 
from Morera's theorem.
\end{proof}

\begin{rem}
The circle $\Gamma$ was chosen in such a manner that $f$ is analytic on and
within it. Therefore, the radius of convergence of the series representation is
the distance between $z_0$ and its nearest singularity.
\end{rem}

Theorem \ref{c5s1t4} considered a function that is analytic at all points on
and inside a circle $\Gamma$. We now consider a situation when a function is
analytic in the annular region between two circles $\Gamma_1$ and $\Gamma_2$
of radii $r_1$ and $r_2$ centred at the point $z_0$. We will show that in
this case, the function can be expanded as a power series with both positive
and negative powers.

\begin{thm}\label{c5e1e5}
Let $f$ be a function which is analytic in the annular region between two
circles $\Gamma_1$ and $\Gamma_2$, each with centre $z_0$ and radius $r_1$ 
and $r_1 < r_2$. Then $f$ can be expressed as
\[
f(z) = \sum_{n = -\infty}^\infty a_n (z - z_0)^n,
\]
at all points in the region of analyticity. The coefficients of the series
are
\[
a_n = \frac{1}{2\pi i}\int_C \frac{f(w)dw}{(w - z_0)^{n+1}}
\]
\end{thm}
\begin{proof}
Let $z$ be a point in the region of analyticity. Dear a circle $\gamma$ centred
at $z$ and of a small enough radius that it is entirely between $\Gamma_1$ and
$\Gamma_2$. Join $\gamma$ with $\Gamma_1$ with two closely spaced lines $L_1$
and $L_2$. Similarly, join $\gamma$ with $\Gamma_2$ with two closely spaced
lines $L_3$ and $L_4$. The curve $\Gamma_2, L_2, \gamma_1, L_4, \Gamma_1, L_3,
\gamma_2, L_1$ form a closed curve $C$ such that the function $f$ is analytic
in its interior. $\gamma_1$ is the semicircle between the points where $L_2$ 
and $L_4$ touch $\gamma$. Similarly, $\gamma_2$ is the semicircle between the
points where $L_3$ and $L_1$ touch $\gamma$.

Since $f$ is analytic in the interior of $C$, so is $f(w)/(w - z)$. Note that
the point $z$ is not on or inside $C$. Therefore, by Cauchy-Goursat theorem,
\[
\int_C \frac{f(w)}{w - z}dw = 0.
\]
Now, the integral along $L_1$ cancels with the integral along $L_2$. Similarly,
the integral along $L_3$ cancels the integral along $L_4$. Therefore,
\[
\int_{\Gamma_2}\frac{f(w)}{w - z}dw - \int_\gamma\frac{f(w)}{w - z}dw - 
\int_{\Gamma_1}\frac{f(w)}{w - z}dw  = 0.
\]
The signs of the integrals around $\gamma$ and $\Gamma_1$ are negative because 
they are traversed clockwise. By Cauchy integral formula the second integral is
$2\pi if(z)$ so that
\begin{equation}\label{c5s1e3}
f(z) = \frac{1}{2\pi i}\int_{\Gamma_2}\frac{f(w)}{w - z}dw - 
\frac{1}{2\pi i}\int_{\Gamma_1}\frac{f(w)}{w - z}dw.
\end{equation}
Write $w - z = w - z_0 - (z - z_0)$. For all points on $\Gamma_2$, $\abs{w-z_0}
> \abs{z - z_0}$. Therefore, write
\begin{equation}\label{c5s1e4}
\frac{1}{w-z} = \frac{1}{w-z_0}\frac{1}{1 - \frac{z - z_0}{w - z_0}} = 
\frac{1}{w - z_0}\sum_{k \ge 0}\left(\frac{z - z_0}{w - z_0}\right)^k 
\text{ on } \Gamma_2.
\end{equation}
For all points on $\Gamma_1$, $\abs{z - z_0} > \abs{w - z_0}$. Therefore, write
\begin{equation}\label{c5s1e5}
\frac{1}{w-z} = -\frac{1}{z-z_0}\frac{1}{1 - \frac{w - z_0}{z - z_0}} = 
\frac{1}{z - z_0}\sum_{k \ge 0}\left(\frac{w - z_0}{z - z_0}\right)^k 
\text{ on } \Gamma_1.
\end{equation}
Substitute equations \eqref{c5s1e4} and \eqref{c5s1e5} in \eqref{c5s1e3} to get
\[
f(z) = \frac{1}{2\pi i}
\int_{\Gamma_2}\sum_{k \ge 0}\frac{f(w)}{(w-z_0)^{k+1}}(z - z_0)^k dw
+ \frac{1}{2\pi i}
\int_{\Gamma_1}\sum_{k \ge 0}\frac{f(w)(w - z_0)^k}{(z - z_0)^{k+1}}dw.
\]
Uniform convergence of the power series in their region of convergence allows
us to interchange the sum and the integral so that
\begin{eqnarray*}
f(z) &=& \frac{1}{2\pi i} 
\sum_{k \ge 0}\int_{\Gamma_2}\frac{f(w)}{(w - z_0)^{k+1}}dw (z-z_0)^k + \\
& & \frac{1}{2\pi i} 
\sum_{k \ge 0}\int_{\Gamma_1}f(w)(w - z_0)^kdw \frac{1}{(z - z_0)^{k+1}}
\end{eqnarray*}
The sum in the second term can be written as
\begin{eqnarray*}
f(z) &=& \frac{1}{2\pi i} 
\sum_{k \ge 0}\int_{\Gamma_2}\frac{f(w)}{(w - z_0)^{k+1}}dw (z-z_0)^k + \\
 & & \frac{1}{2\pi i}
\sum_{k \ge 1}\int_{\Gamma_1}f(w)(w - z_0)^{k-1}dw \frac{1}{(z - z_0)^{k}},
\end{eqnarray*}
or allowing $k$ to run along the negative integers,
\begin{eqnarray*}
f(z) &=& \frac{1}{2\pi i}
\sum_{k \ge 0}\int_{\Gamma_2}\frac{f(w)}{(w - z_0)^{k+1}}dw (z-z_0)^k + \\
 & & \frac{1}{2\pi i}
 \sum_{k=-1}^{-\infty}\int_{\Gamma_1}f(w)(w - z_0)^{-k-1}dw 
\frac{1}{(z - z_0)^{-k}}.
\end{eqnarray*}
We can now combine the two sums as
\[
f(z) = \sum_{k=-\infty}^\infty d_k (z - z_0)^k
\]
where
\[
d_k = \begin{cases}
\frac{1}{2\pi i}
\int_{\Gamma_2}\frac{f(w)}{(w - z_0)^{k+1}}dw & \text{ for } k \ge 0 \\
\frac{1}{2\pi i}
\int_{\Gamma_1}\frac{f(w)}{(w - z_0)^{k+1}}dw & \text{ for } k < 0 .
\end{cases}
\]
Let $C$ be any arbitrary closed curve in the annular region. By joining it
with $\Gamma_1$ with two closely spaced lines one can show that
\[
\int_{\Gamma_1}\frac{f(w)dw}{(w - z_0)^{k+1}} - 
\int_C\frac{f(w)dw}{(w - z_0)^{k+1}} = 0
\]
or 
\[
\int_{\Gamma_1}\frac{f(w)dw}{(w - z_0)^{k+1}} = 
\int_C\frac{f(w)dw}{(w - z_0)^{k+1}} = 0
\]
We can similarly show that
\[
\int_{\Gamma_2}\frac{f(w)dw}{(w - z_0)^{k+1}} = 
\int_C\frac{f(w)dw}{(w - z_0)^{k+1}} = 0.
\]
Thus,
\[
d_k = \frac{1}{2\pi i}\int_C \frac{f(w)}{(w - z_0)^{k + 1}}dw \;\forall k \in
\soi.
\]
\end{proof}
\begin{rem}
The series with both positive and negative powers is called Laurent series.
\end{rem}
