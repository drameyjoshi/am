\chapter{Integration - 2}\label{c5}
\section{Series of complex-valued functions}\label{c5s1}
A sequence of functions $\{f_n\}$ is said to converge to a function $f$ if for
a given $\epsilon > 0$, we can find $n_0 \in \son$ such that for all $n > n_0$,
$\abs{f(z) - f_n(z)} < \epsilon$. In general, $n_0$ depends on $z$. If it does 
not then the convergence is said to be \emph{uniform}.

We will first show that an integral and a limit can be interchanged if the 
convergence of a sequence of functions is uniform.
\begin{thm}\label{c5s1t1}
If $\{f_n\} \rightarrow f$ uniformly in a region $U \subset \soc$ then for
any curve of finite length, entirely in $U$,
\[
\lim_{n \rightarrow \infty}\int_C f_n(z) dz = \int_C \lim_{n \rightarrow \infty}
f_n(z)dz.
\]
\end{thm}
\begin{proof}
Let $f_n \rightarrow f$ uniformly. Therefore, given an $\epsilon > 0$,
there exists $n_0 \in \son$ such that
\[
\abs{f_n(z) - f(z)} \le \frac{\epsilon}{L}
\]
for all $z \in U$ and for all $n > n_0$, $L$ being the length of the curve $C$. 
Thus,
\[
f(z) - \frac{\epsilon}{L} \le f_n(z) \le f(z) + \frac{\epsilon}{L}
\]
so that
\[
\int_C f(z)dz - \epsilon \le \int_C f(z)dz \le \int_C f(z)dz + \epsilon
\]
or
\[
\abs{\int_C f_n(z)dz - \int_C f(z)dz} \le \epsilon
\]
for all $n_0 \ge N$. Therefore,
\[
\lim_{n \rightarrow \infty}\int_C f_n(z)dz = \int_C f(z)dz = 
\int_C \lim_{n \rightarrow \infty} f_n(z)dz.
\]
\end{proof}

\begin{cor}\label{c5s1c1}
If $f_n \rightarrow f$ and the convergence if uniform then if all $f_n$ are
analytic then so is $f$.
\end{cor}
\begin{proof}
By theorem \ref{c5s1t1},
\[
\int_C f(z)dz = \lim_{n \rightarrow \infty} \int_C f_n(z)dz
\]
for any closed curve $C$. Since $f_n$ are analytic, their integral is $0$ so
that
\[
\int_C f(z) dz = 0
\]
for any closed curve. Analyticity of $f$ follows from Morera's theorem.
\end{proof}

We next show that the operations of differentiation and limit can also be
interchanged if the convergence of a sequence of functions is uniform. To 
prove it we need another theorem,
\begin{thm}\label{c5s1t2}
If $f_n \rightarrow f$ in a region $U \subset \soc$ and if a function $g$ is
bounded in $U$ then $gf_n \rightarrow gf$ uniformly.
\end{thm}
\begin{proof}
Since $g$ is bounded in $U$, there exists $M > 0$ such that $g(z) < M$ for all
$z \in U$. Fix $\epsilon > 0$. Then, there exists $n_0 \in \son$ such that
for all $n > n_0$,
\[
\abs{f_n(z) - f(z)} \le \frac{\epsilon}{M} \;\forall z \in U.
\]
Therefore,
\[
f(z) - \frac{\epsilon}{M} \le f_n(z) \le f(z) + \frac{\epsilon}{M}
\]
and 
\[
g(z)f(z) - \epsilon \le g(z)f(z) - \frac{\epsilon g(z)}{M} \le g(z)f_n(z) \le
g(z)f(z) + \frac{\epsilon g(z)}{M} \le g(z)f(z) + \epsilon
\]
so that
\[
\abs{g(z)f_n(z) - g(z)f(z)} \le \epsilon \;\forall n \ge n_0, \forall z \in U.
\]
\end{proof}

\begin{thm}[Weierstrass]\label{c5s1t3}
If $f_n \rightarrow f$ uniformly in any region $R \subset \soc$ then $Df_n 
\rightarrow Df$ uniformly in $R$.
\end{thm}
\begin{proof}
Let $z_0$ be an arbitrary point in $R$ and $C$ be a closed curve around
$z_0$ lying entirely in $R$. Then for all $z$ on $C$, the function
\[
\frac{1}{(z - z_0)^2}
\]
is bounded. Therefore, by theorem \ref{c5s1t2},
\[
\frac{f_n(z)}{(z - z_0)^2} \rightarrow \frac{f(z)}{(z - z_0)^2}
\]
uniformly for all $z$ on $C$. By theorem \ref{c5s1t1},
\[
\int_C \frac{f_n(z)}{(z - z_0)^2} dz \rightarrow 
\int_C \frac{f(z)}{(z - z_0)^2} dz
\]
From equation \eqref{c4s1e9},
\[
D_n f(z_0) \rightarrow Df(z_0).
\]
\end{proof}

A real-valued function $f$ is said to be analytic at a point $x_0 \in \sor$ if
it $f(x)$ can be expressed as a power series
\[
f(x) = \sum_{n \ge 0}a_n(x - x_0)^n
\]
in a small neighbourhood of $x_0$. On the other hand, we defined a complex-
valued function $f$ to be analytic at a point $z_0 \in \soc$ if it is 
differentiable at $z_0$. We will show that these two definitions are 
equivalent to each other. If a complex-valued function is differentiable once
it is differentiable any number of times and therefore it can be expressed
as a Taylor series in the neighbourhood of any point in the region of its
analyticity. So the equivalence of the two notions is not surprising. 
Nevertheless we will prove 
\begin{thm}\label{c5s1e4}
Let $f$ be analytic at all points on and inside a circle $\Gamma$ of a point 
$z_0 \in \soc$. Then for all points $z$ in the neighbourhood,
\[
f(z) = \sum_{n \ge 0} a_n(z - z_0)^n
\]
where 
\[
a_n=\frac{D^nf(z_0)}{n!}=\frac{1}{2\pi i}\int_\Gamma \frac{f(z)dz}{(z-z_0)^2}.
\]
\end{thm}
\begin{proof}
By Cauchy integral formula, if $z$ is inside $\Gamma$,
\[
f(z) = \frac{1}{2\pi i}\int_\Gamma \frac{f(w)dw}{w - z}.
\]
Now $w - z = w - z_0 + z_0 - z = w - z_0 - (z - z_0)$ and
\[
\frac{1}{w-z} = \frac{1}{w-z_0}\frac{1}{1 - \frac{z - z_0}{w - z_0}}
\]
Since $w$ is a point on $\Gamma$ and $z$ in inside it, 
\begin{equation}\label{c5s1e1}
\abs{\frac{z - z_0}{w - z_0}} < 1
\end{equation}
so that 
\[
\frac{1}{w-z} = \frac{1}{w-z_0}\sum_{k \ge 0}\left(\frac{z-z_0}{w-z_0}\right)^k
= \sum_{k \ge 0}\frac{(z - z_0)^k}{(w - z_0)^{k+1}}.
\]
This series is uniformly convergent in the region defined by \eqref{c5s1e1}.
Therefore,
\[
f(z) = \frac{1}{2\pi i}\int_\Gamma\sum_{k \ge 0}\frac{f(w)}{(w - z_0)^{k+1}}
(z - z_0)^k dw.
\]
The uniform convergence of the series allows us to interchange the sum and the
integral to get
\[
f(z) = \sum_{k \ge 0}\frac{1}{2\pi i}\int_\Gamma \frac{f(w)dw}{(w-z_0)^{k+1}}
(z - z_0)^k.
\]
Using equation \eqref{c4s1e19}, we get
\[
f(z) = \sum_{k \ge 0}\frac{D^k f(z_0)}{k!}(z - z_0)^k.
\]
\end{proof}

\begin{rem}
The circle $\Gamma$ was chosen in such a manner that $f$ is analytic on and
within it. Therefore, the radius of convergence of the series representation is
the distance between $z_0$ and its nearest singularity.
\end{rem}
