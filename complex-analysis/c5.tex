\chapter{Integration - 2}\label{c5}
\section{Series of complex-valued functions}\label{c5s1}
A sequence of functions $\{f_n\}$ is said to converge to a function $f$ if for
a given $\epsilon > 0$, we can find $n_0 \in \son$ such that for all $n > n_0$,
$\abs{f(z) - f_n(z)} < \epsilon$. In general, $n_0$ depends on $z$. If it does 
not then the convergence is said to be \emph{uniform}.

We will first show that an integral and a limit can be interchanged if the 
convergence of a sequence of functions is uniform.
\begin{thm}\label{c5s1t1}
If $\{f_n\} \rightarrow f$ uniformly in a region $U \subset \soc$ then for
any curve of finite length, entirely in $U$,
\[
\lim_{n \rightarrow \infty}\int_C f_n(z) dz = \int_C \lim_{n \rightarrow \infty}
f_n(z)dz.
\]
\end{thm}
\begin{proof}
Let $f_n \rightarrow f$ uniformly. Therefore, given an $\epsilon > 0$,
there exists $n_0 \in \son$ such that
\[
\abs{f_n(z) - f(z)} \le \frac{\epsilon}{L}
\]
for all $z \in U$ and for all $n > n_0$, $L$ being the length of the curve $C$. 
Thus,
\[
f(z) - \frac{\epsilon}{L} \le f_n(z) \le f(z) + \frac{\epsilon}{L}
\]
so that
\[
\int_C f(z)dz - \epsilon \le \int_C f(z)dz \le \int_C f(z)dz + \epsilon
\]
or
\[
\abs{\int_C f_n(z)dz - \int_C f(z)dz} \le \epsilon
\]
for all $n_0 \ge N$. Therefore,
\[
\lim_{n \rightarrow \infty}\int_C f_n(z)dz = \int_C f(z)dz = 
\int_C \lim_{n \rightarrow \infty} f_n(z)dz.
\]
\end{proof}

\begin{cor}\label{c5s1c1}
If $f_n \rightarrow f$ and the convergence if uniform then if all $f_n$ are
analytic then so is $f$.
\end{cor}
\begin{proof}
By theorem \ref{c5s1t1},
\[
\int_C f(z)dz = \lim_{n \rightarrow \infty} \int_C f_n(z)dz
\]
for any closed curve $C$. Since $f_n$ are analytic, their integral is $0$ so
that
\[
\int_C f(z) dz = 0
\]
for any closed curve. Analyticity of $f$ follows from Morera's theorem.
\end{proof}

We next show that the operations of differentiation and limit can also be
interchanged if the convergence of a sequence of functions is uniform. To 
prove it we need another theorem,
\begin{thm}\label{c5s1t2}
If $f_n \rightarrow f$ uniformly in a region $U \subset \soc$ and if a function 
$g$ is bounded in $U$ then $gf_n \rightarrow gf$ uniformly.
\end{thm}
\begin{proof}
Since $g$ is bounded in $U$, there exists $M > 0$ such that $g(z) < M$ for all
$z \in U$. Fix $\epsilon > 0$. Then, there exists $n_0 \in \son$ such that
for all $n > n_0$,
\[
\abs{f_n(z) - f(z)} \le \frac{\epsilon}{M} \;\forall z \in U.
\]
Therefore,
\[
f(z) - \frac{\epsilon}{M} \le f_n(z) \le f(z) + \frac{\epsilon}{M}
\]
and 
\[
g(z)f(z) - \epsilon \le g(z)f(z) - \frac{\epsilon g(z)}{M} \le g(z)f_n(z) \le
g(z)f(z) + \frac{\epsilon g(z)}{M} \le g(z)f(z) + \epsilon
\]
so that
\[
\abs{g(z)f_n(z) - g(z)f(z)} \le \epsilon \;\forall n \ge n_0, \forall z \in U.
\]
\end{proof}

\begin{thm}[Weierstrass]\label{c5s1t3}
If $f_n \rightarrow f$ uniformly in any region $R \subset \soc$ then $Df_n 
\rightarrow Df$ uniformly in $R$.
\end{thm}
\begin{proof}
Let $z_0$ be an arbitrary point in $R$ and $C$ be a closed curve around
$z_0$ lying entirely in $R$. Then for all $z$ on $C$, the function
\[
\frac{1}{(z - z_0)^2}
\]
is bounded. Therefore, by theorem \ref{c5s1t2},
\[
\frac{f_n(z)}{(z - z_0)^2} \rightarrow \frac{f(z)}{(z - z_0)^2}
\]
uniformly for all $z$ on $C$. By theorem \ref{c5s1t1},
\[
\int_C \frac{f_n(z)}{(z - z_0)^2} dz \rightarrow 
\int_C \frac{f(z)}{(z - z_0)^2} dz
\]
From equation \eqref{c4s1e9},
\[
D_n f(z_0) \rightarrow Df(z_0).
\]
\end{proof}

A real-valued function $f$ is said to be analytic at a point $x_0 \in \sor$ if
it $f(x)$ can be expressed as a power series
\[
f(x) = \sum_{n \ge 0}a_n(x - x_0)^n
\]
in a small neighbourhood of $x_0$. On the other hand, we defined a complex-
valued function $f$ to be analytic at a point $z_0 \in \soc$ if it is 
differentiable at $z_0$. We will show that these two definitions are 
equivalent to each other. If a complex-valued function is differentiable once
it is differentiable any number of times and therefore it can be expressed
as a Taylor series in the neighbourhood of any point in the region of its
analyticity. So the equivalence of the two notions is not surprising. In order
to prove it, we need the lemma
\begin{lem}\label{c5s1l1}
A power series is uniformly convergent in the region of its convergence.
\end{lem}
\begin{proof}
Let the series 
\[
f(z) = \sum_{n \ge 0} a_n(z - z_0)^n
\]
be convergent in the region $|z - z_0| < R$. Choose $r$ such that $\abs{z - 
z_0} = r < R$. Since the series converges, every term of the series is bounded.
That is, there exists $M > 0$ such that 
\[
\abs{a_n(z - z_0)^n} = \abs{a_n}r^n < M.
\]
Now choose $z$ such that $\abs{z - z_0} < r$. Then,
\[
\abs{a_n (z - z_0)^n} = \abs{a_n (z - z_0)^n} = 
\abs{a_n}r^n\abs{\frac{z - z_0}{r}}^n < M\abs{\frac{z - z_0}{r}}^n.
\]
Now the series on the extreme rhs converges because $\abs{z - z_0} < r$. 
Therefore, by Weierstrass M-test,
\[
\sum_{n \ge 0} a_n(z -z_0)^n
\]
converges uniformly.
\end{proof}

We will now prove the equivalance of the two notions of analyticity of 
complex-valued functions. 

\begin{thm}\label{c5s1t4}
Let $f$ be analytic at all points on and inside a circle $\Gamma$ of a point 
$z_0 \in \soc$. Then for all points $z$ in the neighbourhood,
\[
f(z) = \sum_{n \ge 0} a_n(z - z_0)^n
\]
where 
\[
a_n=\frac{D^nf(z_0)}{n!}=
\frac{1}{2\pi i}\int_\Gamma\frac{f(z)dz}{(z-z_0)^{n+1}}.
\]
Conversely, if $f$ has a power series representation in a region $U$ then it
is analytic in $U$.
\end{thm}
\begin{proof}
By Cauchy integral formula, if $z$ is inside $\Gamma$,
\[
f(z) = \frac{1}{2\pi i}\int_\Gamma \frac{f(w)dw}{w - z}.
\]
Now $w - z = w - z_0 + z_0 - z = w - z_0 - (z - z_0)$ and
\[
\frac{1}{w-z} = \frac{1}{w-z_0}\frac{1}{1 - \frac{z - z_0}{w - z_0}}
\]
Since $w$ is a point on $\Gamma$ and $z$ in inside it, 
\begin{equation}\label{c5s1e2}
\abs{\frac{z - z_0}{w - z_0}} < 1
\end{equation}
so that 
\[
\frac{1}{w-z} = \frac{1}{w-z_0}\sum_{k \ge 0}\left(\frac{z-z_0}{w-z_0}\right)^k
= \sum_{k \ge 0}\frac{(z - z_0)^k}{(w - z_0)^{k+1}}.
\]
This series is uniformly convergent in the region defined by \eqref{c5s1e2}.
Therefore,
\[
f(z) = \frac{1}{2\pi i}\int_\Gamma\sum_{k \ge 0}\frac{f(w)}{(w - z_0)^{k+1}}
(z - z_0)^k dw.
\]
The uniform convergence of the series allows us to interchange the sum and the
integral to get
\[
f(z) = \sum_{k \ge 0}\frac{1}{2\pi i}\int_\Gamma \frac{f(w)dw}{(w-z_0)^{k+1}}
(z - z_0)^k.
\]
Using equation \eqref{c4s1e19}, we get
\[
f(z) = \sum_{k \ge 0}\frac{D^k f(z_0)}{k!}(z - z_0)^k.
\]

To prove the converse, consider
\[
\int_C f(z)dz = \int_C\sum_{n\ge 0} a_n(z - z_0)^n,
\]
where $C$ is a closed curve entirely in $U$. Since a power series is uniformly
convergent in its region of convergence, we can interchange the sum with the
integral to get
\[
\int_C f(z)dz = \sum_{n\ge 0} \int_C a_n(z - z_0)^n = 0
\]
because a polynomial is an entire function. The analyticity of $f$ follows 
from Morera's theorem.
\end{proof}

\begin{rem}
The circle $\Gamma$ was chosen in such a manner that $f$ is analytic on and
within it. Therefore, the radius of convergence of the series representation is
the distance between $z_0$ and its nearest singularity.
\end{rem}

Theorem \ref{c5s1t4} considered a function that is analytic at all points on
and inside a circle $\Gamma$. We now consider a situation when a function is
analytic in the annular region between two circles $\Gamma_1$ and $\Gamma_2$
of radii $r_1$ and $r_2$ centred at the point $z_0$. We will show that in
this case, the function can be expanded as a power series with both positive
and negative powers.

\begin{thm}\label{c5e1e5}
Let $f$ be a function which is analytic in the annular region between two
circles $\Gamma_1$ and $\Gamma_2$, each with centre $z_0$ and radius $r_1$ 
and $r_1 < r_2$. Then $f$ can be expressed as
\[
f(z) = \sum_{n = -\infty}^\infty a_n (z - z_0)^n,
\]
at all points in the region of analyticity. The coefficients of the series
are
\[
a_n = \frac{1}{2\pi i}\int_C \frac{f(w)dw}{(w - z_0)^{n+1}}
\]
\end{thm}
\begin{proof}
Let $z$ be a point in the region of analyticity. Dear a circle $\gamma$ centred
at $z$ and of a small enough radius that it is entirely between $\Gamma_1$ and
$\Gamma_2$. Join $\gamma$ with $\Gamma_1$ with two closely spaced lines $L_1$
and $L_2$. Similarly, join $\gamma$ with $\Gamma_2$ with two closely spaced
lines $L_3$ and $L_4$. The curve $\Gamma_2, L_2, \gamma_1, L_4, \Gamma_1, L_3,
\gamma_2, L_1$ form a closed curve $C$ such that the function $f$ is analytic
in its interior. $\gamma_1$ is the semicircle between the points where $L_2$ 
and $L_4$ touch $\gamma$. Similarly, $\gamma_2$ is the semicircle between the
points where $L_3$ and $L_1$ touch $\gamma$.

Since $f$ is analytic in the interior of $C$, so is $f(w)/(w - z)$. Note that
the point $z$ is not on or inside $C$. Therefore, by Cauchy-Goursat theorem,
\[
\int_C \frac{f(w)}{w - z}dw = 0.
\]
Now, the integral along $L_1$ cancels with the integral along $L_2$. Similarly,
the integral along $L_3$ cancels the integral along $L_4$. Therefore,
\[
\int_{\Gamma_2}\frac{f(w)}{w - z}dw - \int_\gamma\frac{f(w)}{w - z}dw - 
\int_{\Gamma_1}\frac{f(w)}{w - z}dw  = 0.
\]
The signs of the integrals around $\gamma$ and $\Gamma_1$ are negative because 
they are traversed clockwise. By Cauchy integral formula the second integral is
$2\pi if(z)$ so that
\begin{equation}\label{c5s1e3}
f(z) = \frac{1}{2\pi i}\int_{\Gamma_2}\frac{f(w)}{w - z}dw - 
\frac{1}{2\pi i}\int_{\Gamma_1}\frac{f(w)}{w - z}dw.
\end{equation}
Write $w - z = w - z_0 - (z - z_0)$. For all points on $\Gamma_2$, $\abs{w-z_0}
> \abs{z - z_0}$. Therefore, write
\begin{equation}\label{c5s1e4}
\frac{1}{w-z} = \frac{1}{w-z_0}\frac{1}{1 - \frac{z - z_0}{w - z_0}} = 
\frac{1}{w - z_0}\sum_{k \ge 0}\left(\frac{z - z_0}{w - z_0}\right)^k 
\text{ on } \Gamma_2.
\end{equation}
For all points on $\Gamma_1$, $\abs{z - z_0} > \abs{w - z_0}$. Therefore, write
\begin{equation}\label{c5s1e5}
\frac{1}{w-z} = -\frac{1}{z-z_0}\frac{1}{1 - \frac{w - z_0}{z - z_0}} = 
\frac{1}{z - z_0}\sum_{k \ge 0}\left(\frac{w - z_0}{z - z_0}\right)^k 
\text{ on } \Gamma_1.
\end{equation}
Substitute equations \eqref{c5s1e4} and \eqref{c5s1e5} in \eqref{c5s1e3} to get
\[
f(z) = \frac{1}{2\pi i}
\int_{\Gamma_2}\sum_{k \ge 0}\frac{f(w)}{(w-z_0)^{k+1}}(z - z_0)^k dw
+ \frac{1}{2\pi i}
\int_{\Gamma_1}\sum_{k \ge 0}\frac{f(w)(w - z_0)^k}{(z - z_0)^{k+1}}dw.
\]
Uniform convergence of the power series in their region of convergence allows
us to interchange the sum and the integral so that
\begin{eqnarray*}
f(z) &=& \frac{1}{2\pi i} 
\sum_{k \ge 0}\int_{\Gamma_2}\frac{f(w)}{(w - z_0)^{k+1}}dw (z-z_0)^k + \\
& & \frac{1}{2\pi i} 
\sum_{k \ge 0}\int_{\Gamma_1}f(w)(w - z_0)^kdw \frac{1}{(z - z_0)^{k+1}}
\end{eqnarray*}
The sum in the second term can be written as
\begin{eqnarray*}
f(z) &=& \frac{1}{2\pi i} 
\sum_{k \ge 0}\int_{\Gamma_2}\frac{f(w)}{(w - z_0)^{k+1}}dw (z-z_0)^k + \\
 & & \frac{1}{2\pi i}
\sum_{k \ge 1}\int_{\Gamma_1}f(w)(w - z_0)^{k-1}dw \frac{1}{(z - z_0)^{k}},
\end{eqnarray*}
or allowing $k$ to run along the negative integers,
\begin{eqnarray*}
f(z) &=& \frac{1}{2\pi i}
\sum_{k \ge 0}\int_{\Gamma_2}\frac{f(w)}{(w - z_0)^{k+1}}dw (z-z_0)^k + \\
 & & \frac{1}{2\pi i}
 \sum_{k=-1}^{-\infty}\int_{\Gamma_1}f(w)(w - z_0)^{-k-1}dw 
\frac{1}{(z - z_0)^{-k}}.
\end{eqnarray*}
We can now combine the two sums as
\[
f(z) = \sum_{k=-\infty}^\infty d_k (z - z_0)^k
\]
where
\[
d_k = \begin{cases}
\frac{1}{2\pi i}
\int_{\Gamma_2}\frac{f(w)}{(w - z_0)^{k+1}}dw & \text{ for } k \ge 0 \\
\frac{1}{2\pi i}
\int_{\Gamma_1}\frac{f(w)}{(w - z_0)^{k+1}}dw & \text{ for } k < 0 .
\end{cases}
\]
Let $C$ be any arbitrary closed curve in the annular region. By joining it
with $\Gamma_1$ with two closely spaced lines one can show that
\[
\int_{\Gamma_1}\frac{f(w)dw}{(w - z_0)^{k+1}} - 
\int_C\frac{f(w)dw}{(w - z_0)^{k+1}} = 0
\]
or 
\[
\int_{\Gamma_1}\frac{f(w)dw}{(w - z_0)^{k+1}} = 
\int_C\frac{f(w)dw}{(w - z_0)^{k+1}} = 0
\]
We can similarly show that
\[
\int_{\Gamma_2}\frac{f(w)dw}{(w - z_0)^{k+1}} = 
\int_C\frac{f(w)dw}{(w - z_0)^{k+1}} = 0.
\]
Thus,
\[
d_k = \frac{1}{2\pi i}\int_C \frac{f(w)}{(w - z_0)^{k + 1}}dw \;\forall k \in
\soi.
\]
\end{proof}
\begin{rem}
The series with both positive and negative powers is called Laurent series.
\end{rem}

\section{Zeros and poles}\label{c5s2}
\begin{defn}\label{c5s2d1}
An analytic function $f: \soc \rightarrow \soc$ is said to have a zero of
order $n \ge 0$ at a point $z_0$ if $D^kf(z_0) = 0$ for all $k = 0, \ldots,
n - 1$ and $D^nf(z_0) \ne 0$.
\end{defn}

\begin{thm}\label{c5s2t1}
Zeros of an analytic function are isolated.
\end{thm}
\begin{proof}
Let $z_0$ be an $n$-th order zero of an analytic function $f:\soc \rightarrow
\soc$ and let $f$ not be identically zero. Then
\[
f(z) = \sum_{k=n}^\infty a_k(z - z_0)^k = (z - z_0)^n \sum_{k=0}^n b_k(z-z_0)^k
\]
where $b_k = a_{n+k}$. If we write 
\[
h(z) = \sum_{k=0}^n(z - z_0)^k
\]
then $f(z) = (z - z_0)^n h(z)$ and $h$ does not have a zero at $z_0$. Further,
$h$ is continuous at $z_0$. Therefore, there is a neighbourhood of $z_0$ where
$h$ is not zero.
\end{proof}

\begin{cor}\label{c5s2c1}
Let $f:U \rightarrow \soc$ be analytic throughout the open set $U$. If $f=0$
on a set $V$ and if $V$ has a limit point in $U$ then $f = 0$ identically in
$U$.
\end{cor}
\begin{proof}
Let $z_0$ be a limit point of $V$ in $U$. Since $U$ is open, there exists
$\epsilon > 0$ such that $\{z : \abs{z - z_0} < \epsilon\} \in U$. By theorem
\ref{c5s2t1} if $f$ is analytic then its zeros are isolated. The only way
an analytic function be zero in $\{z : \abs{z - z_0} < \epsilon\}$ is if it is
identically zero.
\end{proof}

\begin{defn}\label{c5s2d2}
A point at which a function $f: \soc \rightarrow \soc$ is not analytic is 
called a singularity.
\end{defn}

A function can be singular at a point in a complex plane because of several
reasons.
\begin{enumerate}
\item It fails to be continuous, in which case the singularity is called a 
\emph{branch} point;
\item It fails to be differentiable, in which case it is called a \emph{pole}.
\end{enumerate}

We will first consider a pole. Suppose that a function $f$ has a Laurent
expansion
\[
f(z) = \sum_{k=-n}^\infty a_k (z - z_0)^k.
\]
That is, all but a finite number of coefficients with a negative index are zero.
If $n$ is the lowest negative index for which the coefficient is non-zero then
$z_0$ is said to be pole of order $n$. A pole of order $1$ is called a simple
pole. Thus, the Laurent expansion of a function $f$ with a simple pole at 
$z = z_0$ is
\[
f(z) = \frac{a_{-1}}{z - z_0} + a_0 + a_1(z - z_0) + a_2(z - z_0)^2 + \cdots.
\]
The Laurent expansion of a function $g$ with a pole of order $n$ at a point
$z - z_0$ is
\[
g(z) = \frac{a_{-n}}{(z - z_0)^n} + \cdots \frac{a_{-1}}{z - z_0} + a_0 + 
a_1(z - z_0) + a_2(z - z_0)^2 + \cdots.
\]

\begin{defn}\label{c5s2d3}
A function that is analytic at all points of the complex plane except at 
isolated poles is called a meromorphic function.
\end{defn}

Poles are isolated singularities. If $z_0$ is a pole of order $n$ the function 
$f$ then it is a zero of order $n$ of the function $1/f$ and we saw in theorem
\ref{c5s2t1} that zeros are isolated points.

We will now briefly consider branch points. Consider the function $w = z^{1/2}$
along a circular contour of radius $r$ around the origin. If $z=r\exp(i\theta)$
then the point with $\theta = 0$ and $\theta = 2\pi$ are identical in the 
$z$-plane but their images are different in the $w$ plane. It is as if the same
point in the domain corresponds to two different points in the range. The 
relation between $w$ and $z$ ceases to be a function. If we went around a circle
centred at any other point $z = a$, we will not face this problem. The point $z
= 0$ is called a branch point of the function $w = z^{1/2}$. It is called a 
branch point because as we cross the real axis after going around a circle
centred at it, we end up being on a different `branch' of the function. In order
to consider the behaviour of the function at infinity, one considers
how the function with the variable $t = 1/z$ behaves as $t = 0$. The square
root function becomes $w=t^{-1/2}=r^{-1/2}e^{-i\pi\theta/2}$. We observe that
going round a circular contour in the $t$-plane around $t = 0$ creates the same
problem that we observed in the $z$-plane. As a result we say that the function
has a branch point at infinity as well.

Before closing the discussion on singularities, we will examine the nature of 
essential singularities. They are so severe that in their vicinity the function
takes all possible values of a complex variable. The entire complex plane can be
mapped into a small neighbourhood of an essential singularity.


\begin{thm}[Weierstrass theorem]\label{c5s2t2}
If $f$ has an essential singularity at a point $z_0$ then for any $\epsilon > 0$,
and any complex number $a$, we can find $\delta > 0$ such that $\abs{f(z) - a} <
\epsilon$ for some point $z$ such that $0 < \abs{z - z_0} < \delta$.
\end{thm}
\begin{proof}
Let, if possible, there be a complex number $a$ such that $\abs{f(z) - a} > 
\epsilon$ for all points on the punctured disc $0 < \abs{z - z_0} < \delta$.
Therefore, for all these points,
\[
\abs{\frac{1}{f(z) - a}} \le \frac{1}{\epsilon}.
\]
Therefore, the function $1/(f(z) - a)$ is analytic at all points on the 
punctured disc and can be expressed as
\[
\frac{1}{f(z) - a} = \sum_{n = k}^\infty c_n(z - a)^n 
\]
for some $k \ge 0$. If $k = 0$ then
\[
\frac{1}{f(z) - a} = c_0 \ne 0
\] 
at $z = a$. Otherwise, if has a zero of order $k$ at the point $z=a$. In any 
case
\[
f(z) - a = \frac{1}{(z - a)^k}\cdot\frac{1}{c_k + c_{k+1}(z-a) + \cdots} = 
\frac{1}{(z - a)^k}g(z).
\]
The function $g$ is analytic at $z = a$ with $g(a) = 1/c_k$. Thus either $f$
will have a pole of order $k$ or it will be analytic at $z = a$, a contradiction.
\end{proof}

\begin{rem}
A function oscillates very wildly in the neighbourhood of an essential singularity.
\end{rem}

\section{The calculus of residues}\label{c5s3}
Cauchy-Goursat theorem assures us that if a function $f$ is analytic inside a simple
closed curve $C$ then
\[
\int_C f(z)dz = 0.
\]
What if $f$ is not analytic but has a pole of order $m$ inside $C$. Let $z_0$
be the pole. Then the function
\[
g(z) = (z - z_0)^mf(z)
\]
is analytic. Let us write the integral as
\[
\int_C f(z)dz = \int_C \frac{g(z)}{(z - z_0)^m}dz.
\]
From equation \eqref{c4s1e19}, this integral is
\begin{equation}\label{c5s3e1}
\int_C f(z) = \frac{2\pi i}{(m-1)!}D^{m-1}g(z_0) = 
\frac{2\pi i}{(m-1)!}D^{m-1}((z - z_0)^mf(z))
\end{equation}

\begin{defn}\label{c5s3d1}
The quantity
\[
\frac{D^{m-1}((z - z_0)^mf(z))}{(m-1)!}
\]
is called the residue of $f$ at the pole $z_0$ of order $m$ and is denoted by 
$\Res(f,z_0)$.
\end{defn}

Now consider the Laurent expansion of $f$ about $z_0$,
\[
f(z) = \sum_{k=-m}^\infty a_k(z - z_0)^k
\]
where
\[
a_k = \frac{1}{2\pi i}\int_C \frac{f(w)}{(w - z_0)^{k+1}}dw.
\]
At $k = -1$
\begin{equation}\label{c5s3e2}
a_{-1} = \frac{1}{2\pi i}\int_C f(w)dw.
\end{equation}
Comparing equations \eqref{c5s3e1} and \eqref{c5s3e1} we get
\begin{equation}\label{c5s3e3}
\int_C f(z)dz = 2\pi i a_{-1} = \Res(f, z_0).
\end{equation}
Equation \eqref{c5s3e3} is an alternative definition of residue of a 
function at a pole $z_0$.

There is yet another way to get the same result. Since the Laurent series
is uniformly convergent,
\[
\int_C f(z)dz = \int_C \sum_{k=-m}^\infty a_k(z - z_0)^k = 
\int_C a_{-1}\frac{dz}{z - z_0}
\]
because the terms with all other powers vanish. The integral over $C$ is identical 
to that over a circle $\Gamma$ with centre $z_0$ and radius $r$. Thus,
\[
\int_C f(z)dz = a_{-1}\int_0^{2\pi}\frac{iRe^{i\theta}}{Re^{i\theta}}d\theta 
= 2\pi ia_{-1},
\]
which is same as equation \eqref{c5s3e3}. Since $2\pi ia_{-1}$ is the only
term that `remains' when one integrates the function term by term it is called
the `residue' of $f$ at $z_0$. It is easy to generalise this to the situation
when we integrate a function $f$ along a contour $C$ which encloses poles
$z_1, \ldots, z_k$. We then get
\begin{equation}\label{c5s3e4}
\int_C f(z)dz = 2\pi i \sum_{j=1}^k \Res(f, z_j).
\end{equation}

If instead of a pole of order $m$, $z_0$ were an essential singularity then
in a punctured neighbourhood of $z_0$, we have the Laurent expansion,
\[
f(z)dz = \sum_{k=-\infty}^\infty a_k(z - z_0)^k.
\]
Once again we can take advantage of the uniform convergence of the series to
integrate it term by term and conclude that
\begin{equation}\label{c5s3e5}
\int_C f(z)dz = 2\pi i a_{-1}.
\end{equation} 
Thus the integral is still expressed in terms of a residue but one cannot 
calculate it using a formula given in definition \ref{c5s3d1}.

The following lemma is useful while evaluation integrals using residues.
\begin{lem}\label{c5s3l1}
Let $\Gamma$ denote a semicircle in the upper half of the complex plane with
radius $R$ and centre at the origin. Let $f:\soc \rightarrow \soc$ be such that
$f(z) \rightarrow 0$ uniformly as $\abs{z} \rightarrow \infty$ when $0 \le \arg{z} 
\le \pi$. Then for $\alpha \ge 0$,
\[
\lim_{R \rightarrow \infty}\int_\Gamma e^{i\alpha z}f(z)dz = 0.
\]
\end{lem}
\begin{proof}
Note that the semi-circle does not include the real axis. Therefore, we find it
convenient to introduce polar coordinates $z = Re^{i\theta} = R\cos\theta + iR
\sin\theta$. Thus,
\[
I_R = \int_\Gamma e^{i\alpha z}f(z)dz = \int_0^\pi e^{i\alpha R\cos\theta}
e^{-\alpha R\sin\theta}f(Re^{i\theta})(iR)e^{i\theta}d\theta.
\]
Now,
\[
\abs{\int_C f(z)dz} \le \int_C\abs{f(z)}dz
\]
so that
\[
\abs{I_R} \le R\left(
\int_0^{\pi/2} e^{-\alpha R\sin\theta}\abs{f(Re^{i\theta})}d\theta + 
\int_{\pi/2}^\pi e^{-\alpha R\sin\theta}\abs{f(Re^{i\theta})}d\theta\right)
\] 
In the second integral let $\theta \mapsto \theta - \pi/2$ so that
\[
\abs{I_R} \le R\int_0^{\pi/2} e^{-\alpha R\sin\theta}\left(\abs{f(Re^{i\theta})}
+ \abs{f(Re^{i(\theta-\pi/2)})}\right)d\theta.
\]
$f \rightarrow 0$ uniformly as $R \rightarrow \infty$. That is, $f$ is a 
function of $R$ alone. Thus,
\[
\abs{I_R} \le R\abs{f(R)} \int_0^{\pi/2} e^{-\alpha R\sin\theta} d\theta.
\]
In the interval $[0, \pi/2]$, 
\[
\sin\theta \le \frac{2\theta}{\pi}
\]
so that 
\[
-\alpha R\sin\theta \ge -\alpha R \frac{2\theta}{\pi} \Rightarrow
e^{-\alpha R\sin\theta} \le e^{-2\alpha R\theta/\pi}
\]
and hence
\begin{eqnarray*}
\abs{I_R} &\le& R\abs{f(R)} \int_0^{\pi/2} e^{-2\alpha R\sin\theta/\pi} d\theta \\
&\le& R\abs{f(R)}\frac{\pi}{-2\alpha R} \left[e^{-2\alpha R\sin\theta/\pi}\right]_0^{\pi/2} \\
&\le& \frac{\pi}{2\alpha}\abs{f(R)}\left(1 - e^{-2\alpha R/\pi}\right).
\end{eqnarray*}
Since $\abs{f(R)} \rightarrow 0$ as $R \rightarrow \infty$,
\[
\lim_{R \rightarrow \infty}\abs{I_R} = 0.
\]
\end{proof}

A few remarks are in order.
\begin{enumerate}
\item The lemma is useful only when $\abs{f(z)}$ is a function of $\abs{z}$ alone.
\item The lemma is applicable for $\alpha < 0$ as well, except that the contour
should be in the lower half-plane. We should consider the sign of $\alpha$ while
choosing the contour.
\end{enumerate}

\section{Cauchy principle value}\label{c5s4}
Sometimes the integrand has a singularity on the path of integration. In such 
situations it we deform the path slightly to skirt the singularity. Consider 
an integral of the form
\begin{equation}\label{c5s4e1}
I = \int_{-\infty}^\infty \frac{f(x)}{x - x_0}dx,
\end{equation}
where the function $f$ is such that $\abs{f(z)} \rightarrow \infty$ as $\abs{z}
\rightarrow \infty$. One then considers the integral
\[
I_C = \int_C \frac{f(z)}{z - x_0}dz,
\]
where the closed contour $C$ consists of the segment $[-R, R]$ and the semi-circle
of radius $R$ and origin as the centre. However, we cannot apply the residue
theorem directly because the singularity is on the path of integration. We distort
the curve $C$ so that the segment $[-R, R]$ is replaced with the segment $[-R,x_0-r]$,
the semicircle $\gamma$ of radius $r$ and centre $x_0$ and the segment $[x_0+r, R]$.
The semicircle can either be in the upper half-plane or the lower one. Let us choose
it to be in the upper half. Now the distorted curve no longer encloses the singularity
and hence,
\[
\int_C f(z) dz = 0
\]
so that
\[
\int_{-R}^{x_0-r}\frac{f(x)}{x-x_0}dx + \int_\gamma\frac{f(z)}{z-x_0}dz +
\int_{x_0+r}^R\frac{f(x)}{x-x_0}dx + \int_\Gamma\frac{f(z)}{z-x_0}dz = 0.
\]
In the limit $R \rightarrow \infty$, the above equation becomes
\begin{equation}\label{c5s4e2}
\int_{-\infty}^{x_0-r}\frac{f(x)}{x-x_0}dx + \int_\gamma\frac{f(z)}{z-x_0}dz +
\int_{x_0+r}^\infty\frac{f(x)}{x-x_0}dx = 0.
\end{equation}
Now take the limit $r \rightarrow 0$. Along the curve $\gamma$, $z = x_0 + 
re^{i\theta}$ so that $dz = ire^{i\theta}$ and $\theta$ ranges from $\pi$ to
$0$. Thus,
\[
I_\gamma = \int_{-\pi}^0 \frac{f(x_0 + re^{i\theta})}{re^{i\theta}}ire^{i\theta}d\theta
= -i\int_0^\pi f(x+0 + re^{i\theta})d\theta.
\]
In the limit $r \rightarrow 0$, the value of the integrand is $f(x_0)$,
\[
I_\gamma = -i\pi f(x_0)
\]
and equation \eqref{c5s4e2} becomes
\[
\lim_{r\rightarrow 0}\left(\int_{-\infty}^{x_0-r}\frac{f(x)}{x-x_0}dx +
\int_{x_0+r}^\infty\frac{f(x)}{x-x_0}dx\right) -i\pi f(x_0) = 0
\]
or
\begin{equation}\label{c5s4e3}
\lim_{r\rightarrow 0}\left(\int_{-\infty}^{x_0-r}\frac{f(x)}{x-x_0}dx +
\int_{x_0+r}^\infty\frac{f(x)}{x-x_0}dx\right) = i\pi f(x_0).
\end{equation}
The expression on the left hand side of \eqref{c5s4e3} is called the \emph{
Cauchy Principal Value} and is denoted by
\[
\fint_{-\infty}^\infty \frac{f(x)}{x-x_0}dx
\]
Thus,
\begin{equation}\label{c5s4e4}
\fint_{-\infty}^\infty \frac{f(x)}{x-x_0}dx = i\pi f(x_0).
\end{equation}

If, on the other hand, we had chosen the semi-circle $\gamma$ to be in the
lower half-plane then the distorted contour would have enclosed the singularity
and the right hand side of equation \eqref{c5s4e2} would have become $2\pi if(x_0)$.
The limits of $I_\gamma$ would have been from $\pi$ to $2\pi$ and its value
would have been $i\pi f(x_0)$. The equation prior to \eqref{c5s4e3} would have
been
\[
\lim_{r\rightarrow 0}\left(\int_{-\infty}^{x_0-r}\frac{f(x)}{x-x_0}dx +
\int_{x_0+r}^\infty\frac{f(x)}{x-x_0}dx\right) + i\pi f(x_0) = 2\pi if(x_0)
\]
and we would have once again got equation \eqref{c5s4e4}.

\section{Problems and examples}\label{c5s5}
\begin{enumerate}
\item Evaluate \cite{dk}
\[
I = \int_0^\infty \frac{x^2dx}{(x^2 + 1)(x^2 + 4)}.
\]
Clearly,
\[
I = \frac{1}{2}\int_{\infty}^\infty \frac{x^2dx}{(x^2 + 1)(x^2 + 4)}.
\]
Consider
\[
I_C = \frac{1}{2}\int_C \frac{z^2dz}{(z^2 + 1)(z^2 + 4)},
\]
where $C$ is a contour comprising of the segment $[-R, R]$ and the semicircle centred
at the origin and of radius $R$. As $R \rightarrow \infty$, the integrand goes to
zero and $I_C \rightarrow I$. The integrand has simple poles $i, -i, 2i, -2i$ of which
only the first and the third are in the interior of $C$. Therefore,
\[
I = 2\pi i\left(\Res(f, i) + \Res(f, 2i)\right),
\]
where $f$ denotes the integrand. 
\[
\Res(f, i) = \lim_{z \rightarrow i}(z - i)f(z) = 
\lim_{z \rightarrow i}\frac{z^2}{(z+i)(z^2+4)} = \frac{-1}{6i}
\]
and
\[
\Res(f, 2i) = \lim_{z \rightarrow 2i}(z - 2i)f(z) = 
\lim_{z \rightarrow 2i}\frac{z^2}{(z^2+1)(z+2i)} = \frac{-4}{(-3)(4i)} = \frac{1}{3i}
\]
so that
\[
I = \frac{1}{2}\cdot 2\pi i \cdot \left(\frac{-1}{6i} + \frac{1}{3i}\right) = \frac{\pi}{6}.
\]

\item Evaluate \cite{dk}
\[
I = \int_0^{2\pi}\frac{d\theta}{1 + a\sin\theta}, 0 < a^2 \le 1.
\]
If $z = e^{i\theta}$ then $dz = ie^{i\theta}d\theta = izd\theta$ and
\[
\sin\theta = \frac{z + \bar{z}}{2i} = \frac{1}{2i}\left(z - \frac{1}{z}\right) =
\frac{1}{2i}\frac{z^2 - 1}{z}
\]
so that
\begin{equation}\label{c5s5e1}
I = 2\int_C \frac{dz}{az^2+2iz-a},
\end{equation}
where $C$ is the unit circle around the origin. The integrand has two simple poles
\[
z_1 = -\frac{i}{a}(1 + \sqrt{1 - a^2}), z_2 = -\frac{i}{a}(1 - \sqrt{1 - a^2}).
\]
Since $0 < a^2 \le 1$, we also have $a^2 < a$ so that $1 + a^2 - 2a < 1 - a^2$ and
\[
1 - a < \sqrt{1 - a^2} \Rightarrow 1 - \sqrt{1 - a^2} < a \Rightarrow
\frac{1 - \sqrt{1 - a^2}}{a} < 1.
\]
Therefore, $z_2$ lies inside $C$.

Since $a^2 < a$, $-2a^2 > -2a$, $1 - 2a^2 > 1 - 2a$ or $1 - a^2 > (1 - a)^2$ so that
\[
\sqrt{1 - a^2} > 1 - a > a - 1 \Rightarrow \frac{ 1 + \sqrt{1 - a^2}}{a} > 1
\]
so that $z_1$ lies outside $C$. Therefore
\[
I = 2\pi i \cdot 2 \cdot \Res(f, z_2),
\]
where the additional factor of $2$ comes from \eqref{c5s5e1}. The integrand can be 
written as 
\[
f(z) = \frac{1}{a}\frac{1}{(z-z_1)(z-z_2)}
\]
so that the residue is
\[
\frac{1}{a}\frac{1}{z_2 - z_1} = \frac{1}{2i\sqrt{1-a^2}}
\]
and hence
\[
I = \frac{2\pi}{\sqrt{1 - a^2}}.
\]

\item Evaluate 
\[
I = \int_0^{2\pi}\exp\left(-\frac{2\cos\theta}{a}\right)
\cos\left(\theta + \frac{2\sin\theta}{a}\right)d\theta.
\]
We observe that
\begin{eqnarray*}
I &=& \re\left(\int_0^{2\pi}\exp\left(-\frac{2\cos\theta}{a}\right)
\exp\left(i\left(\theta + \frac{2\sin\theta}{a}\right)\right)d\theta\right) \\
 &=& \re\left(\int_0^{2\pi}e^{i\theta}\exp\left(-\frac{2}{a}e^{-i\theta}\right)d\theta\right)
\end{eqnarray*}
If $z = e^{i\theta}$ then
\[
I = \re\left(-i\int_C\exp\left(-\frac{2}{az}\right) dz \right),
\]
where $C$ is the unit circle with centre at the origin. The Laurent expansion of
the integrand is
\[
\exp\left(-\frac{2}{az}\right) = 1 - \frac{2}{az} + \sum_{k \ge 2}\frac{1}{k!}\left(\frac{2}{a}\right)^kz^{-k}
\]
so that its residue at the origin is $-2/a$ and hence
\[
I = 2\pi i \cdot (-i) \cdot \frac{-2}{a} = -\frac{4\pi}{a}.
\]

Since
\[
\im\left(-i\int_C\exp\left(-\frac{2}{az}\right) dz \right) = 0,
\]
we have
\[
\int_0^{2\pi}\exp\left(-\frac{2\cos\theta}{a}\right)
\cos\left(\theta + \frac{2\sin\theta}{a}\right)d\theta = 0.
\]

\item Evaluate
\[
I = \int_0^\infty \frac{\sin x}{x}dx.
\]
The integrand is an even function about the origin, therefore,
\[
I = \frac{1}{2}\int_{-\infty}^\infty \frac{\sin x}{x}dx =
\frac{1}{2}\im\left(\int_{-\infty}^\infty \frac{e^{ix}}{x}dx\right).
\]
Consider the integral,
\[
I_1 = \int_C \frac{e^{iz}}{z}dz,
\]
where $C$ is the contour made up of the segment $[-R, R]$ and a semi-circle of
radius $R$ centred at the origin. The integrand on the semi-circle is
\[
\frac{e^{iR\cos\theta}e^{-R\sin\theta}}{iRe^{i\theta}}.
\]
As $R \rightarrow \infty$, the integrand tends to zero and the integral is
effectively along the entire real axis.

The integrand has a singularity at the origin, which lies on the contour $C$.
Therefore, we cannot use equation \eqref{c5s3e5}.We therefore change $C$ so that
the segment $[-R, R]$ is replaced with a segment $[-R, -r]$, a semicircle in
the upper half plane of radius $r$ and centre $0$ and the segment $[r, R]$. We
will evaluate the integral along the little semi-circle, $\gamma$, of radius 
$r$.
\[
I_\gamma = 
\int_\gamma \frac{e^{iz}}{z}dz = \int_{\pi}^0\frac{1}{z} 
\sum_{k \ge 0}\frac{(iz)^k}{k!} dz.
\]
Note that the limits of integral are from $\pi$ to $0$ because $\gamma$ is
traversed in the clockwise direction. On $\gamma$, $z = re^{i\theta}$ so that
\[
I_\gamma = \int_{\pi}^0 \frac{1}{re^{i\theta}}\sum_{k \ge 0}
\frac{1}{k!}(ire^{i\theta})^k ire^{i\theta}d\theta = 
-i\int_0^\pi \left(1 + \sum_{k \ge 1}\frac{i^k r^k e^{ik\theta}}{k!}\right)
d\theta.
\]
Taking the limit $r \rightarrow 0$, we have
\[
\lim_{r \rightarrow 0}I_\gamma = -i\pi.
\]
Now,
\[
\int_C \frac{e^{iz}}{z}dz = 0
\]
because the contour $C$ was changed to avoid the singularity at $0$. In the
limit $R \rightarrow \infty$ and $r \rightarrow 0$,
\[
\int_{-\infty}^\infty \frac{\sin x}{x}dx - i\pi = 0
\]
so that
\[
\int_0^\infty \frac{\sin x}{x}dx = \frac{\pi}{2}.
\]
\end{enumerate}
 
