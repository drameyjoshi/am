\chapter{Conformal Mappings - 1}\label{c3}
\section{Complex functions as mappings}\label{c3s1}
A complex function of complex variable maps a point on the complex plane to 
another point on the complex plane. A curve $C$ in the domain is transformed
to a curve $C^\prime$ in the range. Likewise, the point of intersection of
curves $C_1$ and $C_2$ in the domain is mapped to the point of intersection of
the images $C_1^\prime$ and $C_2^\prime$. Let $\nu$ be the angle between $C_1$
and $C_2$ at the point of their intersection. Let $\nu^\prime$ be the angle
between their images $C_1^\prime$ and $C_2^\prime$ at the point of their 
intersection. A mapping $z^\prime(z)$ is said to be \emph{angle-preserving} if
$\nu = \nu^\prime$.

Angle-preserving mappings are also called \emph{conformal} mappings.

\begin{thm}\label{c3s1t1}
If $D$ is the domain of the mapping $z^\prime(z) = u(x,y) + iv(x,y)$ and if
for all points $z \in D$,
\[
\frac{dz^\prime}{dz} \ne 0
\]
then the mapping $z^\prime$ is angle-preserving.
\end{thm}
\begin{proof}
Let $z_0$ be the point of intersections of two curves $C_1$ and $C_2$ in the
$z$-plane. Under the transformation $z^\prime$, the two curves are mapped to
$C_1^\prime$ and $C_2^\prime$ on the $z^\prime$ plane. Let them intersect at
$z_0^\prime$.

Let $z_1$ be an arbitrary point on $C_1$ and $z_1^\prime$ be its map on 
$C_1^\prime$. Let
\begin{eqnarray*}
z_1 - z_0 &=& r_1e^{i\theta} \\
z_1^\prime - z_0^\prime &=& r_1^\prime e^{i\theta^\prime} 
\end{eqnarray*}
so that
\[
\frac{z_1^\prime - z_0^\prime}{z_1 - z_0} = 
\frac{r^\prime}{r}e^{i(\theta^\prime- \theta)}
\]
As $z \rightarrow z_0$, the line $z - z_0$ tends to the tangent to the curve
$C_1$ at $z_0$. The angle $\theta$ then tends to $\alpha$ the angle made by
the tangent to $C_1$ with the real axis. Likewise, the angle $\theta^\prime
\rightarrow \alpha^\prime$ the angle made by the tangent to $C_1^\prime$ at
$z_0^\prime$. We can then write
\[
\frac{dz^\prime}{dz} = \abs{\frac{dz^\prime}{dz}}e^{i(\alpha^\prime - \alpha)}
\]
By hypothesis, this derivative is not zero and hence its magnitude has a 
non-zero value, say $d$. Thus,
\[
\frac{dz^\prime}{dz} = de^{i(\alpha^\prime - \alpha)}.
\]
If we carry out the same analysis for the pair $C_2$ and $C_2^\prime$ then we
get
\[
\frac{dz^\prime}{dz} = de^{i(\beta^\prime - \beta)},
\]
where $\beta$ and $\beta^\prime$ are angles made by tangent to the curves
$C_2$ and $C_2^\prime$ at $z_0$ and $z_0^\prime$. As the derivative of a
function is unique, we have
\[
\alpha^\prime - \alpha = \beta^\prime - \beta \Rightarrow \alpha - \beta = 
\beta^\prime - \alpha^\prime.
\] 
The result follows from the fact that $\nu = \abs{\alpha - \beta}$ and 
$\nu^\prime = \abs{\alpha^\prime - \beta^\prime}$.
\end{proof}

Consider the problem of integration of a complex function $f$ along a countour
$C$ whose parametric equation is $z = z(t)$. Let the contour $C$ be limited
by values $t_1$ and $t_2$ of the parameter $t$. Let
\[
I = \int_C f(z)dz
\]
and let $z^\prime$ be a conformal transformation in a domain $D$ in which the
coutour $C$ lies. Therefore,
\[
\frac{dz^\prime}{dz} \ne 0
\]
throughout $D$ and hence
\[
\int_C f(z)dz = \int_{C^\prime}f(z(z^\prime))\frac{dz}{dz^\prime}dz^\prime.
\]
Such a transformation of the integral is permitted only if one can write the
inverse function $z(z^\prime)$ and when $dz/dz^\prime$ exist. Their existence
is guaranteed by the conformal nature of $z^\prime$.

Consider a curve whose parametric representation is $(x(t), y(t))$, $t$ 
being a parameter. If $z(t) = x(t) + iy(t)$ then the variable $z$ represents
a curve in the complex plane. If $x$ and $y$ are also differentiable functions
of $t$ then
\[
\frac{dz}{dt} = \frac{dx}{dt} + i\frac{dy}{dt}.
\]
If $f$ is an analytic function in the domain $D$ then for the range of the
parameter $t$ lying in the domain, if $w(z) = f(z)$ then $w(t) = f(z(t))$ and
\[
\frac{dw}{dt} = f^\prime(z_0)\frac{dz}{dt}
\]
and hence
\begin{equation}\label{c3s1e1}
\arg(w^\prime) = \arg(f^\prime) + \arg(z^\prime).
\end{equation}
Now, $z^\prime(t_0)$ is the tangent to the curve at $t_0$, $w^\prime(t_0)$ is
the tangent to the image of the curve at $t_0$. The argument of these complex
quantities is the angle they make with the positive real axis. The difference 
between the angles made by the curve and its image under $f$ is quantity
$\arg(f^\prime)$. If $z_1(t)$ is another curve and $w_1$ is its image under
the same function then
\begin{equation}\label{c3s1e2}
\arg(w_1^\prime) = \arg(f^\prime) + \arg(z_1^\prime).
\end{equation}
From equations \eqref{c3s1e1} and \eqref{c3s1e2},
\begin{equation}\label{c3s1e3}
\arg(w^\prime) - \arg(w_1^\prime) = \arg(z^\prime) - \arg(z_1^\prime).
\end{equation}
This is another proof of theorem \ref{c3s1t1} \cite{ablowitz2003complex}.

\section{Problems and examples}\label{c3s2}
\begin{enumerate}
\item Consider the transformation $w = z^{-1}$. It is analytic throughout the
complex plane except at the origin. If $w = u + iv$ then
\begin{eqnarray*}
u &=& \frac{x}{x^2 + y^2} \\
v &=& -\frac{y}{x^2 + y^2}.
\end{eqnarray*}
Consider the line $x = c_1$. It transforms as
\begin{eqnarray*}
u &=& \frac{c_1}{c_1^2 + y^2} \\
v &=& -\frac{y}{c_1^2 + y^2}
\end{eqnarray*}
so that
\[
\frac{u}{v} = -\frac{c_1}{y} \rightarrow y = -c_1\frac{v}{u}.
\]
Substituting for $y$ in the previous pair of equations we get
\begin{eqnarray*}
u &=& \frac{c_1 u^2}{c_1^2u^2 + c_1^2v^2} \\
  &=& \frac{u^2}{c_1u^2 + c_1 v^2} \\
  &=& c_1u^2 + c_1v^2
\end{eqnarray*}
which can be written as
\[
\left(u - \frac{1}{2c_1}\right)^2 + v^2 = \frac{1}{4c_1^2}.
\]
This is the equation of a circle of radius $1/(2c_1)$ and centre $(1/(2c_1), 0)$.
The imaginary axis is tangential to it at the origin.

Now consider the line $y = c_2$. It transforms as
\begin{eqnarray*}
u &=& \frac{x}{x^2 + c_2^2} \\
v &=& -\frac{c_2}{x^2 + c_2^2}
\end{eqnarray*}
from which we get
\[
\frac{u}{v} = -\frac{x}{c_2} \Rightarrow x = -c_2\frac{u}{v}.
\]
Substituting for $x$ in the previous pair of equations we get
\[
u^2 + \left(v + \frac{1}{2c_2}\right)^2 = \frac{1}{4c_2^2}.
\]
This is the equation of a circle with radius $1/(2c_2)$ and centre $(0, -1/(2c_2)$.
The real axis is tangential to it at the origin.

\item Find a linear transformation that maps the circle $\abs{z - 1} = 1$ to the
circle $\abs{w - 3i/2} = 2$.

A linear transformation has the form $w = \alpha z + \beta$ where $\alpha, \beta
\in \soc$. Let $\alpha = ae^{i\theta}$ then
\[
w = a\left(e^{i\theta}z - \frac{\beta}{a}\right).
\]
Thus, a linear transformation involves,
\begin{enumerate}
\item Rotation by an angle $\theta$ followed by
\item Translation by a displacement $-\beta/a$ followed by
\item Scaling by a factor of $a$.
\end{enumerate}
In this example, there is no rotation. Therefore $\theta = 0$. There is a stretching
by a factor of $2$. When stretched, the circle $\abs{z - 1} = 1$ becomes $\abs{2z - 2}
= 2$ so that the translation is $3i/2 - 2$. Therefore, let
\[
w = 2z + \frac{3i}{2} - 2
\]
so that
\[
\abs{w - \frac{3i}{2}} = 2\abs{z - 1} = 2.
\]
\end{enumerate}
