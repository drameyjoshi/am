\chapter{Conformal Transformations}\label{c3}
\section{Complex functions as mappings}\label{c3s1}
A complex function of complex variable maps a point on the complex plane to 
another point on the complex plane. A curve $C$ in the domain is transformed
to a curve $C^\prime$ in the range. Likewise, the point of intersection of
curves $C_1$ and $C_2$ in the domain is mapped to the point of intersection of
the images $C_1^\prime$ and $C_2^\prime$. Let $\nu$ be the angle between $C_1$
and $C_2$ at the point of their intersection. Let $\nu^\prime$ be the angle
between their images $C_1^\prime$ and $C_2^\prime$ at the point of their 
intersection. A mapping $z^\prime(z)$ is said to be \emph{angle-preserving} if
$\nu = \nu^\prime$.

Angle-preserving mappings are also called \emph{conformal} mappings.

\begin{thm}\label{c3s1t1}
If $D$ is the domain of the mapping $z^\prime(z) = u(x,y) + iv(x,y)$ and if
for all points $z \in D$,
\[
\frac{dz^\prime}{dz} \ne 0
\]
then the mapping $z^\prime$ is angle-preserving.
\end{thm}
\begin{proof}
Let $z_0$ be the point of intersections of two curves $C_1$ and $C_2$ in the
$z$-plane. Under the transformation $z^\prime$, the two curves are mapped to
$C_1^\prime$ and $C_2^\prime$ on the $z^\prime$ plane. Let them intersect at
$z_0^\prime$.

Let $z_1$ be an arbitrary point on $C_1$ and $z_1^\prime$ be its map on 
$C_1^\prime$. Let
\begin{eqnarray*}
z_1 - z_0 &=& r_1e^{i\theta} \\
z_1^\prime - z_0^\prime &=& r_1^\prime e^{i\theta^\prime} 
\end{eqnarray*}
so that
\[
\frac{z_1^\prime - z_0^\prime}{z_1 - z_0} = 
\frac{r^\prime}{r}e^{i(\theta^\prime- \theta)}
\]
As $z \rightarrow z_0$, the line $z - z_0$ tends to the tangent to the curve
$C_1$ at $z_0$. The angle $\theta$ then tends to $\alpha$ the angle made by
the tangent to $C_1$ with the real axis. Likewise, the angle $\theta^\prime
\rightarrow \alpha^\prime$ the angle made by the tangent to $C_1^\prime$ at
$z_0^\prime$. We can then write
\[
\frac{dz^\prime}{dz} = \abs{\frac{dz^\prime}{dz}}e^{i(\alpha^\prime - \alpha)}
\]
By hypothesis, this derivative is not zero and hence its magnitude has a 
non-zero value, say $d$. Thus,
\[
\frac{dz^\prime}{dz} = de^{i(\alpha^\prime - \alpha)}.
\]
If we carry out the same analysis for the pair $C_2$ and $C_2^\prime$ then we
get
\[
\frac{dz^\prime}{dz} = de^{i(\beta^\prime - \beta)},
\]
where $\beta$ and $\beta^\prime$ are angles made by tangent to the curves
$C_2$ and $C_2^\prime$ at $z_0$ and $z_0^\prime$. As the derivative of a
function is unique, we have
\[
\alpha^\prime - \alpha = \beta^\prime - \beta \Rightarrow \alpha - \beta = 
\beta^\prime - \alpha^\prime.
\] 
The result follows from the fact that $\nu = \abs{\alpha - \beta}$ and 
$\nu^\prime = \abs{\alpha^\prime - \beta^\prime}$.
\end{proof}

Consider the problem of integration of a complex function $f$ along a countour
$C$ whose parametric equation is $z = z(t)$. Let the contour $C$ be limited
by values $t_1$ and $t_2$ of the parameter $t$. Let
\[
I = \int_C f(z)dz
\]
and let $z^\prime$ be a conformal transformation in a domain $D$ in which the
coutour $C$ lies. Therefore,
\[
\frac{dz^\prime}{dz} \ne 0
\]
throughout $D$ and hence
\[
\int_C f(z)dz = \int_{C^\prime}f(z(z^\prime))\frac{dz}{dz^\prime}dz^\prime.
\]
Such a transformation of the integral is permitted only if one can write the
inverse function $z(z^\prime)$ and when $dz/dz^\prime$ exist. Their existence
is guaranteed by the conformal nature of $z^\prime$.

\section{Application of conformal mappings}\label{c3s2}
